% Este trabalho está licenciado sob a Licença Atribuição-CompartilhaIgual 4.0 Internacional Creative Commons. Para visualizar uma cópia desta licença, visite http://creativecommons.org/licenses/by-sa/4.0/deed.pt_BR ou mande uma carta para Creative Commons, PO Box 1866, Mountain View, CA 94042, USA.

\chapter{Arranjos e Matrizes}\label{cap_arr}
\thispagestyle{fancy}

\hl{Um arranjo é uma coleção de objetos} (todos de um mesmo tipo) \hl{em que os elementos são organizados por eixos}. É a estrutura de dados mais utilizada para a alocação de vetores e matrizes, fundamentais na computação matricial.

\section{Arranjos}\label{cap_arr_sec_arr}

\hl{Um arranjo (em inglês, \textit{array}) é uma coleção de objetos (todos do mesmo tipo) em que os elementos são organizados por eixos}. Nesta seção, vamos nos restringir a \hl{\emph{arranjos unidimensionais}} (de apenas um eixo). Esta é a estrutura computacionais usualmente utilizada \hl{para a alocação de vetores}.

\hl{{\numpy} é uma biblioteca {\python} que fornece suporte para a alocação e manipulação de arranjos}. Usualmente, a biblioteca é importada como segue
\begin{lstlisting}
import numpy as np
\end{lstlisting}
Na sequência, vamos assumir que o {\numpy} já está importado como acima.

\subsection{Alocação de Arranjos}

Na linguagem, a \hl{alocação de um arranjo} pode ser feita com o método \hl{{\href{https://numpy.org/doc/stable/reference/generated/numpy.array.html}{\lstinline+np.array(list)+}}}. Como parâmetro de entrada, recebe uma \lstinline+list+ contendo os elementos do arranjo. Por exemplo,
\begin{lstlisting}
>>> v = np.array([-2, 1, 3])
>>> v
array([-2,  1,  3])
>>> type(v)
<class 'numpy.ndarray'>
\end{lstlisting}
aloca o arranho de números inteiros \lstinline+v+. Embora arranjos não sejam vetores, \hl{a modelagem computacional de vetores usualmente é feita utilizando-se {\lstinline+arrays+}}. Por exemplo, em um código {\python}, o vetor
\begin{equation}
  \pmb{v} = (-2, 1, 3)
\end{equation}
pode ser alocado usando-se o \lstinline+array+ \lstinline+v+ acima.

O \hl{tipo dos dados} de um \lstinline+array+ é definido na sua criação. Pode ser feita de forma automática ou explícita pela propriedade \hl{{\href{https://numpy.org/doc/stable/reference/arrays.dtypes.html}{\lstinline+dtype+}}}. Por exemplo,
\begin{lstlisting}
>>> v = np.array([-2, 1, 3])
>>> v.dtype
dtype('int64')
>>> v = np.array([-2., 1, 3])
>>> v.dtype
dtype('float64')
>>> v = np.array([-2, 1, 3], dtype='float')
>>> v.dtype
dtype('float64')
\end{lstlisting}

\begin{ex}
  Aloque o vetor
  \begin{equation}
    \pmb{v} = (\pi, 1, e)
  \end{equation}
  como um \lstinline+array+ do {\numpy}.
\begin{lstlisting}
>>> import numpy as np
>>> v = np.array([np.pi, 1, np.e])
>>> v
array([3.14159265, 1.        , 2.71828183])
\end{lstlisting}
\end{ex}

O {\numpy} conta com métodos úteis para a \hl{\emph{inicialização} de {\lstinline+arrays+}}:
\begin{itemize}
\item \hl{{\lstinline+np.zeros()+}} : arranjo de elementos nulos.
\begin{lstlisting}
>>> np.zeros(3)
array([0., 0., 0.])
\end{lstlisting}
\item \hl{{\lstinline+np.ones()+}} : arranjo de elementos iguais a um.
\begin{lstlisting}
>>> np.ones(2, dtype='int')
array([1, 1])
\end{lstlisting}
\item \hl{{\lstinline+np.empty()+}} : arranjo de elementos não predefinidos.
\begin{lstlisting}
>>> np.empty(3)
array([4.64404327e-310, 0.00000000e+000, 6.93315702e-310])
\end{lstlisting}
\item \hl{{\lstinline+np.linspace(start, stop, num=50)+}} : arranjo de elementos uniformemente espaçados.
\begin{lstlisting}
>>> np.linspace(0, 1, 5)
array([0.  , 0.25, 0.5 , 0.75, 1.  ])
\end{lstlisting}
\end{itemize}

\subsection{Indexação e Fatiamento}\label{cap_arr_sec_arr:ssec:islice}

\hl{Um {\lstinline+array+} é uma coleção de objetos mutável, ordenada e indexada}. Indexação e fatiamento podem ser feitos da mesma forma que para \lstinline+tuples+ e \lstinline+lists+. Por exemplo,
\begin{lstlisting}
>>> v = np.array([-1, 1, 2, 0, 3])
>>> v[0]
-1
>>> v[-1]
3
>>> v[1:4]
array([1, 2, 0])
>>> v[::-1]
array([ 3,  0,  2,  1, -1])
>>> v[3] = 4
>>> v
array([-1,  1,  2,  4,  3])
\end{lstlisting}

\subsection{Reordenamento dos Elementos}

Em programação, o reordenamento (em inglês, \textit{sorting}) de elementos de uma sequência ordenada de números (\lstinline+array+, \lstinline+tuple+, \lstinline+list+, etc.) consiste em alterar a sequência de forma que os elementos sejam organizados do menor para o mair valor. Na sequência, vamos estudar alguns métodos para isso.

\subsubsection{Método Bolha}

Dado um \lstinline+array+\footnote{Ou, um \lstinline+tuple+, \lstinline+list+, etc..}, o método bolha consiste em percorrer o arranjo e permutar dois elementos consecutivos de forma que o segundo seja sempre maior que o primeiro. Uma vez que percorrermos o arranjo, teremos garantido que o maior valor estará na última posição do arranjo e os demais elementos ainda poderão estar desordenados. Então, percorremos o arranjo novamente, permutando elementos dois-a-dois conforme a ordem desejada, o que trará o segundo maior elemento para a penúltima posição. Ou seja, para um arranjo com $n$ elementos, temos garantido o reordenamento de todos os elementos após $n-1$ repetições desse algoritmo.

\begin{ex}
  Na sequência, implementamos o Método Bolha para o reordenamento de arranjos e aplicamos para
  \begin{equation}
    \pmb{v} = (-1, 1, 0, 4, 3).
  \end{equation}
  
\begin{lstlisting}[caption=bubbleSort\_v1.py]
import numpy as np

def bubbleSort(arr):
    arr = arr.copy()
    n = len(arr)
    for k in range(n-1):
        for i in range(n-k-1):
            if (arr[i] > arr[i+1]):
                arr[i], arr[i+1] = arr[i+1], arr[i]
    return arr

v = np.array([-1,1,0,4,3])
w = bubbleSort(v)
print(w)
\end{lstlisting}
\end{ex}

\begin{obs}
  Em geral, para um arranjo de $n$ elementos, o Método Bolha requer $n-1$ repetições para completar o ordenamento. Entretanto, dependendo do caso, o ordenamento dos elementos pode terminar em menos passos.
\end{obs}

\begin{ex}
  Na sequência, implementamos uma nova versão do Método Bolha para o reordenamento de arranjos. Esta versão verifica se há elementos fora de ordem e, caso não haja, interrompe o algoritmo. Como exemplo, aplicamos para
  \begin{equation}
    \pmb{v} = (-1, 1, 0, 4, 3).
  \end{equation}
  
\begin{lstlisting}[caption=bubbleSort\_v2.py]
import numpy as np

def bubbleSort(arr):
    arr = arr.copy()
    n = len(arr)
    for k in range(n-1):
        noUpdated = True
        for i in range(n-k-1):
            if (arr[i] > arr[i+1]):
                arr[i], arr[i+1] = arr[i+1], arr[i]
                noUpdated = False
        if (noUpdated):
            break
    return arr

v = np.array([-1,1,0,4,3])
w = bubbleSort(v)
\end{lstlisting}
\end{ex}

\begin{obs}\normalfont{(\hl{Métodos de Ordenamento}.)}
  Existem vários métodos para o ordenamento de uma sequência. O Método Bolha é um dos mais simples, mas também, em geral, menos eficiente. O {\numpy} tem disponível a função \hl{{\href{https://numpy.org/doc/stable/reference/generated/numpy.sort.html}{\lstinline+np.sort(arr)+}}} para o reordenamento de elementos. Também bastante útil, é a função \hl{{\href{https://numpy.org/doc/stable/reference/generated/numpy.argsort.html\#numpy.argsort}{\lstinline+np.argsort(arr)+}}}, que retorna os índices que reordenam os elementos.
\end{obs}

\subsection{Operações Elemento-a-Elemento}

No {\numpy}, temos os \hl{operadores aritméticos elemento-a-elemento} (em ordem de precedência)
\begin{itemize}
\item \hl{{\lstinline!**!}}
\begin{lstlisting}
>>> v = np.array([-2., 1, 3])
>>> w = np.array([1., -1, 2])
>>> v ** w
array([-2.,  1.,  9.])
\end{lstlisting}
\item \hl{{\lstinline!*!}, {\lstinline!/!}, {\lstinline!//!}}, \lstinline!%!
\begin{lstlisting}
>>> v * w
array([-2., -1.,  6.])
>>> v / w
array([-2. , -1. ,  1.5])
>>> v // w
array([-2., -1.,  1.])
>>> v % w
array([ 0., -0.,  1.])
\end{lstlisting}
\item \hl{{\lstinline!+!}, {\lstinline!-!}}
\begin{lstlisting}
>>> v + w
array([-1.,  0.,  5.])
>>> v - w
array([-3.,  2.,  1.])
\end{lstlisting}
\end{itemize}

\begin{ex}
  Vamos usar \lstinline+arrays+ para alocar os vetores
  \begin{align}
    \pmb{v} = (1., 0, -2),\\
    \pmb{w} = (2., -1, 3).
  \end{align}
  Então, computamos o produto interno
  \begin{subequations}
    \begin{align}
      \pmb{v}\cdot\pmb{w} &:= v_1w_1 + v_2w_2 + v_3w_3\\
                          &= 1\cdot 2 + 0\cdot(-1) + (-2)\cdot 3\\
                          &= -4.
    \end{align}
  \end{subequations}
\begin{lstlisting}
import numpy as np
# vetores
v = np.array([1., 0, -2])
w = np.array([2., -1, 3])
# produto interno
vdw = np.sum(v*w)
\end{lstlisting}
\end{ex}

\begin{obs}\normalfont{(\hl{Concatenação de Arranjos}.)}
  No {\numpy}, a concatenação de arranjos pode ser feita com a função \hl{{\href{https://numpy.org/doc/stable/reference/generated/numpy.concatenate.html}{np.concatenate()}}}. Por exemplo,
\begin{lstlisting}
>>> v = np.array([1,2])
>>> w = np.array([3,4])
>>> np.concatenate((v,w))
array([1, 2, 3, 4])
\end{lstlisting}
\end{obs}

\subsection{Exercícios}

\begin{exer}
  Aloque os seguintes vetores como \lstinline+array+ do {\numpy}:
  \begin{enumerate}[a)]
  \item $\displaystyle\pmb{a} = (0, -2, 4)$
  \item $\displaystyle\pmb{b} = (0.1, -2.7, 4.5)$
  \item $\displaystyle\pmb{c} = (e, \ln(2), \pi)$
  \end{enumerate}
\end{exer}
\begin{resp}
\begin{lstlisting}
>>> import numpy as np
>>> a = np.array([0, -2, 4])
>>> b = np.array([0.1, -2.7, 4.5])
>>> c = np.array([np.e, np.log(2), np.pi])
\end{lstlisting}
\end{resp}

\begin{exer}
  Considere o seguinte \lstinline+array+
\begin{lstlisting}
>>> v = np.array([4, -1, 1, -2, 3]).
\end{lstlisting}
  Sem implementar, escreva os arranjos derivados:
  \begin{enumerate}[a)]
  \item \lstinline+v[1]+
  \item \lstinline+v[1:4]+
  \item \lstinline+v[:3]+
  \item \lstinline+v[1:]+
  \item \lstinline+v[1:4:2]+
  \item \lstinline+v[-2:-5:-1]+
  \item \lstinline+v[::-2]+
  \end{enumerate}
  Então, verifique seus resultados implementando-os.
\end{exer}
\begin{resp}
  Dica: consulte a Subseção \ref{cap_arr_sec_arr:ssec:islice}. 
\end{resp}

\begin{exer}
  Desenvolva uma função \lstinline+argBubbleSort(arr)+, i.e. uma função que retorna os índices que reordenam os elementos do arranjo \lstinline+arr+ em ordem crescente. Teste seu código para o ordenamento de diversos arranjos e compare os resultados com a aplicação da função \lstinline+np.argsort(arr)+.
\end{exer}
\begin{resp}
\begin{lstlisting}
import numpy as np

def argBubbleSort(arr):
    n = len(arr)
    ind = np.arange(n)
    for k in range(n-1):
        noUpdated = True
        for i in range(n-k-1):
            if (arr[ind[i]] > arr[ind[i+1]]):
                ind[i], ind[i+1] = ind[i+1], ind[i]
                noUpdated = False
        if (noUpdated):
            break
    return ind
\end{lstlisting}
\end{resp}

\begin{exer}
  Desenvolva um Método Bolha para o reordenamento dos elementos de um dado arranjo em ordem decrescente. Teste seu código para o reordenamento de diversos arranjos. Como pode-se usar a função \lstinline+np.sort(arr)+ para obter os mesmos resultados?
\end{exer}
\begin{resp}
\begin{lstlisting}
import numpy as np

def emOrdem(x, y):
    return x < y

def bubbleSort(arr, emOrdem=emOrdem):
    arr = arr.copy()
    n = len(arr)
    for k in range(n-1):
        noUpdated = True
        for i in range(n-k-1):
            if not(emOrdem(arr[i], arr[i+1])):
                arr[i], arr[i+1] = arr[i+1], arr[i]
                noUpdated = False
        if (noUpdated):
            break
    return arr
\end{lstlisting}
\end{resp}

\begin{exer}
  Desenvolva uma função \lstinline+argBubbleSort(arr, emOrdem)+, i.e. uma função que retorna os índices que reordenam os elementos do arranjo \lstinline+arr+ na ordem definida pela função \lstinline+emOrdem+. Teste seu código para o ordenamento de diversos arranjos, tanto em ordem crescente como em ordem decrescente. Como pode-se obter os mesmos resultados usando-se a função \lstinline+np.sort(arr)+?
\end{exer}
\begin{resp}
\begin{lstlisting}
import numpy as np

def argBubbleSort(arr, emOrdem=emOrdem):
    n = len(arr)
    ind = np.arange(n)
    for k in range(n-1):
        noUpdated = True
        for i in range(n-k-1):
            if not(emOrdem(arr[ind[i]], arr[ind[i+1]])):
                ind[i], ind[i+1] = ind[i+1], ind[i]
                noUpdated = False
        if (noUpdated):
            break
    return ind
\end{lstlisting}
\end{resp}

\begin{exer}
  Crie uma função \lstinline+media(arr)+ que returna o valor médio do arranjo de números \lstinline+arr+. Teste seu código para diferentes arranjos e compare os resultados com o da função \lstinline+np.mean(arr)+.
\end{exer}
\begin{resp}
\begin{lstlisting}
import numpy as np

def media(arr):
    return np.sum(arr)/len(arr)
\end{lstlisting}
\end{resp}

\begin{exer}
  Desenvolva uma função que retorna o ângulo entre dois vetores $\pmb{v}$ e $\pmb{w}$ dados.
\end{exer}
\begin{resp}
\begin{lstlisting}
import numpy as np

def dot(v, w):
    return np.sum(v*w)

def angulo(v, w):
    # norma de v
    norm_v = np.sqrt(dot(v,v))
    # norma de w
    norm_w = np.sqrt(dot(w,w))
    # cos(theta)
    cosTheta = dot(v,w)/(norm_v*norm_w)
    # theta
    theta = np.acos(cosTheta)
    return theta
\end{lstlisting}
\end{resp}


\section{Vetores e Arranjos}\label{cap_arr_sec_vetor}

\hl{O uso de {\lstinline+arrays+} é uma das formas mais adequadas para fazermos a modelagem computacional de \emph{vetores}}. Entretanto, devemos ficar atentos que \hl{vetores e arranjos não são equivalentes}. Embora, a soma/subtração e multiplicação por escalar são similares, a multiplicação e potenciação envolvendo vetores não estão definidas, mas para arranjos são operações elemento-a-elemento.

No que segue, vamos assumir que a biblioteca {\numpy} está importada, i.e.
\begin{lstlisting}
>>> import numpy as np
\end{lstlisting}

\begin{ex}
  Podemos alocar os vetores
  \begin{align}
    &\pmb{v} = (1, 0, -2),\\
    &\pmb{w} = (2, -1, 3),
  \end{align}
  como \lstinline+arrays+ do {\numpy}
\begin{lstlisting}
>>> v = np.array([1, 0, -2])
>>> w = np.array([2, -1, 3])
\end{lstlisting}

  A soma dos vetores é uma operação elemento-a-elemento
  \begin{subequations}
    \begin{align}
      \pmb{v}+\pmb{w} &= (1, 0, -2) + (2, -1, 3)\\
                      &= \left(1+2, 0+(-1), -2+3\right)\\
                      &= (3, -1, 1)
    \end{align}
  \end{subequations}
  e a dos \lstinline+arrays+ é equivalente
\begin{lstlisting}
>>> v+w
array([ 3, -1,  1])
\end{lstlisting}

  A subtração dos vetores também é uma operação elemento-a-elemento
  \begin{subequations}
    \begin{align}
      \pmb{v}-\pmb{w} &= (1, 0, -2) - (2, -1, 3)\\
                      &= \left(1-2, 0-(-1), -2-3\right)\\
                      &= (-1, 1, -5)
    \end{align}
  \end{subequations}
  e a dos \lstinline+arrays+ é equivalente
\begin{lstlisting}
>>> v-w
array([ -1, 1,  -5])
\end{lstlisting}

  Ainda, a multiplicação por escalar
  \begin{subequations}
    \begin{align}
      2\pmb{v} &= 2(1, 0, -2)\\
               &= \left(2\cdot 1, 2\cdot 0, 2\cdot(-2)\right)\\
               &= (2, 0, -4)
    \end{align}
  \end{subequations}
  também é feita elemento-a-elemento, assim como com \lstinline+arrays+
\begin{lstlisting}
>>> 2*v
array([ 2,  0, -4])
\end{lstlisting}

  Agora, para vetores, a multiplicação $\pmb{v}\pmb{w}$, divisão $\pmb{v}/\pmb{w}$, potenciação $\pmb{v}^2$ não são operações definidas. Diferentemente, para arranjos são operações elemento-a-elemento
\begin{lstlisting}
>>> v*w
array([ 2,  0, -6])
>>> v/w
array([ 0.5       , -0.        , -0.66666667])
>>> v**2
array([1, 0, 4])
\end{lstlisting}
\end{ex}

\subsection{Funções Vetoriais}

Funções vetoriais $f:\mathbb{R}^n\to\mathbb{R}^n$ e funcionais $f:\mathbb{R}^n\to\mathbb{R}$ também podem ser adequadamente modeladas com o emprego de \lstinline+arrays+ do \hl{{\numpy}}. A biblioteca também \hl{conta com várias funções matemáticas predefinidas}, consulte
\begin{center}
  \url{https://numpy.org/doc/stable/reference/routines.math.html}
\end{center}

\begin{ex}\normalfont{(\hl{Função Vetorial}.)}
  A implementação da função vetorial $\pmb{f}:\mathbb{R}^3\to\mathbb{R}^3$
  \begin{equation}
    \pmb{f}(\pmb{x}) = (x_1^2+\sin(x_1), x_2^2+\sin(x_2), x_3^2+\sin(x_3))
  \end{equation}
  para $\pmb{x} = (x_1, x_2, x_3)\in\mathbb{R}^3$, pode ser feita da seguinte forma
\begin{lstlisting}
import numpy as np

def f(x):
    return x**2 + np.sin(x)

# exemplo
x = np.array([0, np.pi, np.pi/2])
print(f'y = {f(x)}')
\end{lstlisting}
  Verifique!
\end{ex}

\subsection{Produto Interno}

O \hl{\emph{produto interno}} (ou, produto escalar) é a operação entre vetores $\pmb{v},\pmb{w}\in\mathbb{R}^n$ definida por
\begin{equation}
  \pmb{v}\cdot\pmb{v} := v_1w_1+v_2w_2+\cdots+v_nw_n.
\end{equation}
A função\hl{{\href{https://numpy.org/doc/stable/reference/generated/numpy.dot.html}{numpy.dot(v,w)}}} computa o produto interno dos \lstinline+arrays+.

\begin{ex}
  Consideramos os vetores
  \begin{align}
    &\pmb{v} = (1, 0, -2),\\
    &\pmb{w} = (2, -1, 3).
  \end{align}
  O produto interno desses vetores é
  \begin{subequations}
    \begin{align}
    \pmb{v}\cdot\pmb{w} &= v_1w_1 + v_2w_2 + v_3w_3\\
                        &= 1\cdot 2 + 0\cdot(-1) + (-2)\cdot 3\\
                        &= 2 + 0 -6 = -4
    \end{align}
  \end{subequations}
  Usando \lstinline+arrays+, temos
\begin{lstlisting}
>>> v = np.array([1, 0, -2])
>>> w = np.array([2, -1, 3])
>>> np.sum(v*w)
-4
>>> np.dot(v,w)
-4
\end{lstlisting}
\end{ex}

\subsection{Norma de Vetores}

A \hl{\emph{norma} $L^2$ de um vetor} $\pmb{v}\in\mathbb{R}^n$ é definida por
\begin{equation}
  \|\pmb{v}\| := \sqrt{v_1^2+v_2^2+\cdots+v_n^2}
\end{equation}
O \hl{módulo {\href{https://numpy.org/doc/stable/reference/routines.linalg.html}{\lstinline+numpy.linalg+}} de Álgebra Linear contém a função {\href{https://numpy.org/doc/stable/reference/generated/numpy.linalg.norm.html}{\lstinline+numpy.linalg.norm(v)+}}} para a computação da norma de \lstinline+arrays+.

\begin{ex}
  A norma do vetor $\pmb{v} = (3, 0, -4)$ é
  \begin{subequations}
    \begin{align}
      \|\pmb{v}\| &= \sqrt{v_1^2 + v_2^2 + v_3^2}\\
                  &= \sqrt{3^2 + 0^2 + (-4)^2}\\
                  &= \sqrt{9 + 0 + 16}\\
                  &= \sqrt{25} = 5.
    \end{align}
  \end{subequations}
  Usando o módulo \lstinline+numpy.linalg+, obtemos
\begin{lstlisting}
>>> import numpy as np
>>> import numpy.linalg as npla
>>> v = np.array([3, 0, -4])
>>> np.sqrt(np.dot(v,v))
5.0
>>> npla.norm(v)
5.0
\end{lstlisting}
\end{ex}

\subsection{Produto Vetorial}

O \hl{\emph{produto vetorial}} entre dois vetores $\pmb{v}, \pmb{w}\in\mathbb{R}^3$ é definido por
\begin{subequations}
  \begin{align}
    \pmb{v}\times\pmb{w} &:=
                           \begin{vmatrix}
                             \pmb{i} & \pmb{j} & \pmb{k}\\
                             v_1 & v_2 & v_3\\
                             w_1 & w_2 & w_3
                           \end{vmatrix}\\
                         &=
                           \begin{vmatrix}
                             v_2 & v_3\\
                             w_2 & w_3
                           \end{vmatrix}\pmb{i}\\
                         &+ \begin{vmatrix}
                             v_1 & v_3\\
                             w_1 & w_3
                           \end{vmatrix}\pmb{j}\\
                         &+ \begin{vmatrix}
                             v_1 & v_2\\
                             w_1 & w_2
                           \end{vmatrix}\pmb{k}\\
  \end{align}
\end{subequations}
A função \hl{{\lstinline+numpy.cross(v,w)+}} computa o produto vetorial entre \lstinline+arrays+ (unidimensionais de 3 elementos).

\begin{ex}
  O produto vetorial entre os vetores $\pmb{v} = (1, -2, 1)$ e $\pmb{w} = (0, 2, -1)$ é
  \begin{subequations}
    \begin{align}
      \pmb{v}\times \pmb{w} &=
                              \begin{vmatrix}
                                \pmb{i} & \pmb{j} & \pmb{k}\\
                                1 & -2 & 1\\
                                0 & 2 & -1
                              \end{vmatrix}\\
                            &= 0\pmb{i} + \pmb{j} + 2\pmb{k}\\
                            &= (0, 1, 2).
    \end{align}
  \end{subequations}
    Com o {\numpy}, temos
\begin{lstlisting}
>>> v = np.array([1, -2, 1])
>>> w = np.array([0, 2, -1])
>>> np.cross(v,w)
array([0, 1, 2])
\end{lstlisting}
\end{ex}

\subsection{Exercícios}

\begin{exer}
  Considere os seguintes vetores
  \begin{align}
    &\pmb{u} = (2, -1, 1)\\
    &\pmb{v} = (1, -3, 2)\\
    &\pmb{w} = (-2, -1, -3)
  \end{align}
  Usando \lstinline+arrays+ do {\numpy}, compute:
  \begin{enumerate}[a)]
  \item $\pmb{u}\cdot\pmb{v}$
  \item $\pmb{u}\cdot (2\pmb{v})$
  \item $\pmb{u}\cdot (\pmb{w} + \pmb{v})$
  \item $\pmb{v}\cdot (\pmb{v} - 2\pmb{u})$
  \end{enumerate}
\end{exer}
\begin{resp}
  Dica: use a função $np.dot()$ e verifique as computações calculando os resultados esperados.
\end{resp}

\begin{exer}
  Considere os seguintes vetores
  \begin{align}
    &\pmb{u} = (2, -1, 1)\\
    &\pmb{v} = (1, -3, 2)\\
    &\pmb{w} = (-2, -1, -3)
  \end{align}
  Usando \lstinline+arrays+ do {\numpy}, compute:
  \begin{enumerate}[a)]
  \item $\|\pmb{u}\|$
  \item $\|\pmb{u} + \pmb{v}\|$
  \item $|\pmb{u}\cdot \pmb{w}|$
  \end{enumerate}
\end{exer}
\begin{resp}
  Dica: do módulo \lstinline+numpy.linalg+ use a função \lstinline+npla.norm+ e verifique as computações calculando os resultados esperados.
\end{resp}

\begin{exer}
  Dados vetores $\pmb{u}$ e $\pmb{v}$, temos que
  \begin{equation}
    \pmb{u}\cdot\pmb{v} = \|\pmb{u}\|\|\pmb{v}\|\cos\theta,
  \end{equation}
  onde $\theta$ é o ângulo entre esses vetores. Implemente uma função que recebe dois vetores e retorna o ângulo entre eles. Teste seu código para diferentes vetores.
\end{exer}
\begin{resp}
\begin{lstlisting}
import numpy as np
import numpy.linalg as npla

def angulo(v, w):
    # \|v\|
    norm_v = npla.norm(v)
    # \|w\|
    norm_w = npla.norm(w)
    # u.v
    vdw = np.dot(v, w)
    # cos \theta
    ct = norm_v*norm_w/udw
    return np.acos(ct)
\end{lstlisting}
\end{resp}

\begin{exer}
  A projeção ortogonal do vetor $\pmb{u}$ na direção do vetor $\pmb{v}$ é definida por
  \begin{equation}
    \proj_{\pmb{v}} \pmb{u} := \frac{\pmb{u}\cdot\pmb{v}}{\|\pmb{v}\|^2}\pmb{v}.
  \end{equation}
  Implemente uma função que recebe dois vetores $\pmb{u}$, $\pmb{v}$ e retorne a projeção de $\pmb{u}$ na direção de $\pmb{v}$. Teste seu código para diferentes vetores.
\end{exer}
\begin{resp}
\begin{lstlisting}
import numpy as np
import numpy.linalg as npla

def proj(u, v):
    # \|v\|
    norm_v = npla.norm(v)
    # u.v
    udv = np.dot(u, v)
    # proj_v u
    proj_vu = udv/norm_v * v
    return proj_vu
\end{lstlisting}
\end{resp}

\begin{exer}
  Considere os vetores
  \begin{align}
    &\pmb{u} = (2, -3, 1),\\
    &\pmb{v} = (1, -2, -1).
  \end{align}
  Usando \lstinline+arrays+ do {\numpy}, compute os seguintes produtos vetoriais:
  \begin{enumerate}[a)]
  \item $\pmb{u}\times\pmb{v}$
  \item $\pmb{v}\times (2\pmb{v})$
  \end{enumerate}
\end{exer}
\begin{resp}
  Dica: use a função \lstinline+numpy.cross()+ e verifique as computações calculando os resultados esperados.
\end{resp}