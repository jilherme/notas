% Este trabalho está licenciado sob a Licença Atribuição-CompartilhaIgual 4.0 Internacional Creative Commons. Para visualizar uma cópia desta licença, visite http://creativecommons.org/licenses/by-sa/4.0/deed.pt_BR ou mande uma carta para Creative Commons, PO Box 1866, Mountain View, CA 94042, USA.

\chapter{Orientação a Objetos}\label{cap_poo}

Programação Orientação a Objetos (POO) é um paradigma de programação baseado no conceito de classes de objetos. A classe define seus atributos (propriedades e métodos) de seus objetos. Todos os objetos de uma classe têm os mesmos atributos, mas são independentes um dos outros, sendo que cada um é uma instância própria da classe contendo seus próprios valores de seus atributos.

\section{Classe e Objeto}\label{cap_ob_sec_class}

\hl{Uma \emph{classe} é uma forma de estrutura que permite a alocação conjunta de dados e funções}. Em {\python}, a sintaxe de definição de uma classe é

\begin{lstlisting}
class NomeDaClasse:
    <bloco-0>
    <bloco-1>
    ...
    <bloco-2>
\end{lstlisting}

Usualmente, \hl{os blocos de programação consistem de definições de funções (métodos)}. Por exemplo,

\begin{lstlisting}
class MinhaClasse:
    def digaOla(self):
        print('Olá, Mundo!')

obj = MinhaClasse()
obj.digaOla()
\end{lstlisting}

Neste código, temos a definição da classe \lstinline+MinhaClasse+ (linhas 1-3). Esta classe contém o método \lstinline+MinhaClasse.digaOla()+ (linhas 2-3). Obrigatoriamente, \hl{na definição de um método de uma classe deve conter o primeiro parâmetro \texttt{self}}. Um objeto desta classe\endnote{Uma nova instância da classe.} e identificado por \lstinline+obj+ é alocado na linha 5. Na linha 6, este objeto chama seu método \lstinline+obj.digaOla()+.

O método especial \lstinline+__init__()+ é executado na construção de cada nova instância da classe (objeto da classe). Por exemplo,

\begin{lstlisting}
class Brasileira:
    pais = 'Brasil'
    def __init__(self, nome):
        self.nome = nome
        
    def digaOla(self):
        print('\nOlá!')
        print(f'Eu me chamo {self.nome}.')
        print(f'Sou do {self.pais}. :)')

x = Brasileira('Fulane')
x.digaOla()
y = Brasileira('Beltrane')
y.digaOla()
\end{lstlisting}

Aqui, o atributo \lstinline+Brasileira.pais+ é compartilhada entre todas as instâncias da classe (objetos), enquanto que \lstinline+Brasileira.nome+ é um atributo de cada objeto. O método \lstinline+__init()__+ (linhas 3-4) é executada no momento da criação de cada nova instância (linhas 11 e 13).

\begin{ex}\label{cap_oo_sec_class:ex:triangulo}
  No seguinte código, começamos a definição de uma classe para a manipulação de triângulos.

\begin{lstlisting}[caption=classTriangulo.py, label=cap_poo_sec_class:cod:classTriangulo]
import matplotlib.pyplot as plt

class Triangulo:
    '''
    Classe Triangulo ABC.
    '''
    num_lados = 3
    def __init__(self, A, B, C):
        # vértices
        self.A = A
        self.B = B
        self.C = C

    def plot(self):
        fig = plt.figure()
        ax = fig.add_subplot()
        # lados
        ax.plot([self.A[0], self.B[0]],
                [self.A[1], self.B[1]], marker='o', color='blue')
        ax.text((self.A[0]+self.B[0])/2,
                (self.A[1]+self.B[1])/2, 'c')
        ax.plot([self.B[0], self.C[0]],
                [self.B[1], self.C[1]], marker='o', color='blue')
        ax.text((self.B[0]+self.C[0])/2,
                (self.B[1]+self.C[1])/2, 'a')
        ax.plot([self.C[0], self.A[0]],
                [self.C[1], self.A[1]], marker='o', color='blue')
        ax.text((self.A[0]+self.C[0])/2,
                (self.A[1]+self.C[1])/2, 'b')
        # vertices
        ax.text(self.A[0], self.A[1], 'A')
        ax.text(self.B[0], self.B[1], 'B')
        ax.text(self.C[0], self.C[1], 'C')
        ax.grid()
        plt.show()

tria = Triangulo((0., 0.),
                 (2., 0.),
                 (1., 1.))
tria.plot()
\end{lstlisting}

\end{ex}

\subsection{Exercícios}

\begin{exer}
  Considere o Código~\ref{cap_poo_sec_class:cod:classTriangulo}. Adicione o método \lstinline+calcLados()+, que computa e aloca o comprimento de cada lado do triângulo.
\end{exer}
\begin{resp}

\begin{lstlisting}
import numpy as np
import matplotlib.pyplot as plt

class Triangulo:
    '''
    Classe Triangulo ABC.
    '''
    num_lados = 3
    def __init__(self, A, B, C):
        # vértices
        self.A = A
        self.B = B
        self.C = C
        # lados
        self.a = 0.
        self.b = 0.
        self.c = 0.

    def calcLados(self):
        self.a = np.sqrt((self.B[0]-self.C[0])**2\
                         + (self.B[1]-self.C[1])**2)
        self.b = np.sqrt((self.A[0]-self.C[0])**2\
                         + (self.A[1]-self.C[1])**2)
        self.c = np.sqrt((self.A[0]-self.B[0])**2\
                         + (self.A[1]-self.B[1])**2)
\end{lstlisting}

\end{resp}

\begin{exer}
  Considere o Código~\ref{cap_poo_sec_class:cod:classTriangulo}. Adicione o método \lstinline+calcPerimetro()+, que computa e retorna o valor do perímetro do triângulo.
\end{exer}
\begin{resp}

\begin{lstlisting}
import numpy as np
import matplotlib.pyplot as plt

class Triangulo:

    ...

    def perimetro(self):
        return self.a + self.b + self.c

    ...
\end{lstlisting}

\end{resp}

\begin{exer}
  Considere o Código~\ref{cap_poo_sec_class:cod:classTriangulo}. Adicione o método \lstinline+calcAngulos()+, que computa e aloca os ângulos do triângulo.
\end{exer}
\begin{resp}
  Dica: use a \href{https://pt.wikipedia.org/wiki/Lei_dos_cossenos}{Lei dos Cossenos}.
\end{resp}

\begin{exer}
  Considere o Código~\ref{cap_poo_sec_class:cod:classTriangulo}. Adicione o método \lstinline+area()+, que computa a área do triângulo.
\end{exer}
\begin{resp}
  Dica: use o \href{https://pt.wikipedia.org/wiki/Teorema_de_Her%C3%A3o}{Teorema de Herão}.
\end{resp}

\begin{exer}
  Similar a classe \lstinline+Triangulo+ (Código~\ref{cap_poo_sec_class:cod:classTriangulo}), implemente uma nova classe \lstinline+Quadrilateros+ com as seguintes propriedades e métodos de quadriláteros $ABCD$:
  \begin{enumerate}[a)]
  \item vértices (\lstinline+tuples+).
  \item lados (\lstinline+floats+).
  \item cálculo do perímetro (método).
  \item cálculo da área (método).
  \item visualização gráfica (método \textit+plot+).
  \end{enumerate}
\end{exer}

\begin{exer}
  Implemente uma classe para a manipulação de polinômios de segundo grau. A classe deve conter as seguintes propriedades e métodos:
  \begin{enumerate}[a)]
  \item coeficientes (\lstinline+floats+).
  \item cálculo do ponto de interseção com o eixo y (método).
  \item cálculo do vértice da parábola associada ao polinômio (método).
  \item cálculo das raízes do polinômio (método).
  \item plotagem do gráfico do polinômio (método).
  \end{enumerate}
\end{exer}
\begin{resp}
  Dica: utilize a notação $p(x) = ax^2 + bx + c$.
\end{resp}

\ifisbook
\subsubsection{Respostas}
\shipoutAnswer
\fi

  

\section{Herança}\label{cap_oo_sec_her}

\hl{Na programação orientada-a-objetos, \emph{herança} consiste na definição de uma classe derivada a partir de uma dada classe base}. A sintaxe de definição de uma classe derivada é

\begin{lstlisting}
class ClasseDerivada(ClasseBase):
    bloco-0
    bloco-1
    ...
    bloco-n
\end{lstlisting}

\hl{A classe derivada herda todos os atributos da classe base}. Por exemplo, consideramos o seguinte código

\begin{lstlisting}
class ClasseBase:
    def __init__(self, nome):
        self.nome = nome
        
    def digaOi(self):
        print(f'{self.nome}: Oi!')

class ClasseDerivada(ClasseBase):
    def digaTchau(self):
        print(f'{self.nome}: Tchau!')

obj = ClasseDerivada('Fulane')
obj.digaOi()
obj.digaTchau()
\end{lstlisting}

Nas linhas 1-6, a classe base é definida contendo dois métodos: \lstinline+self.__init__()+ chamado na criação de um objeto da classe (uma instância) e, \lstinline+self.digaOi()+ que imprime uma saudação. A classe derivada é definida nas linhas 8-10, ela herda os atributos da classe base e contém um novo método \lstinline+self.digaTchau()+, que imprime uma mensagem de despedida.

A criação de uma instância (objeto) de uma classe derivada é feita da mesma forma que de uma classe base. \hl{A referência a um atributo do objeto é, primeiramente, buscada na classe derivada e, se não encontrada, é buscada na classe base}. Este regra aplica-se recursivamente se a classe base também é derivada de outra classe. \hl{Isso permite que uma classe derivada sobreponha atributos de sua classe base}.

\begin{obs}\normalfont{(\hl{\texttt{super()}}.)}
O método \href{https://docs.python.org/3/library/functions.html\#super}{\lstinline+super()+}\endnote{Consulte na \textit{web} \href{https://docs.python.org/3/library/functions.html\#super}{ The Python Standard Library: Built-in Functions: super}.} retorno um objeto \textit{proxy} da classe base, que acessa os atributos desta classe.
\end{obs}

\begin{ex}
  Vamos criar uma classe para manipular triângulo isósceles. Para tanto, vamos derivá-la a partir da classe \lstinline+Triangulo+ definida no Exemplo~\ref{cap_oo_sec_class:ex:triangulo}. Vamos assumir que os triângulos isósceles têm vértices $\Delta ABC$ com lados $b = AC$ e $a = BC$ de mesmo tamanho.

\begin{lstlisting}[caption=classTrianguloIsosceles.py, label=cap_oo_sec_her:cod:classTrianguloIsosceles]
import numpy as np
import matplotlib.pyplot as plt

class Triangulo:
    '''
    Classe Triangulo ABC.
    '''
    num_lados = 3
    def __init__(self, A, B, C):
        # vértices
        self.A = A
        self.B = B
        self.C = C

    def plot(self):
        fig = plt.figure()
        ax = fig.add_subplot()
        # lados
        ax.plot([self.A[0], self.B[0]],
                [self.A[1], self.B[1]], marker='o', color='blue')
        ax.text((self.A[0]+self.B[0])/2,
                (self.A[1]+self.B[1])/2, 'c')
        ax.plot([self.B[0], self.C[0]],
                [self.B[1], self.C[1]], marker='o', color='blue')
        ax.text((self.B[0]+self.C[0])/2,
                (self.B[1]+self.C[1])/2, 'a')
        ax.plot([self.C[0], self.A[0]],
                [self.C[1], self.A[1]], marker='o', color='blue')
        ax.text((self.A[0]+self.C[0])/2,
                (self.A[1]+self.C[1])/2, 'b')
        # vertices
        ax.text(self.A[0], self.A[1], 'A')
        ax.text(self.B[0], self.B[1], 'B')
        ax.text(self.C[0], self.C[1], 'C')
        ax.grid()
        plt.show()


class TrianguloIsosceles(Triangulo):
    def __init__(self,A,B,C):
        # vertices
        super().__init__(A,B,C)
        # lados
        self.a = self.b = self.c = 0.

    def calcLados(self):
        self.a = np.sqrt((self.B[0] - self.C[0])**2\
                         + (self.B[1] - self.C[1])**2)
        self.b = np.sqrt((self.A[0] - self.C[0])**2\
                         + (self.A[1] - self.C[1])**2)
        self.c = np.sqrt((self.B[0] - self.A[0])**2\
                         + (self.B[1] - self.A[1])**2)
        assert(self.a == self.b)

tria = TrianguloIsosceles((1,0),
                          (3,0),
                          (2,1))
tria.plot()
tria.calcLados()
\end{lstlisting}

\end{ex}

\begin{obs}\normalfont{(\hl{Herança Múltipla}.)}
  {\python} suporta a herança múltipla de classes. A sintaxe é

\begin{lstlisting}
class ClasseDerivada(Base1, Base2, ..., BraseN):
    bloco-0
    bloco-1
    ...
    bloco-m
\end{lstlisting}
  
Quando um objeto da classe derivada faz uma referência a um atributo, este é procurado de forma sequencial (e recursiva, caso uma das classe bases seja também uma classe derivada) começando por essa e, caso não encontrado, buscando-se nas classes \lstinline+Base1+, \lstinline+Base2+, ..., \lstinline+BaseN+.
\end{obs}

\subsection{Exercícios}

\begin{exer}
  No Código~\ref{cap_oo_sec_her:cod:classTrianguloIsosceles}, adicione à classe \lstinline+Triangulo+ o método \lstinline+Triangulo.perimetro()+ que computa, aloca e retorna o valor do perímetro do triângulo. Então, sobreponha o método à classe \lstinline+TrianguloIsosceles+. Teste seu código para diferentes triângulos.
\end{exer}
\begin{resp}

\begin{lstlisting}
class Triangulo:
    def __init(self,A,B,C)__:
        ...
        self.p = 0.
        ...
    ...
    def perimetro(self):
        self.p = self.a\
               + self.b\
               + self.c
        return self.p
    ...

class TrianguloIsosceles(Triangulo):
    ...
    def perimetro(self):
        self.p = 2*self.a + self.c
        return self.p
    ...
\end{lstlisting}

\end{resp}

\begin{exer}\label{cap_oo_sec_her:exer:retangulo}
  Implemente uma classe \lstinline+Retangulo(largura, altura)+ para a manipulação de retângulos de \lstinline+largura+ e \lstinline+altura+ dadas. Equipe sua classe com métodos para o cálculo do perímetro, da diagonal e da área de retângulo. Então, implemente a classe derivada \lstinline+Quadrado(lado)+ para a manipulação de quadrados de \lstinline+lado+ dado. Teste sua implementação para diferentes retângulos e quadrados.
\end{exer}
\begin{resp}

\begin{lstlisting}
import math as m

class Retangulo:
    def __init__(self, largura, altura):
        self.largura = largura
        self.altura = altura

    def perimetro(self):
        return self.largura\
               + self.altura

    def diagonal(self):
        return m.sqrt(self.largura**2\
                      + self.altura**2)

    def area(self):
        return self.largura\
               * self.altura

class Quadrado(Retangulo):
    def __init__(self,lado):
        super().__init__(lado,lado)
\end{lstlisting}

\end{resp}

\begin{exer}
  Refaça o Exercício~\ref{cap_oo_sec_her:exer:retangulo} sobrepondo os métodos do cálculo do perímetro, da diagonal e da área para quadrados.
\end{exer}
\begin{resp}
  Dica: para um quadrado de lado $l$, o perímetro é $p = 2l$, por exemplo.
\end{resp}

\begin{exer}
  Implemente uma classe \lstinline+TrianguloEquilatero+, derivada da classe \lstinline+TrianguloIsosceles+ definida no Código~\ref{cap_oo_sec_her:cod:classTrianguloIsosceles}. Adicione métodos para o cálculo do perímetro e da altura de triângulo equiláteros. Teste seu código para diferentes triângulos.
\end{exer}
\begin{resp}
  Dica:

\begin{lstlisting}
...
class Triangulo:
   def __init__(self,A,B,C):
      ...
   ...

class TrianguloIsosceles(Triangulo):
   ...

class TrianguloEquilatero(TrianguloIsosceles):
   def __init__(self,A,B,C):
      super().__init__(A,B,C)

   def perimetro(self):
      ...

   def altura(self):
      ...
   
   def area(self):
      ...
\end{lstlisting}

\end{resp}

\begin{exer}
  Implemente:
  \begin{enumerate}[a)]
  \item Uma classe \lstinline+Quadrilatero+ para a manipulação de quadriláteros de lados $abcd$. Equipe sua classe com um método \lstinline+self.perimetro()+ para o cálculo do perímetro.
  \item Uma classe \lstinline+Retangulo+, derivada da classe \lstinline+Quadrilatero+, para a manipulação de retângulos de lado dado e altura dada. Na classe derivada, sobreponha o método \lstinline+self.perimetro()+ para o cálculo do perímetro e implemente novos métodos para o cálculo da diagonal e da área de retângulos.
  \item Uma classe \lstinline+Quadrado+, derivada da classe \lstinline+Retangulo+, para a manipulação de quadrados de lado dado. Na classe derivada, sobreponha os métodos para os cálculos do perímetro, da diagonal e da área.
  \end{enumerate}
\end{exer}

\ifisbook
\subsubsection{Respostas}
\shipoutAnswer
\fi
