%Este trabalho está licenciado sob a Licença Atribuição-CompartilhaIgual 4.0 Internacional Creative Commons. Para visualizar uma cópia desta licença, visite http://creativecommons.org/licenses/by-sa/4.0/deed.pt_BR ou mande uma carta para Creative Commons, PO Box 1866, Mountain View, CA 94042, USA.

\chapter{Introdução}\label{cap_intro}
\thispagestyle{fancy}

\begin{flushright}
  [Vídeo] | [Áudio] | \href{https://phkonzen.github.io/notas/contato.html}{[Contatar]}
\end{flushright}

A computação paralela e distribuída é uma realidade em todas as áreas de pesquisa aplicadas. À primeira vista, pode-se esperar que as aplicações se beneficiam diretamente do ganho em poder computacional. Afinal, se a carga (processo) computacional de uma aplicação for repartida e distribuída em $n_p>1$ processadores (\emph{instâncias de processamentos}, {\it threads} ou {\it cores}), a computação paralela deve ocorrer em um tempo menor do que se a aplicação fosse computada em um único processador (em serial). Entretanto, a tarefa de repartir e distribuir (\emph{alocação de tarefas}) o processo computacional de uma aplicação é, em muitos casos, bastante desafiadora e pode, em vários casos, levar a códigos computacionais menos eficientes que suas versões seriais.

Repartir e distribuir o processo computacional de uma aplicação sempre é possível, mas nem sempre é possível a computação paralela de cada uma das partes. Por exemplo, vamos considerar a \href{https://phkonzen.github.io/notas/MatematicaNumerica/cap_eq1d_pfixo.html}{iteração de ponto fixo}
\begin{gather}
  x(n) = f(x(n-1)),\quad n\geq 1, \label{eq:intro_in}\\
  x(0) = x_0,
\end{gather}
onde $f:x\mapsto f(x)$ é uma função dada e $x_0$ é o ponto inicial da iteração. Para computar $x(100)$ devemos processar $100$ vezes a iteração \eqref{eq:intro_in}. Se tivéssemos a disposição $n_P=2$ processadores, poderíamos repartir a carga de processamento em dois, distribuindo o processamento das $50$ primeiras iterações para o primeiro processador (o processador $0$) e as demais $50$ para o segundo processador (o processador $1$). Entretanto, pela característica do processo iterativa, o processador $1$ ficaria ocioso, aguardando o processador $0$ computar $x(50)$. Se ambas instâncias de processamento compartilharem a mesma memória computacional (\emph{memória compartilhada}), então, logo que o processador $0$ computar $x(50)$ ele ficará ocioso, enquanto que o processador $1$ computará as últimas $50$ iterações. Ou seja, esta abordagem não permite a computação em paralelo, mesmo que reparta e distribua o processo computacional entre duas instâncias de processamento.

Ainda sobre a abordagem acima, caso as instâncias de processamento sejam de \emph{memória distribuída} (não compartilhem a mesma memória), então o processador $0$ e o processador $1$ terão de se comunicar, isto é, o processador $0$ deverá enviar $x(50)$ para a instância de processamento $1$ e esta instância deverá receber $x(50)$ para, então, iniciar suas computações. A \emph{comunicação} entre as instâncias de processamento levantam outro desafio que é necessidade ou não da \emph{sincronização} () eventual entre elas. No caso de nosso exemplo, é a necessidade de sincronização na computação de $x(50)$ que está minando a computação paralela.

Em resumo, o design de métodos numéricos paralelos deve levar em consideração a \emph{alocação de tarefas}, a \emph{comunicação} e a \emph{sincronização} entre as instâncias de processamentos. Vamos voltar ao caso da iteração \eqref{eq:intro_in}. Agora, vamos supor que $x = (x_0, x_1)$, $f:x\mapsto (f_0(x), f_1(x))$ e a condição inicial $x(0)=(x_{0}(0), x_{1}(0))$ é dada. No caso de termos duas instâncias de processamentos disponíveis, podemos computar as iterações em paralelo da seguinte forma. Iniciamos distribuindo $x$ às duas instâncias de processamento $0$ e $1$. Em paralelo, a instância $0$ computa $x_{0}(1) = f_0(x)$ e a instância $1$ computa $x_{1}(1) = f_1(x)$. Para computar a nova iterada $x(2)$, a instância $0$ precisa ter acesso a $x_{1}(1)$ e a instância $1$ necessita de $x_{0}(1)$. Isto implica na sincronização das instâncias de processamentos, pois uma instância só consegui seguir a computação após a outra instância ter terminado a computação da mesma iteração. Agora, a comunicação entre as instâncias de processamento, depende da arquitetura do máquina. Se as instâncias de processamento compartilham a mesma memória (memória compartilhada), cada uma tem acesso direto ao resultado da outra. No caso de uma arquitetura de memória distribuída, ainda há a necessidade de instruções de comunicação entre as instância, i.e. a instância $0$ precisa enviar $x_{0}(1)$ à instância $1$, a qual precisa receber o valor enviado. A instância $1$ precisa enviar $x_{1}(1)$ à instância $0$, a qual precisa receber o valor enviado. O processo segue análogo para cada iteração até a computação de $x(100)$.

A primeira parte destas notas de aula, restringe-se a implementação de métodos numéricos paralelos em uma arquitetura de memória compartilhada. Os exemplos computacionais são apresentados em linguagem C/C++ com a interface de programação de aplicações (API, {\it Application Programming Interface}) \href{https://www.openmp.org/}{OpenMP}. A segunda parte, dedica-se a implementação paralela em arquitetura de memória distribuída. Os códigos C/C++ são, então, construídos com a API \href{https://www.open-mpi.org/}{OpenMPI}.
