%Este trabalho está licenciado sob a Licença Atribuição-CompartilhaIgual 4.0 Internacional Creative Commons. Para visualizar uma cópia desta licença, visite http://creativecommons.org/licenses/by-sa/4.0/deed.pt_BR ou mande uma carta para Creative Commons, PO Box 1866, Mountain View, CA 94042, USA.

\chapter{Métodos iterativos para sistemas lineares}\label{cap_sl_iter}
\thispagestyle{fancy}

\section{Métodos de Jacobi e de Gauss-Seidel}\label{cap_sl_iter_sec_jgs}

Nesta seção, discutiremos os métodos de Jacobi\footnote{Carl Gustav Jacob Jacobi, 1804 - 1851, matemático alemão. Fonte: \href{https://en.wikipedia.org/wiki/Carl_Gustav_Jacob_Jacobi}{Wikipedia}.} e de Gauss-Seidel\footnote{Johann Carl Friedrich Gauss, 1777 - 1855, matemático alemão. Philipp Ludwig von Seidel, 1821 - 1896, matemático alemão. Fonte: \href{https://en.wikipedia.org/wiki/Philipp_Ludwig_von_Seidel}{Wikipedia}.} para a aproximação da solução de sistemas lineares.

\subsection{Método de Jacobi}

Dado um sistema $A\pmb{x} = \pmb{b}$ com $n$ equações e $n$ incógnitas, consideramos a seguinte decomposição da matriz $A = L + D + U$:
\begin{align}
  A &=
  \begin{bmatrix}
    a_{11} & a_{12} & a_{13} & \ldots & a_{1n}\\
    a_{21} & a_{22} & a_{23} & \ldots & a_{2n}\\
    a_{31} & a_{32} & a_{33} & \ldots & a_{3n}\\
    \vdots & \vdots & \vdots & \ldots & \vdots\\
    a_{n1} & a_{n2} & a_{n3} & \ldots & a_{nn}\\
  \end{bmatrix}\\
    &= \underbrace{\begin{bmatrix}
    0 & 0 & 0 & \ldots & 0\\
    a_{21} & 0 & 0 & \ldots & 0\\
    a_{31} & a_{32} & 0 & \ldots & 0\\
    \vdots & \vdots & \vdots & \ldots & \vdots\\
    a_{n1} & a_{n2} & a_{n3} & \ldots & 0\\
  \end{bmatrix}}_{L}\\
    &+ \underbrace{\begin{bmatrix}
    a_{11} & 0 & 0 & \ldots & 0\\
    0 & a_{22} & 0 & \ldots & 0\\
    0 & 0 & a_{33} & \ldots & 0\\
    \vdots & \vdots & \vdots & \ldots & \vdots\\
    0 & 0 & 0 & \ldots & a_{nn}\\
  \end{bmatrix}}_{D}\\
  &+ \underbrace{\begin{bmatrix}
    0 & a_{12} & a_{13} & \ldots & a_{1n}\\
    0 & 0 & a_{23} & \ldots & a_{2n}\\
    0 & 0 & a_{33} & \ldots & a_{3n}\\
    \vdots & \vdots & \vdots & \ldots & \vdots\\
    0 & 0 & 0 & \ldots & a_{nn}\\
  \end{bmatrix}}_{U}.
\end{align}
Isto é, a matriz $A$ decomposta como a soma de sua parte triangular inferior $L$, de sua diagonal $D$ e de sua parte triangular superior $U$.

Desta forma, podemos reescrever o sistema $A\pmb{x}=b$ da seguinte forma:
\begin{align}
  A\pmb{x} = \pmb{b} &\Leftrightarrow (L + D + U)\pmb{x} = \pmb{b}\\
         &\Leftrightarrow D\pmb{x} = -(L+U)\pmb{x} + \pmb{b}\\
         &\Leftrightarrow \pmb{x} = -D^{-1}(L+U)\pmb{x} + D^{-1}\pmb{b}.
\end{align}
Ou seja, resolver o sistema $A\pmb{x} = \pmb{b}$ é equivalente a resolver o problema de ponto fixo
\begin{equation}
  \pmb{x} = T_J\pmb{x} + \pmb{c}_J,
\end{equation}
onde $T_J = -D^{-1}(L+U)$ é chamada de \emph{matriz de Jacobi}\index{matriz de!Jacobi} e $\pmb{c}_J = D^{-1}\pmb{b}$ é chamado de \emph{vetor de Jacobi}\index{vetor de!Jacobi}.

\begin{ex}\label{ex:jacobi_intro}
  Consideremos o sistema linear $A\pmb{x} = \pmb{b}$ com
  \begin{equation}
    A =
    \begin{bmatrix}
      -4 & 2 & -1 \\
      -2 & 5 & 2 \\
       1 & -1 & -3
    \end{bmatrix},\quad
    \pmb{b} =
    \begin{bmatrix}
      -11\\ -7\\ 0
    \end{bmatrix}.
  \end{equation}
  Este sistema tem solução $\pmb{x} = (2, -1, 1)$. Neste caso, temos a decomposição $A = L + D + U$ com
  \begin{equation}
    L = \begin{bmatrix} 
      0 & 0 & 0 \\
      -2 & 0 & 0 \\
       1 & -1 & 0
     \end{bmatrix},\quad
    D = \begin{bmatrix}
      -4 & 0 & 0 \\
      0 & 5 & 0 \\
       0 & 0 & -3                  
     \end{bmatrix}
  \end{equation}
  e
  \begin{equation}
    U = \begin{bmatrix}
      0 & 2 & -1 \\
      0 & 0 & 2 \\
       0 & 0 & 0
    \end{bmatrix}.
  \end{equation}
  Ainda, observamos que
  \begin{align}
    T_J\pmb{x} + \pmb{c}_J &= -D^{-1}(L+U)\pmb{x} + D^{-1}\pmb{b}\\
    &= \underbrace{\begin{bmatrix}
        0 & 1/2 & 1/4 \\
        2/5 & 0 & -2/5 \\
        1/3 & -1/3 & 0
      \end{bmatrix}}_{T_J}
      \underbrace{\begin{bmatrix}
        2 \\
        -1 \\
        1             
      \end{bmatrix}}_{\pmb{x}} +
      \underbrace{\begin{bmatrix}
       11/4 \\
       -7/5 \\
       0
      \end{bmatrix}}_{\pmb{c}_J}\\
  &= \underbrace{\begin{bmatrix}
        2 \\
        -1 \\
        1             
      \end{bmatrix}}_{\pmb{x}}.
  \end{align}
% \ifisoctave
% No \verb+GNU Octave+, podemos fazer as computações acima com o seguinte \href{https://github.com/phkonzen/notas/blob/master/src/MatematicaNumerica/cap_sl_iter/dados/ex_jacobi_intro/ex_jacobi_intro.m}{código}:
% \verbatiminput{./cap_sl_iter/dados/ex_jacobi_intro/ex_jacobi_intro.m}
% \fi
\end{ex}

Com o exposto acima, o \emph{método de Jacobi} consiste na seguinte iteração de ponto fixo
\begin{align}
  \pmb{x}^{(1)} &= \text{aprox. inicial},\\
  \pmb{x}^{(k+1)} &= T_J\pmb{x}^{(k)} + \pmb{c}_J,\label{eq:iter_jacobi_mat}
\end{align}
onde $\pmb{x}^{(k)} = (x_1^{(k)}, x_2^{(k)}, \dotsc, x_n^{(k)})$ é a $k$-ésima aproximação (ou iterada) de Jacobi.

A iteração \eqref{eq:iter_jacobi_mat} pode ser equivalentemente escrita na seguinte forma algébrica
\begin{equation}
  x_i^{(k+1)} = \frac{{\displaystyle b_i - \sum_{\overset{j=1}{j\neq i}}^n a_{ij}x_j^{(k)}}}{a_{ii}},~i=1, 2, \dotsc, n,
\end{equation}
a qual não requer a computação da matriz $T_J$ e $\pmb{c}_J$.

\begin{ex}\label{ex:jacobi_exec}
  Consideremos o sistema $A\pmb{x} = \pmb{b}$ com
  \begin{equation}
    A =
    \begin{bmatrix}
      -4 & 2 & -1 \\
      -2 & 5 & 2 \\
       1 & -1 & -3
    \end{bmatrix},\quad
    \pmb{b} =
    \begin{bmatrix}
      -11\\ -7\\ 0
    \end{bmatrix}.
  \end{equation}
  Aplicando o método de Jacobi com aproximação inicial $\pmb{x}^{(1)} = (0, 0, 0)$ obtemos os resultados da Tabela \ref{tab:ex_jacobi_exec}.

  \begin{table}[h!]
    \centering
    \begin{tabular}{l|cc}
      k & $\pmb{x}^{(k)}$ & $\|A\pmb{x}^{(k)}-\pmb{b}\|$\\\hline
      1 & $(0,0,~0,0,~0,0)$ & $1,3\E+1$\\
      2 & $(2,8,~-1,4,~0,0)$ & $7,4\E+0$ \\
      3 & $(2,0,~-0,3,~1,4)$ & $4,6\E+0$ \\
      4 & $(2,3,~-1,1,~0,8)$ & $2,2\E+0$ \\
      5 & $(2,0,~-0,8,~1,1)$ & $1,4\E+0$ \\
      6 & $(2,1,~-1,1,~0,9)$ & $6,9\E-1$ \\
      7 & $(2,0,~-0,9,~1,0)$ & $4,2\E-1$ \\
      8 & $(2,0,~-1,0,~1,0)$ & $2,2\E-1$ \\
      9 & $(2,0,~-1,0,~1,0)$ & $1,3\E-1$ \\
      10 & $(2,0,~-1,0,~1,0)$ & $6,9\E-2$ \\\hline
    \end{tabular}
    \caption{Resultados referentes ao Exemplo \ref{ex:jacobi_exec}.}
    \label{tab:ex_jacobi_exec}
  \end{table}

% \ifisoctave
% No \verb+GNU Octave+, podemos obter os resultados reportados na Tabela \ref{tab:ex_jacobi_exec} com o seguinte \href{https://github.com/phkonzen/notas/blob/master/src/MatematicaNumerica/cap_sl_iter/dados/ex_jacobi_exec/ex_jacobi_exec.m}{código}:
% \verbatiminput{./cap_sl_iter/dados/ex_jacobi_exec/ex_jacobi_exec.m}
% \fi
\end{ex}

\subsection{Método de Gauss-Seidel}

Como acima, começamos considerando um sistema linear $A\pmb{x} = \pmb{b}$ e a decomposição $A = L + D + U$, onde $L$ é a parte triangular inferior de $A$, $D$ é sua parte diagonal e $U$ sua parte triangular superior. Então, observamos que
\begin{align}
  A\pmb{x} = \pmb{b} &\Leftrightarrow (L + D + U)\pmb{x} = \pmb{b}\\
  &\Leftrightarrow (L+D)\pmb{x} = -U\pmb{x} + \pmb{b}\\
  &\Leftrightarrow \pmb{x} = -(L+D)^{-1}U\pmb{x} + (L+D)^{-1}\pmb{b}.
\end{align}
Isto nos leva a iteração de Gauss-Seidel
\begin{align}
  \pmb{x}^{(1)} = \text{aprox. inicial},\\
  \pmb{x}^{(k+1)} = T_G\pmb{x}^{(k)} + \pmb{c}_G,\label{eq:iter_gs_mat}
\end{align}
onde $T_G = -(L+D)^{-1}U$ é a chamada \emph{matriz de Gauss-Seidel}\index{matriz de!Gauss-Seidel} e $\pmb{c}_G = (L+D)^{-1}\pmb{b}$ é o chamado \emph{vetor de Gauss-Seidel}\index{vetor de!Gauss-Seidel}.

Observamos, também, que a iteração \eqref{eq:iter_gs_mat} pode ser reescrita na seguinte forma algébrica
\begin{equation}
  x_i^{(k+1)} = \frac{{\displaystyle b_i - \sum_{j=1}^{i-1} a_{ij}x_j^{(k+1)} - \sum_{j=i+1}^{n} a_{ij}x_j^{(k)}}}{a_{ii}},~i=1, 2, \dotsc, n.
\end{equation}

\begin{ex}\label{ex:gs_exec}
  Consideremos o sistema $A\pmb{x} = \pmb{b}$ com
  \begin{equation}
    A =
    \begin{bmatrix}
      -4 & 2 & -1 \\
      -2 & 5 & 2 \\
       1 & -1 & -3
    \end{bmatrix},\quad
    \pmb{b} =
    \begin{bmatrix}
      -11\\ -7\\ 0
    \end{bmatrix}.
  \end{equation}
  Aplicando o método de Gauss-Seidel com aproximação inicial $\pmb{x}^{(1)} = (0, 0, 0)$ obtemos os resultados da Tabela \ref{tab:ex_gs_exec}.

  \begin{table}[h!]
    \centering
    \begin{tabular}{l|cc}
      k & $\pmb{x}^{(k)}$ & $\|A\pmb{x}^{(k)}-\pmb{b}\|$\\\hline
      1 & $(0,0,~0,0,~0,0)$ & $1,3\E+1$ \\
      2 & $(2,8,~-0,3,~1,0)$ & $2,6\E+0$ \\
      3 & $(2,3,~-0,9,~1,1)$ & $1,2\E+0$ \\
      4 & $(2,0,~-1,0,~1,0)$ & $2,5\E-1$ \\
      5 & $(2,0,~-1,0,~1,0)$ & $4,0\E-2$ \\\hline
    \end{tabular}
    \caption{Resultados referentes ao Exemplo \ref{ex:gs_exec}.}
    \label{tab:ex_gs_exec}
  \end{table}

% \ifisoctave
% No \verb+GNU Octave+, podemos obter os resultados reportados na Tabela \ref{tab:ex_gs_exec} com o seguinte \href{https://github.com/phkonzen/notas/blob/master/src/MatematicaNumerica/cap_sl_iter/dados/ex_gs_exec/ex_gs_exec.m}{código}:
% \verbatiminput{./cap_sl_iter/dados/ex_gs_exec/ex_gs_exec.m}
% \fi
\end{ex}

\subsection{Análise de convergência}

Observamos que ambos os métodos de Jacobi e de Gauss-Seidel consistem de iterações da forma
\begin{equation}
  \pmb{x}^{(k+1)} = T\pmb{x}^{(k)} + \pmb{c},~k=1, 2, \ldots,\label{eq:jgs_iter}
\end{equation}
com $x^{(1)}$ uma aproximação inicial dada, $T$ e $c$ a matriz e o vetor de iteração, respectivamente. O seguinte teorema nos fornece uma condição suficiente e necessária para a convergência de tais métodos.

\begin{teo}
  Para qualquer $\pmb{x}^{(1)}\in\mathbb{R}^n$, temos que a sequência $\{\pmb{x}^{(k+1)}\}_{k=1}^{\infty}$ dada por
  \begin{equation}
    \pmb{x}^{(k+1)} = T\pmb{x}^{(k)} + \pmb{c},
  \end{equation}
  converge para a solução única de $\pmb{x} = T\pmb{x} + \pmb{c}$ se, e somente se, $\rho(T) < 1$\footnote{$\rho(T)$ é o raio espectral da matriz $T$, i.e. o máximo dos módulos dos autovalores de $T$.}.
\end{teo}
\begin{dem}
  Veja \cite[Cap. 7, Sec. 7.3]{Burden2015a}.
\end{dem}

\begin{obs}(\normalfont{Taxa de convergência})
  Para uma iteração da forma \eqref{eq:jgs_iter}, vale
  \begin{equation}
    \|\pmb{x}^{(k)}-\pmb{x}\| \approx \rho(T)^{k-1}\|\pmb{x}^{(1)}-\pmb{x}\|,
  \end{equation}
onde $\pmb{x}$ é a solução de $\pmb{x} = T\pmb{x} + \pmb{c}$.
\end{obs}

\begin{ex}\label{ex:jacobi_exec}
  Consideremos o sistema $A\pmb{x} = \pmb{b}$ com
  \begin{equation}
    A =
    \begin{bmatrix}
      -4 & 2 & -1 \\
      -2 & 5 & 2 \\
       1 & -1 & -3
    \end{bmatrix},\quad
    \pmb{b} =
    \begin{bmatrix}
      -11\\ -7\\ 0
    \end{bmatrix}.
  \end{equation}
  Nos Exemplos \ref{ex:jacobi_exec} e \ref{ex:gs_exec} vimos que ambos os métodos de Jacobi e de Gauss-Seidel eram convergentes, sendo que este convergiu aproximadamente duas vezes mais rápido que esse. Isto é confirmado pelos raios espectrais das respectivas matrizes de iteração
  \begin{equation}
    \rho(T_J) \approx 0,56,\quad\rho(T_G) \approx 0,26.
  \end{equation}

% \ifisoctave
% No \verb+GNU Octave+, podemos obter os raios espectrais das matrizes de iteração de Jacobi e Gauss-Seidel com o seguinte \href{https://github.com/phkonzen/notas/blob/master/src/MatematicaNumerica/cap_sl_iter/dados/ex_jgs_conv/ex_jgs_conv.m}{código}:
% \verbatiminput{./cap_sl_iter/dados/ex_jgs_conv/ex_jgs_conv.m}
% \fi
\end{ex}

\begin{obs}{\normalfont{Matriz estritamente diagonal dominante}}
  Pode-se mostrar que se $A$ é uma matriz estritamente diagonal dominante, i.e. se
  \begin{equation}
    |a_{ii}| > \sum_{\overset{j=1}{j\neq i}}^n |a_{ij}|,~\forall i=1, 2, \ldots, n,
  \end{equation}
então ambos os métodos de Jacobi e de Gauss-Seidel são convergentes.
\end{obs}

\subsection*{Exercícios}

\begin{exer}\label{exer:jacobi_exec}
  Considere o seguinte sistema linear
  \begin{align}
    -4x_1 + x_2 + x_3 - x_4 &= -1\\
    5x_2 -x_3 + 2x_4 &= 3\\
    -x_1 + 4x_3 - 2x_4 &= -2\\
    x_1 -x_2 -5x_4 &= 1
  \end{align}
  Compute a quinta iterada $x^{(5)}$ do método de Jacobi aplicado a este sistema com aproximação inicial $x^{(1)} = (1, 1, -1, -1)$. Também, compute $\|Ax^{(5)} - b\|$.
\end{exer}
\begin{resp}
  %   \ifisoctave 
  %   \href{https://github.com/phkonzen/notas/blob/master/src/MatematicaNumerica/cap_sl_iter/dados/exer_jacobi_exec/exer_jacobi_exec.m}{Código.} 
  % \fi
  $x^{(5)} = (-1,00256,~2,95365,~-1,95347,~0,97913)$; $\|Ax^{(5)}-b\| = 0,42244$
\end{resp}

\begin{exer}\label{exer:gs_exec}
  Considere o seguinte sistema linear
  \begin{align}
    -4x_1 + x_2 + x_3 - x_4 &= -1\\
    5x_2 -x_3 + 2x_4 &= 3\\
    -x_1 + 4x_3 - 2x_4 &= -2\\
    x_1 -x_2 -5x_4 &= 1
  \end{align}
  Compute a quinta iterada $x^{(5)}$ do método de Gauss-Seidel aplicado a este sistema com aproximação inicial $x^{(1)} = (1, 1, -1, -1)$. Também, compute $\|Ax^{(5)} - b\|$.
\end{exer}
\begin{resp}
  %   \ifisoctave 
  %   \href{https://github.com/phkonzen/notas/blob/master/src/MatematicaNumerica/cap_sl_iter/dados/exer_gs_exec/exer_gs_exec.m}{Código.} 
  % \fi
  $x^{(5)} = (-1,00423,~3,00316,~-2,00401,~0,99852)$; $\|Ax^{(5)}-b\| = 0,025883$
\end{resp}

\section{Método do gradiente}\label{cap_sl_iter_sec_metg}

Começamos observando que se $A$ é uma matriz $n\times n$ positiva definida\footnote{$A$ é simétrica e $x^TAx > 0$ para todo $x\neq 0$.}, temos que $\pmb{x}\in\mathbb{R}^n$ é solução de
\begin{equation}\label{eq:metg_sislin}
  A\pmb{x} = \pmb{b}
\end{equation}
se, e somente se, $\pmb{x}$ é solução do seguinte problema de minimização
\begin{equation}\label{eq:metg_minprob}
  \min_{\pmb{x}\in\mathbb{R}^n}f(\pmb{x}) := \frac{1}{2}\pmb{x}^TA\pmb{x}-\pmb{x}^T\pmb{b}.
\end{equation}

O método do gradiente é um algoritmo da forma: dada uma aproximação inicial $\pmb{x}^{(1)}$ da solução de \eqref{eq:metg_minprob} (ou, equivalentemente, de \eqref{eq:metg_sislin}), computamos novas aproximações da forma iterativa
\begin{equation}
  \pmb{x}^{(k+1)} = \pmb{x}^{(k)} + \alpha^{(k)}\pmb{d}^{(k)},\quad k=1, 2, \ldots,
\end{equation}
onde $\alpha^{(k)}$ é o tamanho do passo (um escalar) e $\pmb{d}^{(k)}\in\mathbb{R}^n$ é a direção de busca.

Para escolhermos a direção $\pmb{d}^{(k)}$, tomamos a fórmula de Taylor de $f$ em torno da aproximação $\pmb{x}^{(k)}$
\begin{equation}\label{eq:metg_taylor}
  f(\pmb{x}^{(k+1)}) = f(\pmb{x}^{(k)}) + \alpha^{(k)}\nabla f(\pmb{x}^{(k)})\cdot \pmb{d}^{(k)} + O\left((\alpha^{(k)})^2\right),
\end{equation}
com $\alpha^{(k)}\to 0$, onde $\nabla f$ denota o gradiente de $f$, i.e.
\begin{align}
  \nabla f(\pmb{x}^{(k)}) &= \left(\frac{\p f}{\p x_1}(\pmb{x}^{(k)}), \frac{\p f}{\p x_2}(\pmb{x}^{(k)}), \dotsc, \frac{\p f}{\p x_n}(\pmb{x}^{(k)})\right)\\
  &= A\pmb{x}^{(k)}-\pmb{b}.
\end{align}


De \eqref{eq:metg_taylor}, segue que se
\begin{equation}
  \nabla f(\pmb{x}^{(k)})\cdot \pmb{d}^{(k)} < 0,
\end{equation}
então $f(\pmb{x}^{(k+1)}) < f(\pmb{x}^{(k)})$ se $\alpha^{(k)}$ é suficientemente pequeno. Em particular, podemos escolher
\begin{equation}
  \pmb{d}^{(k)} = -\nabla f(\pmb{x}^{(k)}),
\end{equation}
se $\nabla f(\pmb{x}^{(k)})\neq 0$.

Do exposto acima, temos a \pmb{iteração do método do gradiente}
\begin{align}
  \pmb{x}^{(1)} &= \text{aprox. inicial}\\
  \pmb{x}^{(k+1)} &= \pmb{x}^{(k)} - \alpha^{(k)}\pmb{r}^{(k)},~k=1, 2, \ldots,
\end{align}
onde $\pmb{r}^{(k)}$ é o resíduo da iterada $k$ dado por
\begin{equation}
  \pmb{r}^{(k)} = A\pmb{x^{(k)}}-\pmb{b}.
\end{equation}

\begin{ex}\label{ex:metg_pc}
  Consideremos o sistema $Ax = b$ com
  \begin{equation}
    A =
    \begin{bmatrix}
      2 & -1 & 0 & 0\\
      -1 & 2 & -1 & 0\\
      0 & -1 & 2 & -1 \\
      0 & 0 & -1 & 2
    \end{bmatrix},\quad
    b =
    \begin{bmatrix}
      -3\\
      2\\
      2\\
      -3
    \end{bmatrix}.
  \end{equation}
  Na Tabela \ref{tab:metg_pc} temos os resultados do emprego do método do gradiente com $\pmb{x}^{(1)} = (0, 0, 0, 0)$ e com passo constante $\alpha^{(k)}\equiv 0,5$.

  \begin{table}[h!]
    \centering
    \caption{Resultados referentes ao Exemplo \ref{ex:metg_pc}.}
    \label{tab:metg_pc}
    \begin{tabular}{l|c|c}
      k & $\pmb{x}^{(k)}$ & $\|A\pmb{x}^{(k)}-\pmb{b}\|$\\\hline
      1 & $(0,0,~0,0,~0,0,~0,0)$ & $5,1\E+0$\\
      2 & $(-1,5,~1,0,~1,0,~-1,5)$ & $1,6\E+0$\\
      3 & $(-1,0,~0,8,~0,8,~-1,0)$ & $5,0\E-1$\\
      4 & $(-1,1,~0,9,~0,9,~-1,1)$ & $1,8\E-1$\\
      5 & $(-1,1,~0,9,~0,9,~-1,1)$ & $8,8\E-2$\\
      6 & $(-1,1,~0,9,~0,9,~-1,1)$ & $6,2\E-2$\\
      7 & $(-1,0,~0,9,~0,9,~-1,0)$ & $4,9\E-2$\\
      8 & $(-1,0,~0,9,~0,9,~-1,0)$ & $4,0\E-2$\\
      9 & $(-1,0,~0,9,~0,9,~-1,0)$ & $3,2\E-2$\\
      10 & $(-1,0,~1,0,~1,0,~-1,0)$ & $2,6\E-2$\\
      11 & $(-1,0,~1,0,~1,0,~-1,0)$ & $2,1\E-2$\\\hline
    \end{tabular}
  \end{table}

% \ifisoctave
% No \verb+GNU Octave+, podemos fazer as computações acima com o seguinte \href{https://github.com/phkonzen/notas/blob/master/src/MatematicaNumerica/cap_sl_iter/dados/ex_metg_pc/ex_metg_pc.m}{código}:
% \verbatiminput{./cap_sl_iter/dados/ex_metg_pc/ex_metg_pc.m}
% \fi
\end{ex}

\subsection{Escolha do passo}

Da iteração do método do gradiente, temos que a melhor escolha do passo $\alpha^{(k)}$ é tal que
\begin{equation}
  f(\pmb{x}^{(k)}+\alpha^{(k)}\pmb{r}^{(k)}) = \min_{\alpha > 0} f(\pmb{x}^{(k)}+\alpha\pmb{r}^{(k)}).
\end{equation}
Desta forma,
\begin{align}
  \frac{\dd}{\dd \alpha}f(\pmb{x}^{(k)}+\alpha \pmb{r}^{(k)}) = 0 &\Rightarrow \nabla f(\pmb{x}^{(k+1)})\cdot \pmb{r}^{(k)} = 0,\\
  &\Rightarrow \left(A(\pmb{x}^{(k)}+\alpha^{(k)}\pmb{r}^{(k)})-b\right)\cdot\pmb{r}^{(k)} = 0,\\
  &\Rightarrow (A\pmb{x}^{(k)}-\pmb{b})\cdot\pmb{r}^{(k)}+\alpha^{(k)}\pmb{r}^{(k)}\cdot A\pmb{r}^{(k)} = 0,
\end{align}
donde
\begin{equation}
  \alpha^{(k)} = - \frac{\pmb{r}^{(k)}\cdot\pmb{r}^{(k)}}{\pmb{r}^{(k)}\cdot A\pmb{r}^{(k)}}.
\end{equation}

\begin{ex}\label{ex:metg_alpha}
  Consideremos o sistema $Ax = b$ com
  \begin{equation}
    A =
    \begin{bmatrix}
      2 & -1 & 0 & 0\\
      -1 & 2 & -1 & 0\\
      0 & -1 & 2 & -1 \\
      0 & 0 & -1 & 2
    \end{bmatrix},\quad
    b =
    \begin{bmatrix}
      -3\\
      2\\
      2\\
      -3
    \end{bmatrix}.
  \end{equation}
  Na Tabela \ref{tab:metg_alpha} temos os resultados do emprego do método do gradiente com $\pmb{x}^{(1)} = (0, 0, 0, 0)$ e com passo
  \begin{equation}
    \alpha^{(k)} = - \frac{\pmb{r}^{(k)}\cdot\pmb{r}^{(k)}}{\pmb{r}^{(k)}\cdot A\pmb{r}^{(k)}}.
\end{equation}

  \begin{table}[h!]
    \centering
    \caption{Resultados referentes ao Exemplo \ref{ex:metg_alpha}.}
    \label{tab:metg_alpha}
    \begin{tabular}{l|c|c}
      k & $\pmb{x}^{(k)}$ & $\|A\pmb{x}^{(k)}-\pmb{b}\|$\\\hline
      1 & $(0,0,~0,0,~0,0,~0,0)$ & $5,1\E+0$\\
      2 & $(-1,1,~0,8,~0,8,~-1,1)$ & $1,5\E-1$\\
      3 & $(-1,0,~1,0,~1,0,~-1,0)$ & $3,0\E-2$\\
      4 & $(-1,0,~1,0,~1,0,~-1,0)$ & $8,8\E-4$\\
      5 & $(-1,0,~1,0,~1,0,~-1,0)$ & $1,8\E-4$\\\hline
    \end{tabular}
  \end{table}

% \ifisoctave
% No \verb+GNU Octave+, podemos fazer as computações acima com o seguinte \href{https://github.com/phkonzen/notas/blob/master/src/MatematicaNumerica/cap_sl_iter/dados/ex_metg_alpha/ex_metg_alpha.m}{código}:
% \verbatiminput{./cap_sl_iter/dados/ex_metg_alpha/ex_metg_alpha.m}
% \fi
\end{ex}

\subsection*{Exercícios}

\emconstrucao

\section{Método do gradiente conjugado}\label{cap_sl_iter_sec_metgc}

O método do gradiente conjugado é uma variação do método do gradiente (veja Seção \ref{cap_sl_iter_sec_metg}). Aqui, a solução de um dado sistema $A\pmb{x}=\pmb{b}$, com $A$ uma matriz positiva definida, é computada de forma iterativa por
\begin{align}
  \pmb{x}^{(1)} &= \text{aprox. inicial},\\
  \pmb{d}^{(1)} &= \pmb{r}^{(1)},\\
  &\\
  \pmb{x}^{(k+1)} &= \pmb{x}^{(k)} + \alpha_k\pmb{d}^{(k)},\\
  \alpha^{(k)} &= -\frac{\pmb{r}^{(k)}\cdot \pmb{d}^{(k)}}{\pmb{d}^{(k)}\cdot A\pmb{d}^{(k)}},\\
  \pmb{d}^{(k+1)} &= -\pmb{r}^{(k+1)}+\beta_k\pmb{d}^{(k)},\\
  \beta^{(k)} &= \frac{\pmb{r}^{(k+1)}\cdot A\pmb{d}^{(k)}}{\pmb{d}^{(k)}\cdot A\pmb{d}^{(k)}},
\end{align}
para $k = 1, 2, \ldots$, e $\pmb{r}^{(k)} = A\pmb{x}^{(k)}-\pmb{b}$.

\begin{ex}\label{ex:metgc_exec}
  Consideremos o sistema $Ax = b$ com
  \begin{equation}
    A =
    \begin{bmatrix}
      2 & -1 & 0 & 0\\
      -1 & 2 & -1 & 0\\
      0 & -1 & 2 & -1 \\
      0 & 0 & -1 & 2
    \end{bmatrix},\quad
    b =
    \begin{bmatrix}
      -3\\
      2\\
      2\\
      -3
    \end{bmatrix}.
  \end{equation}
  Na Tabela \ref{tab:metgc_exec} temos os resultados do emprego do método do gradiente conjugado com $\pmb{x}^{(1)} = (0, 0, 0, 0)$.

  \begin{table}[h!]
    \centering
    \caption{Resultados referentes ao Exemplo \ref{ex:metgc_exec}.}
    \label{tab:metgc_exec}
    \begin{tabular}{l|c|c}
      k & $\pmb{x}^{(k)}$ & $\|A\pmb{x}^{(k)}-\pmb{b}\|$\\\hline
      1 & $(0,~0,~0,~0)$ & $5.1\E+0$\\
      2 & $(-1,1,~0,8,~0,8,~-1,1)$ & $1,5\E-1$\\
      3 & $(-1,0,~1,0,~1,0,~-1,0)$ & $0,0\E+0$\\\hline
    \end{tabular}
  \end{table}

% \ifisoctave
% No \verb+GNU Octave+, podemos fazer as computações acima com o seguinte \href{https://github.com/phkonzen/notas/blob/master/src/MatematicaNumerica/cap_sl_iter/dados/ex_metgc_exec/ex_metgc_exec.m}{código}:
% \verbatiminput{./cap_sl_iter/dados/ex_metgc_exec/ex_metgc_exec.m}
% \fi
\end{ex}

\subsection*{Exercícios}

\emconstrucao