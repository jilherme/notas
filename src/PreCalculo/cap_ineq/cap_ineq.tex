%Este trabalho está licenciado sob a Licença Atribuição-CompartilhaIgual 4.0 Internacional Creative Commons. Para visualizar uma cópia desta licença, visite http://creativecommons.org/licenses/by-sa/4.0/deed.pt_BR ou mande uma carta para Creative Commons, PO Box 1866, Mountain View, CA 94042, USA.

\chapter{Equações e inequações}\label{cap_ineq}
\thispagestyle{fancy}

\begin{flushright}
  [Vídeo] | [Áudio] | \href{https://phkonzen.github.io/notas/contato.html}{[Contatar]}
\end{flushright}

\section{Equações}\label{cap_ineq_sec_eq}

Uma equação é uma declaração de que duas expressões são iguais. Escrevemos
\begin{equation}
  E_e = E_d
\end{equation}
para estabelecer que a expressão à esquerda $E_e$ é igual a expressão à direira $E_d$.

\begin{ex}
  Estudemos os seguintes casos:
  \begin{enumerate}[a)]
  \item $2^2 = 4$
  \item $2x - 1 = 0$
  \item $e^{x+y} = e^xe^y$
  \item $\displaystyle \frac{x^2-1}{x+1} = x - 1$
  \end{enumerate}

  \begin{ifispython}
    No \python, podemos declarar as equações com a função \lstinline{https://docs.sympy.org/latest/modules/core.html?highlight=equality#sympy.core.relational.Equality}{Eq}. Os casos são implementados como segue:
    \begin{lstlisting}
      >>> from sympy import *
      >>> Eq(2**2, 4)
      True
      >>> x = Symbol('x')
      >>> Eq(2*x - 1, 0)
      Eq(2*x - 1, 0)
      >>> y = Symbol('y')
      >>> Eq(exp(x+y), exp(x)*exp(y))
      Eq(exp(x + y), exp(x)*exp(y))
      >>> Eq((x**2-1)/(x+1), x-1)
      Eq((x**2 - 1)/(x + 1), x - 1)
    \end{lstlisting}
  \end{ifispython}
\end{ex}

\subsection{Solução de uma equação}

Equação é uma poderosa ferramenta matemática para impor uma condição sobre uma ou mais \emph{incógnitas} (ou \emph{variáveis}). Por exemplo, quando escrevemos
\begin{equation}
  2^x = 4
\end{equation}
estamos impondo que a incógnita $x$ seja aquela a satisfazer esta equação. No caso, $x=2$ satisfaz a equação, pois ao substituirmos $x$ por $2$ nela, obtemos
\begin{gather}
  2^2 = 4
  \Leftrightarrow 4 = 4.
\end{gather}
Usualmente, dizemos que $x=2$ é \emph{solução} da equação. O procedimento de encontrar a(s) solução(ões) de uma equação é chamado de \emph{resolução} da equação, i.e. o procedimento de resolver a equação.

\begin{obs}
  Uma equação pode ter uma única solução, várias soluções, infinitas soluções ou nenhuma solução.
\end{obs}

\begin{ex}
  Estudemos os seguintes casos:
  \begin{enumerate}[a)]
  \item $x - 1 = 0$ tem solução única $x=1$.
  \item $y^2 - 1 = 0$ têm soluções $y=-1$ ou $y=1$.
  \item $x^2 = -1$ não tem solução.
  \item $(u+1)^2 = u^2 + 2u + 1$, qualquer $u\in\mathbb{R}$ é solução.
  \end{enumerate}

  \ifispython
  No \python, podemos resolver estas equação com o comando \href{https://docs.sympy.org/latest/tutorial/solvers.html#solving-equations-algebraically}{solve ou solveset}. Estudemos as seguintes entradas e saídas:
  \begin{lstlisting}
    >>> from sympy import *
    >>> x = Symbol('x', real=True)
    >>> solve(x-1, domain=S.Reals)
    [1]
    >>> solveset(x-1, domain=S.Reals)
    FiniteSet(1)
    >>> y,u = symbols('y,u', real=True)
    >>> solve(y**2-1, domain=S.Reals)
    [-1, 1]
    >>> solve(Eq(x**2, -1), domain=S.Reals)
    []
    >>> solveset(Eq(x**2, -1), domain=S.Reals)
    EmptySet
    >>> solveset(Eq((u+1)**2, u**2 + 2*u + 1), domain=S.Reals)
    Reals
  \end{lstlisting}
  \fi
\end{ex}

Não existe um procedimento único para a resolução de equações em geral. Em síntese, a resolução, quando possível, é obtida da aplicação das seguintes propriedades. Sendo $E_1, E_2 e E_3$ expressões matemáticas, temos
\begin{itemize}
\item
  \begin{equation}
    {\color{red}E_1} = {\color{olive}E_2} \Leftrightarrow {\color{olive}E_2} = {\color{red}E_1}
  \end{equation}
\item
  \begin{equation}
    E_1 = E_2 \Leftrightarrow E_1 + {\color{blue}E_3} = E_2 + {\color{blue}E_3}
\end{equation}
\item Para $E_3\in\mathbb{R}^*$
  \begin{equation}
    E_1 = E_2 \Leftrightarrow E_1\cdot {\color{blue}E_3} = E_2 \cdot {\color{blue}E_3}
  \end{equation}
\end{itemize}

\subsection{Equações lineares}

Equação algébricas lineares de uma incógnita são aquelas que podem ser escritas na seguinte forma
\begin{equation}
  ax + b = 0,
\end{equation}
onde, são conhecidos (dados) os parâmetros $a\in\mathbb{R}^*$ e $b\in\mathbb{R}$. Sua resolução pode ser feita da seguinte forma
\begin{gather}
  ax + b = 0 \\
  ax + b - b = 0 - b \\
  ax = -b \\
  \frac{1}{a}\cdot ax = \frac{1}{a}\cdot (-b) \\
  1\cdot x = -\frac{b}{a} \\
  x = -\frac{b}{a}.
\end{gather}

\begin{ex}
  Vamos resolver
  \begin{equation}
    2x -4 = 5 - x
  \end{equation}
  Esta é uma equação linear, pois
  \begin{gather}
    2x - 4 - 5 = 5 - x - 5 \\
    2x -9 = -x \\
    x + 2x - 9 = x - x \\
    3x - 9 = 0.
  \end{gather}
  Logo, a solução é
  \begin{equation}
    x = \frac{9}{3} = 3.
  \end{equation}

  \ifispython
  No \python, podemos resolver esta equação com
  \begin{lstlisting}
    >>> from sympy import *
    >>> x = Symbol('x', real=True)
    >>> solve(Eq(2*x - 4, 5 - x), domain=S.Reals)
    [3]
  \end{lstlisting}
  \fi
\end{ex}

\subsection{Equação quadrática}

Uma equação algébrica quadrática de um incógnita é aquela que pode ser escrita na forma
\begin{equation}\label{eq:ineq_eqquad}
  ax^2 + bx + c = 0,
\end{equation}
com $a\in\mathbb{R}^*$ e $b,c\in\mathbb{R}$.

Para resolver tal equação, vamos, primeiro, lembrar que
\begin{equation}
  (a + b)^2 = a^2 + 2ab + b^2,
\end{equation}
para quaisquer $a,b\in\mathbb{R}$. Daí, a ideia é \emph{completar os quadrados} na equação \eqref{eq:ineq_eqquad}, fazendo
\begin{gather}
  ax^2 + bx + c - c= 0 - c \\
  ax^2 + bx = -c \\
  \left(ax^2 + bx\right)\frac{1}{a} = -c\cdot\frac{1}{a} \\
  x^2 + \frac{b}{a}x = -\frac{c}{a} \\
  x^2 + \frac{b}{a}x + \left(\frac{b}{2a}\right)^2 = -\frac{c}{a}  + \left(\frac{b}{2a}\right)^2 \\
  \left(x + \frac{b}{2a}\right)^2 = -\frac{c}{a} + \frac{b^2}{4a^2} \\
  \left(x + \frac{b}{2a}\right)^2 = \frac{b^2 - 4ac}{4a^2} \\  
  \sqrt{\left(x + \frac{b}{2a}\right)^2} = \sqrt{\frac{b^2 - 4ac}{4a^2}} \\
  x + \frac{b}{2a} = \pm\frac{b^2 - 4ac}{2a} \\
  x + \frac{b}{2a} - \frac{b}{2a} = - \frac{b}{2a} \pm\frac{b^2 - 4ac}{2a}\\
\end{gather}
donde temos a chamada \href{https://pt.wikipedia.org/wiki/Bhaskara\_II}{fórumla de Bhaskara}
\begin{equation}
  x = \frac{-b \pm \sqrt{b^2 - 4ac}}{2a}.
\end{equation}

\begin{ex}
  Vamos resolver
  \begin{equation}
    x^2 = x + 2.
  \end{equation}
  Esta é uma equação quadrática, pois
  \begin{gather}
    x^2 - x - 2 = x + 2 - x - 2 \\
    x^2 -x - 2 = 0.
  \end{gather}
  Logo, da fórmula da Bhaskara, obtemos
  \begin{gather}
    x = \frac{-(-1) \pm \sqrt{(-1)^2 - 4\cdot 1 \cdot (-2)}}{2\cdot 1} \\
    x = \frac{1 \pm \sqrt{1 + 8}}{2} \\
    x = \frac{1 \pm \sqrt{9}}{2} \\
    x = \frac{1 \pm 3}{2}
  \end{gather}
  Donde,
  \begin{gather}
    x = \frac{1 - 3}{2} \\
    x = \frac{-2}{2} \\
    x = -1
  \end{gather}
  ou
  \begin{gather}
    x = \frac{1 + 3}{2} \\
    x = \frac{4}{2} \\
    x = 2
  \end{gather}
  Concluímos que a equação tem soluções $x=-1$ ou $x=2$.

  \ifispython
  No \python, podemos resolver esta equação com
  \begin{lstlisting}
    >>> from sympy import *
    >>> x = Symbol('x', real=True)
    >>> solve(Eq(x**2, x + 2), domain=S.Reals)
    [-1, 2]
  \end{lstlisting}
  \fi
\end{ex}

\subsection{Equações exponenciais}

Um equação exponencial é aquela em que a incógnita aparece como expoente em um ou mais termos. Tais equações não tem formato único, nem procedimento geral de resolução. Quando possível, a ideia é reescrever todos os termos da equação em uma base comum.

\begin{obs}
  Lembramos que\footnote{Quando bem definido.}:
  \begin{itemize}
  \item $\displaystyle b^x = b^y \Leftrightarrow x=y$
  \item $\displaystyle b^{x+y} = b^x\cdot b^y$
  \item $\displaystyle b^{xy} = \left(b^x\right)^y$
  \item $\displaystyle b^{-x} = \frac{1}{b^x}$
  \item $\displaystyle b^{\frac{x}{y}} = \sqrt[y]{b^x}$
  \end{itemize}
\end{obs}


\begin{ex}
  Vamos resolver
  \begin{equation}
    5^{x+3} = 25.
  \end{equation}
  Para resolver esta equação, vamos escrever $25$ como potência de $5$, i.e.
  \begin{equation}
    25 = 5^2.
  \end{equation}
  Logo, a equação é equivalente a
  \begin{equation}
    5^{x+3} = 5^2
  \end{equation}
  donde
  \begin{gather}
    x+3 = 2 \\
    x = -1.
  \end{gather}
  Ou seja, a solução é $x=-1$.

  \ifispython
  No \python:
  \begin{lstlisting}
    >>> from sympy import *
    >>> x = Symbol('x', real=True)
    >>> solve(Eq(5**(x+3), 25), domain=S.Reals)
    [-1]
  \end{lstlisting}
  \fi
\end{ex}

\begin{ex}
  Vamos resolver
  \begin{equation}
    5^{x+3} = 5^{-x} + 20.
  \end{equation}
  Notamos que esta equação é equivalente a
  \begin{equation}
    5^x\cdot 5^3 = \left(5^x\right)^{-1} + 20.
  \end{equation}
  Fazemos, então, a seguinte \emph{mudança de variável}
  \begin{equation}
    y = 5^x.
  \end{equation}
  Com isso, a equação se resume a
  \begin{equation}
    y\cdot 5^3 = y^{-1} + 20
  \end{equation}
  Resolvemos esta equação como segue
  \begin{gather}
    125y = \frac{1}{y} + 20 \\
    125y^2 = 1 + 20y \\
    125y^2 - 20y - 1 = 0
  \end{gather}
  Usando a fórmula de Bhaskara, obtemos
  \begin{gather}
    y = \frac{20 \pm \sqrt{20^2 - 4\cdot 125\cdot (-1)}}{2\cdot 125}\\
    y = \frac{20 \pm \sqrt{900}}{250} \\
    y = \frac{20 - \pm 30}{250}
  \end{gather}
  Ou seja, $y = -1/25$ ou $y = 1/5$. Observando que $y=5^x$ e, portanto positivo, temos
  \begin{equation}
    5^x = \frac{1}{5} = 5^{-1}.
  \end{equation}
  Concluímos que $x = -1$.

  \ifispython
  No \python:
  \begin{lstlisting}
    >>> from sympy import *
    >>> x = Symbol('x', real=True)
    >>> solve(Eq(5**(x+3), 25), domain=S.Reals)
    [-1]
  \end{lstlisting}
  \fi  
\end{ex}

\subsection*{Exercícios}

\emconstrucao

\section*{Inequações}\label{cap_ineq_sec_ineq}

\emconstrucao

\subsection*{Exercícios}

\emconstrucao
