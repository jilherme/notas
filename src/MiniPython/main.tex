%Este trabalho está licenciado sob a Licença Atribuição-CompartilhaIgual 4.0 Internacional Creative Commons. Para visualizar uma cópia desta licença, visite http://creativecommons.org/licenses/by-sa/4.0/deed.pt_BR ou mande uma carta para Creative Commons, PO Box 1866, Mountain View, CA 94042, USA.

\documentclass[12pt]{article}

\input ../preambulo.tex

\usepackage{url}

\makeindex

\begin{document}

\ifispython
\lstset { %
  language=Python,
  numbers=right,
  numberstyle=\small,
  stepnumber=1,    
  firstnumber=1,
  xleftmargin=-1em,
  extendedchars=true,
  inputencoding=utf8,
  upquote=true,
  basicstyle=\ttfamily,
  keywordstyle=\ttfamily,
  stringstyle=\ttfamily,
  commentstyle=\ttfamily,
  showspaces=false,
  showstringspaces=false,
  showtabs=false,
}
\fi

\title{Minicurso de Python para Matemática}
\author{Pedro H A Konzen}
\date{\today}
% \ifishtml
% \else
% \addcontentsline{toc}{chapter}{Capa}
% \fi

\maketitle

\tableofcontents

\section*{Licença}\label{sec_licenca}

Este trabalho é uma adaptação livre a partir está licenciado sob a Licença Atribuição-CompartilhaIgual 4.0 Internacional Creative Commons. Para visualizar uma cópia desta licença, visite http://creativecommons.org/licenses/by-sa/4.0/deed.pt\_BR ou mande uma carta para Creative Commons, PO Box 1866, Mountain View, CA 94042, USA.


\section{Sobre a linguagem}\label{sec_sobrepy}

{\python} é uma linguagem de programação de alto nível e multi-paradigma. Ou seja, é relativamente próxima das linguagens humanas naturais, é desenvolvida para aplicações diversas e permite a utilização de diferentes paradigmas de programação (programação estruturada, orientada a objetos, orientada a eventos, paralelização, etc.).

\begin{itemize}
\item Site oficial: \href{https://www.python.org/}{https://www.python.org/}
\end{itemize}

\subsection{Instalação e execução}

Para executar um código {\python} é necessário instalar um interpretador. No \href{https://www.python.org/}{site oficial do \python} estão disponível para {\it download} interpretadores gratuitos e com licença para uso livre. Neste minicurso, vamos utilizar {\python} 3 instalado em um sistema \verb+Linux+. Para outros sistemas, pode ser necessário fazer algumas pequenas adequações.

\subsection{Utilização}

A execução de códigos \python pode ser feita de três formas básicas:
\begin{itemize}
\item em modo iterativa em um console \python;
\item por execução de um código \verb+.py+ em um console \python;
\item por execução de um cógido \verb+.py+ em um terminal;
\end{itemize}

\begin{ex}
  Implemente o seguinte pseudocódigo.
\begin{verbatim}
s = "Ola, mundo!".
imprime(s). (imprime a string s)
\end{verbatim}
  \begin{enumerate}[a)]
  \item em modo iterativo no console;
  \item escrevendo o código \verb+.py+ e executando-o no console;
  \item escrevendo o código \verb+.py+ e executando-o no terminal.
  \end{enumerate}
  
  \noindent{\bf Resolução.} Seguem as implementações em cada caso.

  \begin{enumerate}[a)]
  \item Em modo iterativo.

    Iniciamos um console {\python} em terminal digitando
\begin{verbatim}
$ python3
\end{verbatim}
    Então, digitamos
    \begin{lstlisting}[xleftmargin=-3em]
      >>> s = "Ola, Mundo!"
      >>> print(s) #imprime a string s
      Ola, Mundo!
    \end{lstlisting}
    Para encerrar o console, digitamos
    \begin{lstlisting}[xleftmargin=-3em]
      >>> quit()
    \end{lstlisting}
    
  \item Executando {\it script} \verb+.py+ no console.

    Primeiramente, escrevemos o código
    \begin{lstlisting}
      s = "Ola, Mundo!"
      print(s) # imprime a string s
    \end{lstlisting}
    em um editor de texto (ou no seu IDE de preferência) e salvamo-lo em \verb+/pasta/codigo.py+. Então, executamo-lo no console {\python} com
    \begin{lstlisting}[xleftmargin=-3em]
      >>> exec(open('/pasta/codigo.py').read())
      Ola, mundo!
    \end{lstlisting}

  \item Executando o código em terminal.

    Considerando que já temos o código salvo em \verb+/pasta/codigo.py+, executamo-lo com
\begin{verbatim}
$ python3 /pasta/codigo.py
Olá, mundo!
\end{verbatim}
  \end{enumerate}
\end{ex}

\section{Elementos da linguagem}\label{sec_elem}

{\python} é uma linguagem de programação dinâmica em que as variáveis são declaradas automaticamente ao receberem um valor. Por exemplo, consideremos as seguintes instruções
\begin{lstlisting}
  >>> x = 1
  >>> y = x * 2.0
\end{lstlisting}
Na primeira instrução, a variável \lstinline+x+ é recebe o valor inteiro $1$ e, então, é armazenado na memória do computador como um objeto da classe \lstinline+int+. Na segunda instrução, \lstinline+y+ recebe o valor decimal $2.0$ (resultado de $1\times 2.0$) e é armazenado como um objeto da classe \lstinline+float+ (ponto flutuante de 64-{\it bits}). Podemos verificar isso, com as seguintes instruções
\begin{lstlisting}
  >>> print(x, y)
  1 2.0
  >>> print(type(x), type(y))
  <class 'int'> <class 'float'>
\end{lstlisting}

Códigos {\python} admitem comentários e continuação de linha como no seguinte exemplo
\begin{lstlisting}
  >>> # isso eh um comentario
  >>> s = "isso eh uma \
  ... string"
  >>> print(s)
  isso eh uma string
  >>> type(s)
  <class 'str'>
\end{lstlisting}

\begin{obs}(Notação científica)
  O {\python} aceita notação científca, por exemplo $5.2\times 10^{-2}$ é digitado como
  \begin{lstlisting}
    >>> 5.2e-2
    0.052
  \end{lstlisting}
\end{obs}

\begin{obs}
  Além dos tipos numéricos e {\it string}, {\python} também conta com os tipos de dados \lstinline+list+ (lista), \lstinline+tuple+ ($n$-upla) e \lstinline+dict+ (dicionário). Estudaremos estes tipos mais adiante neste minicurso.
\end{obs}

\subsection{Operações aritméticas elementares}

Os operadores aritméticos elementares são:
\begin{itemize}
\item[]\lstinline-+-: adição
\item[]\lstinline+-+: subtração
\item[]\lstinline+*+: multiplicação
\item[]\lstinline+/+: divisão
\item[]\lstinline+**+: potenciação
\item[]\lstinline+%+: módulo
\item[]\lstinline+//+: módulo
\end{itemize}

Consideremos o seguinte exemplo
\begin{lstlisting}
  >>> 2+8*3/2**2-1
  7.0
\end{lstlisting}
Observamos que as operações \lstinline+**+ tem precedência sobre as operações \lstinline+*, /+, as quais têm precedência sobre as operações \lstinline!+, -!. Operações de mesma precedência seguem a ordem da esquerda para direita, conforme escritas na linha de comando. Usa-se parênteses para alterar a precedência entre as operações, por exemplo
\begin{lstlisting}
  >>> (2+8*3)/2**2-1
  5.5
\end{lstlisting}
Consulte mais informações sobre a precedência de operadores em \href{https://docs.python.org/3/reference/expressions.html#operator-precedence}{Python Docs}.

\begin{exr}
  Compute as raízes do seguinte polinômio quadrático
  \begin{equation}
    x^2 - x - 2
  \end{equation}
  usando a fórmula de Bhaskara\footnote{Bhaskara Akaria, 1114 - 1185, matemático e astrônomo indiano. Fonte: \href{https://pt.wikipedia.org/wiki/Bhaskara\_II}{Wikipédia}.}
\end{exr}

O operador \lstinline+%+ módulo computa o resto da divisão e o operador \lstinline+//+ a divisão inteira, por exemplo
\begin{lstlisting}
  >>> 5 % 2
  1
  >>> 5 // 2
  2
\end{lstlisting}

\begin{exr}
  Use o {\python} para verificar se $14/21$ é menor ou igual a $15/23$. Então, compute o resto da divisão do maior quociente.
\end{exr}

\subsection{Funções e constantes elementares}

O módulo Python \href{https://docs.python.org/3/library/math.html}{math} disponibiliza várias funções e constantes elementares. Para usá-las, precisamos importar o módulo para nossa seção. Fazemos isso com a instrução
\begin{lstlisting}
  >>> import math
\end{lstlisting}
Com isso, temos acesso a todas as definições e declarações contidas neste módulo. Por exemplo
\begin{lstlisting}
  >>> math.pi
  3.141592653589793
  >>> math.cos(math.pi)
  -1.0
  >>> math.sqrt(2)
  1.4142135623730951
  >>> math.log(math.e)
  1.0
\end{lstlisting}

\begin{obs}
  Notemos que \lstinline+math.log+ é a função logaritmo natural, i.e. $\ln(x) = \log_e(x)$. A implementação {\python} para o logaritmo de base 10 é \lstinline+math.log(x,10)+ ou, mais acurado, \lstinline+math.log10+.
\end{obs}

\begin{exr}
  Compute $e^{\log_3(\pi)}$.
\end{exr}

\subsection{Operadores de comparação elementares}

Os operadores de comparação elementares são
\begin{itemize}
\item[]\lstinline+==+: igual a
\item[]\lstinline+!=+: diferente de
\item[]\lstinline+>+: maior que
\item[]\lstinline+<+: menor que
\item[]\lstinline+>=+: maior ou igual que
\item[]\lstinline+<=+: menor ou igual que
\end{itemize}
Estes operadores retornam os valores lógicos \lstinline+True+ (verdadeiro) ou \lstinline+False+ (falso).

Por exemplo, temos
\begin{lstlisting}
  >>> x = 2
  >>> x + x == 4
  True
\end{lstlisting}

\begin{exr}
  Atribua a variável \lstinline+x+ o valor $\sqrt{3}$. Então, verifique se o valor computado de $x^2$ é maior que $3$. Em caso negativo, verifique se $x^2$ é menor que 3. Comente o resultado obtido.
\end{exr}

\subsection{Operadores lógicos elementares}

Os operadores lógicos elementares são:
\begin{itemize}
\item[]\lstinline+and+: e lógico
\item[]\lstinline+or+: ou lógico
\item[]\lstinline+not+: não lógico
\end{itemize}

A tabela booleana\footnote{George Boole, 1815 - 1864, matemático e filósofo britânico. Fonte: \href{https://pt.wikipedia.org/wiki/George\_Boole}{Wikipédia}.} do ``e'' lógico é
\begin{center}
  \begin{tabular}[H]{ll|l}
    Valor & Valor & Resultado \\\hline
    True & True & True \\
    True & False & False \\
    False & True & False \\
    False & False & False \\\hline
  \end{tabular}
\end{center}
Podemos verificar isso no {\python} como segue
\begin{lstlisting}
  >>> True and True
  True
  >>> True and False
  False
  >>> False and True
  False
  >>> False and False
  False
\end{lstlisting}

\begin{exr}
  Construa as tabelas booleanas do operador \lstinline+or+ e do \lstinline+not+.
\end{exr}

\begin{exr}
  Use {\python} para verificar se $1.4 <= \sqrt{2} < 1.5$. E, também, verifique se $\sqrt{3} > 1.7$ ou $\sqrt{3} >= 1.7321$.
\end{exr}

\subsection{Conjuntos}

{\python} tem conjuntos finitos como um tipo básico de variável. Um conjunto é uma coleção de itens \emph{não ordenada} e \emph{imutável} e \emph{não admite itens duplicados}. Por exemplo,
\begin{lstlisting}
  >>> a = {1, 2, 3}
  >>> type(a)
  <class 'set'>
  >>> b = set((2, 1, 3, 3))
  >>> b
  {1, 2, 3}
  >>> a == b
  True
  >>> # conjunto vazio
  >>> e = set()
\end{lstlisting}
aloca o conjunto $a = {1,2 3}$. Note que o conjunto $b$ é igual a $a$. Observamos que o conjunto vazio deve ser construído com a instrução \verb+set()+ e não com \verb+{}+\footnote{Isso constrói um dicionário vazio, como introduziremos logo mais.}.

\begin{obs}
  A função {\python} \verb+len+ retorna o número de elementos de um conjunto. Por exemplo,
  \begin{lstlisting}
    >>> len(a)
    3
  \end{lstlisting}
\end{obs}

\begin{itemize}
\item Operadores envolvendo conjuntos:
  \begin{itemize}
  \item[] \verb+-+: diferença entre conjuntos;
  \item[] \verb+|+: união de conjuntos;
  \item[] \verb+&+: interseção de conjuntos;
  \item[] \verb+^+: diferença simétrica;
  \end{itemize}
\end{itemize}

\begin{ex}
  Sejam os conjuntos
  \begin{gather}
    A = \{2, \pi, -0.25, 3, \text{'banana'}\}\\
    B = \{\text{'laranja'}, 3, \operatorname{arc cos}(-1), -1\}
  \end{gather}
  Compute
  \begin{enumerate}[a)]
  \item $A\setminus B$
  \item $A\cup B$
  \item $A\cap B$
  \item $A\Delta B = (A\setminus B) \cup (B\setminus A)$
  \end{enumerate}

  \noindent{\bf Resolução.} Começamos alocando os conjuntos como segue
  \begin{lstlisting}
    >>> import math
    >>> A = {2, math.pi, -0.25, 3, 'banana'}
    >>> B = {'laranja', 3, math.acos(-1), -1}
  \end{lstlisting}
  
  \begin{enumerate}[a)]
  \item $A\setminus B$
    \begin{lstlisting}
      >>> A - B
      {-0.25, 2, 'banana'}
    \end{lstlisting}
  \item $A\cup B$
    \begin{lstlisting}
      >>> A | B
      {-0.25, 2, 3, 3.141592653589793, \
      'laranja', 'banana', -1}
    \end{lstlisting}
  \item $A\cap B$
    \begin{lstlisting}
      >>> A & B
      {3, 3.141592653589793}
    \end{lstlisting}
  \item $A\Delta B$
    \begin{lstlisting}
      >>> A ^ B
      {-0.25, 2, 'laranja', 'banana', -1}
    \end{lstlisting}
  \end{enumerate}
\end{ex}

\begin{obs}
  {\python} disponibiliza a sintaxe de compreensão de conjuntos. Por exemplo,
  \begin{lstlisting}
    >>> {x for x in A if type(x) == type('')}
    {'banana'}
  \end{lstlisting}
\end{obs}

\begin{exr}
  Considere o conjunto
  \begin{equation}
    Z = {-3, -2, -1, 0, 1, 2, 3}.
  \end{equation}
  Faça um código {\python} para extrair os números pares do conjunto $Z$.
\end{exr}

\subsection{Listas}

O tipo {\python} \verb+list+ permite alocar em uma única variável uma lista de itens ordenada. Por exemplo, observemos as seguintes listas
\begin{lstlisting}
  >>> x = [-1, 2, -3, -5]
  >>> type(x)
  <class 'list'>
  >>> y = ['a', 'b', 'a']
  >>> y
  ['a', 'b', 'a']
  >>> vazia = []
  >>> len(x)
  4
\end{lstlisting}

Os elementos de uma lista são indexados, o índice $0$ corresponde ao primeiro elemento, o índice $1$ ao segundo elemento e assim por diante. Desta forma é possível o acesso direto a um elemento de uma lista usando-se sua posição. Por exemplo,
\begin{lstlisting}
  >>> x[0]
  -1
  >>> y[2] = 'c'
  >>> y
  ['a', 'b', 'c']
\end{lstlisting}
Pode-se fazer um corte de elementos de uma lista usando o operador \verb+:+. Por exemplo,
\begin{lstlisting}
  >>> x = [1,2,-1,3,-2]
  >>> x[1:3]
  [2,-1]
\end{lstlisting}

\begin{itemize}
\item Operadores básicos:
  \begin{itemize}
  \item[] \lstinline-+-: concatenação
    \begin{lstlisting}
      >>> [1, 2] + [3, 4, 5]
      [1, 2, 3, 4, 5]
    \end{lstlisting}
  \item[] \lstinline+*+: repetição
    \begin{lstlisting}
      >>> [1, 2] * 2
      [1, 2, 1, 2]
    \end{lstlisting}
  \item[] \lstinline+in+: pertencimento
    \begin{lstlisting}
      >>> 1 in [-1, 0, 1, 2]
      True
    \end{lstlisting}
  \end{itemize}
\end{itemize}

\begin{obs}
  Listas contam com várias funções prontas para a execução de diversas tarefas práticas como, por exemplo, inserir/deletar itens, contar ocorrências, ordenar itens, etc. Consulte \href{https://docs.python.org/3/tutorial/datastructures.html#more-on-lists}{Python Docs}.
\end{obs}

\begin{obs}(Alocação {\it versus} cópia)
  Estude o seguinte exemplo
  \begin{lstlisting}
    >>> x = [2, 3, 1]
    >>> y = x
    >>> y[1] = 0
    >>> x
    [2, 0, 1]
  \end{lstlisting}
  Notamos que \lstinline+y+ aponta para o mesmo endereço de memória de \lstinline+x+. Para copiar uma lista e alocá-la em um novo endereço de memória, deve-se usar a função \lstinline+list.copy()+, como segue
  \begin{lstlisting}
    >>> x = [2, 3, 1]
    >>> y = x.copy()
    >>> y[1] = 0
    >>> x
    [2, 3, 1]
    >>> y
    [2, 0, 1]
  \end{lstlisting}
\end{obs}

\begin{exr}
  \emconstrucao
\end{exr}

\subsection{$n$-uplas}

\emconstrucao

\subsection{Dicionários}

\emconstrucao


%references
\nocite{*}
% \bibliographystyle{plain}
\bibliography{main}
\addcontentsline{toc}{section}{Referências Bibliográficas}

\end{document}
