%Este trabalho está licenciado sob a Licença Atribuição-CompartilhaIgual 4.0 Internacional Creative Commons. Para visualizar uma cópia desta licença, visite http://creativecommons.org/licenses/by-sa/4.0/deed.pt_BR ou mande uma carta para Creative Commons, PO Box 1866, Mountain View, CA 94042, USA.

\chapter{Equações de ordem 1}\label{cap_ead1}

Neste capítulo, discutimos de forma introdutória sobre {\bf equações a diferenças de primeira ordem}. Tais equações podem ser escritas na forma
\begin{equation}
  f\left(y(n+1),y(n);n\right)=0,
\end{equation}
onde $n=0, 1, \ldots$ e $y:n\mapsto y(n)$ é função discreta (incógnita).

\section{Equações lineares}\label{cap_ead1_sec_eqlin}

Nesta seção, discutimos sobre equações a diferenças de ordem 1 e lineares. Tais equações podem ser escritas na seguinte forma
\begin{equation}
  y(n+1) = a(n)y(n) + g(n),
\end{equation}
onde $n=n_0, n_0+1, \ldots$, $n_0$ número inteiro, $a:n\mapsto a(n)$ e $g:n\mapsto g(n)$ é o termo fonte. A equação é dita ser {\bf homogênea} quando $g\equiv 0$ e, caso contrário, é dita ser {\bf não homogênea}.

\subsection{Equação homogênea}

A solução de uma equação a diferenças de ordem 1, linear e homogênea
\begin{equation}\label{eq:ead1_linear_eqh}
  y(n+1) = a(n)y(n),\quad n\geq n_0,
\end{equation}
pode ser obtida por iterações diretas. Para $n\geq n_0$, temos
\begin{align}
  y(n+1) &= a(n)y(n) \\
          &= a(n)a(n-1)y(n-1)\\
          &= a(n)a(n-1)a(n-2)y(n-2) \\
          &\vdots\\
          &= a(n)a(n-1)\cdots a(n_0)y(n_0).
\end{align}
Ou seja, dado o valor inicial $y(n_0)$, temos a solução\footnote{A demonstração por ser feita por indução matemática.}
\begin{equation}\label{eq:ead1_linear_eqh_sol}
  y(n) = \left[\prod_{i=n_0}^{n-1}a(i)\right]y(n_0),
\end{equation}
assumindo a notação de que $\prod_{i=n+1}^na(i)=1$.

\begin{ex}
  Vamos calcular a solução de
  \begin{equation}\label{eq:ex_ead1_lin_eh}
    y(n+1) = 2y(n),\quad n\geq 0.
  \end{equation}

  Comparando com \eqref{eq:ead1_linear_eqh}, temos $a(n) = 2$ para todo $n$. Calculando a solução por iterações diretas, temos
  \begin{align}
    y(n+1) &= 2y(n) \\
            &= 2\cdot 2y(n-1) \\
            &= 2^2y(n-1) \\
            &= 2^2\cdot 2y(n-2) \\
            &= 2^3y(n-2) \\
            &\cdots \\
            &= 2^{n+1}y(0)
  \end{align}
  Equivalentemente, por \eqref{eq:ead1_linear_eqh_sol}, temos
  \begin{align}
    y(n) &= \left[\prod_{i=0}^{n-1}2\right]y(0) \\
         &= 2^{n}y(0).
  \end{align}
  A solução vale para qualquer valor inicial $y(0)$.

  \ifispython
  No \python, podemos computar a solução da equação a diferenças \eqref{eq:ex_ead1_lin_eh} com os seguintes comandos:
\begin{verbatim}
In : n = symbols('n', integer=true)
In : y = symbols('y', cls=Function)
In : ead = Eq(y(n+1),2*y(n))
In : rsolve(ead, y(n))
Out: 2**n*C0
\end{verbatim}
  \fi
\end{ex}

\subsection{Equação não homogênea}

A solução de uma equação a diferenças de ordem 1, linear e não homogênea
\begin{equation}\label{eq:ead1_linear_eqnh}
  y(n+1) = a(n)y(n) + g(n),\quad n\geq n_0,
\end{equation}
pode ser obtida por iterações diretas.

Vejamos, para $n\geq n_0$ temos
\begin{align*}
  y(n+1) &= a(n)y_n + g(n) \\
         &= a(n)\left[a(n-1)y(n-1)+g(n-1)\right] + g(n) \\
         &= a(n)a(n-1)y(n-1)+a(n)g(n-1) + g(n) \\
         &= a(n)a(n-1)\left[a(n-2)y(n-2)+g(n-2)\right] \\
         &+ a(n)g(n-1) + g(n) \\
         &= a(n)a(n-1)a(n-2)y(n-2) \\
         &+ a(n)a(n-1)g(n-2) + a(n)g(n-1) + g(n) \\
         &\vdots
\end{align*}
Com isso, podemos inferir\footnote{A demonstração por ser feita por indução matemática.} que
\begin{align}
  y(n+1) &= \left[\prod_{i=n_0}^{n} a(i)\right]y(n_0) \\
         &+ \sum_{i=n_0}^n\left[\prod_{j=i+1}^{n} a(i)\right]g(i).
\end{align}
No último termo, consideramos a notação $\sum_{j=i+1}^i a(i) = 0$. Ou equivalentemente,
\begin{align}
  {\color{blue}y(n)} &{\color{blue}= \left[\prod_{i=n_0}^{n-1} a(i)\right]y(n_0)} \nonumber\\
       &{\color{blue}+ \sum_{i=n_0}^{n-1}\left[\prod_{j=i+1}^{n-1} a(i)\right]g(i)}. \label{eq:ead1_linear_eqnh_sol}
\end{align}


\begin{ex}
  Vamos calcular a solução de
  \begin{equation}\label{eq:ex_ead1_lin_enh}
    y(n+1) = 2y(n) - 1,\quad n\geq 0.
  \end{equation}

  Comparando com \eqref{eq:ead1_linear_eqnh}, temos $a(n) = 2$ e $g(n)=-1$ para todo $n$. Calculando a solução por iterações diretas, temos
  \begin{align}
    y(n+1) &= 2y(n)-1 \\
            &= 2\cdot \left[2y(n-1)-1\right]-1 \\
            &= 2^2y(n-1)-2-1 \\
            &= 2^2\cdot \left[2y(n-2)-1\right]-2-1 \\
            &= 2^3y(n-2)-2^2-2-1 \\
            &\cdots \\
            &= 2^{n+1}y(0)-\sum_{i=0}^{n}2^i
  \end{align}
  Este último termo, é a soma dos termos da {\bf progressão geométrica} de razão $q=2$ (veja Subseção \ref{cap_ead1_sec_eqlin_subsec_sd}), i.e.
  \begin{equation}
    \sum_{i=0}^n q^i = \frac{1-q^{n+1}}{1-q}.
  \end{equation}
  Logo, temos que a solução de \eqref{eq:ead1_linear_eqnh} é
  \begin{align}
    y(n+1) &= 2^{n+1}y(0) - \frac{1-2^{n+1}}{1-2} \\
           &= 2^{n+1}y(0) -2^{n+1}+1.
  \end{align}
  
  Equivalentemente, por \eqref{eq:ead1_linear_eqnh_sol}, temos
  \begin{align}
    y(n) &= \left[\prod_{i=n_0}^{n-1} a(i)\right]y(n_0) \\
         &+ \sum_{i=n_0}^{n-1}\left[\prod_{j=i+1}^{n-1} a(i)\right]g(i) \\
         &= \left[\prod_{i=0}^{n-1} 2\right]y(0) \\
         &+ \sum_{i=0}^{n-1}\left[\prod_{j=i+1}^{n-1} 2\right](-1) \\
         &= 2^ny(0) - \sum_{i=0}^{n-1}2^{n-i-1}\\
         &= 2^ny(0) - 2^{n-1}\sum_{i=0}^{n-1}2^{-i}\\
         &= 2^ny(0) - 2^{n} + 1. \label{eq:ex_ead1_linear_eqnh_sol}
  \end{align}
  A solução vale para qualquer valor inicial $y(0)$.

  \ifispython
  No \python, podemos computar a solução da equação a diferenças \eqref{eq:ex_ead1_lin_eh} com os seguintes comandos:
\begin{verbatim}
In : n = symbols('n', integer=true)
In : y = symbols('y', cls=Function)
In : ead = Eq(y(n+1),2*y(n)-1)
In : rsolve(ead, y(n))
Out: 2**n*C0 + 1
\end{verbatim}
  Observamos que esta solução é equivalente à \eqref{eq:ex_ead1_linear_eqnh_sol}, pois
  \begin{align}
    y(n) &= 2^ny(0) - 2^{n} + 1 \\
         &= 2^n\left[y(0)-1\right]+1,
  \end{align}
  onde $y(0)$ é um valor inicial arbitrário.
  \fi
\end{ex}

\subsection{Somas definidas}\label{cap_ead1_sec_eqlin_subsec_sd}

Seguem algumas somas definidas que podem ser úteis na resolução de equações a diferenças.

\begin{align}
  \sum_{k=1}^{n} k &= \frac{n(n+1)}{2}\\
  \sum_{k=1}^{n} k^2 &= \frac{n(n+1)(2n+1)}{6}\\
  \sum_{k=1}^n k^3 &= \left[\frac{n(n+1)}{2}\right]^2\\
  \sum_{k=1}^n k^4 &= \frac{n(6n^4+15n^3+10n^2-1)}{30}\\
  \sum_{k=0}^{n-1}q^k &= \frac{(1-q^n)}{1-q},\quad q\neq 1\\
  \sum_{k=1}^n kq^k &= \frac{(q-1)(n+1)q^{n+1}-q^{n+2}+q}{(q-1)^2}
\end{align}

\subsection*{Exercícios resolvidos}

\begin{exeresol}
  Calcule a solução da equação à diferenças
  \begin{align}
    y(n+1) &= \frac{1}{2}y(n),\quad n\geq 0,\\
    y(0) &= 1.
  \end{align}
\end{exeresol}
\begin{resol}
  De \eqref{eq:ead1_linear_eqh_sol}, temos
  \begin{align}
    y(n) &= \left[\prod_{i=0}^{n-1} \frac{1}{2}\right]y(0)\\
         &= \left(\frac{1}{2}\right)^{n}\cdot 1 \\
         &= 2^{-n}.
  \end{align}

  \ifispython
  No \python, podemos computar a solução deste exercício com os seguintes comandos:
\begin{verbatim}
In : n = symbols('n', integer=true)
In : y = symbols('y', cls=Function)
In : ead = Eq(y(n+1),1/2*y(n))
In : rsolve(ead, y(n), {y(0):1})
Out: 0.5**n
\end{verbatim}
  \fi
\end{resol}

\begin{exeresol}
  Calcule a solução de
  \begin{align}
    y(n+1) &= 2y(n) + \frac{1}{2}^n,\quad n\geq 0, \\
    y(0) &= 0.
  \end{align}
\end{exeresol}
\begin{resol}
  De \eqref{eq:ead1_linear_eqnh_sol}, temos
\begin{align}
  y(n) &= \left[\prod_{i=0}^{n-1} 2\right]y(0) \nonumber\\
       &+ \sum_{i=0}^{n-1}\left[\prod_{j=i+1}^{n-1} 2\right]\cdot\left(\frac{1}{2}\right)^i \\
       &= \sum_{i=0}^{n-1} 2^{n-1-i}\cdot 2^{-i} \\
       &= \sum_{i=0}^{n-1} 2^{n-1}\cdot 2^{-2i} \\
       &= 2^{n-1}\sum_{i=0}^{n-1}\left(\frac{1}{4}\right)^{i} \\
       &= 2^{n-1}\cdot \frac{\left[1-\left(\frac{1}{4}\right)^n\right]}{1-\frac{1}{4}} \\
       &=2^{n-1}\cdot\frac{4}{3}\cdot\left(1-\frac{1}{4^n}\right)\\
       &= \frac{4}{3}\left(2^{n-1} - \frac{2^{n-1}}{4^n}\right)\\
       &= \frac{4}{3}\left(2^{n-1} - 2^{n-1}2^{-2n}\right)\\
       &= \frac{4}{3}\left(2^{n-1}-2^{-n-1}\right)\\
       &= \frac{2}{3}\left(2^n-2^{-n}\right).
\end{align}

  \ifispython
  No \python, podemos computar a solução deste exercício com os seguintes comandos:
\begin{verbatim}
In : n = symbols('n', integer=true)
In : y = symbols('y', cls=Function)
In : ead = Eq(y(n+1),2*y(n)+(1/2)**n)
In : rsolve(ead, y(n), {y(0):0})
Out: -0.666666666666667*0.5**n + 0.666666666666667*2**n
\end{verbatim}
  \fi  
\end{resol}

\subsection*{Exercícios}

\begin{exer}
  Calcule a solução de
  \begin{equation}
    y(n+1) = 3y(n),\quad n\geq 0.
  \end{equation}
\end{exer}
\begin{resp}
  $y(n) = 3^ny(0)$
\end{resp}

\begin{exer}
  Calcule a solução de
  \begin{align}
    y(n+1) &= \frac{1}{3}y(n),\quad n\geq 0,\\
    y(0) &= -1.
  \end{align}
\end{exer}
\begin{resp}
  $y(n) = -\frac{1}{3^n}$
\end{resp}

\begin{exer}
  Calcule a solução de
  \begin{align}
    y(n+1) &= 3y(n) -3,\quad n\geq 0,\\
    y(0) &= 2.
  \end{align}
\end{exer}
\begin{resp}
  $y(n) = \frac{1}{2}(3^n+3)$
\end{resp}

\begin{exer}
  Calcule a solução de
  \begin{align}
    y(n+1) &= ny(n)+n!,\quad n\geq 0,\\
    y(0) &= 1.
  \end{align}
\end{exer}
\begin{resp}
  $y(n)=n!$
\end{resp}

\emconstrucao