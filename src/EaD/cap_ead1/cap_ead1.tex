%Este trabalho está licenciado sob a Licença Atribuição-CompartilhaIgual 4.0 Internacional Creative Commons. Para visualizar uma cópia desta licença, visite http://creativecommons.org/licenses/by-sa/4.0/deed.pt_BR ou mande uma carta para Creative Commons, PO Box 1866, Mountain View, CA 94042, USA.

\chapter{Equações de ordem 1}\label{cap_ead1}

Neste capítulo, discutimos de forma introdutória sobre {\bf equações a diferenças de primeira ordem}. Tais equações podem ser escritas na forma
\begin{equation}
  f\left(y(n+1),y(n);n\right)=0,
\end{equation}
onde $n=0, 1, \ldots$ e $y:n\mapsto y(n)$ é função discreta (incógnita).

\section{Equações lineares}\label{cap_ead1_sec_eqlin}

Nesta seção, discutimos sobre equações a diferenças de ordem 1 e lineares. Tais equações podem ser escritas na seguinte forma
\begin{equation}
  y(n+1) = a(n)y(n) + g(n),
\end{equation}
onde $n=n_0, n_0+1, \ldots$, sendo $n_0$ um número inteiro, $a:n\mapsto a(n)$ e $g:n\mapsto g(n)$ é o termo fonte. A equação é dita ser {\bf homogênea} quando $g\equiv 0$ e, caso contrário, é dita ser {\bf não homogênea}.

\subsection{Equação homogênea}

\begin{flushright}
  \href{https://youtu.be/Y9VWFtlE3k4}{[YouTube]} | \href{https://archive.org/details/ead-o1h}{[Vídeo]} | [Áudio] | \href{https://phkonzen.github.io/notas/contato.html}{[Contatar]}
\end{flushright}

A solução de uma equação a diferenças de ordem 1, linear e homogênea
\begin{equation}\label{eq:ead1_linear_eqh}
  y(n+1) = a(n)y(n),\quad n\geq n_0,
\end{equation}
pode ser obtida por iterações diretas. Para $n\geq n_0$, temos
\begin{gather}
  y(n+1) = a(n)y(n) \\
  = a(n)a(n-1)y(n-1)\\
  = a(n)a(n-1)a(n-2)y(n-2) \\
  \vdots\\
  = a(n)a(n-1)\cdots a(n_0)y(n_0).
\end{gather}
Ou seja, dado o valor inicial $y(n_0)$, temos a solução\footnote{A demonstração por ser feita por indução matemática.}
\begin{equation}
  y(n) = a(n_0)a(n_0+1)\cdots a(n-1)y(n_0).
\end{equation}
A fim de termos uma notação mais prática, vamos usar a notação de produtório\footnote {Veja mais em \href{https://pt.wikipedia.org/wiki/Produt\%C3\%B3rio}{Wiki: Produtório}.}
\begin{equation}
  \prod_{i=n_0}^{n-1} a(n) = a(n_0)a(n_0+1)\cdots a(n-1).
\end{equation}
Com esta notação, a solução de \eqref {eq:ead1_linear_eqh} pode ser escrita como segue
\begin{equation}\label{eq:ead1_linear_eqh_sol}
  {\color{blue} y(n) = \left[\prod_{i=n_0}^{n-1}a(i)\right]y(n_0)},
\end{equation}
assumindo a notação de que $\prod_{i=n+1}^na(i)=1$.

\begin{ex}
  Vamos calcular a solução de
  \begin{equation}\label{eq:ex_ead1_lin_eh}
    y(n+1) = 2y(n),\quad n\geq 0.
  \end{equation}

  \begin{enumerate}[a)]
  \item Por iterações diretas.

    Comparando com \eqref{eq:ead1_linear_eqh}, temos $a(n) = 2$ para todo $n$. Calculando a solução por iterações diretas, temos
    \begin{gather}
      y(n+1) = 2y(n) \\
      = 2\cdot 2y(n-1) \\
      = 2^2y(n-1) \\
      = 2^2\cdot 2y(n-2) \\
      = 2^3y(n-2) \\
      \vdots \nonumber\\
      = 2^{n+1}y(0)
    \end{gather}
    ou, equivalentemente, temos a solução
    \begin{equation}
      y(n) = 2^ny(0).
    \end{equation}

  \item Por \eqref{eq:ead1_linear_eqh_sol}.
    \begin{gather}
      y(n) = \left[\prod_{i=0}^{n-1}2\right]y(0) \\
      = (\underbrace{2\cdot 2\cdot \cdots \cdot 2}_{\text{n vezes}}) y(0)\\
      = 2^{n}y(0).
    \end{gather}
  \end{enumerate}
  A solução vale para qualquer valor inicial $y(0)$.

  \ifispython
  No \python, podemos computar a solução da equação a diferenças \eqref{eq:ex_ead1_lin_eh} com os seguintes comandos:
  \begin{lstlisting}
    In : from sympy import *
    In : n = symbols('n', integer=true)
    In : y = symbols('y', cls=Function)
    In : ead = Eq(y(n+1),2*y(n))
    In : rsolve(ead, y(n))
    Out: 2**n*C0
  \end{lstlisting}
  \fi
\end{ex}

\subsection{Equação não homogênea}

\begin{flushright}
  \href{https://youtu.be/tYwlWmEeMFo}{[YouTube]} | \href{https://archive.org/details/ead-o1nh}{[Vídeo]} | [Áudio] | \href{https://phkonzen.github.io/notas/contato.html}{[Contatar]}
\end{flushright}

A solução de uma equação a diferenças de ordem 1, linear e não homogênea
\begin{equation}\label{eq:ead1_linear_eqnh}
  y(n+1) = a(n)y(n) + g(n),\quad n\geq n_0,
\end{equation}
pode ser obtida por iterações diretas.

Vejamos, para $n\geq n_0$ temos
\begin{gather}
  y(n+1) = a(n){\color{blue}y(n)} + g(n) \\
  = a(n)\left[{\color{blue}a(n-1)y(n-1)+g(n-1)}\right] + g(n) \\
  = a(n)a(n-1){\color{blue}y(n-1)}+a(n)g(n-1) + g(n) \\
  = a(n)a(n-1)\left[{\color{blue}a(n-2)y(n-2)+g(n-2)}\right] \nonumber\\
  + a(n)g(n-1) + g(n) \\
  = a(n)a(n-1)a(n-2)y(n-2) \nonumber\\
  + a(n)a(n-1)g(n-2) + a(n)g(n-1) + g(n) \\
  \vdots\nonumber\\
  = {\color{blue}a(n)a(n-1)\cdots a(n_0)}y(n_0) \nonumber\\
  + {\color{red}a(n_0+1)a(n_0+2)\cdots a(n)}g(n_0) \nonumber\\
  + {\color{red}a(n_0+2)a(n_0+3)\cdots a(n)}g(n_0+1) \nonumber\\
  + \cdots + {\color{red}a(n)}g(n-1) + g(n)
\end{gather}
Logo, podemos inferir que a solução é dada por\footnote{A demonstração por ser feita por indução matemática.}
\begin{gather}
  y(n) = {\color{blue}a(n_0)a(n_0+1)\cdots a(n-1)}y(n_0) \nonumber\\
  + {\color{red}a(n_0+1)a(n_0+2)\cdots a(n-1)}g(n_0) \nonumber\\
  + {\color{red}a(n_0+2)a(n_0+3)\cdots a(n-1)}g(n_0+1) \nonumber\\
  + \cdots + {\color{red}a(n-1)}g(n-2) + g(n-1)
\end{gather}
Aqui, por maior praticidade, vamos empregar a notação de somatório\footnote{Veja mais em \href{https://pt.wikipedia.org/wiki/Somat\%C3\%B3rio}{Wiki: Somatório}}
\begin{equation}
  \sum_{i=n_0}^{n} a(i) = a(n_0) + a(n_0+1) + \cdots + a(n).
\end{equation}
Com isso, a solução de \eqref{eq:ead1_linear_eqnh} pode ser escrita como segue
\begin{gather}
  {\color{blue}y(n) = \left[\prod_{i=n_0}^{n-1} a(i)\right]y(n_0)} \nonumber\\
  {\color{blue}+ \sum_{i=n_0}^{n-1}\left[\prod_{j=i+1}^{n-1} a(j)\right]g(i)}. \label{eq:ead1_linear_eqnh_sol}
\end{gather}
No último termo, consideramos a notação $\sum_{j=i+1}^i a(i) = 0$.

\begin{ex}
  Vamos calcular a solução de
  \begin{equation}\label{eq:ex_ead1_lin_enh}
    y(n+1) = 2y(n) - 1,\quad n\geq 0.
  \end{equation}

  Comparando com \eqref{eq:ead1_linear_eqnh}, temos $a(n) = 2$ e $g(n)=-1$ para todo $n$.

  \begin{enumerate}
  \item Cálculo por iterações diretas.

    Calculando a solução por iterações diretas, temos
    \begin{gather}
      y(n+1) = 2y(n)-1 \\
      = 2\cdot \left[2y(n-1)-1\right]-1 \\
      = 2^2y(n-1)-2-1 \\
      = 2^2\cdot \left[2y(n-2)-1\right]-2-1 \\
      = 2^3y(n-2)-2^2-2-1 \\
      \cdots \nonumber\\
      = 2^{n+1}y(0)-\sum_{i=0}^{n}2^i
    \end{gather}
    Logo, temos
    \begin{equation}
      y(n) = 2^ny(0) - {\color{blue}\sum_{i=0}^{n-1} 2^i}.
    \end{equation}
    Este último termo, é a soma dos termos da {\bf progressão geométrica}\footnote{Veja mais em \href{https://pt.wikipedia.org/wiki/Progress\%C3\%A3o\_geom\%C3\%A9trica}{Wiki:Progressão geométrica.}} de razão $q=2$ e termo inicial $1$ (veja Subseção \ref{cap_ead1_sec_eqlin_subsec_sd}), i.e.
    \begin{equation}
      \sum_{i=0}^{n-1} q^i = \frac{1-q^n}{1-q}.
    \end{equation}
    Portanto, a solução \eqref{eq:ead1_linear_eqnh} é
    \begin{gather}
      y(n) = 2^ny(0) - {\color{blue}\frac{1-2^{n}}{1-2}} \\
      = 2^{n}y(0) -2^{n}+1.
    \end{gather}

  \item Cálculo por \eqref{eq:ead1_linear_eqnh_sol}.
    
    \begin{gather}
      y(n) = {\color{blue}\left[\prod_{i=n_0}^{n-1} a(i)\right]}y(n_0) \\
      + {\color{red}\sum_{i=n_0}^{n-1}\left[\prod_{j=i+1}^{n-1} a(j)\right]}g(i) \\
      = {\color{blue}\left[\prod_{i=0}^{n-1} 2\right]}y(0) \\
      + {\color{red}\sum_{i=0}^{n-1}\left[\prod_{j=i+1}^{n-1} 2\right]}(-1) \\
      = {\color{blue}2^n}y(0) -{\color{red} \sum_{i=0}^{n-1}2^{n-1-i}}\\
      = {\color{blue}2^n}y(0) - {\color{red}2^{n-1}\sum_{i=0}^{n-1}2^{-i}}\label{eq:ead1_lineqnh_ae}
    \end{gather}
    Este último somatório é a soma dos termos da progressão geométrica de razão $q=1/2$ e termo inicial $1$ ((veja Subseção \ref{cap_ead1_sec_eqlin_subsec_sd}), equação \eqref{eq:ead1_eqlin_spg}). Logo,
    \begin{gather}
      \sum_{i=0}^{n-1}2^{-i} = \frac{1-\left(\frac{1}{2}\right)^n}{1-\frac{1}{2}}\\
      = 2\left(1-2^{-n}\right).
    \end{gather}
    Retornando a \eqref{eq:ead1_lineqnh_ae}, temos
    \begin{gather}
      y(n) = {\color{blue}2^n}y(0) - {\color{red}2^{n-1}\cdot 2\cdot \left(1 - 2^{-n}\right)}\\
      = 2^ny(0) - 2^{n} + 1. \label{eq:ex_ead1_linear_eqnh_sol}
    \end{gather}
    A solução vale para qualquer valor inicial $y(0)$.
  \end{enumerate}
  \ifispython
  No \python, podemos computar a solução da equação a diferenças \eqref{eq:ex_ead1_lin_eh} com os seguintes comandos:
  \begin{lstlisting}
    In : from sympy import *
    In : n = symbols('n', integer=true)
    In : y = symbols('y', cls=Function)
    In : ead = Eq(y(n+1),2*y(n)-1)
    In : rsolve(ead, y(n))
    Out: 2**n*C0 + 1
  \end{lstlisting}
  Observamos que esta solução é equivalente à \eqref{eq:ex_ead1_linear_eqnh_sol}, pois
  \begin{align}
    y(n) &= 2^ny(0) - 2^{n} + 1 \\
    &= 2^n\left[y(0)-1\right]+1,
  \end{align}
  onde $y(0)$ é um valor inicial arbitrário.
  \fi
\end{ex}

\subsection{Somas definidas}\label{cap_ead1_sec_eqlin_subsec_sd}

\begin{flushright}
  [Vídeo] | [Áudio] | \href{https://phkonzen.github.io/notas/contato.html}{[Contatar]}
\end{flushright}

Seguem algumas somas definidas que podem ser úteis na resolução de equações a diferenças.

\begin{align}
  \sum_{k=1}^{n} k &= \frac{n(n+1)}{2}\\
  \sum_{k=1}^{n} k^2 &= \frac{n(n+1)(2n+1)}{6}\\
  \sum_{k=1}^n k^3 &= \left[\frac{n(n+1)}{2}\right]^2\\
  \sum_{k=1}^n k^4 &= \frac{n(6n^4+15n^3+10n^2-1)}{30}\\
  \sum_{k=0}^{n-1}q^k &= \frac{(1-q^n)}{1-q},\quad q\neq 1\label{eq:ead1_eqlin_spg}\\
  \sum_{k=1}^n kq^k &= \frac{(q-1)(n+1)q^{n+1}-q^{n+2}+q}{(q-1)^2}
\end{align}

\subsection*{Exercícios resolvidos}

\begin{flushright}
  [Vídeo] | [Áudio] | \href{https://phkonzen.github.io/notas/contato.html}{[Contatar]}
\end{flushright}

\begin{exeresol}
  Calcule a solução da equação à diferenças
  \begin{align}
    y(n+1) &= \frac{1}{2}y(n),\quad n\geq 0,\\
    y(0) &= 1.
  \end{align}
\end{exeresol}
\begin{resol}
  De \eqref{eq:ead1_linear_eqh_sol}, temos
  \begin{align}
    y(n) &= \left[\prod_{i=0}^{n-1} \frac{1}{2}\right]y(0)\\
    &= \left(\frac{1}{2}\right)^{n}\cdot 1 \\
    &= 2^{-n}.
  \end{align}

  \ifispython
  No \python, podemos computar a solução deste exercício com os seguintes comandos:
  \begin{lstlisting}
    In : from sympy import *
    In : n = symbols('n', integer=true)
    In : y = symbols('y', cls=Function)
    In : ead = Eq(y(n+1),S(1)/2*y(n))
    In : rsolve(ead, y(n), {y(0):1})
    Out: 0.5**n
  \end{lstlisting}
  \fi
\end{resol}

\begin{exeresol}
  Calcule a solução de
  \begin{align}
    y(n+1) &= 2y(n) + \left(\frac{1}{2}\right)^n,\quad n\geq 0, \\
    y(0) &= 0.
  \end{align}
\end{exeresol}
\begin{resol}
  De \eqref{eq:ead1_linear_eqnh_sol}, temos
  \begin{align}
    y(n) &= \left[\prod_{i=0}^{n-1} 2\right]y(0) \nonumber\\
    &+ \sum_{i=0}^{n-1}\left[\prod_{j=i+1}^{n-1} 2\right]\cdot\left(\frac{1}{2}\right)^i \\
    &= \sum_{i=0}^{n-1} 2^{n-1-i}\cdot 2^{-i} \\
    &= \sum_{i=0}^{n-1} 2^{n-1}\cdot 2^{-2i} \\
    &= 2^{n-1}\sum_{i=0}^{n-1}\left(\frac{1}{4}\right)^{i} \\
    &= 2^{n-1}\cdot \frac{\left[1-\left(\frac{1}{4}\right)^n\right]}{1-\frac{1}{4}} \\
    &=2^{n-1}\cdot\frac{4}{3}\cdot\left(1-\frac{1}{4^n}\right)\\
    &= \frac{4}{3}\left(2^{n-1} - \frac{2^{n-1}}{4^n}\right)\\
    &= \frac{4}{3}\left(2^{n-1} - 2^{n-1}2^{-2n}\right)\\
    &= \frac{4}{3}\left(2^{n-1}-2^{-n-1}\right)\\
    &= \frac{2}{3}\left(2^n-2^{-n}\right).
  \end{align}

  \ifispython
  No \python, podemos computar a solução deste exercício com os seguintes comandos:
  \begin{lstlisting}
    In : from sympy import *
    In : n = symbols('n', integer=true)
    In : y = symbols('y', cls=Function)
    In : ead = Eq(y(n+1),2*y(n)+(1/2)**n)
    In : rsolve(ead, y(n), {y(0):0})
    Out: 2*2**n/3 - 2*2**(-n)/3
  \end{lstlisting}
  \fi  
\end{resol}

\subsection*{Exercícios}

\begin{flushright}
  [Vídeo] | [Áudio] | \href{https://phkonzen.github.io/notas/contato.html}{[Contatar]}
\end{flushright}

\begin{exer}
  Calcule a solução de
  \begin{equation}
    y(n+1) = 3y(n),\quad n\geq 0.
  \end{equation}
\end{exer}
\begin{resp}
  $y(n) = 3^ny(0)$
\end{resp}

\begin{exer}
  Calcule a solução de
  \begin{align}
    y(n+1) &= \frac{1}{3}y(n),\quad n\geq 0,\\
    y(0) &= -1.
  \end{align}
\end{exer}
\begin{resp}
  $y(n) = -\frac{1}{3^n}$
\end{resp}

\begin{exer}
  Considere um empréstimo de $\$ 100$ a uma taxa mensal de $1\%$. Considerando $y(0)=100$, qual o valor de $y(n)$ no $n$-ésimo mês? Modele o problema como uma equação à diferenças e calcule sua solução. Então, calcule o valor da dívida no 36º mês.
\end{exer}
\begin{resp}
  $y(n+1)=1,01\cdot y(n),~y(0)=100$; $y(n)=100\cdot 1,01^n$; $y(36)\approx 143,08$
\end{resp}

\begin{exer}
  Calcule a solução de
  \begin{align}
    y(n+1) &= 3y(n) -3,\quad n\geq 0,\\
    y(0) &= 2.
  \end{align}
\end{exer}
\begin{resp}
  $y(n) = \frac{1}{2}(3^n+3)$
\end{resp}

\begin{exer}
  Calcule a solução de
  \begin{align}
    y(n+1) &= ny(n)+n!,\quad n\geq 0,\\
    y(0) &= 1.
  \end{align}
\end{exer}
\begin{resp}
  $y(n)=n!$
\end{resp}

\begin{exer}
  Calcule a solução de
  \begin{align}
    y(n+1) &= 2y(n)+2^n,\quad n\geq 0,\\
    y(0) &= 2.
  \end{align}
\end{exer}
\begin{resp}
  $\displaystyle y(n)=2^n\left(\frac{n}{2}+2\right)$
\end{resp}

\begin{exer}
  Considere um empréstimo de $\$ 100$ a uma taxa mensal de $1\%$ e com parcelas mensais fixas de $\$ 1$. Considerando $y(0)=100$, qual o valor de $y(n)$ no $n$-ésimo mês? Modele o problema como uma equação à diferenças e calcule sua solução.
\end{exer}
\begin{resp}
  $y(n+1)=1,01\cdot y(n)-1,~y(0)=100$; $y(n)=100$;
\end{resp}

\begin{exer}
  Calcule a solução de
  \begin{equation}
    y(n+1) = ay(n) + b,\quad n\geq 0,
  \end{equation}
  onde $a$ e $b$ são constantes com $a\neq 1$.
\end{exer}
\begin{resp}
  $\displaystyle y(n) = \left(y(0)-\frac{b}{1-a}\right)a^n + \frac{b}{1-a}$
\end{resp}

\section{Estudo assintótico de equações lineares}\label{cap_ead1_sec_eael}

\begin{flushright}
  \href{https://archive.org/details/ead-ponto-equilibrio}{[Vídeo]} | [Áudio] | \href{https://phkonzen.github.io/notas/contato.html}{[Contatar]}
\end{flushright}

Nesta seção, vamos introduzir aspectos básicos sobre o comportamento assintótico de soluções de equações a diferenças de primeira ordem e lineares.

Seja
\begin{equation}\label{eq:ead1_eael_pt}
  y(n+1) = f(y(n),n),\quad n\geq n_0,
\end{equation}
uma equação a diferenças com valor inicial $y(n_0)$. Dizemos que $y^*$ é {\bf ponto de equilíbrio} da equação, quando $y^*$ é tal que
\begin{equation}\label{eq:ead1_eael_eq}
  f(y^*,n) = y^*,
\end{equation}
para todo $n\geq n_0$.  Neste caso, ao escolhermos $y(n_0)=y^*$, então a solução de equação a diferenças \eqref{eq:ead1_eael_pt} é
\begin{equation}
  y(n) = y^*.
\end{equation}

\begin{ex}\label{ex:ead1_eqlin_pe}
  Vamos calcular o(s) ponto(s) de equilíbrio de
  \begin{equation}\label{eq:ex_ead1_eael_calc}
    y(n+1) = \frac{4}{3}y(n)-1,\quad n\geq 0.
  \end{equation}

  Neste caso, por comparação com \eqref{eq:ead1_eael_pt}, temos $\displaystyle f(y(n),n) = \frac{4}{3}y(n)-1$. Para calcularmos o(s) ponto(s) de equilíbrio, resolvemos
  \begin{gather}
    f(y^*,n) = y^* \\
    \frac{4}{3}y^*-1 = y^* \\
    \left(\frac{4}{3}-1\right)y^* = 1 \\
    \frac{1}{3}y^* = 1 \\
    y^* = 3.
  \end{gather}
  Com isso, concluímos que $y^*=3$ é o único ponto de equilíbrio de \eqref{eq:ex_ead1_eael_calc}.

  Notamos que, de fato, ao escolhermos $y(0)=3$, temos
  \begin{align}
    y(1) &= \frac{4}{3}y(0) - 1 = 3 \\
    y(2) &= \frac{4}{3}y(1) - 1 = 3 \\
    y(3) &= \frac{4}{3}y(1) - 1 = 3 \\
    &\vdots \\
    y(n) &= 3.
  \end{align}
\end{ex}

Seja a equação a diferenças de primeira ordem, linear e com coeficientes constantes
\begin{equation}\label{eq:ead1_eael_cc}
  y(n+1) = ay(n) + b,\quad n\geq n_0,
\end{equation}
Se $a=1$ e $b=0$, então todo número real $y^*$ é ponto de equilíbrio de \eqref{eq:ead1_eael_cc}. Agora, se $a=1$ e $b\neq 0$, então \eqref{eq:ead1_eael_cc} não tem ponto de equilíbrio. Por fim, se $a\neq 1$, então
\begin{equation}
  y^* = \frac{b}{1-a}
\end{equation}
é o único ponto de equilíbrio de \eqref{eq:ead1_eael_cc}. Este é o caso do Exemplo \ref{ex:ead1_eqlin_pe}.

Um ponto de equilíbrio é um {\bf atrator global} quando
\begin{equation}
  \lim_{n\to\infty} y(n) = y^*,
\end{equation}
para qualquer valor inicial $y(n_0)$. Neste caso, também dizemos que $y^*$ é um ponto de equilíbrio {\bf assintoticamente globalmente estável}.

Uma equação a diferenças da forma
\begin{equation}
  y(n+1) = ay(n),\quad n\geq n_0,
\end{equation}
com $-1<a<1$, tem $y^*=0$ como atrator global. De fato, a solução desta equação a diferenças é
\begin{align}
  y(n) &= \left[\prod_{i=n_0}^{n-1}a\right]y(n_0) \\
  &= a^{n-n_0}y(n_0).
\end{align}
Logo, temos
\begin{align}
  \lim_{n\to\infty} y(n) &= \lim_{n\to\infty} \cancelto{0}{a^{n-n_0}}y(n_0)\\
  &= 0.
\end{align}

\begin{ex}
  Para a equação a diferenças
  \begin{equation}
    y(n+1) = \frac{1}{2}y(n),\quad n\geq 0,
  \end{equation}
  temos que $y^*=0$ é um ponto de equilíbrio assintoticamente globalmente estável.
\end{ex}

Um equação a diferenças da forma
\begin{equation}\label{eq:ead1_ape_eqlin_cc2}
  y(n+1) = ay(n) + b,\quad n\geq n_0,
\end{equation}
com $-1<a<1$, tem
\begin{equation}
  y^* = \frac{b}{1-a}
\end{equation}
como ponto de equilíbrio assintoticamente globalmente estável. De fato, a solução desta equação é
\begin{gather}
  y(n) = \left[\prod_{i=n_0}^{n-1} a\right]y(n_0) \nonumber\\
  + \sum_{i=n_0}^{n-1}\left[\prod_{j=i+1}^{n-1} a\right]b \\
  = a^{n-n_0}y(n_0) + \sum_{i=n_0}^{n-1}a^{n-1-i}b\\
  = a^{n-n_0}y(n_0) + a^{n-1}b\sum_{i=n_0}^{n-1}a^{-i} \\
  = a^{n-n_0}y(n_0) + a^{n-1}b\sum_{j=0}^{n-n_0-1}a^{-j-n_0}\\
  = a^{n-n_0}y(n_0) + a^{n-n_0-1}b\sum_{j=0}^{n-n_0-1}a^{-j}\\
  = a^{n-n_0}y(n_0) + a^{n-n_0-1}b\frac{(1-a^{-(n-n_0)})}{1-a^{-1}}\\
  = a^{n-n_0}y(n_0) + a^{n-n_0-1}b\frac{\frac{a^{n-n_0}-1}{a^{n-n_0}}}{\frac{a-1}{a}}\\
  = a^{n-n_0}y(n_0) + b\frac{1-a^{n-n_0}}{1-a} \\
  = \left(y(n_0) - \frac{b}{1-a}\right)a^{n-n_0} + \frac{b}{1-a}.
\end{gather}
Observamos que esta última equação, confirma que
\begin{equation}
  y^* = \frac{b}{1-a}
\end{equation}
é ponto de equilíbrio de \eqref{eq:ead1_ape_eqlin_cc2} e é assintoticamente globalmente estável, pois
\begin{align}
  \lim_{n\to\infty} y(n) &= \lim_{n\to\infty} \left[\left(y(n_0) - \frac{b}{1-a}\right)\cancelto{0}{a^{n-n_0}} + \frac{b}{1-a}\right] \\
  &=  \frac{b}{1-a}.
\end{align}

\begin{ex}
  A equação a diferenças
  \begin{equation}
    y(n+1) = 4y(n) - 1,\quad n\geq 0,
  \end{equation}
  tem $y^* = 1/3$ como ponto de equilíbrio, o qual não é um atrator global. De fato, para qualquer escolha de $y(0)\neq y^*$, temos
  \begin{equation}
    y(n) = \underbrace{\left(y(0) - \frac{1}{3}\right)}_{\neq 0}4^n + \frac{1}{3}.
  \end{equation}
  Logo, vemos que $y(n)\to\pm\infty$ quando $n\to\infty$, onde o sinal é igual ao do termo $y(0)-1/3$.

  \ifispython
  Observamos as seguintes computações no \python:
  \begin{lstlisting}
    In : y=1/3
    ...: for i in range(1,31):
    ...:     y=4*y-1
    ...: 
    ...: y
    Out: -21.0
  \end{lstlisting}
  Ou seja, $y(30)=-21.0$ computando por iterações recorrentes, enquanto que o valor esperado é $y(30)=1/3$, sendo este um ponto de equilíbrio da equação a diferenças.

  O que está ocorrendo nestas computações é um fenômeno conhecido como cancelamento catastrófico em máquina. No computador, o valor inicial $y(0)=1/3$ é computado com um pequeno erro de arredondamento. Do que vimos acima, se $y(0)\neq 1/3$, então $y(n)\to\pm\infty$ quando $n\to\infty$.

  No \python, podemos fazer as computações exatas na aritmética dos números racionais. Para tanto, podemos usar o seguinte código:
  \begin{lstlisting}
    In : from sympy import Rational
    ...: y=Rational(1,3)
    ...: for i in range(1,31):
    ...:     y=4*y-1
    ...: 
    ...: y
    Out: 1/3
  \end{lstlisting}
  \begin{lstlisting}

  \end{lstlisting}
  \fi
\end{ex}

\subsection*{Exercícios resolvidos}

\begin{flushright}
  [Vídeo] | [Áudio] | \href{https://phkonzen.github.io/notas/contato.html}{[Contatar]}
\end{flushright}

\begin{exeresol}
  Calcule os pontos de equilíbrio de
  \begin{equation}
    y(n+1) = ny(n),\quad n\geq 0.
  \end{equation}
\end{exeresol}
\begin{resol}
  Temos que $y^*$ é ponto de equilíbrio da equações a diferenças, quando
  \begin{gather}
    y^* = ny^*\\
    (1-n)y^* = 0
  \end{gather}
  para todo $n\geq 0$. Logo, $y^*=0$ é ponto de equilíbrio da equação a diferenças.
\end{resol}

\begin{exeresol}
  Verifique se $y^*=0$ é ponto de equilíbrio assintoticamente globalmente estável de
  \begin{equation}
    y(n+1) = \frac{1}{n+1}y(n),\quad n\geq 0.
  \end{equation}
\end{exeresol}
\begin{resol}
  Primeiramente, confirmamos que $y^* = 0$ é ponto de equilíbrio, pois
  \begin{equation}
    \frac{1}{n+1}y^* = 0 = y^*,\quad n\geq 0.
  \end{equation}

  Por fim, a solução da equação a diferenças é
  \begin{align}
    y(n) &= \left[\prod_{i=0}^{n-1}\frac{1}{i+1}\right]y(0)\\
    &= \frac{1}{n!}y(0).
  \end{align}
  Daí, vemos que
  \begin{equation}
    \lim_{n\to\infty} \frac{1}{n!}y(0) = 0 = y^*.
  \end{equation}
  Logo, concluímos que $y^*=0$ é ponto de equilíbrio assintoticamente globalmente estável da equação a diferenças dada. 
\end{resol}

\subsection*{Exercícios}

\begin{flushright}
  [Vídeo] | [Áudio] | \href{https://phkonzen.github.io/notas/contato.html}{[Contatar]}
\end{flushright}

\begin{exer}\label{exer:ead1_pe_0}
  Calcule o ponto de equilíbrio de
  \begin{equation}
    y(n+1) = -y(n) + 1 
  \end{equation}
\end{exer}
\begin{resp}
  $1/2$
\end{resp}

\begin{exer}
  O ponto de equilíbrio da equação a diferenças do Exercício \ref{exer:ead1_pe_0} é um atrator global? Justifique sua resposta.
\end{exer}
\begin{resp}
  não
\end{resp}

\begin{exer}
  Encontre o ponto de equilíbrio de
  \begin{equation}
    y(n+1) = \frac{1}{2}y(n) + \frac{1}{2},\quad n\geq 2,
  \end{equation}
  e diga se ele é um atrator global. Justifique sua resposta.
\end{exer}
\begin{resp}
  $y^*=1$; atrator global
\end{resp}

\begin{exer}
  Encontre o ponto de equilíbrio de
  \begin{equation}
    y(n+1) = 2y(n) + 1,\quad n\geq 2,
  \end{equation}
  e diga se ele é assintoticamente globalmente estável. Justifique sua resposta.
\end{exer}
\begin{resp}
  $y^*=-1$; não é assintoticamente globalmente estável
\end{resp}

\begin{exer}
  Considere um financiamento de valor $\$ 100$ com taxa de juros $1\%$ a.m. e amortizações fixas mensais de valor $\$ a$. O valor devido $y(n+1)$ no $n+1$-ésimo mês pode ser modelado pela seguinte equações a diferenças
  \begin{equation}
    y(n+1) = 1,01y(n)-a,\quad n\geq 0,
  \end{equation}
  com valor inicial $y(0)=100$. Calcule o valor $a$ mínimo a ser amortizado mensalmente de forma que o valor devido permaneça sempre constante.
\end{exer}
\begin{resp}
  $a=1$
\end{resp}

\section{Alguns aspectos sobre equações não lineares}\label{cap_ead1_sec_eqnlin}

\begin{flushright}
  [Vídeo] | [Áudio] | \href{https://phkonzen.github.io/notas/contato.html}{[Contatar]}
\end{flushright}

O estudo de equações a diferenças não lineares é bastante amplo, podendo chegar ao estado da arte. Nesta seção, vamos abordar alguns conceitos fundamentais para a análise de equações de primeira ordem e não lineares, i.e. equações da forma
\begin{equation}\label{eq:ead1_enl}
  f\left(y(n+1),y(n);n\right)=0,\quad n\geq n_0\geq 0,
\end{equation}
onde $f$ é uma função não linear nas incógnitas $y(n+1)$ ou $y(n)$.

\subsection{Solução}

\begin{flushright}
  [Vídeo] | [Áudio] | \href{https://phkonzen.github.io/notas/contato.html}{[Contatar]}
\end{flushright}

A variedade de formas que uma equação a diferenças não linear pode ter é enorme e não existem formas fechadas para a solução da grande maioria delas. No entanto, sempre pode-se buscar calcular a solução por iteração direta, i.e.
\begin{gather}
  y(n_0) = \text{valor inicial},\\
  f\left(y(n+1),y(n);n\right)=0, ~ n= n_0, n_0+1, n_0+2, \ldots
\end{gather}

\begin{ex}
  Vamos calcular a solução da seguinte equação a diferenças não linear
  \begin{equation}\label{eq:ead1_enl_ex_sol}
    y(n+1) = y^2(n),\quad n\geq 0.
  \end{equation}

  A partir do valor inicial $y(0)$ e por iterações diretas, temos
  \begin{align}
    y(1) &= y^2(0),\\
    y(2) &= [y(1)]^2 \\
    &= \left[y^2(0)\right]^2 \\
    &= y^{2^2}(0),\\
    y(3) &= [y(2)]^2 \\
    &= \left[y^{2^2}(0)\right]^2 \\
    &= y^{2^3}(0) \\
    &\vdots
  \end{align}
  Disso, podemos inferir que a solução de \ref{eq:ead1_enl_ex_sol} é
  \begin{equation}
    y(n) = y^{2^n}(0).
  \end{equation}
\end{ex}

\subsection{Pontos de equilíbrio}

\begin{flushright}
  [Vídeo] | [Áudio] | \href{https://phkonzen.github.io/notas/contato.html}{[Contatar]}
\end{flushright}

Introduzimos pontos de equilíbrio na Seção \ref{cap_ead1_sec_eael} e, aqui, vamos estudá-los no contexto de equação a diferenças de primeira ordem e não lineares. Um dos primeiros aspectos a serem notados é que equação não lineares podem ter vários pontos de equilíbrio, ter somente um ou não ter.

\begin{ex}(Ponto de equilíbrio)
  Vejamos os seguintes casos:
  \begin{enumerate}[a)]
  \item $y(n+1) = y(n)^2+1,~n\geq 0$

    Se $y^*$ é ponto de equilíbrio, então
    \begin{gather}
      y^* = \left(y^*\right)^2 + 1 \\
      \left(y^*\right)^2 - y^* + 1 = 0,
    \end{gather}
    a qual não admite solução real. Ou seja, a equação a diferenças deste item não tem ponto de equilíbrio.

  \item $y(n+1) = y(n)^2,~n\geq 0$

    \begin{gather}
      y^* = \left(y^*\right)^2 \\
      \left(y^*\right)^2 - y^* = 0 \\
      y^*\left(y^* - 1\right) = 0,
    \end{gather}
    Neste caso, a equação a diferenças tem dois pontos de equilíbrio, a saber, $y_1^* = 0$ e $y_2^* = 1$.

  \item $\left[y(n+1)-1\right]\cdot \left[y(n)-1\right]=0,~n\geq 0$

    \begin{gather}
      \left(y^*-1\right)\cdot \left(y^*-1\right) = 0 \\
      \left(y^*-1\right)^2 = 0 \\
      y^* = 1
    \end{gather}
    Concluímos que esta equação tem $y^*=1$ como seu único ponto de equilíbrio.

  \item $y(n+1) = y(n)\cos\left(y(n)\right), ~ n\geq 0$

    \begin{gather}
      y^* = y^*\cos\left(y^*\right) \\
      \left[\cos\left(y^*\right) - 1\right]y^* = 0 \\
      \cos\left(y^*\right) = 1
    \end{gather}
    Disso, temos que $y^* = 2k\pi$, $k\in\mathbb{Z}$, são pontos de equilíbrio da equação a diferenças dada. 
  \end{enumerate}
\end{ex}

Equações a diferenças não lineares podem ter pontos de equilíbrio eventuais. Mais especificamente, uma equação a diferenças
\begin{equation}\label{eq:ead1_enl_pe_pt}
  f(y(n+1),y(n);n)=0,\quad n\geq n_0,
\end{equation}
tem $y^*$ como {\bf ponto de equilíbrio eventual} quando existe $n_1>n_0$ tal que
\begin{equation}
  y(n) = y^*,\quad n\geq n_1.
\end{equation}

\begin{ex}(Ponto de equilíbrio eventual)
  A equação a diferenças
  \begin{align}
    y(n+1) = |2y(n)-2|,\quad n\geq 0,\\
    y(0) = 1,
  \end{align}
  tem $y^* = 2$ como ponto de equilíbrio eventual. De fato, por iterações diretas temos
  \begin{align}
    y(1) &= |2y(0)-2| \\
    &= |2\cdot 1 - 2| = 0 \\
    y(2) &= |2y(1)-2| \\
    &= |2\cdot 0 - 2| = 2 \\
    y(3) &= |2y(2) - 2| \\
    &= |2\cdot 2 - 2| = 2 \\
    &\vdots \\
    y(n) &= 2,\quad n\geq 2.
  \end{align}
\end{ex}

Um ponto de equilíbrio $y^*$ de \eqref{eq:ead1_enl_pe_pt} é dito ser {\bf estável} quando, para cada $\epsilon>0$ existe $\delta>0$ tal que
\begin{equation}
  |y(0)-y^*|<\delta \Rightarrow |y(n)-y^*|<\epsilon,
\end{equation}
para todo $n>0$. Em outras palavras, para todo $n$, a solução $y(n)$ está arbitráriamente próxima de $y^*$ para toda escolha de valor inicial $y(0)\neq y^*$ suficientemente próximo de $y^*$. Quando este não é o caso, $y^*$ é dito ser ponto de equilíbrio {\bf instável}.

\begin{ex}
  Vamos estudar os pontos de equilíbrio de
  \begin{equation}
    y(n+1) = \left(y^2(n)-1\right)^2 + 1,\quad n\geq 0.
  \end{equation}

  Vamos calcular os pontos de equilíbrio.
  \begin{gather}
    y^* = \left[\left(y^*\right)^2-1\right]^2 + 1 \\
    y^* = \left(y^*\right)^2 - 2y^* + 2 \\
    \left(y^*\right)^2 - 3y^* + 2 = 0 \\
    y_1^* = 1,\quad y_2^* = 2
  \end{gather}

  Tomamos o ponto de equilíbrio $y^*=1$. Seja $\epsilon > 0$ e escolhemos $0<\delta<1$ tal que $\delta<\epsilon$. Então, para qualquer valor inicial
  \begin{equation}
    y(0) = 1 \pm \delta
  \end{equation}
  temos
  \begin{align}
    y(1) &= \left(y(0)-1\right)^2+1\\
    &= \delta^2 + 1 < 1 + \epsilon
  \end{align}
\end{ex}

\subsection*{Exercícios resolvidos}

\begin{flushright}
  [Vídeo] | [Áudio] | \href{https://phkonzen.github.io/notas/contato.html}{[Contatar]}
\end{flushright}

\emconstrucao

\subsection*{Exercícios}

\begin{flushright}
  [Vídeo] | [Áudio] | \href{https://phkonzen.github.io/notas/contato.html}{[Contatar]}
\end{flushright}

\emconstrucao
