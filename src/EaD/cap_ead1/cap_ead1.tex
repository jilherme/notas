%Este trabalho está licenciado sob a Licença Atribuição-CompartilhaIgual 4.0 Internacional Creative Commons. Para visualizar uma cópia desta licença, visite http://creativecommons.org/licenses/by-sa/4.0/deed.pt_BR ou mande uma carta para Creative Commons, PO Box 1866, Mountain View, CA 94042, USA.

\chapter{Equações de ordem 1}\label{cap_ead1}

Neste capítulo, discutimos de forma introdutória sobre {\bf equações a diferenças de primeira ordem}. Tais equações podem ser escritas na forma
\begin{equation}
  f(y_{n+1},y_n;n)=0,
\end{equation}
onde $n=0, 1, \ldots$.

\section{Equações lineares}\label{cap_ead1_sec_eqlin}

Nesta seção, discutimos sobre equações a diferenças de ordem 1 e lineares. Tais equações podem ser escritas na seguinte forma
\begin{equation}
  y_{n+1} = a(n)y_n + g(n),
\end{equation}
onde $n=n_0, n_0+1, \ldots$, $n_0$ número inteiro, $a:n\mapsto a(n)$ e $g:\mapsto g(n)$ é o termo fonte. A equação é dita ser {\bf homogênea} quando $g\equiv 0$ e, caso contrário, é dita ser {\bf não homogênea}.

\subsection{Equação homogênea}

A solução de uma equação a diferenças de ordem 1, linear e homogênea
\begin{equation}\label{eq:ead1_linear_eqh}
  y_{n+1} = a(n)y_n,\quad n\geq n_0,
\end{equation}
pode ser obtida pela iterações diretas. Para $n\geq n_0$, temos
\begin{align}
  y_{n+1} &= a(n)y_n \\
          &= a(n)a(n-1)y_{n-1}\\
          &= a(n)a(n-1)a(n-2)y_{n-2} \\
          &\vdots\\
          &= a(n)a(n-1)\cdots a(n_0)y_{n_0}.
\end{align}
Ou seja, dada o valor inicial $y_{n_0}$, temos a solução
\begin{equation}\label{eq:ead1_linear_eqh_sol}
  y_{n+1} = \left[\prod_{i=n_0}^{n}a(i)\right]y_{n_0}.
\end{equation}

\begin{ex}
  Vamos calcular a solução de
  \begin{equation}
    y_{n+1} = 2y_n,\quad n\geq 0.
  \end{equation}

  Comparando com \eqref{eq:ead1_linear_eqh}, temos $a(n) = 2$ para todo $n$. Calculando a solução por iterações diretas, temos
  \begin{align}
    y_{n+1} &= 2y_n \\
            &= 2\cdot 2y_{n-1} \\
            &= 2^2y_{n-1} \\
            &= 2^2\cdot 2y_{n-2} \\
            &= 2^3y_{n-2} \\
            &\cdots \\
            &= 2^{n+1}y_0
  \end{align}
  Equivalentemente, por \eqref{eq:ead1_linear_eqh_sol}, temos
  \begin{align}
    y_{n+1} &= \left[\prod_{i=0}^{n}2\right]y_{0} \\
            &= 2^{n+1}y_0.
  \end{align}
\end{ex}

\emconstrucao

\subsection*{Exercícios resolvidos}

\emconstrucao

\subsection*{Exercícios}

\emconstrucao