%Este trabalho está licenciado sob a Licença Atribuição-CompartilhaIgual 4.0 Internacional Creative Commons. Para visualizar uma cópia desta licença, visite http://creativecommons.org/licenses/by-sa/4.0/deed.pt_BR ou mande uma carta para Creative Commons, PO Box 1866, Mountain View, CA 94042, USA.

\chapter{Introdução}\label{cap_intro}

Neste capítulo, introduzimos conceitos e definições elementares sobre \emph{equações a diferenças}. Por exemplo, definimos tais equações, apresentamos alguns exemplos de modelagem matemática e problemas relacionados.

\ifispython
\begin{obs}\label{obs:python}
Ao longo das notas de aula, contaremos com o suporte de alguns códigos \python\, com o seguinte preâmbulo:
\begin{verbatim}
from sympy import *
\end{verbatim}
\end{obs}
\fi


\section{Equações a diferenças}\label{cap_intro_sec_ead}

Equações a diferenças são aquelas que podem ser escritas na seguinte forma
\begin{equation}\label{eq:intro_ead}
  f(y_{n+k},y_{n+k-1},\dotsc,y_n;n) = 0,
\end{equation}
onde $n=0, 1, 2, \ldots$ e $k\geq 0$ número natural.

\begin{ex}\label{ex:intro_modelos}
  Vejamos os seguintes exemplos.
  \begin{enumerate}[a)]
  \item \emph{Modelo de juros compostos}
    \begin{equation}
      y_{n+1} = (1+r)y_n
    \end{equation}
    Esta equação a diferenças modela uma aplicação corrigida a juros compostos com taxa $r$ por período de tempo $n$ (dia, mês, ano, etc.). Mais especificamente, seja $y_0$ o valor da aplicação inicial, então
    \begin{equation}
      y_1 = (1+r)y_0
    \end{equation}
    é o valor corrigido a taxa $r$ no primeiro período (dia, mês, ano). No segundo período, o valor corrigido é
    \begin{equation}
      y_2 = (1+r)y_1
    \end{equation}
    e assim por diante.
  \item \emph{Equação logística}
    \begin{equation}
      y_{n+1} = r y_n\left(1 - \frac{y_n}{K}\right),
    \end{equation}
    onde $y_n$ representa o tamanho da população no período $n$, $r$ é a taxa de crescimento e $K$ um limiar de saturação.
  \item \emph{Sequência de Fibonacci}\footnote{Fibonacci, c. 1170 - c. 1240, matemático italiano. Fonte: \href{https://en.wikipedia.org/wiki/Fibonacci}{Wikipedia}.}
    \begin{equation}
      y_{n+2} = y_{n+1} + y_n,
    \end{equation}
    onde $y_0=0$ e $y_1=1$.
  \end{enumerate}
\end{ex}

Uma equação a diferenças \eqref{eq:intro_ead} é dita ser de {\bf ordem $k$} (ou de $k$-ésima ordem). É dita ser {\bf linear} quando $f$ é função linear nas variáveis dependentes $y_{n+k}, y_{n+k-1}, \dotsc, y_n$, noutro caso é dita ser {\bf não linear}.

\begin{ex}
  No Exemplo \ref{ex:intro_modelos}, temos
  \begin{enumerate}[a)]
  \item O modelo de juros compostos é dado por equação a diferenças de primeira ordem e linear.
  \item A equação logística é uma equação a diferenças de primeira ordem e não linear.
  \item A sequência equação de Fibonacci é descrita por uma equação a diferenças de segunda ordem e linear.
  \end{enumerate}
\end{ex}

A solução de uma equação a diferenças \eqref{eq:intro_ead} é uma sequência de números $(y_n)_{n=0}^\infty = (y_0, y_1, \dotsc, y_n, \ldots)$ que satisfazem a equação. Em alguns casos é possível escrever a solução como uma forma fechada
\begin{equation}
  y_n = g(n),
\end{equation}
onde $n = 0, 1, \ldots$.

\begin{ex}
  Vamos encontrar a solução para o modelo de juros compostos
  \begin{equation}
    y_{n+1} = (1+r)y_n,\quad n\geq 0.
  \end{equation}

  A partir do valor inicial $y_0$, temos
  \begin{align}
    y_1 &= (1+r)y_0\\
    y_2 &= (1+r)y_1\\
        &= (1+r)(1+r)y_0\\
        &= (1+r)^2y_0 \\
    y_3 &= (1+r)y_2\\
        &= (1+r)(1+r)^2y_0\\
        &= (1+r)^3y_0\\
        &\vdots
  \end{align}
  Com isso, podemos inferir que a solução é dada por
  \begin{equation}
    y_n = (1+r)^ny_0,
  \end{equation}
  onde o valor inicial $y_0$ é arbitrário.
\end{ex}

\subsection*{Exercícios resolvidos}

\begin{exeresol}
  Calcule $y_{10}$, sendo que
  \begin{align}
    y_{n+1} = 1,05y_n,\quad n\geq0,
    y_0 = 1000.
  \end{align}
\end{exeresol}
\begin{resol}
  Observamos que
  \begin{align}
    y_1 &= 1,05y_0\\
    y_2 &= 1,05y_1\\
        &= 1,05\cdot 1,05y_0 \\
        &= 1,05^2y_0\\
    y_3 &= 1,05y_2\\
        &=1,05\cdot 1,05^2y_0\\
        &= 1,05^3y_0\\
        &\vdots
  \end{align}
  Com isso, temos que a solução da equação a diferenças é
  \begin{equation}
    y_n = 1,05^ny_0.
  \end{equation}
  Portanto,
  \begin{align}
    y_{10} &= 1,05^{10}y_0 \\
           &= 1,05^{10}\cdot 1000 \\
           &\approx 1628,89.
  \end{align}
\end{resol}

\begin{exeresol}
  Uma semente plantada produz uma flor com uma semente no final do primeiro ano e uma flor com duas sementes no final de cada ano consecutivo. Supondo que cada semente é plantada tão logo é produzida, escreva a equação de diferenças que modela o número de flores $y_n$ no final do $n$-ésimo ano.
\end{exeresol}
\begin{resol}
  No final do ano $n+2\geq 0$, o número de flores é igual a
  \begin{equation}
    y_{n+2} = 2u_{n+2} + 3d_{n+2},
  \end{equation}
  onde $u_{n+2}$ é o número de flores plantadas a um ano e $d_{n+2}$ é o número de flores plantas a pelo menos dois anos. Ainda, temos
  \begin{equation}
    u_{n+2} = u_{n+1} + 2d_{n+1}
  \end{equation}
  e
  \begin{equation}
    d_{n+2} = u_{n+1} + d_{n+1}.
  \end{equation}
  Com isso, temos
  \begin{align}
    y_{n+2} &= 2(u_{n+1}+2d_{n-1}) + 3(u_{n+1}+d_{n-1}) \\
          &= 2y_{n+1} + u_{n+1} + d_{n+1} \\
          &= 2y_{n+1} + \underbrace{u_{n} + 2d_{n}}_{u_{n+1}} + \underbrace{u_{n} + d_{n}}_{d_{n+1}} \\
          &= 2y_{n+1} + 2u_{n} + 3d_{n} \\
          &= 2y_{n} + y_{n}.
  \end{align}
  Desta forma, concluímos que o número de plantas é modelado pela seguinte equação a diferenças de segunda ordem e linear
  \begin{equation}
    y_{n+2} = 2y_{n+1} + y_{n+2}.
  \end{equation}
\end{resol}

\subsection*{Exercícios}

\begin{exer}
  Classifique as seguintes equações a diferenças quanto a ordem e linearidade.
  \begin{enumerate}
  \item $\displaystyle y_{n+1}-\sqrt{2}y_{n} = 1$
  \item $\displaystyle ny_{n+1} = y_{n}\ln(n+1)$
  \item $\displaystyle y_{n} = y_{n+1} + 2y_{n+2} - 1$
  \item $\displaystyle y_{n+1} - (1-y_n)(1+y_n) = 0$
  \item $\displaystyle y_{n+2} = n\sqrt{y_n}$
  \end{enumerate}
\end{exer}
\begin{resp}
  a) ordem 1, linear; b) ordem 1, linear; c) ordem 2, linear; d) ordem 1, não linear; e) ordem 2, não linear;
\end{resp}

\begin{exer}
  Encontre a equação a diferenças que modela o saldo devedor anual de uma cliente de cartão de crédito com taxa de juros de 200\% a.a. (ao ano), considerando uma dívida inicial no valor de $y_0$ reais e que o cartão não está mais em uso.
\end{exer}
\begin{resp}
  $y_{n+1} = 3y_n$.
\end{resp}

\begin{exer}
  Considere uma espécie de seres vivos monogâmicos que após um mês de vida entram na fase reprodutiva. Durante a fase reprodutiva, cada casal produz um novo casal por mês. Desconsiderando outros fatores (por exemplo, mortalidade, perda de fertilidade, etc.), encontre a equação a diferenças que modela o número de casais no $n$-ésimo mês. 
\end{exer}
\begin{resp}
  Sequência de Fibonacci
\end{resp}
