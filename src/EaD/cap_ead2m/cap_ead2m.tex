%Este trabalho está licenciado sob a Licença Atribuição-CompartilhaIgual 4.0 Internacional Creative Commons. Para visualizar uma cópia desta licença, visite http://creativecommons.org/licenses/by-sa/4.0/deed.pt_BR ou mande uma carta para Creative Commons, PO Box 1866, Mountain View, CA 94042, USA.

\chapter{Equações de ordem 2 ou mais alta}\label{cap_ead2m}

Neste capítulo, temos uma rápida introdução a equações a diferenças de ordem 2 ou mais alta.

\section{Equações lineares de ordem 2}\label{cap_ead2m_sec_ead2lin}

\begin{flushright}
  \href{https://youtu.be/Y8-0DOGM-i8}{[YouTube]} | \href{https://archive.org/details/ead-o2h}{[Vídeo]} | [Áudio] | \href{https://phkonzen.github.io/notas/contato.html}{[Contatar]}
\end{flushright}

Aqui, vamos considerar equações lineares de ordem 2 com coeficientes constantes e homogêneas, i.e. equações da forma
\begin{equation}\label{eq:ead2m_2cch}
  {\color{blue}y(n+2)+p_1y(n+1)+p_2y(n)=0},
\end{equation}
onde $p_1,p_2\in\mathbb{R}$.

A ideia para resolver uma tal equação é de buscar por soluções da forma
\begin{equation}
  {\color{blue}y(n)=\lambda^n},
\end{equation}
onde $\lambda$ é um escalar não nulo (número real ou complexo). Substituindo em \eqref{eq:ead2m_2cch}, obtemos
\begin{align}
  \lambda^{n+2}+p_1\lambda^{n+1} + p_2\lambda^n = 0\\
  \lambda^n\left(\lambda^{2}+p_1\lambda + p_2\right) = 0.
\end{align}
Ou seja, $\lambda$ deve satisfazer a \emph{equação característica}
\begin{equation}\label{eq:ead2m_2ec}
  {\color{blue}\lambda^{2}+p_1\lambda + p_2 = 0}.
\end{equation}

\subsection{Caso de raízes reais distintas}

\begin{flushright}
  [Vídeo] | [Áudio] | \href{https://phkonzen.github.io/notas/contato.html}{[Contatar]}
\end{flushright}

Aqui, vamos encontrar a solução geral para \eqref{eq:ead2m_2cch} quando a equação característica associada \eqref{eq:ead2m_2ec} tem raízes reais distintas. As raízes podem ser obtidas da fórmula de Bhaskara, i.e.
\begin{equation}
  {\color{blue}\lambda_1,\lambda_2 = \frac{-p_1 \pm \sqrt{p_1^2-4p_2}}{2}},
\end{equation}
onde $p_1^2 - 4p_2 > 0$. Com isso, temos as soluções
\begin{gather}
  y_1(n) = \lambda_1^n,\\
  y_2(n) = \lambda_2^n.
\end{gather}
Estas são chamadas de \emph{soluções fundamentais}, pois pode-se mostrar que qualquer solução da equação a diferenças \eqref{eq:ead2m_2cch} pode ser escrita como combinação linear de $y_1(n)$ e $y_2(n)$. Ou seja, a solução geral de \eqref{eq:ead2m_2cch} é
\begin{equation}
  {\color{blue}y(n) = c_1\underbrace{\,\lambda_1^n\,}_{y_1(n)} + c_2\underbrace{\,\lambda_2^n\,}_{y_2(n)}},
\end{equation}
onde $c_1$ e $c_2$ são constantes indeterminadas.

\begin{ex}
  Vamos encontrar a solução geral de
  \begin{equation}
    y(n+2) - 4y(n) = 0.
  \end{equation}
  Para tanto, resolvemos a \emph{equação característica} associada
  \begin{gather}
    \lambda^2 - 4 = 0\\
    \lambda^2 = 4\\
    \lambda = \pm 2
  \end{gather}
  Com isso, temos as soluções fundamentais $y_1(n)=(-2)^n$ e $y_2(n)=2^n$. A \emph{solução geral} é
  \begin{equation}
    y(n) = c_1\cdot (-2)^n + c_2\cdot 2^n.
  \end{equation}
\end{ex}

\subsection{Caso de raízes reais duplas}

\begin{flushright}
  [Vídeo] | [Áudio] | \href{https://phkonzen.github.io/notas/contato.html}{[Contatar]}
\end{flushright}

Agora, vamos encontrar a solução geral para \eqref{eq:ead2m_2cch} quando a equação característica associada \eqref{eq:ead2m_2ec} tem raízes reais duplas, i.e.
\begin{equation}
  {\color{blue}\lambda_{1,2} = -\frac{p_1}{2}}.
\end{equation}

Neste caso, múltiplos de
\begin{equation}
  {\color{blue}y_1(n) = \lambda^n_{1,2}}
\end{equation}
não nos fornecem todas as soluções possíveis da equação a diferenças. Entretanto, temos que
\begin{equation}
  {\color{blue}y_2(n) = n\lambda_{1,2}^{n-1}},
\end{equation}
também é solução. De fato, substituindo em \eqref{eq:ead2m_2cch}, obtemos
\begin{gather}
  y_2(n+2) + p_1y_2(n+1) + p_2y_2(n) = 0\\
  (n+2)\lambda_{1,2}^{n+1} + p_1\cdot(n+1)\lambda_{1,2}^n + p_2\cdot n\lambda_{1,2}^{n-1} = 0\\
  n\lambda_{1,2}^{-1}\left(\underbrace{\lambda_{1,2}^{n+2}+p_1\cdot\lambda_{1,2}^{n+1}+p_2\lambda_{1,2}^n}_{=0}\right) + 2\lambda_{1,2}^{n+1}+p_1\lambda_{1,2}^n = 0\\
  2\left(-\frac{p_1}{2}\right)^{n+1}+p_1\left(-\frac{p_1}{2}\right)^n = 0\\
  (-1)^{n+1}\frac{p_1^{n+1}}{2^n} + (-1)^n\frac{p_1^{n+1}}{2^n} = 0\\
  0 = 0.
\end{gather}

Com isso, temos que a \emph{solução geral} da equação a diferenças é dada por
\begin{equation}
  {\color{blue}y(n) = c_1\lambda_{1,2}^n + c_2n\lambda_{1,2}^{n-1}}.
\end{equation}

\begin{ex}
  Vamos encontrar a solução geral de
  \begin{equation}
    y(n+2)+4y(n+1)+4y(n)=0.
  \end{equation}
  Começamos encontrando as soluções da equação característica associada
  \begin{gather}
    \lambda^2 + 4\lambda + 4 = 0\\
    (\lambda+2)^2 = 0 \\
    \lambda_{1,2} = -2.
  \end{gather}
  Desta forma, temos as soluções fundamentais
  \begin{gather}
    y_1(n) = (-2)^n\\
    y_2(n) = n\cdot (-2)^{n-1}
  \end{gather}
  e a solução geral
  \begin{gather}
    y(n) = c_1\cdot (-2)^n + c_2\cdot n\cdot (-2)^{n-1}\\
    y(n) = c_1\cdot (-2)^n + c_2\cdot n\cdot \frac{(-2)^n}{-2}\\
    y(n) = c_1\cdot(-2)^n + c_2\cdot n \cdot (-2)^n\\
    y(n) = (-2)^n\cdot \left(c_1 + c_2\cdot n\right)
  \end{gather}
\end{ex}

\subsection{Caso de raízes complexas}

\begin{flushright}
  [Vídeo] | [Áudio] | \href{https://phkonzen.github.io/notas/contato.html}{[Contatar]}
\end{flushright}

Agora, vamos encontrar a solução geral para \eqref{eq:ead2m_2cch} quando a equação característica associada \eqref{eq:ead2m_2ec} tem raízes complexas, i.e.
\begin{equation}
  {\color{blue}\lambda_{1,2} = \alpha \pm i\beta}.
\end{equation}
Neste caso, temos a \emph{solução geral}
\begin{equation}
  {\color{blue}y(n) = c_1(\alpha-i\beta)^n + c_2(\alpha + i\beta)^n}.
\end{equation}

\begin{ex}
  Vamos encontrar a solução geral de
  \begin{equation}
    y(n+2) + 4y(n) = 0.
  \end{equation}
  Resolvemos a equação característica associada.
  \begin{gather}
    \lambda^2 + 4 = 0\\
    \lambda^2 = -4 \\
    \lambda_{1,2} = \pm 2i
  \end{gather}
  Com isso, temos a solução geral
  \begin{equation}
    y(n) = c_1\cdot (-2i)^n + c_2\cdot (2i)^n.
  \end{equation}
\end{ex}

\subsection*{Exercícios resolvidos}

\begin{flushright}
  [Vídeo] | [Áudio] | \href{https://phkonzen.github.io/notas/contato.html}{[Contatar]}
\end{flushright}

\begin{exeresol}
  A \emph{sequência de Fibonacci}\footnote{Leonardo Fibonacci, c.1170 - c1250, matemático italiano. Fonte: \href{https://pt.wikipedia.org/wiki/Leonardo_Fibonacci}{Wikipédia}.}
  \begin{equation}
    1, 1, 2, 3, 5, 8, 13, \ldots
  \end{equation}
  tem valores iniciais $y(1)=1$, $y(2)=1$ e os demais valores $y(n+2)=y(n+1)+y(n)$. Logo, a sequência é solução da equação a diferenças
  \begin{align}
    &y(n+2)-y(n+1)-y(n)=0,\quad n\geq 1,\\
    &y(1)=1,\quad y(2)=1.
  \end{align}
  Resolva esta equação a diferença de forma a obter uma forma fechada para $y(n)$, i.e. o $n$-ésimo valor na sequência de Fibonacci.
\end{exeresol}
\begin{resol}
  A equação a diferenças
  \begin{equation}
    y(n+2)-y(n+1)-y(n)=0
  \end{equation}
  é linear e com coeficientes constantes. Desta forma, temos a equação característica associada
  \begin{equation}
    \lambda^2 - \lambda -1 = 0
  \end{equation}
  a qual tem raízes reais distintas
  \begin{align*}
    \lambda_1 = \frac{1-\sqrt{5}}{2},\\
    \lambda_2 = \frac{1+\sqrt{5}}{2}.
  \end{align*}
  Logo, a solução geral desta equação é
  \begin{equation}
    y(n) = c_1\left(\frac{1-\sqrt{5}}{2}\right)^n + c_2\left(\frac{1+\sqrt{5}}{2}\right)^n,\quad n\geq 1.
  \end{equation}
  Agora, aplicando os valores iniciais $y(1)=1$ e $y(2)=2$, obtemos
  \begin{align*}
    y(1)&=1\Rightarrow c_1\left(\frac{1-\sqrt{5}}{2}\right) + c_2\left(\frac{1+\sqrt{5}}{2}\right) = 1\\
    y(2)&=1\Rightarrow c_1\left(\frac{1-\sqrt{5}}{2}\right)^2 + c_2\left(\frac{1+\sqrt{5}}{2}\right)^2 = 1
  \end{align*}
  Resolvendo, obtemos
  \begin{gather}
    c_1 = -\frac{1}{\sqrt{5}},\\
    c_2 = \frac{1}{\sqrt{5}}.
  \end{gather}
  Concluímos que a solução é
  \begin{equation}
    y(n) = -\frac{1}{\sqrt{5}}\left(\frac{1-\sqrt{5}}{2}\right)^n + \frac{1}{\sqrt{5}}\left(\frac{1+\sqrt{5}}{2}\right)^n,\quad n\geq 1.
  \end{equation}
\end{resol}

\begin{exeresol}
  Entre a solução da seguinte equações a diferenças
  \begin{align}
    &y(n+2)-2y(n+1)+y(n) = 0,\quad n\geq 0,\\
    &y(0)=1,\quad y(1)=1.
  \end{align}
\end{exeresol}
\begin{resol}
  Trata-se de uma equação a diferenças de ordem 2 com coeficientes constantes e homogênea. A equação característica associada é
  \begin{equation}
    \lambda^2 - 2\lambda + 1 = 0
  \end{equation}
  com raízes reais duplas $\lambda_{1,2} = 1$. Assim sendo, a solução geral é
  \begin{align}
    y(n) &= c_1\cdot 1^n + c_2\cdot n \cdot 1^n\\
    &= c_1 + c_2\cdot n.
  \end{align}
  Aplicando os valores iniciais, obtemos
  \begin{align}
    y(0)=1 \Rightarrow c_1 = 1\\
    y(1)=1 \Rightarrow 1 + c_2 = 1
  \end{align}
  Logo, temos $c_1=1$ e $c_2=0$. Concluímos que a solução é a sequência constante
  \begin{equation}
    y(n)=1.
  \end{equation}
\end{resol}

\begin{exeresol}
  Resolva a seguinte equação a diferenças
  \begin{equation}
    y(n+2) - 2y(n+1) + 2 = 0.
  \end{equation}
\end{exeresol}
\begin{resol}
  Sendo a equação a diferenças linear homogênea com coeficientes constantes, resolvemos a equação característica
  \begin{gather}
    \lambda^2 - 2\lambda + 2 = 0\\
    \lambda_{1,2} = \frac{2\pm\sqrt{2^2-4\cdot 2}}{2}\\
    \lambda_{1,2} = 1\pm i
  \end{gather}
  Sendo estas as raízes, temos a solução geral
  \begin{equation}
    y(n) = c_1(1-i)^n + c_2(1+i)^n.
  \end{equation}
\end{resol}

\subsection*{Exercícios}

\begin{flushright}
  [Vídeo] | [Áudio] | \href{https://phkonzen.github.io/notas/contato.html}{[Contatar]}
\end{flushright}

\begin{exer}
  Calcule a solução geral de
  \begin{equation}
    y(n+2) - 5y(n+1) + 6y(n) = 0
  \end{equation}
\end{exer}
\begin{resp}
  $y(n) = c_1\cdot 2^n + c_2\cdot 3^n$
\end{resp}

\begin{exer}
  Calcule a solução geral de
  \begin{equation}
    y(n+2) - 4y(n+1) + 4y(n) = 0
  \end{equation}
\end{exer}
\begin{resp}
  $y(n) = 2^n(c_1 + c_2\cdot n)$
\end{resp}

\begin{exer}
  Calcule a solução geral de
  \begin{equation}
    y(n+2) + 4y(n+1) + 13y(n) = 0
  \end{equation}
\end{exer}
\begin{resp}
  $y(n) = c_1(-2-3i)^n + c_2(-2+3i)^n$
\end{resp}

\begin{exer}
  Resolva
  \begin{align}
    &y(n+2) - 2y(n+1) - 8y(n) = 0,\quad n\geq 0,\\
    &y(0)=2,\quad y(1)=-1.
  \end{align}
\end{exer}
\begin{resp}
  $y(n) = \frac{3}{2}(-2)^n + \frac{1}{2}4^n$
\end{resp}
