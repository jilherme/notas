%Este trabalho está licenciado sob a Licença Atribuição-CompartilhaIgual 4.0 Internacional Creative Commons. Para visualizar uma cópia desta licença, visite http://creativecommons.org/licenses/by-sa/4.0/deed.pt_BR ou mande uma carta para Creative Commons, PO Box 1866, Mountain View, CA 94042, USA.

\chapter{Perceptron}\label{cap_perceptron}
\thispagestyle{fancy}

\section{Unidade de Processamento}

A unidade básica de processamento (neurônio artificial) que exploramos nestas notas é baseado no \emph{perceptron} (consultemos a Fig. \ref{fig:perceptron}). Consiste na composição de uma \emph{função de ativação} $f:\mathbb{R}\to\mathbb{R}$ com a \emph{pré-ativação}
\begin{align}
  p &= \pmb{x}\cdot\pmb{w}_{:n} + w_{n}\\
    &= x_0w_0 + x_1w_1 + \cdots + x_{n-1}w_{n-1} + w_{n}
\end{align}
onde, $\pmb{x}\in\mathbb{R}^{n}$ é o vetor de entrada, $\pmb{w}\in\mathbb{R}^{n+1}$ é o vetor de pesos\footnote{$w_{n}$ é chamado de \emph{bias}.}. Escolhida uma função de ativação, a saída do neurônio é dada por
\begin{align}
  y &= f(p)\\
    &= f(\pmb{x}\cdot\pmb{w}_{:n} + w_n)
\end{align}
O treinamento (calibração) consiste em determinar os pesos $\pmb{w}$ de forma que o neurônio forneça as saídas $y$ esperadas os valores de entrada $\pmb{x}$.

\begin{figure}[H]
  \centering
  
  \caption{Esquema de um perceptron: unidade de processamento.}
  \label{fig:perceptron}
\end{figure}

Uma das vantagens deste modelo de neurônio é sua generalidade, i.e. pode ser aplicado a diferentes problemas. Na sequência, vamos aplicá-lo na resolução de um problema de classificação e noutro de regressão.

\subsection{Um problema de classificação}

Vamos desenvolver um perceptron que faça a operação do \verb+e+-lógico. I.e, receba como entrada dois valores lógicos $A_0$ e $A_1$ (V, verdadeiro ou F, falso) e forneça como saída o valor lógico $R = A_0 \verb+e+ A_1$. Consultemos a seguinte tabela verdade:

\begin{center}
  \begin{tabular}{cc|c}
    $A_0$ & $A_1$ & R\\\hline
    V & V & V\\
    V & F & F\\
    F & V & F\\
    F & F & F\\\hline
  \end{tabular}
\end{center}

\subsubsection{Modelo}

Nosso modelo de neurônio será um perceptron com duas entradas $\pmb{x}\in \{-1,1\}^2$ e a função de identidade $f(p)=p$ como função de ativação, i.e.
\begin{equation}
  y = \pmb{x}\cdot\pmb{w}_{:2} + w_2,
\end{equation}
onde $\pmb{w}\in\mathbb{R}^3$ são pesos a determinar.
    
\subsubsection{Pré-processamento}

Uma vez que nosso modelo recebe valores $\pmb{x}\in \{-1,1\}^2$ e retorna $y\in\mathbb{R}$, precisamos pré-processar os dados do problema de forma a utilizá-lo. Uma forma, é assumir que todo valor negativo está associado ao valor lógico $F$ (falso) e valor positivo ao valor lógico $V$ (verdadeiro). Desta forma, os dados podem ser interpretados como na seguinte tabela

\begin{center}
  \begin{tabular}{rr|c}
    $x_0$ & $x_1$ & y\\\hline
    1 & 1 & +\\
    1 & -1 & -\\
    -1 & 1 & -\\
    -1 & -1 & -\\\hline
  \end{tabular}
\end{center}
    
    
\subsubsection{Treinamento}

Agora, nos falta treinar nosso neurônio para fornecer o valor de $y$ esperado para cada dada uma entrada $\pmb{x}$. Isso consiste em um método para escolhermos os pesos $\pmb{w}$ que sejam adequados para esta tarefa. Vamos explorar mais sobre isso na sequência do texto e, aqui, apenas escolhemos
\begin{equation}
  \pmb{w} = (1, 1, -0.5)
\end{equation}
Com isso, nosso perceptron é
\begin{align}
  y &= f(\pmb{x}\cdot\pmb{w}_{:n} + w_n)\\
    &= x_0 + x_1 - 1.5
\end{align}
Verifique que ele satisfaz a tabela verdade acima!

\subsubsection{Implementação}

\ifispython
Podemos implementar nosso perceptron com o {\pytorch} como segue

\lstinputlisting{./cap_perceptron/dados/py_e/main.py}

Verifique! Nesta implementação utilização vários métodos para a manipulação de tensores (\lstinline+torch.Tensor+). Consulte \href{https://pytorch.org/docs/stable/torch.html}{PyTorch Docs: torch} para mais informações sobre estes e muitos outros métodos disponíveis.
\fi

\subsubsection{Interpretação geométrica}

\emconstrucao

\subsection{Problema de regressão}

\emconstrucao

\section{Treinamento}

\emconstrucao
