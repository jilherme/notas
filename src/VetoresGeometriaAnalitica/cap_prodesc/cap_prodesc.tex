%Este trabalho está licenciado sob a Licença Atribuição-CompartilhaIgual 4.0 Internacional Creative Commons. Para visualizar uma cópia desta licença, visite http://creativecommons.org/licenses/by-sa/4.0/deed.pt_BR ou mande uma carta para Creative Commons, PO Box 1866, Mountain View, CA 94042, USA.

\chapter{Produto escalar}\label{cap_prodesc}
\thispagestyle{fancy}

\section{Produto escalar}\label{cap_prodesc_sec_prodesc}

Ao longo desta seção, assumiremos $B = (\vec{i},\vec{j},\vec{k})$ uma base ortonormal no espaço. O {\bf produto escalar} dos vetores $\vec{u} = (u_1,u_2,u_3)$ e $\vec{v}=(v_1,v_2,v_3)$ é o número real
\begin{equation}
  \vec{u}\cdot\vec{v} = u_1v_1+u_2v_2+u_3v_3.
\end{equation}

\begin{ex}
  Se $\vec{u}=(2,-1,3)$ e $\vec{v}=(-3,-4,2)$, então
  \begin{equation}
    \vec{u}\cdot\vec{v} = 2\cdot(-3)+(-1)\cdot(-4)+3\cdot 2 = 4.
  \end{equation}
\end{ex}

\subsection{Propriedades do produto interno}

Quaisquer que sejam $\vec{u}$, $\vec{v}$, $\vec{w}$ e qualquer número real $\alpha$, temos:
\begin{itemize}
\item Comutatividade: $\vec{u}\cdot\vec{v}=\vec{v}\cdot\vec{u}$.

  Dem.:
  \begin{align}
    \vec{u}\cdot\vec{v} &= (u_1,u_2,u_3)\cdot(v_1,v_2,v_3)\\
                        &= u_1v_1+u_2v_2+u_3v_3 \\
                        &= v_1u_1+v_2u_2+v_3u_3 \\
                        &= \vec{v}\cdot\vec{u}.
  \end{align}

\item Distributividade com multiplicação por escalar:
  \begin{equation}
  (\alpha\vec{u})\cdot\vec{v}=\vec{u}\cdot(\alpha\vec{v})=\alpha(\vec{u}\cdot\vec{v}).
\end{equation}


  Dem.:
  \begin{align}
    (\alpha\vec{u})\cdot\vec{v} &= (\alpha u_1,\alpha u_2, \alpha u_3)\cdot (v_1,v_2,v_3)\\
                                &= (\alpha u_1)v_1+(\alpha u_2)v_2 + (\alpha u_3)v_3 \\
                                &= \alpha (u_1v_1)+\alpha (u_2v_2)+\alpha (u_3v_3) \\
                                &= \alpha (u_1v_1+u_2v_2+u_3v_3) = \alpha(\vec{u}\cdot\vec{v})\\
                                &= u_1(\alpha v_1) + u_2(\alpha v_2) + u_3(\alpha v_3) \\
                                &= (u_1,u_2,u_3)\cdot(\alpha v_1,\alpha v_2,\alpha v_3) \\
                                &= \vec{u}\cdot(\alpha\vec{v}).
  \end{align}

\item Distributividade com a adição: $\vec{u}\cdot(\vec{v}+\vec{w}) = \vec{u}\cdot\vec{v}+\vec{u}\cdot\vec{w}$.

  Dem.:
  \begin{align}
    \vec{u}\cdot(\vec{v}+\vec{w}) &= (u_1,u_2,u_3)\cdot\left((v_1,v_2,v_3)+(w_1,w_2,w_3)\right) \\
                                  &= (u_1,u_2,u_3)\cdot [(v_1+w_1,v_2+w_2,v_3+w_3)] \\
                                  &= u_1(v_1+w_1) + u_2(v_2+w_2) + u_2(v_2+w_2) \\
                                  &= u_1v_1+u_1w_1+u_2v_2+u_2w_2+u_3v_3+u_3w_3 \\
                                  &= u_1v_1+u_2v_2+u_3v_3 + u_1w_1+u_2w_2+u_3w_3 \\
                                  &= \vec{u}\cdot\vec{v}+\vec{u}\cdot\vec{w}.
  \end{align}

\item Sinal: $\vec{u}\cdot\vec{u}\geq 0$ e $\vec{u}\cdot\vec{u}=0 \Leftrightarrow \vec{u}=\vec{0}$.

  Dem.:
  \begin{align}
    \vec{u}\cdot\vec{u} = u_1^2+u_2^2+u_3^2 \geq 0.
  \end{align}
  Além disso, observamos que a soma de números não negativos é nula se, e somente se, os números forem zeros.

\item Norma: $|u|^2 = \vec{u}\vec{u}$.

  Dem.:
  Como fixamos uma base ortonormal $B$, a Proposição \ref{prop:bo_norma} nos garante que
  \begin{equation}
    |u|^2 = u_1^2+u_2^2+u_3^2 = \vec{u}\cdot\vec{u}.
  \end{equation}
\end{itemize}

\begin{ex}
  Sejam $\vec{u}=(-1,2,1)$, $\vec{v}=(2,-1,3)$ e $\vec{w}=(1,0,-1)$. Vejamos os seguintes casos:
  \begin{itemize}
  \item Comutatividade:
    \begin{align}
      \vec{u}\cdot\vec{v} &= -1\cdot 2 + 2\cdot (-1) + 1\cdot 3 = -1,\\
      \vec{v}\cdot\vec{u} &= 2\cdot(-1) + (-1)\cdot 2 + 3\cdot 1 = -1.            
    \end{align}
  \item Distributividade com a multiplicação por escalar:
    \begin{align}
      (2\vec{u})\cdot\vec{v} &= (-2,4,2)\cdot(2,-1,3) = -4-4+6=-2,\\
      2(\vec{u}\vec{v}) &= 2(-2-2+3) = -2,\\
      \vec{u}\cdot(2\vec{v}) &= (-1,2,1)\cdot(4,-2,6) = -2.
    \end{align}
  \item Distributividade com a adição:
    \begin{align}
      \vec{u}\cdot(\vec{v}+\vec{w})) &= (-1,2,1)\cdot(3,-1,2) = -3-2+2=-3,\\
      \vec{u}\cdot\vec{v}+\vec{u}\cdot\vec{w} &= (-2-2+3)+(-1+0-1) = -3.
    \end{align}
  \item Sinal:
    \begin{align}
      \vec{w}\vec{w} = 1+0+1 = 2 \geq 0.
    \end{align}
  \item Norma:
    \begin{align}
      |u|^2 &= (-1)^2+2^2+1^2 = 6,\\
      \vec{u}\cdot\vec{u} &= (-1)\cdot(-1)+2\cdot 2+1\cdot 1 = 6.
    \end{align}
  \end{itemize}
\end{ex}

\subsection{Ângulo entre dois vetores}

O {\bf ângulo formado entre dois vetores} $\vec{u}$ e $\vec{v}$ não nulos, é definido como o menor ângulo determinado entre quaisquer representações $\vec{u} = \overrightarrow{OA}$ e $\vec{v} = \overrightarrow{OB}$.

\begin{prop}\label{prop:angulo_prodesc}
  Dados $\vec{u}$ e $\vec{v}$, temos
  \begin{equation}
    \vec{u}\cdot\vec{v}=|\vec{u}||\vec{v}|\cos\alpha,
  \end{equation}
  onde $\alpha$ é o ângulo entre os vetores $\vec{u}$ e $\vec{v}$.
\end{prop}
\begin{dem}
  Tomamos as representações $\vec{u} = \overrightarrow{OA}$ e $\vec{v} = \overrightarrow{OB}$. Observamos que $\vec{u}-\vec{v} = \overrightarrow{BA}$. Então, aplicando a lei dos cossenos no triângulo $\triangle OAB$, obtemos
  \begin{equation}
    |\overrightarrow{BA}|^2 = |\overrightarrow{OA}|^2 + |\overrightarrow{OB}|^2 - 2|\overrightarrow{OA}||\overrightarrow{OB}|\cos\alpha,
  \end{equation}
  ou, equivalentemente,
  \begin{align}
    |\vec{u}-\vec{v}|^2 &= |\vec{u}|^2+|\vec{v}|^2-2|\vec{u}||\vec{v}|\cos\alpha\\
    (\vec{u}-\vec{v})\cdot(\vec{u}-\vec{v}) &= |\vec{u}|^2+|\vec{v}|^2-2|\vec{u}||\vec{v}|\cos\alpha\\
    \vec{u}\cdot\vec{u}-2\vec{u}\cdot\vec{v}+\vec{v}\cdot\vec{v} &= |\vec{u}|^2+|\vec{v}|^2-2|\vec{u}||\vec{v}|\cos\alpha\\
    |\vec{u}|^2+|\vec{v}|^2-2\vec{u}\cdot\vec{v} &= |\vec{u}|^2+|\vec{v}|^2-2|\vec{u}||\vec{v}|\cos\alpha
  \end{align}
  donde
  \begin{equation}
    \vec{u}\cdot\vec{v} = |\vec{u}||\vec{v}|\cos\alpha.
  \end{equation}
\end{dem}

\begin{ex}
  Vamos determinar ângulo entre os vetores $\displaystyle \vec{u}=\left(\frac{\sqrt{3}}{2},\frac{1}{2},0\right)$ e $\displaystyle \vec{u}=\left(\frac{1}{2},\frac{\sqrt{3}}{2},0\right)$. Da Proposição \ref{prop:angulo_prodesc}, temos
    \begin{align}
      \cos\alpha &= \frac{\vec{u}\cdot\vec{v}}{|u|\cdot|v|}\\
      &= \frac{\frac{\sqrt{3}}{2}}{1\cdot 1} = \frac{\sqrt{3}}{2}.
    \end{align}
    Portanto, temos $\alpha = \pi/6$.
\end{ex}

\subsection*{Exercícios}

\emconstrucao