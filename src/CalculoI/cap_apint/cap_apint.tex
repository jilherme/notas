%Este trabalho está licenciado sob a Licença Atribuição-CompartilhaIgual 4.0 Internacional Creative Commons. Para visualizar uma cópia desta licença, visite http://creativecommons.org/licenses/by-sa/4.0/deed.pt_BR ou mande uma carta para Creative Commons, PO Box 1866, Mountain View, CA 94042, USA.

\chapter{Aplicações da integral}\label{cap_apint}
\thispagestyle{fancy}

\ifispython
\begin{obs}\label{obs:cap_apint_python}
  Nos códigos \verb+Python+ apresentados neste capítulo, assumimos o seguinte preâmbulo:
\begin{verbatim}
from sympy import *
var('x',real=True)
\end{verbatim}
\end{obs}
\fi

\section{Cálculo de áreas}\label{cap_apint_sec_areas}

A integral definida $\int_a^b f(x)\,dx$ está associada a área entre o gráfico da função $f$ e o eixo das abscissas no intervalo $[a,b]$. Ocorre que se $f$ for não negativa, então $\int_a^b f(x)\,dx \geq 0$. Se $f$ for negativa, então $\int_a^b f(x)\,dx < 0$. Por isso, dizemos que $\int_a^b f(x)\,dx$ é a área líquida (ou com sinal) entre o gráfico de $f$ e o eixo das abscissas.

\emconstrucao

\subsection{Áreas entre curvas}

Observamos que se $f(x)\geq g(x)$ no intervalo $[a, b]$, então
\begin{equation}
  \int_a^b f(x)-g(x)\,dx
\end{equation}
corresponde à área entre as curvas $f(x)$ e $g(x)$ restritas ao intervalo $[a,b]$.

\emconstrucao

\subsection*{Exercícios resolvidos}

\begin{exeresol}
  Calcule a área entre o gráfico de $f(x) = x^3-x$ e o eixo das abscissas no intervalo $[-1,1]$.
\end{exeresol}
\begin{resol}
  Para calcularmos a área entre o gráfico de $f(x)$ e o eixo das abscissas no intervalo $[-1,1]$, fazemos:
  \begin{enumerate}[1.]
  \item O estudo de sinal de $f$ no intervalo $[-1,1]$.
    \begin{enumerate}
    \item Cálculo das raízes de $f$ no intervalo $[-1,1]$.
      \begin{align}
        x^3-x=0 &\Rightarrow x(x^2-1)=0\\
                &\Rightarrow x(x-1)(x+1)=0\\
                &\Rightarrow x=-1\text{ ou }x=0\text{ ou }x=1.
      \end{align}
    \item Os sinais de $f(x)$.
      \begin{align}
        -1\leq x \leq 0 \Rightarrow f(x)\geq 0\\
        0\leq x \leq 1 \Rightarrow f(x)\leq 0.
      \end{align}
    \end{enumerate}
  \item Cálculo da área usando integrais definidas.
    \begin{enumerate}
    \item Cálculo da integral indefinida.
      \begin{align}
        \int f(x)\,dx &= \int x^3-x\,dx\\
                      &= \int x^3\,dx - \int x\,dx\\
                      &= \frac{x^4}{4} - \frac{x^2}{2} + C.
      \end{align}
    \item Cálculo da área.
    \begin{align}
      A &= \int_{-1}^0 f(x)\,dx - \int_{0}^{1} f(x)\,dx \\
        &= \left[\frac{x^4}{4} - \frac{x^2}{2}\right]_{-1}^0 - \left[\frac{x^4}{4} - \frac{x^2}{2}\right]_{0}^1\\
        &= \frac{1}{2}.
    \end{align}
    \end{enumerate}
  \end{enumerate}

  \ifispython
  Podemos computar a solução deste exercícios usando os seguintes comandos do \verb+Sympy+\footnote{Veja Observação \ref{obs:cap_apint_python}.}. Para o estudo de sinal, podemos utilizar
\begin{verbatim}
f = lambda x: x*(x-1)*(x+1)
reduce_inequalities(f(x)>=0)
\end{verbatim}
  Então, para o cálculo da área, podemos utilizar
\begin{verbatim}
integrate(f(x),(x,-1,0))-integrate(f(x),(x,0,1))
\end{verbatim}
  \fi
\end{resol}

\begin{exeresol}
  Cálculo a área entre a reta $y=1$ e o gráfico de $f(x)=x^2$ restritas ao intervalo $[0,1]$.
\end{exeresol}
\begin{resol}
  Observamos que a medida desta área corresponde à área do quadrado $\{0\leq x \leq 1\}\times \{0\leq y \leq 1\}$ descontada a área sob o gráfico de $f(x)=x^2$ restrita ao intervalo $[0,1]$. Isto é,
  \begin{align}
    A &= 1 - \int_0^1 x^2\,dx\\
      &= 1 - \left[\frac{x^3}{3}\right]_0^1\\
      &= 1 - \frac{2}{3} = \frac{1}{3}.
  \end{align}
\end{resol}

\begin{exeresol}
  Calcule a área entre as curvas $y=x^2$, $y=x$, $x=0$ e $x=1$.
\end{exeresol}
\begin{resol}
  O problema é equivalente a calcular a área entre os gráficos das funções $f(x)=x$ e $g(x)=x^2$ restritas ao intervalo $[0,1]$. Como $f(x)\geq g(x)$ neste intervalo, temos
  \begin{align}
    A &= \int_0^1 f(x)-g(x)\,dx\\
      &= \int_0^1 x-x^2\,dx\\
      &= \left.\frac{x^2}{2}-\frac{x^3}{3}\right|_0^1\\
      &= \frac{1}{6}.
  \end{align}
\end{resol}

\emconstrucao

\subsection*{Exercícios}

\begin{exer}
  Calcule a área entre o gráfico de $f(x)=x^3$ e a reta $y=1$ restritas ao intervalo $[-1,1]$.
\end{exer}
\begin{resp}
  $2$
\end{resp}

\begin{exer}
  Calcule a área entre as curvas $y=x$, $y=x^2$, $x=0$ e $x=2$.
\end{exer}
\begin{resp}
  Observamos o problema é equivalente a calcular a área entre os gráficos das funções $f(x)=x^2$ e $g(x)=x$ restritas no intervalo $[0,2]$. Como $g(x)\geq f(x)$ no intervalo $[0,1]$ e $f(x)\geq g(x)$ no intervalo $[1,2]$, temos
  \begin{align}
    A &= \int_0^1 g(x)-f(x)\,dx + \int_1^2 f(x)-g(x)\,dx\\
      &= \int_0^1 x-x^2\,dx + \int_1^2 x^2-x\,dx\\
      &= \left[\frac{x^2}{2}-\frac{x^3}{3}\right]_0^1 + \left[\frac{x^3}{3}-\frac{x^2}{2}\right]_1^2\\
      &= 1
  \end{align}
\end{resp}

\emconstrucao

\section{Volumes por fatiamento e rotação}\label{cap_apint_sec_volfat}

\emconstrucao

\subsection*{Exercícios resolvidos}

\emconstrucao

\subsection*{Exercícios}

\emconstrucao