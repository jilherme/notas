%Este trabalho está licenciado sob a Licença Atribuição-CompartilhaIgual 4.0 Internacional Creative Commons. Para visualizar uma cópia desta licença, visite http://creativecommons.org/licenses/by-sa/4.0/deed.pt_BR ou mande uma carta para Creative Commons, PO Box 1866, Mountain View, CA 94042, USA.

\chapter{Introdução}\label{cap_intro}
\thispagestyle{fancy}

\hl{Uma rede neural artificial é um modelo de aprendizagem profunda (\emph{deep learning})}, uma área da aprendizagem de máquina (\emph{machine learning}). O termo tem origem no início dos desenvolvimentos de inteligência artificial, em que modelos matemáticos e computacionais foram inspirados no cérebro biológico (tanto de humanos como de outros animais). Muitas vezes desenvolvidos com o objetivo de compreender o funcionamento do cérebro, também tinham a intensão de emular a inteligência.

Nestas notas de aula, estudamos um dos modelos de redes neurais usualmente aplicados. A \hl{unidade básica de processamento} data do modelo de neurônio de McCulloch-Pitts (McCulloch and Pitts, 1943), conhecido como \hl{\emph{perceptron}} (Rosenblatt, 1958, 1962), o primeiro com um algoritmo de treinamento para problemas de classificação linearmente separável. Um modelo similiar é o ADALINE (do inglês, {\it adaptive linear element}, Widrow and Hoff, 1960), desenvolvido para a predição de números reais. Pela questão histórica, vamos usar o termo \emph{perceptron} para designar a unidade básica (o neurônio), mesmo que o modelo de neurônio a ser estudado não seja restrito ao original.

\hl{Métodos de aprendizagem profunda são técnicas de treinamento (calibração) de composições em múltiplos níveis}, aplicáveis a problemas de aprendizagem de máquina que, muitas vezes, não têm relação com o cérebro ou neurônios biológicos. Um exemplo, é a rede neural que mais vamos explorar nas notas, o \hl{\emph{perceptron multicamada}} (MLP, em inglês \textit{multilayer perceptron}), \hl{um modelo de progressão (em inglês, \textit{feedfoward}) de rede profunda em que a informação é processada pela composição de camadas de perceptrons}. Embora a ideia de fazer com que a informação seja processada através da conexão de múltiplos neurônios tenha inspiração biológica, usualmente a escolha da disposição dos neurônios em uma MLP é feita por questões algorítmicas e computacionais. I.e., baseada na eficiente utilização da arquitetura dos computadores atuais.
