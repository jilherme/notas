%Este trabalho está licenciado sob a Licença Atribuição-CompartilhaIgual 4.0 Internacional Creative Commons. Para visualizar uma cópia desta licença, visite http://creativecommons.org/licenses/by-sa/4.0/ ou mande uma carta para Creative Commons, PO Box 1866, Mountain View, CA 94042, USA.


%%%%%%%%%%%%%%%%%%%%%%%%%%%%%%%%%
%   Predefinicoes
%%%%%%%%%%%%%%%%%%%%%%%%%%%%%%%%%

\newif\ifisbook         % O layout será book?
\newif\ifishtml         % O layout será html?

\newif\ifisoctave       % Códigos em octave?
\newif\ifispython       % Códigos em python
\newif\ifismaxima       % Códigos em maxima?
\newif\ifiscc           % Códigos em C/C++?

\def\tfn{config.knd}     % Arquivo que guarda as definições do tipo de saída
\def \tdata{}          % Definições do tipo de saída: book, slide ou html.

\openin1=\tfn\relax    % Leitura das definições de saída
\read1 to \tdata
\closein1

\tdata                 % Definições de saída

%%%%%%%%%%%%%%%%%%%%%%%%%%%%%%%%%
%   Opcões de Linguagem
%%%%%%%%%%%%%%%%%%%%%%%%%%%%%%%%%
\usepackage[latin1,utf8]{inputenc}
\usepackage[T1]{fontenc}
\usepackage[portuguese]{babel}

% %%%%%%%%%%%%%%%%%%%%%%%%%%%%%%%%%
% %   Bibliografia
% %%%%%%%%%%%%%%%%%%%%%%%%%%%%%%%%%
% \bibliographystyle{abbrv}

%%%% copy and paste from PDF (correctly) %%%%
\usepackage{lmodern}
\usepackage{textcomp}

% %%%%%%%%%%%%%%%%%%%%%%%%%%%%%%%%%
% %   Parágrafos e indentação
% %%%%%%%%%%%%%%%%%%%%%%%%%%%%%%%%%
% \setlength{\parindent}{0pt}
% \nonzeroparskip
\usepackage[skip=1em, indent=0em]{parskip}

%%%% ams-latex %%%%
%\usepackage[fleqn]{amsmath}
\usepackage{amsmath}
\usepackage{amssymb}
\usepackage{amsthm}

%%% quotes %%%
\usepackage{upquote}

%%% bold symbols %%%
\usepackage{bm}

%%% landscape environment%%%
\usepackage{lscape}

%%%% float H option%%%%
\usepackage{float}

%%%% graphics %%%%
\usepackage[export]{adjustbox}
\usepackage{graphicx}
\usepackage{caption}
\usepackage{subcaption}

%%%% links %%%%
\usepackage[hyphens,spaces,obeyspaces]{url}
\usepackage[pdfborder={0 0 0 [0 0]},colorlinks=true,linkcolor=blue,citecolor=blue,filecolor=blue,urlcolor=blue]{hyperref}

%%% notas %%%
\usepackage{endnotes}


% colors
\usepackage[dvipsnames]{xcolor}

%%%% code insert (verbatim) %%%%
\usepackage{verbatim}
% \definecolor{backcolour}{rgb}{0.95,0.95,0.92}
\definecolor{backcolour}{rgb}{0.969, 0.969, 0.969}
\usepackage{listings}
\renewcommand{\lstlistingname}{Código}% Listing -> Código
\renewcommand{\lstlistlistingname}{Lista de \lstlistingname s}% List of Listings -> Lista of Códigos
%%% listing %%%
\lstset { %
  belowskip=0em,
  backgroundcolor=\color{backcolour},
  keywordstyle=\color{blue},
  linewidth=\textwidth,
  numbers=left,
  numberstyle=\small\color{gray},
  numbersep=3pt,
  stepnumber=1,
  firstnumber=1,
  xleftmargin=1em,
  framexleftmargin=0in,
  framexrightmargin=0in,
  % frame=single,
  extendedchars=true,
  inputencoding=utf8,
  basicstyle=\ttfamily,
  commentstyle=\ttfamily\itshape\color{gray},
  stringstyle=\ttfamily,
  showspaces=false,
  showstringspaces=false,
  resetmargins=true,
  breaklines=true,
  breakindent=0pt,             
  keepspaces=true,
  upquote=true,
  literate      =        % Support additional characters
      {á}{{\'a}}1  {é}{{\'e}}1  {í}{{\'i}}1 {ó}{{\'o}}1  {ú}{{\'u}}1
      {Á}{{\'A}}1  {É}{{\'E}}1  {Í}{{\'I}}1 {Ó}{{\'O}}1  {Ú}{{\'U}}1
      {à}{{\`a}}1  {è}{{\`e}}1  {ì}{{\`i}}1 {ò}{{\`o}}1  {ù}{{\`u}}1
      {À}{{\`A}}1  {È}{{\'E}}1  {Ì}{{\`I}}1 {Ò}{{\`O}}1  {Ù}{{\`U}}1
      {ä}{{\"a}}1  {ë}{{\"e}}1  {ï}{{\"i}}1 {ö}{{\"o}}1  {ü}{{\"u}}1
      {Ä}{{\"A}}1  {Ë}{{\"E}}1  {Ï}{{\"I}}1 {Ö}{{\"O}}1  {Ü}{{\"U}}1
      {â}{{\^a}}1  {ê}{{\^e}}1  {î}{{\^i}}1 {ô}{{\^o}}1  {û}{{\^u}}1
      {Â}{{\^A}}1  {Ê}{{\^E}}1  {Î}{{\^I}}1 {Ô}{{\^O}}1  {Û}{{\^U}}1
      {œ}{{\oe}}1  {Œ}{{\OE}}1  {æ}{{\ae}}1 {Æ}{{\AE}}1  {ß}{{\ss}}1
      {ç}{{\c c}}1 {Ç}{{\c C}}1 {ø}{{\o}}1  {Ø}{{\O}}1   {å}{{\r a}}1
      {Å}{{\r A}}1 {ã}{{\~a}}1  {õ}{{\~o}}1 {Ã}{{\~A}}1  {Õ}{{\~O}}1
      {ñ}{{\~n}}1  {Ñ}{{\~N}}1  {¿}{{?`}}1  {¡}{{!`}}1
      {°}{{\textdegree}}1 {º}{{\textordmasculine}}1 {ª}{{\textordfeminine}}1
      % acentuação somente funciona com \lstlisting, não deve-se usar
      % \lstinputlisting      
    }


%%%% citation %%%%
\usepackage{cite}

%%%% lists %%%%
\usepackage{enumerate}

%%%% index %%%%
\usepackage{imakeidx}

%%%% miscellaneous %%%%
\usepackage{multicol}
\usepackage{multirow}
\usepackage[normalem]{ulem}
\usepackage{cancel}
% \usepackage{soulutf8}

% tables
\usepackage{booktabs}

%%%%%%%%%%%%%%%%%%%%%%%%%%%%%%%%%%%%%%%%%%%%%%%%%%
% MACROS E NOVOS COMANDOS
%%%%%%%%%%%%%%%%%%%%%%%%%%%%%%%%%%%%%%%%%%%%%%%%%%

% highlight & emphasis
\usepackage{soulutf8}

\newcommand{\hleq}[1]{\color{blue}#1}
\DeclareTextFontCommand{\emph}{\bfseries}

\newcommand{\hlemph}[1]{\hl{\textbf{#1}}}

% \usepackage{realboxes}
% \newcommand{\hlcode}[1]{\Colorbox{yellow}{#1}}

% math funs & ops
\newcommand{\arc}{\operatorname{arc}}
\newcommand{\arctg}{\operatorname{arctg}}
\newcommand{\cotg}{\operatorname{cotg}}
\newcommand{\cosec}{\operatorname{cosec}}
\newcommand{\cossec}{\operatorname{cossec}}
\newcommand{\dist}{\operatorname{dist}}
\newcommand{\ddiv}{\operatorname{div}}
\newcommand{\mmc}{\operatorname{mmc}}
\newcommand{\p}{\partial}
\newcommand{\dd}{\mathrm{d}}
\newcommand{\Dom}{\operatorname{Dom}}
\newcommand{\diag}{\operatorname{diag}}
\newcommand{\proj}{\operatorname{proj}}
\newcommand{\rank}{\operatorname{rank}}
\newcommand{\sign}{\operatorname{sign}}
\newcommand{\spn}{\operatorname{span}}
\newcommand{\sen}{\operatorname{sen}}
\newcommand{\senh}{\operatorname{senh}}

\newcommand{\tg}{\operatorname{tg}}
\newcommand{\tr}{\operatorname{tr}}

\newcommand{\elu}{\operatorname{elu}}
\newcommand{\sigmoid}{\operatorname{sigmoid}}

% FENiCS link
\newcommand{\fenics}{\href{https://fenicsproject.org/}{FEniCS}}

% FEniCSx link
\newcommand{\fenicsx}{\href{https://fenicsproject.org/}{FEniCSx}}

% GSL link
\newcommand{\gsl}{\href{https://www.gnu.org/software/gsl/}{GSL}}

% Geogebra link
\newcommand{\geogebra}{\href{https://www.geogebra.org/}{Geogebra}}

% Google
\newcommand{\google}{\href{https://www.google.com.br}{Google}}

% Google Colab
\newcommand{\colab}{\href{https://colab.google.com}{Google Colab}}

% Jupyter
\newcommand{\jupyter}{\href{https://jupyter.org/}{Jupyter}}

% Kaggle
\newcommand{\kaggle}{\href{https://www.kaggle.com/}{Kaggle}}

% Linux
\newcommand{\linux}{\href{https://www.kernel.org/}{Linux}}

% Matplotlib
\newcommand{\matplotlib}{\href{https://matplotlib.org/}{Matplotlib}}

% NumPy link
\newcommand{\numpy}{\href{https://numpy.org/}{NumPy}}

% OpenMP
\newcommand{\omp}{\href{https://www.openmp.org/}{OpenMP}}

% OpenMPI
\newcommand{\ompi}{\href{https://www.open-mpi.org/}{Open MPI}}

% Python link
\newcommand{\python}{\href{https://www.python.org}{Python}}

% PyTorch link
\newcommand{\pytorch}{\href{https://pytorch.org/}{PyTorch}}

% SciPy link
\newcommand{\scipy}{\href{https://scipy.org/}{SciPy}}

% SymPy link
\newcommand{\sympy}{\href{https://www.sympy.org}{SymPy}}

% Spyder link
\newcommand{\spyder}{\href{https://www.spyder-ide.org/}{Spyder}}


% Giants
\newcommand{\bernoulli}{\footnote{Jacob Bernoulli, 1655-1705, matemático suíço. Fonte: \href{https://pt.wikipedia.org/wiki/Jakob_Bernoulli}{Wikipédia: Jakob Bernoulli}.}}

\newcommand{\bhaskara}{\footnote{Bhaskara Akaria, 1114 - 1185, matemático e astrônomo indiano. Fonte: \href{https://pt.wikipedia.org/wiki/Bhaskara\_II}{Wikipédia: Bhaskara II}.}}

\newcommand{\boole}{\footnote{George Boole, 1815 - 1864, matemático britânico. Fonte: \href{https://pt.wikipedia.org/wiki/George_Boole}{Wikipédia: George Boole}.}}

\newcommand{\burgers}{\footnote{Jan Burgers, 1895 - 1981, físico neerlandês. Fonte: \href{https://pt.wikipedia.org/wiki/Jan_Burgers}{Wikipédia: Jan Burgers}.}}

\newcommand{\cauchy}{\footnote{Augustin-Louis Cauchy, 1789-1857, matemático francês. Fonte: \href{https://pt.wikipedia.org/wiki/Augustin-Louis_Cauchy}{Wikipédia: Augustin-Louis Cauchy}.}}

\newcommand{\cotes}{\footnote{Roger Cotes, 1682 - 1716, matemático inglês. Fonte: \href{https://pt.wikipedia.org/wiki/Roger_Cotes}{Wikipédia: Roger Cotes}.}}

\newcommand{\cramer}{\footnote{Gabriel Cramer, 1704 - 1752, matemático suíço. Fonte: \href{https://pt.wikipedia.org/wiki/Gabriel_Cramer}{Wikipédia: Gabriel Cramer}.}}

\newcommand{\descartes}{\footnote{René Descartes, 1596 - 1650, matemático e filósofo francês.  Fonte: \href{https://pt.wikipedia.org/wiki/Ren\%C3\%A9_Descartes}{Wikipédia: René Descartes}.}}

\newcommand{\dirichlet}{\footnote{Johann Peter Gustav Lejeune Dirichlet, 1805 - 1859, matemático alemão. Fonte: \href{https://pt.wikipedia.org/wiki/Johann_Peter_Gustav_Lejeune_Dirichlet}{Wikipédia: Johann Peter Gustav Lejeune Dirichlet}.}}

\newcommand{\euclides}{\footnote{Euclides de Alexandria, 300 a.C., matemático grego. Fonte: \href{https://pt.wikipedia.org/wiki/Euclides}{Wikipédia: Euclides}.}}

\newcommand{\euler}{\footnote{Leonhard Paul Euler, 1707-1783, matemático e físico suíço. Fonte: \href{https://pt.wikipedia.org/wiki/Ronald_Fisher}{Wikipédia: Ronald Fisher}.}}

\newcommand{\fibonacci}{\footnote{Leonardo Fibonacci, 1170 - 1250, matemático italiano. Fonte: \href{https://pt.wikipedia.org/wiki/Leonardo_Fibonacci}{Wikipédia: Leonardo Fibonacci}.}}

\newcommand{\fisher}{\footnote{Ronald Aylmer Fisher, 1890-1962, biólogo inglês. Fonte: \href{https://pt.wikipedia.org/wiki/Ronald_Fisher}{Wikipédia: Ronald Fisher}.}}

\newcommand{\gauss}{\footnote{Johann Carl Friedrich Gauss, 1777 - 1855, matemático alemão. Fonte: \href{https://pt.wikipedia.org/wiki/Carl_Friedrich_Gauss}{Wikipédia: Carl Friedrich Gauss}.}}

\newcommand{\green}{\footnote{George Green, 1793 - 1841, matemático britânico. Fonte: \href{https://pt.wikipedia.org/wiki/George_Green}{Wikipédia:George Green }.}}

\newcommand{\heron}{\footnote{Heron de Alexandria, 10 - 80, matemático grego. Fonte: \href{https://pt.wikipedia.org/wiki/Heron_de_Alexandria}{Wikipédia: Heron de Alexandria}.}}

\newcommand{\jacobi}{\footnote{Carl Gustav Jakob Jacobi, 1804 - 1851, matemático alemão. Fonte: \href{https://pt.wikipedia.org/wiki/Carl_Gustav_Jakob_Jacobi}{Wikipédia: Carl Gustav Jakob Jacobi}.}}

\newcommand{\lovelace}{\footnote{Augusta Ada Byron King, Condessa de Lovelace, 1815 - 1852, matemática inglesa. Fonte: \href{https://pt.wikipedia.org/wiki/Ada_Lovelace}{Wikipédia: Ada Lovelace}.}}

\newcommand{\kutta}{\footnote{Martin Wilhelm Kutta, 1867 - 1944, matemático alemão. Fonte: \href{https://pt.wikipedia.org/wiki/Martin_Wilhelm_Kutta}{Wikipédia: Martin Wilhelm Kutta}.}}

\newcommand{\lagrange}{\footnote{Joseph-Louis Lagrange, 1736 - 1813, matemático italiano. Fonte: \href{https://pt.wikipedia.org/wiki/Joseph-Louis_Lagrange}{Wikipédia: Joseph-Louis Lagrange}.}}

\newcommand{\laplace}{\footnote{Pierre-Simon Laplace, 1749 - 1827, matemático francês. Fonte: \href{https://pt.wikipedia.org/wiki/Pierre-Simon_Laplace}{Wikipédia: Pierre-Simon Laplace}.}}

\newcommand{\neumann}{\footnote{Carl Gottfried Neumann, 1832 - 1925, matemático alemão. Fonte: \href{https://pt.wikipedia.org/wiki/Carl_Neumann}{Wikipédia: Carl Neumann}.}}

\newcommand{\newton}{\footnote{Isaac Newton, 1642 - 1727, matemático, físico, astrônomo, teólogo e autor inglês. Fonte: \href{https://pt.wikipedia.org/wiki/Isaac_Newton}{Wikipédia: Isaac Newton}.}}

\newcommand{\poisson}{\footnote{Siméon Denis Poisson, 1781 - 1840, matemático francês. Fonte: \href{https://pt.wikipedia.org/wiki/Sim\%C3\%A9on_Denis_Poisson}{Wikipédia:Siméon Denis Poisson}.}}

\newcommand{\riemann}{\footnote{Georg Friedrich Bernhard Riemann, 1826 - 1866, matemático alemão. Fonte: \href{https://pt.wikipedia.org/wiki/Bernhard_Riemann}{Wikipédia: Bernhard Riemann}.}}

\newcommand{\robin}{\footnote{Victor Gustave Robin, 1855 - 1897, matemático francês. Fonte: \href{https://en.wikipedia.org/wiki/Victor_Gustave_Robin}{Wikipedia: Victor Gustave Robin}.}}

\newcommand{\rossum}{\footnote{Guido van Rossum, 1956-, matemático e programador de computadores holandês. Fonte: \href{https://pt.wikipedia.org/wiki/Guido\_van\_Rossum}{Wikipédia: Guido van Rossum}.}}

\newcommand{\runge}{\footnote{Carl David Tolmé Runge, 1856 - 1927, matemático alemão. Fonte: \href{https://pt.wikipedia.org/wiki/Carl_Runge}{Wikipédia: Carl Runge}.}}

\newcommand{\seidel}{\footnote{Philipp Ludwig von Seidel, 1821 - 1896, matemático alemão. Fonte: \href{https://pt.wikipedia.org/wiki/Philipp_Ludwig_von_Seidel}{Wikipédia: Philipp Ludwig von Seidel}.}}

\newcommand{\simpson}{\footnote{Thomas Simpson, 1710 - 1761, matemático britânico. Fonte: \href{https://pt.wikipedia.org/wiki/Thomas_Simpson}{Wikipédia: Thomas Simpson}.}}

\newcommand{\schwarz}{\footnote{Karl Hermann Amandus Schwarz, 1843-1921, matemático alemão. Fonte: \href{https://pt.wikipedia.org/wiki/Hermann_Amandus_Schwarz}{Wikipédia: Hermann Amandus Schwarz}.}}

\newcommand{\taylor}{\footnote{Brook Taylor, 1685 - 1731, matemático britânico. Fonte: \href{https://pt.wikipedia.org/wiki/Brook_Taylor}{Wikipédia:Brook Taylor}.}}

\newcommand{\vandermonde}{\footnote{Alexandre-Théophile Vandermonde, 1735 - 1796, matemático francês. Fonte: \href{https://pt.wikipedia.org/wiki/Alexandre-The\%C3\%B3phile_Vandermonde}{Wikipédia: Alexandre-Theóphile Vandermonde}.}}

\newcommand{\vonNeumann}{\footnote{John von Neumann, 1903 - 1957, matemático húngaro, naturalizado estadunidense. Fonte: \href{https://pt.wikipedia.org/wiki/John_von_Neumann}{Wikipédia: John von Neumann}.}}


%E = 10^
\def\E#1{\mathrm{e}\!#1\!}
\def\e#1{\mathrm{e}\!#1\!}
%NaN
\def\NaN{\mathrm{NaN}\!}

%%%%%%%%%%%%%%%%%%%%%%%%%%%%%%%%%%%%%%%%%%%%%%%%%%

\newcommand{\emconstrucao}{\vspace{0.25cm}Em construção ...\vspace{0.25cm}}

%%%%%%%%%%%%%%%%%%%%%%%%%%%%%%%%%%%%%%%%%%%%%%%%%%
\theoremstyle{plain}
\newtheorem{teo}{Teorema}[section]
\newtheorem{teorema}{Teorema}[section]
\newtheorem{lem}{Lema}[section]
\newtheorem{lema}{Lema}[section]
\newtheorem{prop}{Proposição}[section]
\newtheorem{proposicao}{Proposição}[section]
\newtheorem{corol}{Corolário}[section]
\newtheorem{corolario}{Corolário}[section]
\theoremstyle{definition}
\newtheorem{defn}{Definição}[section]
\newtheorem{definicao}{Definição}[section]
\newtheorem{obs}{Observação}[section]
\newtheorem{observacao}{Observação}[section]

\newtheorem{ex}{Exemplo}[section]

\newenvironment{dem}{\begin{proof}}{\end{proof}}
\newenvironment{demonstracao}{\begin{proof}}{\end{proof}}


% exercícios para minicursos (document class article)
\newtheorem{exr}{Exercício}[subsection]

%Exercícios Resolvidos

\newtheorem{exeresol}{ER}[section]

\newcommand{\exerref}[1]{E.\ref{#1}}
\newcommand{\exeresolref}[1]{ER.\ref{#1}}


% %%%%%%%%%%%%%%%%%%%%%%%%%%%%%%%%%%%%%%%%%%%%%%%%%%
% % + INTRUCOES PARA O FORMATO EPUB
% %%%%%%%%%%%%%%%%%%%%%%%%%%%%%%%%%%%%%%%%%%%%%%%%%%
% \ifisepub

% % allow page breaks in equation enviroments
% \allowdisplaybreaks

% % badges
% \usepackage[most]{tcolorbox}
% \newtcbox{\badge}[1][red]{
%   on line, 
%   arc=2pt,
%   colback=#1,
%   colframe=#1,
%   fontupper=\color{black},
%   % boxrule=1pt, 
%   % boxsep=0pt,
%   left=3pt,
%   right=3pt,
%   top=2pt,
%   bottom=2pt
% }

% \newcommand{\badgeConstrucao}{\badge[yellow]{Em construção}}
% \newcommand{\badgeRevisar}{\badge[yellow]{Em revisão}}
% \newcommand{\badgeRemover}{\badge[red]{Em remoção}}


% %%%%%%%%%%%%%%%%%%%%%%%%%%%%%%%
% %%%% Exercises and Answers %%%%

% % exerícios para notas de aula
% \usepackage[answerdelayed,lastexercise]{exercise}
% \renewcommand{\ExerciseName}{E.}
% \usepackage{chngcntr}
% \counterwithin{Exercise}{section}
% \counterwithin{Answer}{section}
% \renewcommand{\ExerciseHeader}{\noindent\textbf{\ExerciseName\ExerciseHeaderNB.}~}
% \renewcommand{\AnswerHeader}{\textbf{\ExerciseName\ExerciseHeaderNB.}~}
% \newenvironment{exer}{\begin{Exercise}}{\end{Exercise}}
% \newenvironment{resp}{\begin{Answer}}{\end{Answer}}

% % %%%%%%%%%%%%%%%%%%%%%%%%%%%%%%


% \newenvironment{resol}
% {{\bfseries Solução.}}
% {
%   \begin{flushright}
%     $\Diamond$
%   \end{flushright}
% }

% \fi
% %%%%%%%%%%%%%%%%%%%%%%%%%%%%%%%%%%%%%%%%%%%%%%%%%%

%%%%%%%%%%%%%%%%%%%%%%%%%%%%%%%%%%%%%%%%%%%%%%%%%%
% + INTRUCOES PARA O FORMATO PDF
%%%%%%%%%%%%%%%%%%%%%%%%%%%%%%%%%%%%%%%%%%%%%%%%%%
\ifisbook
\input ../preambulo_book.tex
\fi
%%%%%%%%%%%%%%%%%%%%%%%%%%%%%%%%%%%%%%%%%%%%%%%%%%

%%%%%%%%%%%%%%%%%%%%%%%%%%%%%%%%%%%%%%%%%%%%%%%%%%
% + INTRUCOES PARA O FORMATO HTML
%%%%%%%%%%%%%%%%%%%%%%%%%%%%%%%%%%%%%%%%%%%%%%%%%%
\ifishtml
\input ../preambulo_html.tex
\fi
%%%%%%%%%%%%%%%%%%%%%%%%%%%%%%%%%%%%%%%%%%%%%%%%%%