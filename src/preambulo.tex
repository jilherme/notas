%Este trabalho está licenciado sob a Licença Atribuição-CompartilhaIgual 4.0 Internacional Creative Commons. Para visualizar uma cópia desta licença, visite http://creativecommons.org/licenses/by-sa/4.0/ ou mande uma carta para Creative Commons, PO Box 1866, Mountain View, CA 94042, USA.


%%%%%%%%%%%%%%%%%%%%%%%%%%%%%%%%%
%   Predefinicoes
%%%%%%%%%%%%%%%%%%%%%%%%%%%%%%%%%

\newif\ifisbook         % O layout será book?
\newif\ifishtml         % O layout será html?

\newif\ifisoctave       % Códigos em octave?
\newif\ifispython       % Códigos em python
\newif\ifismaxima       % Códigos em maxima?
\newif\ifiscc           % Códigos em C/C++?

\def\tfn{config.knd}     % Arquivo que guarda as definições do tipo de saída
\def \tdata{}          % Definições do tipo de saída: book, slide ou html.

\openin1=\tfn\relax    % Leitura das definições de saída
\read1 to \tdata
\closein1

\tdata                 % Definições de saída

%%%%%%%%%%%%%%%%%%%%%%%%%%%%%%%%%
%   Opcões de Linguagem
%%%%%%%%%%%%%%%%%%%%%%%%%%%%%%%%%
\usepackage[utf8]{inputenc}
\usepackage[T1]{fontenc}
\usepackage[portuguese]{babel}
%\usepackage[fixlanguage]{babelbib}

%%%%%%%%%%%%%%%%%%%%%%%%%%%%%%%%%
%   Bibliografia
%%%%%%%%%%%%%%%%%%%%%%%%%%%%%%%%%
\bibliographystyle{abbrv}

%%%% copy and paste from PDF (correctly) %%%%
\usepackage{upquote}
\usepackage{lmodern}
\usepackage{textcomp}

%%%% comma as a decimal separator %%%%
\usepackage{icomma}

% %%%%%%%%%%%%%%%%%%%%%%%%%%%%%%%%%
% %   Parágrafos sem indentação
% %%%%%%%%%%%%%%%%%%%%%%%%%%%%%%%%%
% \newlength\tindent
% \setlength{\tindent}{\parindent}
\usepackage{parskip}
\usepackage{indentfirst}
\setlength{\parindent}{1em}
% \renewcommand{\indent}{\hspace*{\tindent}}

\usepackage{fancyhdr}
\pagestyle{fancy}
\fancyhead[R]{\thepage}
%\renewcommand{\headrulewidth}{0pt}
%\fancyfoot[RE,RO]{\thepage}
\fancyfoot[C]{{\small \href{https://notaspedrok.com.br}{Notas de Aula - Pedro Konzen} */* \href{https://creativecommons.org/licenses/by-sa/4.0/deed.pt_BR}{Licença CC-BY-SA 4.0}}}

%%%% no blank pages between chapters %%%%
\let\cleardoublepage\clearpage

%%%% ams-latex %%%%
\usepackage[fleqn]{amsmath}
\usepackage{amssymb}
\usepackage{amsthm}

%%% bold symbols %%%
\usepackage{bm}

% \usepackage{mathtools}
\usepackage{bbding}

%%% landscape environment%%%
\usepackage{lscape}

%%%% float H option%%%%
\usepackage{float}

%%%% graphics %%%%
\usepackage{graphicx}
\usepackage{caption}
\usepackage{subcaption}

%\usepackage{graphics}
%\usepackage{caption}

%%%% links %%%%
\usepackage[hyphens,spaces,obeyspaces]{url}
\usepackage[pdfborder={0 0 0 [0 0]},colorlinks=true,linkcolor=blue,citecolor=blue,filecolor=blue,urlcolor=blue]{hyperref}

% colors
\usepackage{xcolor}


%%%% code insert (verbatim) %%%%
\usepackage{verbatim}
\usepackage{listings, lstautogobble}
\lstset { %
  keywordstyle=\color{blue},
  linewidth=\textwidth,
  numbers=left,
  numberstyle=\small,
  stepnumber=1,    
  firstnumber=1,
  numberfirstline=true,
  extendedchars=true,
  inputencoding=utf8,
  upquote=true,
  basicstyle=\small\ttfamily,
  commentstyle=\ttfamily\itshape\color{gray},
  stringstyle=\ttfamily,
  showspaces=false,
  showstringspaces=false,
  showtabs=false,
  autogobble=true,
literate      =        % Support additional characters
      {á}{{\'a}}1  {é}{{\'e}}1  {í}{{\'i}}1 {ó}{{\'o}}1  {ú}{{\'u}}1
      {Á}{{\'A}}1  {É}{{\'E}}1  {Í}{{\'I}}1 {Ó}{{\'O}}1  {Ú}{{\'U}}1
      {à}{{\`a}}1  {è}{{\`e}}1  {ì}{{\`i}}1 {ò}{{\`o}}1  {ù}{{\`u}}1
      {À}{{\`A}}1  {È}{{\'E}}1  {Ì}{{\`I}}1 {Ò}{{\`O}}1  {Ù}{{\`U}}1
      {ä}{{\"a}}1  {ë}{{\"e}}1  {ï}{{\"i}}1 {ö}{{\"o}}1  {ü}{{\"u}}1
      {Ä}{{\"A}}1  {Ë}{{\"E}}1  {Ï}{{\"I}}1 {Ö}{{\"O}}1  {Ü}{{\"U}}1
      {â}{{\^a}}1  {ê}{{\^e}}1  {î}{{\^i}}1 {ô}{{\^o}}1  {û}{{\^u}}1
      {Â}{{\^A}}1  {Ê}{{\^E}}1  {Î}{{\^I}}1 {Ô}{{\^O}}1  {Û}{{\^U}}1
      {œ}{{\oe}}1  {Œ}{{\OE}}1  {æ}{{\ae}}1 {Æ}{{\AE}}1  {ß}{{\ss}}1
      {ç}{{\c c}}1 {Ç}{{\c C}}1 {ø}{{\o}}1  {Ø}{{\O}}1   {å}{{\r a}}1
      {Å}{{\r A}}1 {ã}{{\~a}}1  {õ}{{\~o}}1 {Ã}{{\~A}}1  {Õ}{{\~O}}1
      {ñ}{{\~n}}1  {Ñ}{{\~N}}1  {¿}{{?`}}1  {¡}{{!`}}1
      {°}{{\textdegree}}1 {º}{{\textordmasculine}}1 {ª}{{\textordfeminine}}1
      % acentuação somente funciona com \lstlisting, não deve-se usar
      % \lstinputlisting      
    }
\renewcommand{\lstlistingname}{Código}% Listing -> Código
\renewcommand{\lstlistlistingname}{Lista de \lstlistingname s}% List of Listings -> Lista of Códigos


%%%% citation %%%%
\usepackage{cite}

%%%% lists %%%%
\usepackage{enumerate}

%%%% index %%%%
\usepackage{makeidx}


%%%% miscellaneous %%%%
\usepackage{multicol}
\usepackage{multirow}
\usepackage[normalem]{ulem}
\usepackage{cancel}
\usepackage{soulutf8}

% tables
\usepackage{booktabs}

%%%%%%%%%%%%%%%%%%%%%%%%%%%%%%%%%%%%%%%%%%%%%%%%%%
% MACROS E NOVOS COMANDOS
%%%%%%%%%%%%%%%%%%%%%%%%%%%%%%%%%%%%%%%%%%%%%%%%%%

% % highlight
\usepackage[customcolors]{hf-tikz}
\newcommand{\hleq}[1]{\color{blue}#1}
%\newcommand{\hl}[1]{\colorbox{yellow}{#1}}

%emphasis \emph
\DeclareTextFontCommand{\emph}{\bfseries}
\newcommand{\hlemph}[1]{\hl{\textbf{#1}}}

\newcommand{\arc}{\operatorname{arc}}
\newcommand{\arctg}{\operatorname{arctg}}
\newcommand{\cotg}{\operatorname{cotg}}
\newcommand{\cosec}{\operatorname{cosec}}
\newcommand{\cossec}{\operatorname{cossec}}
\newcommand{\dist}{\operatorname{dist}}
\newcommand{\ddiv}{\operatorname{div}}
\newcommand{\mmc}{\operatorname{mmc}}
\newcommand{\p}{\partial}
\newcommand{\dd}{\mathrm{d}}
\newcommand{\Dom}{\operatorname{Dom}}
\newcommand{\diag}{\operatorname{diag}}
\newcommand{\proj}{\operatorname{proj}}
\newcommand{\rank}{\operatorname{rank}}
\newcommand{\sign}{\operatorname{sign}}
\newcommand{\spn}{\operatorname{span}}
\newcommand{\sen}{\operatorname{sen}}
\newcommand{\senh}{\operatorname{senh}}

\newcommand{\tg}{\operatorname{tg}}
\newcommand{\tr}{\operatorname{tr}}

\newcommand{\elu}{\operatorname{elu}}
\newcommand{\sigmoid}{\operatorname{sigmoid}}

% FENiCS link
\newcommand{\fenics}{\href{https://fenicsproject.org/}{FEniCS}}

% GSL link
\newcommand{\gsl}{\href{https://www.gnu.org/software/gsl/}{GSL}}

% Geogebra link
\newcommand{\geogebra}{\href{https://www.geogebra.org/}{Geogebra}}

% Google
\newcommand{\google}{\href{https://www.google.com.br}{Google}}

% Google Colab
\newcommand{\colab}{\href{https://colab.google.com}{Google Colab}}

% Jupyter
\newcommand{\jupyter}{\href{https://jupyter.org/}{Jupyter}}

% Kaggle
\newcommand{\kaggle}{\href{https://www.kaggle.com/}{Kaggle}}

% Linux
\newcommand{\linux}{\href{https://www.kernel.org/}{Linux}}

% Matplotlib
\newcommand{\matplotlib}{\href{https://matplotlib.org/}{Matplotlib}}

% NumPy link
\newcommand{\numpy}{\href{https://numpy.org/}{NumPy}}

% OpenMP
\newcommand{\omp}{\href{https://www.openmp.org/}{OpenMP}}

% OpenMPI
\newcommand{\ompi}{\href{https://www.open-mpi.org/}{Open MPI}}

% Python link
\newcommand{\python}{\href{https://www.python.org}{Python}}

% PyTorch link
\newcommand{\pytorch}{\href{https://pytorch.org/}{PyTorch}}

% SciPy link
\newcommand{\scipy}{\href{https://scipy.org/}{SciPy}}

% SymPy link
\newcommand{\sympy}{\href{https://www.sympy.org}{SymPy}}

% Spyder link
\newcommand{\spyder}{\href{https://www.spyder-ide.org/}{Spyder}}


% Giants
\newcommand{\bhaskara}{\footnote{Bhaskara Akaria, 1114 - 1185, matemático e astrônomo indiano. Fonte: \href{https://pt.wikipedia.org/wiki/Bhaskara\_II}{Wikipédia}.}}

\newcommand{\boole}{\footnote{George Boole, 1815 - 1864, matemático britânico. Fonte: \href{https://pt.wikipedia.org/wiki/George_Boole}{Wikipédia}.}}

\newcommand{\cauchy}{\footnote{Augustin-Louis Cauchy, 1789-1857, matemático francês. Fonte: \href{https://pt.wikipedia.org/wiki/Augustin-Louis_Cauchy}{Wikipédia}.}}

\newcommand{\cotes}{\footnote{Roger Cotes, 1682 - 1716, matemático inglês. Fonte: \href{https://pt.wikipedia.org/wiki/Roger_Cotes}{Wikipédia}.}}

\newcommand{\cramer}{\footnote{Gabriel Cramer, 1704 - 1752, matemático suíço. Fonte: \href{https://pt.wikipedia.org/wiki/Gabriel_Cramer}{Wikipédia}.}}

\newcommand{\descartes}{\footnote{René Descartes, 1596 - 1650, matemático e filósofo francês.  Fonte: \href{https://pt.wikipedia.org/wiki/Ren\%C3\%A9_Descartes}{Wikipédia}.}}

\newcommand{\euclides}{\footnote{Euclides de Alexandria, 300 a.C., matemático grego. Fonte: \href{https://pt.wikipedia.org/wiki/Euclides}{Wikipédia}.}}

\newcommand{\euler}{\footnote{Leonhard Paul Euler, 1707-1783, matemático e físico suíço. Fonte: \href{https://pt.wikipedia.org/wiki/Leonhard_Euler}{Wikipédia}.}}

\newcommand{\fibonacci}{\footnote{Leonardo Fibonacci, 1170 - 1250, matemático italiano. Fonte: \href{https://pt.wikipedia.org/wiki/Leonardo_Fibonacci}{Wikipédia}.}}

\newcommand{\gauss}{\footnote{Johann Carl Friedrich Gauss, 1777 - 1855, matemático alemão. Fonte: \href{https://pt.wikipedia.org/wiki/Carl_Friedrich_Gauss}{Wikipédia}.}}

\newcommand{\jacobi}{\footnote{Carl Gustav Jakob Jacobi, 1804 - 1851, matemático alemão. Fonte: \href{https://pt.wikipedia.org/wiki/Carl_Gustav_Jakob_Jacobi}{Wikipédia}.}}

\newcommand{\kutta}{\footnote{Martin Wilhelm Kutta, 1867 - 1944, matemático alemão. Fonte: \href{https://pt.wikipedia.org/wiki/Martin_Wilhelm_Kutta}{Wikipédia}.}}

\newcommand{\lagrange}{\footnote{Joseph-Louis Lagrange, 1736 - 1813, matemático italiano. Fonte: \href{https://pt.wikipedia.org/wiki/Joseph-Louis_Lagrange}{Wikipédia}.}}

\newcommand{\laplace}{\footnote{Pierre-Simon Laplace, 1749 - 1827, matemático francês. Fonte: \href{https://pt.wikipedia.org/wiki/Pierre-Simon_Laplace}{Wikipédia}.}}

\newcommand{\neumann}{\footnote{Carl Gottfried Neumann, 1832 - 1925, matemático alemão. Fonte: \href{https://pt.wikipedia.org/wiki/Carl_Neumann}{Wikipédia}.}}

\newcommand{\newton}{\footnote{Isaac Newton, 1642 - 1727, matemático, físico, astrônomo, teólogo e autor inglês. Fonte: \href{https://pt.wikipedia.org/wiki/Isaac_Newton}{Wikipédia}.}}

\newcommand{\poisson}{\footnote{Siméon Denis Poisson, 1781 - 1840, matemático francês. Fonte: \href{https://pt.wikipedia.org/wiki/Sim\%C3\%A9on_Denis_Poisson}{Wikipédia}.}}

\newcommand{\riemann}{\footnote{Georg Friedrich Bernhard Riemann, 1826 - 1866, matemático alemão. Fonte: \href{https://pt.wikipedia.org/wiki/Bernhard_Riemann}{Wikipédia}.}}

\newcommand{\runge}{\footnote{Carl David Tolmé Runge, 1856 - 1927, matemático alemão. Fonte: \href{https://pt.wikipedia.org/wiki/Carl_Runge}{Wikipédia}.}}

\newcommand{\seidel}{\footnote{Philipp Ludwig von Seidel, 1821 - 1896, matemático alemão. Fonte: \href{https://pt.wikipedia.org/wiki/Philipp_Ludwig_von_Seidel}{Wikipédia}.}}

\newcommand{\simpson}{\footnote{Thomas Simpson, 1710 - 1761, matemático britânico. Fonte: \href{https://pt.wikipedia.org/wiki/Thomas_Simpson}{Wikipédia}.}}

\newcommand{\schwarz}{\footnote{Karl Hermann Amandus Schwarz, 1843-1921, matemático alemão. Fonte: \href{https://pt.wikipedia.org/wiki/Hermann_Amandus_Schwarz}{Wikipédia}.}}

\newcommand{\taylor}{\footnote{Brook Taylor, 1685 - 1731, matemático britânico. Fonte: \href{https://en.wikipedia.org/wiki/Brook_Taylor}{Wikipédia}.}}

\newcommand{\vandermonde}{\footnote{Alexandre-Théophile Vandermonde, 1735 - 1796, matemático francês. Fonte: \href{https://en.wikipedia.org/wiki/Alexandre-Th\%C3\%A9ophile_Vandermonde}{Wikipédia}.}}

%E = 10^
\def\E#1{\mathrm{e}\!#1\!}
\def\e#1{\mathrm{e}\!#1\!}
%NaN
\def\NaN{\mathrm{NaN}\!}

%%%%%%%%%%%%%%%%%%%%%%%%%%%%%%%%%%%%%%%%%%%%%%%%%%

\newcommand{\emconstrucao}{\vspace{0.25cm}Em construção ...\vspace{0.25cm}}

%%%%%%%%%%%%%%%%%%%%%%%%%%%%%%%%%%%%%%%%%%%%%%%%%%
\theoremstyle{plain}
\newtheorem{teo}{Teorema}[section]
\newtheorem{lem}{Lema}[section]
\newcommand{\lema}{\lem}
\newtheorem{prop}{Proposição}[section]
\newtheorem{corol}{Corolário}[section]
\theoremstyle{definition}
\newtheorem{defn}{Definição}[section]
\newtheorem{obs}{Observação}[section]

\newtheorem{ex}{Exemplo}[section]

% exercícios para minicursos (document class article)
\newtheorem{exr}{Exercício}[subsection]

\newenvironment{dem}{\begin{proof}}{\end{proof}}


%Exercícios Resolvidos

\newtheorem{exeresol}{ER}[section]

% font awesome
\usepackage{fontawesome5}


%%%%%%%%%%%%%%%%%%%%%%%%%%%%%%%%%%%%%%%%%%%%%%%%%%
% + INTRUCOES PARA O FORMATO LIVRO
%%%%%%%%%%%%%%%%%%%%%%%%%%%%%%%%%%%%%%%%%%%%%%%%%%
\ifisbook
\input ../preambulo_book.tex
\fi
%%%%%%%%%%%%%%%%%%%%%%%%%%%%%%%%%%%%%%%%%%%%%%%%%%


%%%%%%%%%%%%%%%%%%%%%%%%%%%%%%%%%%%%%%%%%%%%%%%%%%
% + INTRUCOES PARA O FORMATO HTML
%%%%%%%%%%%%%%%%%%%%%%%%%%%%%%%%%%%%%%%%%%%%%%%%%%
\ifishtml
\input ../preambulo_html.tex
\fi
%%%%%%%%%%%%%%%%%%%%%%%%%%%%%%%%%%%%%%%%%%%%%%%%%%
