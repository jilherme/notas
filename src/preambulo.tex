%Este trabalho está licenciado sob a Licença Atribuição-CompartilhaIgual 4.0 Internacional Creative Commons. Para visualizar uma cópia desta licença, visite http://creativecommons.org/licenses/by-sa/4.0/ ou mande uma carta para Creative Commons, PO Box 1866, Mountain View, CA 94042, USA.


%%%%%%%%%%%%%%%%%%%%%%%%%%%%%%%%%
%   Predefinicoes
%%%%%%%%%%%%%%%%%%%%%%%%%%%%%%%%%

\newif\ifisbook         % O layout será book?
\newif\ifishtml         % O layout será html?

\newif\ifisoctave       % Códigos em octave?
\newif\ifispython       % Códigos em python
\newif\ifismaxima       % Códigos em maxima?
\newif\ifiscc           % Códigos em C/C++?

\def\tfn{config.knd}     % Arquivo que guarda as definições do tipo de saída
\def \tdata{}          % Definições do tipo de saída: book, slide ou html.

\openin1=\tfn\relax    % Leitura das definições de saída
\read1 to \tdata
\closein1

\tdata                 % Definições de saída

%%%%%%%%%%%%%%%%%%%%%%%%%%%%%%%%%
%   Opcões de Linguagem
%%%%%%%%%%%%%%%%%%%%%%%%%%%%%%%%%
\usepackage[brazil]{babel}
\usepackage[utf8]{inputenc}
\usepackage[T1]{fontenc}
\usepackage[fixlanguage]{babelbib}

%%%%%%%%%%%%%%%%%%%%%%%%%%%%%%%%%
%   Bibliografia
%%%%%%%%%%%%%%%%%%%%%%%%%%%%%%%%%
% \selectbiblanguage{brazil}
% \bibliographystyle{babunsrt}
\bibliographystyle{plain}


%%%% copy and paste from PDF (correctly) %%%%
\usepackage{upquote}
\usepackage{lmodern}
\usepackage{textcomp}

%%%% comma as a decimal separator %%%%
\usepackage{icomma}

% %%%%%%%%%%%%%%%%%%%%%%%%%%%%%%%%%
% %   Parágrafos sem indentação
% %%%%%%%%%%%%%%%%%%%%%%%%%%%%%%%%%
% \newlength\tindent
% \setlength{\tindent}{\parindent}
\usepackage{parskip}
\usepackage{indentfirst}
\setlength{\parindent}{1em}
% \renewcommand{\indent}{\hspace*{\tindent}}

\usepackage{fancyhdr}
\pagestyle{fancy}
\fancyhead[R]{\thepage}
%\renewcommand{\headrulewidth}{0pt}
%\fancyfoot[RE,RO]{\thepage}
\fancyfoot[C]{{\small \href{https://phkonzen.github.io/notas}{Notas de Aula - Pedro Konzen} */* \href{https://creativecommons.org/licenses/by-sa/4.0/deed.pt_BR}{Licença CC-BY-SA 4.0}}}

%%%% no blank pages between chapters %%%%
\let\cleardoublepage\clearpage

%%%% ams-latex %%%%
\usepackage{amsmath}
\usepackage{amssymb}
\usepackage{amsthm}

%%% bold symbols %%%
\usepackage{bm}

% \usepackage{mathtools}
\usepackage{bbding}

%%% landscape environment%%%
\usepackage{lscape}

%%%% float H option%%%%
\usepackage{float}

%%%% graphics %%%%
\usepackage{graphicx}

%\usepackage{graphics}
%\usepackage{caption}

%%%% links %%%%
\usepackage[hyphens,spaces,obeyspaces]{url}
\usepackage[pdfborder={0 0 0 [0 0]},colorlinks=true,linkcolor=blue,citecolor=blue,filecolor=blue,urlcolor=blue]{hyperref}


%%%% code insert (verbatim) %%%%
\usepackage{verbatim}
\usepackage{listings}
\lstset { %
  numbers=left,
  numberstyle=\small,
  stepnumber=1,    
  firstnumber=1,
  numberfirstline=true,
  extendedchars=true,
  inputencoding=utf8,
  upquote=true,
  basicstyle=\ttfamily\small,
  showspaces=false,
  showstringspaces=false,
  showtabs=false,
}

%%%% citation %%%%
\usepackage{cite}

%%%% lists %%%%
\usepackage{enumerate}

%%%% index %%%%
\usepackage{makeidx}


%%%% miscellaneous %%%%
\usepackage{multicol}
\usepackage{multirow}
\usepackage[normalem]{ulem}
\usepackage{cancel}
\usepackage{xcolor}

% \renewcommand{\arraystretch}{1.5} %space between rows in tables
% \usepackage{array,booktabs}
% \usepackage{tikz}

%emphasis \emph
\DeclareTextFontCommand{\emph}{\bfseries}

%%%%%%%%%%%%%%%%%%%%%%%%%%%%%%%%%%%%%%%%%%%%%%%%%%
% MACROS E NOVOS COMANDOS
%%%%%%%%%%%%%%%%%%%%%%%%%%%%%%%%%%%%%%%%%%%%%%%%%%
\newcommand{\mmc}{\operatorname{mmc}}
\newcommand{\arc}{\operatorname{arc}}
\newcommand{\sen}{\operatorname{sen}}
\newcommand{\senh}{\operatorname{senh}}
\newcommand{\tg}{\operatorname{tg}}
\newcommand{\arctg}{\operatorname{arctg}}
\newcommand{\cotg}{\operatorname{cotg}}
\newcommand{\cosec}{\operatorname{cosec}}
\newcommand{\cossec}{\operatorname{cossec}}
\newcommand{\p}{\partial}
\newcommand{\dd}{\mathrm{d}}
\newcommand{\Dom}{\operatorname{Dom}}
\newcommand{\diag}{\operatorname{diag}}
\newcommand{\proj}{\operatorname{proj}}
\newcommand{\dist}{\operatorname{dist}}
\newcommand{\sign}{\operatorname{sign}}
\newcommand{\spn}{\operatorname{span}}

% SumPy link
\newcommand{\sympy}{\href{https://www.sympy.org}{SymPy}}

% Python link
\newcommand{\python}{\href{https://www.python.org}{Python}}

% NumPy link
\newcommand{\numpy}{\href{https://numpy.org/}{NumPy}}

% SciPy link
\newcommand{\scipy}{\href{https://scipy.org/}{SciPy}}

% FENiCS link
\newcommand{\fenics}{\href{https://fenicsproject.org/}{FEniCS}}

% Geogebra link
\newcommand{\geogebra}{\href{https://www.geogebra.org/}{Geogebra}}

% Giants

\newcommand{\newton}{\footnote{Isaac Newton, 1642 - 1727, matemático, físico, astrônomo, teólogo e autor inglês. Fonte: \href{https://pt.wikipedia.org/wiki/Isaac_Newton}{Wikipédia}.}}

%E = 10^
\def\E#1{\mathrm{E}\!#1\!}
%NaN
\def\NaN{\mathrm{NaN}\!}

%%%%%%%%%%%%%%%%%%%%%%%%%%%%%%%%%%%%%%%%%%%%%%%%%%

\newcommand{\emconstrucao}{\vspace{0.25cm}Em construção ...\vspace{0.25cm}}

%%%%%%%%%%%%%%%%%%%%%%%%%%%%%%%%%%%%%%%%%%%%%%%%%%
\theoremstyle{plain}
\newtheorem{teo}{Teorema}[section]
\newtheorem{lem}{Lema}[section]
\newtheorem{prop}{Proposição}[section]
\newtheorem{corol}{Corolário}[section]
\theoremstyle{definition}
\newtheorem{defn}{Definição}[section]
\newtheorem{obs}{Observação}[section]
\newtheorem{ex}{Exemplo}[section]

\newenvironment{dem}{\begin{proof}}{\end{proof}}


%Exercícios Resolvidos

\newtheorem{exeresol}{ER}[section]

% font awesome
\usepackage{fontawesome}


%%%%%%%%%%%%%%%%%%%%%%%%%%%%%%%%%%%%%%%%%%%%%%%%%%
% + INTRUCOES PARA O FORMATO LIVRO
%%%%%%%%%%%%%%%%%%%%%%%%%%%%%%%%%%%%%%%%%%%%%%%%%%
\ifisbook
\input ../preambulo_book.tex
\fi
%%%%%%%%%%%%%%%%%%%%%%%%%%%%%%%%%%%%%%%%%%%%%%%%%%


%%%%%%%%%%%%%%%%%%%%%%%%%%%%%%%%%%%%%%%%%%%%%%%%%%
% + INTRUCOES PARA O FORMATO HTML
%%%%%%%%%%%%%%%%%%%%%%%%%%%%%%%%%%%%%%%%%%%%%%%%%%
\ifishtml
\input ../preambulo_html.tex
\fi
%%%%%%%%%%%%%%%%%%%%%%%%%%%%%%%%%%%%%%%%%%%%%%%%%%
