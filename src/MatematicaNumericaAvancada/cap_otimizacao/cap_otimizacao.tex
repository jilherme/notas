%Este trabalho está licenciado sob a Licença Atribuição-CompartilhaIgual 4.0 Internacional Creative Commons. Para visualizar uma cópia desta licença, visite http://creativecommons.org/licenses/by-sa/4.0/deed.pt_BR ou mande uma carta para Creative Commons, PO Box 1866, Mountain View, CA 94042, USA.

\chapter{Sistemas Não Lineares e Otimização}\label{cap_otimizacao}
\thispagestyle{fancy}

\begin{flushright}
  [Vídeo] | [Áudio] | \href{https://phkonzen.github.io/notas/contato.html}{[Contatar]}
\end{flushright}

Neste capítulo, apresentam-se métodos numéricos para a resolução de problemas de otimização. Salvo explicitado ao contrário, assume-se que os problemas são bem definidos.

\section{Método de Newton}\label{cap_otimizacao_sec_newton}

Consideremos o seguinte problema: $F:D\subset \mathbb{R}^n\to\mathbb{R}^n$ encontrar $x^*\in\mathbb{R}^n$ tal que
\begin{equation}
  F(x^*) = 0.
\end{equation}
Salvo explicitado ao contrário, assumiremos que $F\in C^1(D)$, i.e. $F$ é uma função continuamente diferenciável. Vamos, também, denotar por $J_F(x) = [\jmath_{i,j}(x)]_{i,j=1}^{n,n}$ a \emph{matriz jacobiana}\footnote{Carl Gustav Jakob Jacobi, 1804 - 1851, matemático alemão. Fonte: \href{https://pt.wikipedia.org/wiki/Carl_Gustav_Jakob_Jacobi}{Wikipédia}.} com
\begin{equation}
  \jmath_{i,j}(x) = \frac{\p f_i(x)}{\p x_j},
\end{equation}
onde $F(x) = (f_1(x), f_2(x), \dotsc, f_n(x))$ e $x = (x_1, x_2,\dotsc, x_n)$.

A iteração básica do \emph{Método de Newton}\footnote{Isaac Newton, 1642 - 1727, matemático, físico, astrônomo, teólogo e autor inglês. Fonte: \href{https://pt.wikipedia.org/wiki/Isaac_Newton}{Wikipédia}.} consiste em: dada uma aproximação inicial $x^{(0)}\in\mathbb{R}^n$,
\begin{align}
  &\text{resolver}\quad J_F\left(x^{(k)}\right)\delta^{(k)} = -F\left(x^{(k)}\right)\\
  &\text{calcular}\quad x^{(k+1)} = x^{(k)} + \delta^{(k)}
\end{align}
para $k=0,1,2,\ldots$ até convergência $x^{(k)}\to x^*$.

\begin{ex}\label{ex:burgers}
  Consideremos a \emph{Equação de Burgers}\footnote{Johannes Martinus Burgers, 1895 - 1981, físico holandês. Fonte: \href{https://en.wikipedia.org/wiki/Jan_Burgers}{Wikipedia}.}
  \begin{equation}
    \frac{\p u}{\p t} + u\frac{\p u}{\p x} = \nu\frac{\p^2 u}{\p x^2}
  \end{equation}
  com $\nu>0$, condição inicial
  \begin{equation}
    u(0,x) = \sin(\pi x)
  \end{equation}
  e condições de contorno de Dirichlet\footnote{Johann Peter Gustav Lejeune Dirichlet, 1805 - 1859, matemático alemão. Fonte: \href{https://pt.wikipedia.org/wiki/Johann_Peter_Gustav_Lejeune_Dirichlet}{Wikipédia}.} homogêneas
  \begin{equation}
    u(t,0) = u(t,1) = 0.
  \end{equation}

  Aplicando o \emph{Método de Rothe}\footnote{Erich Hans Rothe, 1895 - 1988, matemático alemão. Fonte: \href{https://pt.wikipedia.org/wiki/Erich_Rothe}{Wikipédia}.} com aproximação de Euler\footnote{Leonhard Paul Euler, 1707 - 1783, matemático e físico suíço. Fonte: \href{https://pt.wikipedia.org/wiki/Leonhard_Euler}{Wikipédia}.} implícita, obtemos
  \begin{equation}
    \frac{u(t+h_t,x) - u(t,x)}{h_t} + u(t+h_t,x)u_x(t+h_t,x) \approx \nu u_{xx}(t+h_t,x),
  \end{equation}
  onde $h_t>0$ é o passo no tempo. Agora, aplicamos diferenças finitas para obter
  \begin{gather}
    \frac{u(t+h_t,x_i) - u(t,x_i)}{h_t} + u(t+h_t, x_i)\frac{u(t+h_t,x_{i+1})-u(t+h_t,x_i)}{h_x} \\
    \approx \nu\frac{u(t+h_t,x_{i-1}) - 2u(t+h_t,x_i) + u(t+h_t,x_{i+1})}{h_x^2},
  \end{gather}
  onde, $x_i=(i-1)h_x$, $i=1,\dotsc,n_x$ e $h_x=1/(n_x-1)$ é o tamanho da malha.

  Rearranjando os termos e denotando $w^{(k)}_i\approx u(t_k, x_i)$, $t_k = (k-1)h$, obtemos o seguinte \emph{problema discreto}
  \begin{gather}
    w^{(k+1)}_1 = 0\\
    ~\nonumber\\
    \frac{1}{h_t}w^{(k+1)}_i - \frac{1}{h_t}w^{(k)}_i + \frac{1}{h_x}w^{(k+1)}_iw^{(k+1)}_{i+1} - \frac{1}{h_x}w^{(k+1)}_iw^{(k+1)}_i \\
    - \frac{\nu}{h_x^2}w^{(k+1)}_{i-1} + 2\frac{\nu}{h_x^2}w^{(k+1)}_i - \frac{\nu}{h_x}w^{(k+1)}_{i+1} = 0,\\
    ~\nonumber\\
    w^{(k+1)}_{n_x} = 0,
  \end{gather}
  sendo $w^{(0)}_i = \sen(\pi x_i)$, $i=1,\dotsc,n_x$ e $k=1,2,\ldots$.

  Este problema pode ser reescrito como segue: para cada $k=1,2,\ldots$, encontrar $v\in\mathbb{R}^{n_x}$, tal que
  \begin{equation}
    F(v; v^{(0)}) = 0, 
  \end{equation}
  onde $v_{i} \approx w^{(k+1)}_i$, $v_{i}^{(0)} \approx w^{(k)}_{i}$ e
  \begin{gather}
    f_1(v,v^{(0)}) = v_1,\\
    f_{i}(v;v^{(0)}) = \frac{1}{h_t}v_i - \frac{1}{h_t}v^{(0)}_i + \frac{1}{h_x}v_iv_{i+1} - \frac{1}{h_x}v_iv_i \\
    - \frac{\nu}{h_x^2}v_{i-1} + 2\frac{\nu}{h_x^2}v_i - \frac{\nu}{h_x}v_{i+1},\\
    f_{n_x}(v;v^{(0)}) = v_{n_x}.
  \end{gather}
  A matriz jacobiana associada $J=[\jmath_{i,j}]_{i,j}^{n_x,n_x}$ contém
  \begin{gather}
    \jmath_{i,j} = 0,\quad j\neq i-1,i,i+1,\\
    \jmath_{1,1} = 1,\\
    \jmath_{1,2} = 0,\\
    \jmath_{i,i-1} = -\frac{\nu}{h_x^2},\\
    \jmath_{i,i} = \frac{1}{h_t} + \frac{1}{h_x}v_{i+1} - \frac{2}{h_x}v_i + \frac{2\nu}{h_x^2},\\
    \jmath_{i,i+1} = \frac{1}{h_x}v_{i} - \frac{\nu}{h_x^2},\\
    \jmath_{n_x,n_x-1} = 0\\
    \jmath_{n_x,n_x} = 1.
  \end{gather}
\end{ex}

\begin{exer}\label{exer:burgers}
  Considere o problema discreto apresentado no Exemplo \ref{ex:burgers} para diferentes valores do coeficiente de difusão $\nu=1., 0.5, 0.1, 0.01, 0.001$. Simule o problema com cada uma das seguintes estratégias e as compare quanto ao desempenho computacional.
  \begin{enumerate}[a)]
  \item Simule-o aplicando o Método de Newton com o {\it solver} linear \lstinline+npla.solve+.
  \item Observe que a jacobiana é uma matriz tridiagonal. Simule o problema aplicando o Método de Newton com o {\it solver} linear \lstinline+npla.solve_banded+.
  \item Aloque a jacobiana como uma matriz esparsa. Então, simule o problema aplicando o Método de Newton com {\it solver} linear adequado para matrizes esparsas.
  \end{enumerate}
\end{exer}
