%Este trabalho está licenciado sob a Licença Atribuição-CompartilhaIgual 4.0 Internacional Creative Commons. Para visualizar uma cópia desta licença, visite http://creativecommons.org/licenses/by-sa/4.0/deed.pt_BR ou mande uma carta para Creative Commons, PO Box 1866, Mountain View, CA 94042, USA.

\chapter{EDO linear de ordem 2 ou mais alta}\label{cap_edolin}
\thispagestyle{fancy}

\begin{flushright}
  [Vídeo] | [Áudio] | \href{https://phkonzen.github.io/notas/contato.html}{[Contatar]}
\end{flushright}

Neste capítulo, discutimos sobre EDOs lineares, as quais podem ser escritas na seguinte forma
\begin{equation}
  p_n(x)y^{(n)} + p_{n-1}(x)y^{(n-1)} + \cdots + p_0(x)y = g(x),
\end{equation}
sendo $y = y(x)$, $p_n(x)\not\equiv 0$ e $n\geq 2$.

\section{EDO de ordem 2: fundamentos}\label{cap_edolin_sec_edo2I}

\begin{flushright}
  \href{https://archive.org/details/introducao-edo-ordem-2}{[Vídeo]} | [Áudio] | \href{https://phkonzen.github.io/notas/contato.html}{[Contatar]}
\end{flushright}

Nesta seção, vamos nos restringir a \emph{EDOs lineares de segunda ordem, homogêneas e com coeficientes constantes}, i.e. EDOs da forma
\begin{equation}\label{eq:edo2_hom_coefconst}
  {\color{blue}ay'' + by' + cy = 0},
\end{equation}
onde $y = y(t)$ e $a, b, c$ são parâmetros constantes (números reais).

Vamos buscar por soluções da forma ${\color{blue}y(t) = e^{rt}}$, onde $r$ é constante. Substituindo na equação \eqref{eq:edo2_hom_coefconst}, obtemos
\begin{align}
  & a\left(e^{rt}\right)'' + b\left(e^{rt}\right)' + ce^{rt} = 0 \\
  & \Rightarrow ar^2e^{rt} + bre^{rt} + ce^{rt} = 0 \\
  & \Rightarrow {\color{blue}(ar^2 + br + c)e^{rt} = 0}.
\end{align}
Ou seja, $y(t) = e^{rt}$ é solução de \eqref{eq:edo2_hom_coefconst} quando
\begin{equation}
  {\color{blue}ar^2 + br + c = 0}.
\end{equation}
Esta é chamada de \emph{equação característica} de \eqref{eq:edo2_hom_coefconst}.

\begin{ex}\label{ex:ed2o_homcc}
  Vamos buscar por soluções de
  \begin{equation}\label{eq:ex_ed2o_hom_coefconst1}
    y'' - 4y = 0.
  \end{equation}
  Buscamos por $r$ tal que $y(t) = e^{rt}$ seja solução desta equação. Substituindo na equação, obtemos
  \begin{align}
    \left(e^{rt}\right)'' - 4e^{rt} = 0 \Rightarrow (r^2 - 4)e^{rt} = 0
  \end{align}
  o que nos fornece a equação característica
  \begin{equation}
    r^2 - 4 = 0.
  \end{equation}
  As soluções desta equação são $r_1 = -2$ e $r_2 = 2$. Ou seja, obtemos as seguintes \emph{soluções particulares} da EDO
  \begin{equation}
    y_1(t) = e^{-2t}\quad\text{e}\quad y_2(t) = e^{2t}.
  \end{equation}

  Observamos, ainda, que para quaisquer constantes $c_1$ e $c_2$,
  \begin{equation}
    y(t) = c_1e^{-2t} + c_2e^{2t}
  \end{equation}
  também é solução da EDO \eqref{eq:ex_ed2o_hom_coefconst1}. De fato, temos
  \begin{align}
    y'' - 4y &= \left(c_1e^{-2t} + c_2e^{2t}\right)'' -4\left(c_1e^{-2t}+c_2e^{2t}\right) \\
    &= 4c_1e^{-2t} + 4c_2e^{2t} -4c_1e^{-2t} -4c_2e^{2t} \\
    &= 0.
  \end{align}

  Como veremos logo mais,
  \begin{equation}
    y(t) = c_1e^{-2t} + c_2e^{2t}
  \end{equation}
  é a \emph{solução geral} de \eqref{eq:ex_ed2o_hom_coefconst1}.
\end{ex}

\subsection{Conjunto fundamental de soluções}

\begin{flushright}
  \href{https://archive.org/details/wronskiano-edo-de-ordem-2}{[Vídeo]} | [Áudio] | \href{https://phkonzen.github.io/notas/contato.html}{[Contatar]}
\end{flushright}

Sejam $y_1 = y_1(t)$ e $y_2 = y_2(t)$ soluções de
\begin{equation}\label{eq:ed2o_homcc}
  ay'' + by' + cy = 0.
\end{equation}
Então,
\begin{equation}
  y(t) = c_1y_1(t) + c_2y_2(t)
\end{equation}
também é solução de \eqref{eq:ed2o_homcc}.

De fato, basta verificar que
\begin{align}
  ay'' + by' + cy &= a(c_1y_1+c_2y_2)'' \\
  &+ b(c_1y_1 + c_2y_2)' \\
  &+ c(c_1y_1 + c_2y_2) \\
  &= c_1(ay_1'' + by_1' + cy_1) \\
  &+ c_2(ay_2'' + by_2' + cy_2) \\
  &= 0.
\end{align}

Suponhamos, ainda, que as soluções $y_1 = y_1(t)$ e $y_2 = y_2(t)$ são tais que o chamado \emph{wronskiano}
\begin{equation}
  {\color{blue}W(y_1,y_2;t) := \left|
    \begin{array}{cc}
      y_1 & y_2 \\
      y_1' & y_2'
    \end{array}
    \right| \neq 0},
\end{equation}
para todo $t$.

Neste caso, sempre é possível escolher as constantes $c_1$ e $c_2$ tais que
\begin{equation}\label{eq:od2o_homcc_sg}
  y(t) = c_1y_1(t) + c_2y_2(t)
\end{equation}
satisfaça o problema de valor inicial
\begin{align}
  &ay'' + by' + cy = 0,\\
  &y(t_0) = y_0,\quad y'(t_0) = y_0',
\end{align}
para quaisquer dados valores $y_0$ e $y'_0$.

De fato, já sabemos que \eqref{eq:od2o_homcc_sg} satisfaz a EDO. Então, $c_1$ e $c_2$ deve satisfazer o seguinte sistema linear
\begin{align}
  y(t_0) &= c_1y_1(t_0) + c_2y_2(t_0) = y_0\\
  y'(t_0) &= c_1y_1'(t_0) + c_2y_2'(t_0) = y_0'.
\end{align}
Do método de Cramer\footnote{Gabriel Cramer, 1704 - 1752, matemático suíço.}, temos
\begin{equation}
  c_1 = \frac{\left|
    \begin{matrix}
      y_0 & y_2(t_0) \\
      y'_0 & y_2'(t_0)
    \end{matrix}
    \right|}{\left|
    \begin{matrix}
      y_1(t_0) & y_2(t_0) \\
      y'_1(t_0) & y_2'(t_0)
    \end{matrix}
    \right|}
\end{equation}
e
\begin{equation}
  c_2 = \frac{\left|
    \begin{matrix}
      y_1(t_0) & y_0\\
      y_1'(t_0) & y'_0
    \end{matrix}
    \right|}{\left|
    \begin{matrix}
      y_1(t_0) & y_2(t_0) \\
      y'_1(t_0) & y_2'(t_0)
    \end{matrix}
    \right|}.
\end{equation}
O \emph{wronskiano não nulo} nos garante a existência de $c_1$ e $c_2$.

Por fim, afirmamos que todas as soluções de \eqref{eq:ed2o_homcc} podem ser escritas como combinação linear de $y_1 = y_1(t)$ e $y_2 = y_2(t)$, i.e. têm a forma
\begin{equation}
  y(t) = c_1y_1(t) + c_2y_2(t).
\end{equation}

De fato, seja $\psi = \psi(t)$ uma solução de \eqref{eq:ed2o_homcc}. Então, $\psi$ é solução do seguinte PVI
\begin{align}
  &ay'' + by' + cy = 0,\\
  &y(t_0) = \psi(t_0),\quad y'(t_0) = \psi'(t_0),
\end{align}
para quaisquer $t_0$ dado. Agora, pelo que vimos acima e lembrando que o wronskiano $W(y_1,y_2;t)\neq 0$, temos que existem constantes $c_1$ e $c_2$ tais que
\begin{equation}
  y(t) = c_1y_1(t) + c_2y_2(t).
\end{equation}
também é solução deste PVI. Da \emph{unicidade de solução}\footnote{Embora não tenha sido apresentada aqui, a unicidade de solução pode ser demonstrada.}, segue que
\begin{equation}
  \psi(t) = c_1y_1(t) + c_2y_2(t).
\end{equation}

Do que vimos aqui, a \emph{solução geral} de \eqref{eq:ed2o_homcc} é
\begin{equation}
  y(t) = c_1y_1(t) + c_2y_2(t)
\end{equation}
dadas quaisquer soluções  $y_1 = y_1(t)$ e $y_2 = y_2(t)$ com wronskiano $W(y_1,y_2;t)\neq 0$ para todo $t$.

\begin{ex}
  No Exemplo \ref{ex:ed2o_homcc}, vimos que
  \begin{equation}
    y_1(t) = e^{-2t}\quad\text{e}\quad y_2(t) = e^{2t}
  \end{equation}
  são soluções particulares de
  \begin{equation}\label{eq:ex_ed2o_homcc}
    y'' - 4y = 0.
  \end{equation}
  Como
  \begin{align}
    W(y_1,y_2;t) &= \left|\begin{matrix}
    y_1 & y_2 \\
    y_1' & y_2' \\
    \end{matrix}\right| \\
    &= \left|\begin{matrix}
    e^{-2t} & e^{2t} \\
    -2e^{-2t} & 2e^{2t} \\
    \end{matrix}\right| \\
    &= 4 \neq 0,
  \end{align}
  temos que
  \begin{equation}
    y(t) = c_1e^{-2t} + c_2e^{2t}
  \end{equation}
  é solução geral de \eqref{eq:ex_ed2o_homcc}.
\end{ex}

\subsection{Raízes reais distintas}

\begin{flushright}
  \href{https://archive.org/details/edo-ordem-2-rrd}{[Vídeo]} | [Áudio] | \href{https://phkonzen.github.io/notas/contato.html}{[Contatar]}
\end{flushright}

Uma EDO da forma
\begin{equation}
  {\color{blue}ay'' + by' + cy = 0}
\end{equation}
tem \emph{solução geral}
\begin{equation}
  y(t) = c_1e^{r_1t} + c_2e^{r_2t}
\end{equation}
quando sua \emph{equação característica}
\begin{equation}
  {\color{blue}ar^2 + br + c = 0}
\end{equation}
tem $r_1$ e $r_2$ como suas raízes reais distintas.

\begin{ex}
  Vamos resolver o seguinte PVI
  \begin{align}
    y'' - 3y' + 2y = 0,\\
    y(0) = 3,\quad y'(0) = 5.
  \end{align}

  Começamos resolvendo a equação característica associada
  \begin{equation}
    r^2 -3r + 2 = 0.
  \end{equation}
  As soluções são
  \begin{align}
    r &= \frac{3 \pm \sqrt{9 - 8}}{2} \\
    &= \frac{3 \pm 1}{2}.
  \end{align}
  Ou seja, $r_1 = 1$ e $r_2 = 2$. Logo,
  \begin{equation}
    y(t) = c_1e^t + c_2e^{2t}
  \end{equation}
  é solução geral da EDO.

  Agora, aplicando as condições iniciais, temos
  \begin{align}
    y(0) = 3 &\Rightarrow c_1 + c_2 = 3,\\
    y'(0) = 5 &\Rightarrow c_1 + 2c_2 = 5.
  \end{align}
  Resolvendo este sistema linear, obtemos $c_1 = 1$ e $c_2 = 2$. Concluímos que
  \begin{equation}
    y(t) = e^t + 2e^{2t}
  \end{equation}
  é a solução do PVI.
\end{ex}

\subsection*{Exercícios resolvidos}

\begin{flushright}
  [Vídeo] | [Áudio] | \href{https://phkonzen.github.io/notas/contato.html}{[Contatar]}
\end{flushright}

\begin{exeresol}
  Calcule a solução geral de
  \begin{equation}
    2y'' + 2y' - 4y = 0.
  \end{equation}
\end{exeresol}
\begin{resol}
  A equação característica associada é
  \begin{equation}
    2r^2 + 2r -4 = 0.
  \end{equation}
  Suas soluções são
  \begin{align}
    r &= \frac{-2 \pm \sqrt{4 - 4\cdot 2\cdot (-4)}}{2} \\
    &= \frac{-2 \pm 6}{4},
  \end{align}
  i.e. $r_1 = -2$ e $r_2 = 1$. Como a equação característica tem raízes reais distintas, concluímos que
  \begin{equation}
    y(t) = c_1e^{-2t} + c_2e^{t}
  \end{equation}
  é solução geral da EDO.
\end{resol}

\begin{exeresol}
  Mostre que se $y_1(t) = e^{-2t}$ e $y_2(t) = e^t$, então o wronskiano
  \begin{equation}
    W(y_1,y_2; t) \neq 0.
  \end{equation}
\end{exeresol}
\begin{resol}
  Calculamos
  \begin{align}
    W(y_1,y_2; t) &= \left|
    \begin{matrix}
      y_1 & y_2 \\
      y_1' & y_2'
    \end{matrix}
    \right|
    &= \left|
    \begin{matrix}
      e^{-2t} & e^t \\
      -2e^{-2t} & e^{t}
    \end{matrix} \right| \\
    &= e^{-2t}e^t + 2e^{-2t}e^t \\
    &= 3e^{-t}.
  \end{align}
  Como $e^{-t} \neq 0$ para todo $t$, temos que $W(y_1,y_2; t)\neq 0$ para todo $t$.
\end{resol}

\subsection*{Exercícios}

\begin{flushright}
  [Vídeo] | [Áudio] | \href{https://phkonzen.github.io/notas/contato.html}{[Contatar]}
\end{flushright}

\begin{exer}
  Calcule a solução geral de
  \begin{equation}
    -2y'' + 2y' + 4y = 0.
  \end{equation}
\end{exer}
\begin{resp}
  $y(t) = c_1e^{-t} + c_2e^{2t}$
\end{resp}

\begin{exer}
  Resolva o seguinte PVI
  \begin{align}
    &y'' = 7y' - 12y,\\
    &y(0) = 0,\quad y'(0) = -1.
  \end{align}
\end{exer}
\begin{resp}
  $y(t) = e^{3t} - e^{4t}$
\end{resp}

\begin{exer}
  Resolva o seguinte PVI
  \begin{align}
    &y'' - 3y' + 2y = 0,\\
    &y(\ln 2) = -2,\quad y'(\ln 2) = -6.
  \end{align}
\end{exer}
\begin{resp}
  $y(t) = e^t - e^{2t}$
\end{resp}

\begin{exer}
  Calcule o wronskiano de $y_1(t) = \cos(t)$ e $y_2(t) = \sen(t)$.
\end{exer}
\begin{resp}
  $1$
\end{resp}

\begin{exer}
  Mostre que se $r_1$ e $r_2$ são raízes reais distintas da equação
  \begin{equation}
    ar^2 + br + c = 0,
  \end{equation}
  então
  \begin{equation}
    y(t) = c_1e^{r_1t} + c_2e^{r_2t}
  \end{equation}
  é solução geral de
  \begin{equation}
    ay'' + by' + cy = 0.
  \end{equation}
\end{exer}
\begin{resp}
  Mostre que $y_1(t) = e^{r_1t}$ e $y_2(t) = e^{r_2t}$ são soluções da EDO com $W(y_1,y_2; t) \neq 0$.
\end{resp}


\section{EDO de ordem 2: raízes complexas ou repetidas}\label{cap_edolin_sec_edo2II}

\begin{flushright}
  [Vídeo] | [Áudio] | \href{https://phkonzen.github.io/notas/contato.html}{[Contatar]}
\end{flushright}

Na Seção \ref{cap_edolin_sec_edo2I} introduzimos as propriedades fundamentais de EDOs lineares de segunda ordem com coeficientes constantes. Em particular, tratamos o caso em que a equação característica tem raízes reais distintas. Nesta seção, estudamos os casos em que a equação característica tem raízes complexas ou raízes duplas.

\subsection{Raízes complexas}

\begin{flushright}
  \href{https://archive.org/details/edo-ordem-2-rc}{[Vídeo]} | [Áudio] | \href{https://phkonzen.github.io/notas/contato.html}{[Contatar]}
\end{flushright}

Consideramos
\begin{equation}\label{eq:ed2o_hcc_rc}
  {\color{blue}ay'' + by' + cy = 0},
\end{equation}
cuja \emph{equação característica}
\begin{equation}
  {\color{blue}ar^2 + br + c = 0}
\end{equation}
tem \emph{raízes complexas}
\begin{equation}
  {\color{blue}r_1 = \lambda + i\mu,\quad r_2 = \lambda - i\mu}.
\end{equation}
As soluções particulares associadas são
\begin{align}
  &y_1(t) = e^{r_1t} = e^{(\lambda + i\mu)t} \\
  &y_2(t) = e^{r_2t} = e^{(\lambda - i\mu)t}.
\end{align}

Da \emph{fórmula de Euler}\footnote{Leonhard Euler, 1707-1783, matemático suíço. Fonte: \href{https://en.wikipedia.org/wiki/Leonhard_Euler}{Wikipedia}.}, temos
\begin{align}
  e^{a+bi} &= e^ae^{bi}\\
  &= e^a(\cos b + i\sen b).
\end{align}
Ou seja, as soluções particulares podem ser reescritas da forma\footnote{Lembre-se que seno é uma função ímpar, i.e. $\sen(-x) = -\sen(x)$.}
\begin{align}
  &y_1(t) = e^{\lambda t}\left[\cos(\mu t) + i\sen(\mu t)\right], \\
  &y_2(t) = e^{\lambda t}\left[\cos(\mu t) - i\sen(\mu t)\right]
\end{align}

Agora, se denotarmos
\begin{equation}
  u(t) = e^{\lambda t}\cos(\mu t)\quad\text{e}\quad v(t) = e^{\lambda t}\sen(\mu t),
\end{equation}
temos
\begin{equation}
  y_1(t) = u(t) + iv(t),\quad y_2(t) = u(t) - iv(t).
\end{equation}
Para concentrar a escrita, vamos denotar
\begin{equation}
  y(t) = u(t) \pm iv(t).
\end{equation}
Substituindo $y = y(t)$ na EDO, obtemos
\begin{align}
  0 &= ay'' + by' + cy \\
  &= a(u'' \pm iv'') \\
  &+ b(u' \pm iv') \\
  &+ c(u + \pm iv) \\
  &= (au'' + bu' + cu) \\
  &\pm i(av'' + bv' + cv).
\end{align}
Ou seja,
\begin{align}
  au'' + bu' + cu &= 0 \\
  av'' + bv' + cv &= 0.
\end{align}
Desta forma, concluímos que $u(t) = e^{\lambda t}\cos(\mu t)$ e $v(t) = e^{\lambda t}\sen(\mu t)$ são soluções particulares da EDO \eqref{eq:ed2o_hcc_rc}. Ainda mais, pode-se mostrar que o wronskiano $W(u,v;t)\neq 0$, i.e. $u$ e $v$ formam um conjunto fundamental de soluções. Do que vimos na Seção \ref{cap_edolin_sec_edo2I}, concluímos que
\begin{equation}
  {\color{blue}y(t) = e^{\lambda t}\left[c_1\cos(\mu t) + c_2\sen(\mu t)\right]}
\end{equation}
é \emph{solução geral} de \eqref{eq:ed2o_hcc_rc}.

\begin{ex}
  Vamos resolver
  \begin{equation}
    y'' + 2y' + 5y = 0.
  \end{equation}

  Começamos identificando a equação característica associada
  \begin{equation}
    r^2 + 2r + 5 = 0.
  \end{equation}
  Suas raízes são
  \begin{align}
    r &= \frac{-2 \pm \sqrt{4 - 4\cdot 1\cdot 5}}{2} \\
    &= -1 \pm 2i.
  \end{align}
  Logo, a solução geral é
  \begin{equation}
    y(t) = e^{-t}\left[c_1\cos(2t) + c_2\sen(2t)\right].
  \end{equation}
\end{ex}

\subsubsection{Modelagem: sistema massa-mola não amortecido}

\begin{flushright}
  [Vídeo] | [Áudio] | \href{https://phkonzen.github.io/notas/contato.html}{[Contatar]}
\end{flushright}

Consideremos um sistema massa-mola não amortecido e sem ação de força externa. Denotamos por $m>0$ a massa, $k>0$ a constante da mola. Desta forma, a lei de Newton do movimento nos fornece o seguinte modelo matemático
\begin{equation}
  ms''(t) = -ks(t),
\end{equation}
onde $s = s(t)$ é a posição da massa ($s=0$ é a posição de repouso, $s>0$ a mola está esticada e $s<0$ a mola está contraída). Ou seja, trata-se de uma EDO de segunda ordem homogênea e com coeficientes constantes.

Supondo que, no tempo inicial $t=0$, a massa está na posição inicial $s_0$ e velocidade $v_0$, temos que a situação física é modelada pelo seguinte PVI
\begin{align}
  & s'' + \frac{k}{m}s = 0,\quad t>0,\\
  & s(0) = s_0,\quad s'(0) = v_0.
\end{align}

Como $k/m>0$, temos que a equação característica associada têm raízes imaginárias
\begin{equation}
  r = \pm \sqrt{\frac{k}{m}}i.
\end{equation}
Logo, a solução geral é
\begin{equation}
  s(t) = c_1\cos\left(\sqrt{\frac{k}{m}}t\right) + c_2\sen\left(\sqrt{\frac{k}{m}}t\right).
\end{equation}
Agora, aplicando as condições iniciais, obtemos
\begin{equation}
  s(t) = s_0\cos\left(\sqrt{\frac{k}{m}}t\right) + v_0\sqrt{\frac{m}{k}}\sen\left(\sqrt{\frac{k}{m}}t\right).
\end{equation}

\subsection{Raízes repetidas}

\begin{flushright}
  \href{https://archive.org/details/edo-ordem-2-rrr}{[Vídeo]} | [Áudio] | \href{https://phkonzen.github.io/notas/contato.html}{[Contatar]}
\end{flushright}

Seja a equação
\begin{equation}\label{eq:ed2o_hcc_rr}
  {\color{blue}ay'' + by' + cy = 0},
\end{equation}
cuja equação característica
\begin{equation}
  {\color{blue}ar^2 + br + c = 0}
\end{equation}
tem raiz dupla\footnote{$b^2-4ac = 0$}
\begin{equation}
  r = \frac{-b}{2a}.
\end{equation}
Neste caso, podemos verificar que
\begin{equation}
  {\color{blue}y_1(t) = e^{-\frac{b}{2a}t}}
\end{equation}
é solução particular de \eqref{eq:ed2o_hcc_rr}.

Vamos usar o \emph{método de redução de ordem} para encontrar uma segunda solução particular $y_2 = y_2(t)$ de \eqref{eq:ed2o_hcc_rr}, lembrando que o wronskiano $W(y_1,y_2; t)$ deve ser não nulo. O método consiste em buscar por uma solução da forma
\begin{align}
  {\color{blue}y_2(t)} &{\color{blue}= u(t)y_1(t)} \\
  &= u(t)e^{-\frac{b}{2a}t}.
\end{align}
Substituindo $y_2$ na EDO \eqref{eq:ed2o_hcc_rr}, obtemos
\begin{align}
  0 &= ay_2'' + by_2' + cy_2 \\
  &= a(u''y_1 + 2u'y_1' + uy_1'') \\
  &+ b(u'y_1 + uy_1') + cuy_1 \\
  &= au''y_1 + (2ay_1' + by_1)u' \\
  &+ (ay_1'' + by_1' + cy_1)u \\
  &= au''y_1 + \left(-\frac{2ab}{2a}e^{-\frac{b}{2a}} + be^{-\frac{b}{2a}}\right)u' \\
  &= au''y_1.
\end{align}
Segue que
\begin{align}
  u'' = 0  &\Rightarrow u' = c_1 \\
  &\Rightarrow u = c_1 + c_2t.
\end{align}
Podemos escolher $c_1$ e $c_2$ arbitrariamente, desde que o wronskiano
\begin{equation}
  W(y_1,y_2;t) \neq 0.
\end{equation}
A escolha mais simples é $c_1 = 0$ e $c_2 = 1$, donde segue que
\begin{equation}
  {\color{blue}y_2(t) = te^{-\frac{b}{2a}t}}.
\end{equation}
Concluímos que a \emph{solução geral} de \eqref{eq:ed2o_hcc_rr} é
\begin{equation}
  {\color{blue}y(t) = (c_1 + c_2t)e^{-\frac{b}{2a}t}}.
\end{equation}

\begin{ex}
  Vamos resolver
  \begin{equation}
    y'' -2y' + y = 0.
  \end{equation}

  Da equação característica
  \begin{equation}
    r^2 - 2r + 1 = 0
  \end{equation}
  obtemos a raiz dupla
  \begin{equation}
    r = 1.
  \end{equation}
  Logo, a solução geral da EDO é
  \begin{equation}
    y(t) = (c_1 + c_2t)e^{t}.
  \end{equation}
\end{ex}

\subsection*{Exercícios resolvidos}

\begin{flushright}
  [Vídeo] | [Áudio] | \href{https://phkonzen.github.io/notas/contato.html}{[Contatar]}
\end{flushright}

\begin{exeresol}
  Resolva
  \begin{align}
    &y'' - 4y' + 5y = 0,\\
    &y(0) = 2,\quad y'(0)=0.
  \end{align}
\end{exeresol}
\begin{resol}
  Resolvendo a equação característica
  \begin{equation}
    r^2 - 4r + 5 = 0,
  \end{equation}
  obtemos as raízes
  \begin{align}
    r &= \frac{4 \pm \sqrt{16 - 4\cdot 1\cdot 5}}{2} \\
    &= 2 \pm i.
  \end{align}
  Logo, a solução geral é
  \begin{equation}
    y(t) = e^{2t}\left[c_1\cos(t) + c_2\sen(t)\right].
  \end{equation}

  Por fim, aplicamos as condições iniciais
  \begin{align}
    y(0) = 2 &\Rightarrow e^{2\cdot 0}\left[c_1\cos(0) + c_2\sen(0)\right] = 2 \\
    &\Rightarrow c_1 = 2.
  \end{align}
  e, observando que
  \begin{equation}
    y'(t) = e^{2t}\left[(2c_1 + c_2)\cos(t) + (2c_2 - c_1)\sen(t)\right]
  \end{equation}
  temos
  \begin{align}
    y'(0) = 0 &\Rightarrow e^{2\cdot 0}(2\cdot 2 + c_2) = 0 \\
    &\Rightarrow 4 + c_2 = 0 \\
    &\Rightarrow c_2 = -4.
  \end{align}
  Concluímos que a solução do PVI é
  \begin{equation}
    y(t) = e^{2t}\left[2\cos(t) - 4\sen(t)\right].
  \end{equation}
\end{resol}

\begin{exeresol}
  Resolva
  \begin{align}
    &y'' + 4y' + 4y = 0,\\
    &y(0) = 0,\quad y'(0)=1.    
  \end{align}
\end{exeresol}
\begin{resol}
  Resolvemos a equação característica
  \begin{equation}
    r^2 + 4r + 4 = 0,
  \end{equation}
  de modo que obtemos uma raiz dupla
  \begin{equation}
    r = -2.
  \end{equation}
  Logo, a solução geral da EDO é
  \begin{equation}
    y(t) = (c_1 + c_2t)e^{-2t}.
  \end{equation}

  Agora, aplicamos as condições iniciais
  \begin{equation}
    y(0) = 0 \Rightarrow c_1 = 0
  \end{equation}
  e, observando que
  \begin{equation}
    y'(t) = (c_2-2c_1-2c_2t)e^{-2t}
  \end{equation}
  \begin{align}
    y'(0) = 1 &\Rightarrow (c_2-2\cdot 0-2c_2\cdot 0)e^{-2\cdot 0} = 1 \\
    &\Rightarrow c_2 = 1.
  \end{align}
  Concluímos que a solução do PVI é
  \begin{equation}
    y(t) = te^{-2t}.
  \end{equation}
\end{resol}

\begin{exeresol}(Sistema massa-mola amortecido)
  Um sistema massa-mola amortecido sem força externa pode ser modelado pelo seguinte PVI
  \begin{align}
    & ms'' + \gamma s' + ks = 0,\quad t>0,\\
    & s(0) = s_0,\quad s'(0) = v_0,
  \end{align}
  onde $s = s(t)$ é a posição da massa ($s=0$ posição de repouso, $s>0$ mola estendida, $m<0$ mola contraída), $m>0$ massa, $\gamma>0$ coeficiente de resistência do meio, $k>0$ constante da mola, $s_0$ posição inicial e $v_0$ velocidade inicial da massa.

  Mostre que $s(t)\to 0$ quando $t\to\infty$, i.e. a massa tende ao repouso ao passar do tempo.
\end{exeresol}
\begin{resol}
  A equação característica associada é
  \begin{equation}
    mr^2 + \gamma r + k = 0,
  \end{equation}
  cujas raízes são
  \begin{equation}
    r_1, r_2 = \frac{-\gamma \pm\sqrt{\gamma^2 - 4mk}}{2m}.
  \end{equation}
  Vejamos as seguintes possibilidades:
  \begin{enumerate}[a)]
  \item $\gamma^2 - 4mk\geq 0$.

    Como $m,k>0$, temos que $\gamma^2-4mk < \gamma^2$ e, portanto, $\sqrt{\gamma^2 - 4mk} < \gamma$. Segue que $r_1, r_2 < 0$. Se $r_1 \neq r_2$, a solução geral é
    \begin{equation}
      s(t) = c_1e^{r_1t} + c_2e^{r_2t}.
    \end{equation}
    Se $r_1=r_2$, a solução geral é
    \begin{equation}
      s(t) = (c_1 + c_2t)e^{-\frac{\gamma}{2m}t}.
    \end{equation}
    Em ambos os casos, $s(t)\to 0$ quando $t\to 0$, devido aos expoentes negativos.

  \item $\gamma^2 -4mk < 0$.

    Neste caso, a solução geral é
    \begin{equation}
      s(t) = e^{-\frac{\gamma}{2m}t}\left[c_1\cos\left(\frac{\gamma^2-4mk}{2m}t\right) + c_2\sen\left(\frac{\gamma^2-4mk}{2m}t\right)\right].
    \end{equation}
    Novamente, como seno e cosseno são funções limitadas, temos que o termo exponencial domina para $t\to 0$. Ou seja, $s(t)\to 0$ quando $t\to 0$.
  \end{enumerate}
\end{resol}

\subsection*{Exercícios}

\begin{flushright}
  [Vídeo] | [Áudio] | \href{https://phkonzen.github.io/notas/contato.html}{[Contatar]}
\end{flushright}

\begin{exer}
  Encontre a solução geral de
  \begin{equation}
    2y'' - 4y' + 4y = 0.
  \end{equation}
\end{exer}
\begin{resp}
  $y(t) = [c_1\sen(t) + c_2\cos(t)]e^{t}$
\end{resp}

\begin{exer}
  Resolva
  \begin{align}
    &2y'' + 12y' = -26y,\\
    &y(0) = 0,\quad y'(0) = 2.
  \end{align}
\end{exer}
\begin{resp}
  $y(t) = e^{-3t}\sen(2t)$
\end{resp}

\begin{exer}
  Encontre a solução geral de
  \begin{equation}
    3y'' + 27y = 18y'
  \end{equation}
\end{exer}
\begin{resp}
  $y(t) = (c_1 + c_2t)e^{3t}$
\end{resp}

\begin{exer}
  Resolva
  \begin{align}
    &-y = 2y' + y'',\\
    &y(0) = 2,\quad y'(0) = 0.
  \end{align}
\end{exer}
\begin{resp}
  $y(t) = (2 + 2t)e^{-t}$
\end{resp}

\begin{ex}
  Mostre que o wronskiano de $y_1(t) = e^{\lambda t}\cos(\mu t)$ e $y_2(t) = e^{\lambda t}\sen(\mu t)$ é não nulo para qualquer $\mu\neq 0$.
\end{ex}
\begin{resp}
  $W(y_1,y_2;t) = \mu e^{2\lambda t} \neq 0$
\end{resp}

\begin{ex}
  Mostre que o wronskiano de $y_1(t) = e^{rt}$ e $y_2(t) = te^{rt}$ é não nulo para qualquer $r$.  
\end{ex}
\begin{resp}
  $W(y_1,y_2;t) = e^{2rt} \neq 0$
\end{resp}

\section{EDO de ordem 2 não homogênea}\label{cap_edolin_sec_edo2III}

\begin{flushright}
  [Vídeo] | \href{https://archive.org/details/s-3-3-edo-ordem-2-nao-homogenea}{[Áudio]} | \href{https://phkonzen.github.io/notas/contato.html}{[Contatar]}
\end{flushright}

Nesta seção, vamos discutir o caso de EDOs lineares de segunda ordem, não-homogêneas e com coeficientes constantes. Tais EDOs têm a forma
\begin{equation}\label{eq:ed2o_nh}
  {\color{blue}y'' + ay' + by = g(t)},
\end{equation}
e pode-se mostrar que sua solução geral é dada como
\begin{equation}
  {\color{blue}y(t) = c_1y_1(t) + c_2y_2(t) + y_p(t)},
\end{equation}
onde $y_1$ e $y_2$ formam um conjunto fundamental de soluções\footnote{São soluções da equação homogênea associada e $W(y_1,y_2;t)\neq 0$.} da equação homogênea associada
\begin{equation}
  y'' + ay' + by = 0
\end{equation}
e $y_p$ é uma solução particular qualquer de \eqref{eq:ed2o_nh}.

\subsection{Método da variação dos parâmetros}

\begin{flushright}
  \href{https://archive.org/details/edo-ordem-2-mvp}{[Vídeo]} | \href{https://archive.org/details/ss-3-3-1-metodo-da-variacao-dos-parametros}{[Áudio]} | \href{https://phkonzen.github.io/notas/contato.html}{[Contatar]}
\end{flushright}

O \emph{método da variação dos parâmetros} consiste em calcular uma solução particular de \eqref{eq:ed2o_nh} da forma
\begin{equation}
  {\color{blue}y_p(t) = u_1(t)y_1(t) + u_2(t)y_2(t)},
\end{equation}
onde $y_1$ e $y_2$ é um conjunto fundamental de soluções da equação homogênea associada, enquanto $u_1$ e $u_2$ são funções a serem determinadas.

Observamos que a única condição que temos para determinar $u_1$ e $u_2$ é a equação \eqref{eq:ed2o_nh}. Ou seja, temos uma equação e duas incógnitas. Para fechar o problema, impomos a seguinte condição extra
\begin{equation}\label{eq:ed2o_nh_mv_1}
  {\color{blue}u_1'y_1 + u_2'y_2 = 0}.
\end{equation}

Com isso, temos
\begin{align}
  y_p'(t) &=  u_1'y_1 + u_1y_1' + u_2'y_2 + u_2y_2' \\
  &= u_1y_1' + u_2y_2'
\end{align}
e
\begin{align}
  y_p''(t) &= u_1'y_1' + u_1y_1'' \\
  &+ u_2'y_2' + u_2y_2''.
\end{align}

Substituindo $y_p$ em \eqref{eq:ed2o_nh}, temos
\begin{align}
  g(t) &= y_p'' + ay_p' + by_p \\
  &= \left(u_1'y_1' + {\color{blue}u_1y_1''} + u_2'y_2' + {\color{red}u_2y_2''} \right) \\
  &+ a({\color{blue}u_1y_1'} + {\color{red}u_2y_2'}) \\
  &+ b({\color{blue}u_1y_1} + {\color{red}u_2y_2}) \\
  &= u_1'y_1' + u_2'y_2' \\
  &+ \underbrace{{\color{blue}u_1(y_1'' + ay_1' + by_1)}}_{=0} \\
  &+ \underbrace{{\color{red}u_2(y_2'' + ay_2' + by_2)}}_{=0} \\
  &= u_1'y_1' + u_2'y_2'. \label{eq:ed2o_nh_mv_2}
\end{align}

Ou seja, \eqref{eq:ed2o_nh_mv_1} e \eqref{eq:ed2o_nh_mv_2} formam o seguinte sistema de equações
\begin{align}
  u_1'y_1 + u_2'y_2 &= 0\\
  u_1'y_1' + u_2'y_2' &= g(t)
\end{align}
que têm $u_1'$ e $u_2'$ como incógnitas. Aplicando o método de Cramer\footnote{Gabriel Cramer, 1704 - 1752, matemático suíço. Fonte: \href{https://en.wikipedia.org/wiki/Gabriel_Cramer}{Wikipedia}.}, obtemos
\begin{align}
  u_1' &= \frac{
    \begin{vmatrix}
      0 & y_2\\
      g(t) & y_2'
    \end{vmatrix}
  }{
    \begin{vmatrix}
      y_1 & y_2\\
      y_1' & y_2'
    \end{vmatrix}
  }\\
  &= -\frac{y_2(t)g(t)}{W(y_1,y_2;t)}
\end{align}
e
\begin{align}
  u_2' &= \frac{
    \begin{vmatrix}
      y_1 & 0 \\
      y_1' & g(t)
    \end{vmatrix}
  }{
    \begin{vmatrix}
      y_1 & y_2\\
      y_1' & y_2'                    
    \end{vmatrix}
  } \\
  &= \frac{y_1(t)g(t)}{W(y_1,y_2;t)}.
\end{align}
Ou, ainda, por integração temos
\begin{equation}\label{eq:ed2o_nh_vp_u1}
  {\color{blue}u_1(t) = -\int \frac{y_2(t)g(t)}{W(y_1,y_2;t)}\,dt}
\end{equation}
e
\begin{equation}\label{eq:ed2o_nh_vp_u2}
  {\color{blue}u_2(t) = \int \frac{y_1(t)g(t)}{W(y_1,y_2;t)}\,dt}.
\end{equation}

Por tudo isso, concluímos que uma solução particular de \eqref{eq:ed2o_nh} é dada por
\begin{align}
  y_p(t) &= -y_1(t)\int \frac{y_2(t)g(t)}{W(y_1,y_2;t)}\,dt \\
  &+ y_2(t)\int \frac{y_1(t)g(t)}{W(y_1,y_2;t)}\,dt.
\end{align}

\begin{ex}\label{ex:ed2o_nh_vp}
  Vamos calcular a solução geral de
  \begin{equation}\label{eq:ex_ed2o_nh_vp}
    y'' - y = e^{2t}.
  \end{equation}

  Começamos determinando um conjunto fundamental de soluções $y_1 = y_1(t)$ e $y_2 = y_2(t)$ da equação homogênea associada
  \begin{equation}
    y'' - y = 0.
  \end{equation}
  A equação característica associada é
  \begin{equation}
    r^2 - 1 = 0,
  \end{equation}
  cujas raízes são $r_1=-1$ e $r_2=1$. Segue que
  \begin{equation}
    y_1(t) = e^{-t}\quad\text{e}\quad y_2(t) = e^t.
  \end{equation}

  Agora, buscamos por uma solução particular de \eqref{eq:ex_ed2o_nh_vp} da forma
  \begin{equation}
    y_p(t) = u_1(t)y_1(t) + u_2(t)y_2(t),
  \end{equation}
  onde $u_1$ é dada em \eqref{eq:ed2o_nh_vp_u1} e $u_2$ por \eqref{eq:ed2o_nh_vp_u2}. Ambas expressões requer o cálculo do wronskiano
  \begin{align}
    W(y_1,y_2;t) &= \left|
    \begin{matrix}
      y_1 & y_2 \\
      y_1' & y_2'
    \end{matrix}
    \right| \\
    &= y_1y_2' - y_2y_1' \\
    &= e^{-t}e^t + e^te^{-t} \\
    &= 2.
  \end{align}
  Com isso, temos
  \begin{align}
    u_1(t) &= -\int \frac{y_2(t)g(t)}{W(y_1,y_2;t)}\,dt \\
    &= -\int \frac{e^te^{2t}}{2}\,dt \\
    &= -\frac{1}{2}\int e^{3t}\,dt \\
    &= -\frac{1}{6}e^{3t}
  \end{align}
  e
  \begin{align}
    u_2(t) &= \int \frac{y_1(t)g(t)}{W(y_1,y_2;t)}\,dt \\
    &= \int \frac{e^{-t}e^{2t}}{2}\,dt \\
    &= \frac{1}{2}\int e^{t}\,dt \\
    &= \frac{1}{2}e^{t}
  \end{align}
  Desta forma, obtemos a solução particular
  \begin{align}
    y_p(t) &= u_1(t)y_1(t) + u_2(t)y_2(t) \\
    &= -\frac{1}{6}e^{3t}e^{-t} + \frac{1}{2}e^{t}e^{t} \\
    &= -\frac{1}{6}e^{2t} + \frac{1}{2}e^{2t} \\
    &= \frac{1}{3}e^{2t}.
  \end{align}
  Observamos que a solução particular é um múltiplo do termo não homogêneo da EDO \eqref{eq:ex_ed2o_nh_vp}. Isso não é apenas um acaso e vamos explorar isso mais adiante no texto.

  Por fim, concluímos que a solução geral de \eqref{eq:ex_ed2o_nh_vp} é
  \begin{align}
    y(t) &= c_1y_1(t) + c_2y_2(t) + y_p(t) \\
    &= c_1e^{-t} + c_2e^t + \frac{1}{3}e^{2t}.
  \end{align}
\end{ex}

\subsection{Método dos coeficientes a determinar}\label{subsec:edolin_o2_mcc}

\begin{flushright}
  \href{https://archive.org/details/edo-ordem-2-mcd}{[Vídeo]} | \href{https://archive.org/details/ss-3-3-2-metodo-dos-coeficientes-a-determinar}{[Áudio]} | \href{https://phkonzen.github.io/notas/contato.html}{[Contatar]}
\end{flushright}

O métodos dos coeficientes a determinar consiste em buscar por uma solução particular na forma de uma combinação linear de funções elementares apropriadas. Tais funções são inferidas a partir do termo não homogêneo da equação.

\subsubsection{$\pmb{g(t) = ce^{st}}$}

\begin{flushright}
  [Vídeo] | [Áudio] | \href{https://phkonzen.github.io/notas/contato.html}{[Contatar]}
\end{flushright}

Uma equação da forma
\begin{equation}
  y'' + ay' + by = ce^{st}
\end{equation}
com $s\neq r_1,r_2$, onde $r_1$ e $r_2$ são raízes da equação característica, admite solução particular
\begin{equation}
  y_p(t) = Ae^{st},
\end{equation}
onde $A$ é uma constante a determinar.

\begin{ex}
  Vamos calcular uma solução particular para
  \begin{equation}
    y'' - y = e^{2t}.
  \end{equation}

  Pelo método dos coeficientes a determinar, buscamos por uma solução particular da forma
  \begin{equation}
    y_p(t) = Ae^{2t},
  \end{equation}
  observando que $r_1=-1$ e $r_2=1$ são raízes da equação característica associada.

  Substituindo $y_p$ na EDO, obtemos
  \begin{align}
    e^{2t} &= y'' - y \\
    &= \left(Ae^{2t}\right)'' - Ae^{2t} \\
    &= (4A - A)e^{2t} \\
    &= 3Ae^{2t}.
  \end{align}
  Segue que
  \begin{equation}
    3A = 1 \Rightarrow A = \frac{1}{3}.
  \end{equation}
  Daí, concluímos que
  \begin{equation}
    y_p(t) = \frac{1}{3}e^{2t}
  \end{equation}
  é solução particular da EDO.
\end{ex}

\begin{obs}
  \begin{enumerate}[a)]
  \item $s = r_1$.
    Uma equação da forma
    \begin{equation}
      y'' + ay' + by = ce^{r_1t},
    \end{equation}
    onde $r_1$ é raiz simples da equação característica associada, admite solução particular
    \begin{equation}
      y_p(t) = Ate^{r_1t}.
    \end{equation}
  \item $s = r$.
    Uma equação da forma
    \begin{equation}
      y'' + ay' + by = ce^{rt},
    \end{equation}
    onde $r$ é raiz dupla da equação característica associada, admite solução particular
    \begin{equation}
      y_p(t) = At^2e^{rt}.
    \end{equation}
  \end{enumerate}
\end{obs}

\subsubsection{$\pmb{g(t) = c_nt^n + c_{n-1}t^{n-1} + \cdots + c_0}$}

\begin{flushright}
  [Vídeo] | [Áudio] | \href{https://phkonzen.github.io/notas/contato.html}{[Contatar]}
\end{flushright}

Uma equação da forma
\begin{equation}
  y'' + ay' + by = c_nt^n + c_{n-1}t^{n-1} + \cdots + c_0
\end{equation}
admite solução particular
\begin{equation}
  y_p(t) = A_nt^n + A_{n-1}t^{n-1} + \cdots + A_0,
\end{equation}
onde $A_n$, $A_{n-1}$, $\dotsc$, $A_0$ são constantes a determinar.

\begin{ex}
  Vamos calcular uma solução particular para
  \begin{equation}
    y'' - 4y = t.
  \end{equation}

  Pelo método dos coeficientes a determinar, buscamos por uma solução particular da forma
  \begin{equation}
    y_p(t) = A_1t + A_0.
  \end{equation}

  Substituindo $y_p$ na EDO, obtemos
  \begin{align}
    t &= y'' - 4y \\
    &= \left(A_1t + A_0\right)'' - 4(A_1t + A_0) \\
    &= -4A_1t - 4A_0.
  \end{align}
  Segue que
  \begin{align}
    -4A_1 = 1 &\Rightarrow A_1 = -\frac{1}{4},\\
    -4A_0 = 0 &\Rightarrow A_0 = 0.
  \end{align}
  Daí, concluímos que
  \begin{equation}
    y_p(t) = -\frac{1}{4}t
  \end{equation}
  é solução particular da EDO.
\end{ex}

\subsubsection{$\pmb{g(t) = c_1\sen(\beta t) + c_2\cos(\beta t)}$}

\begin{flushright}
  [Vídeo] | [Áudio] | \href{https://phkonzen.github.io/notas/contato.html}{[Contatar]}
\end{flushright}

Uma equação da forma
\begin{equation}
  y'' + ay' + by = c_1\sen(\beta t) + c_2\cos(\beta t)
\end{equation}
admite solução particular
\begin{equation}
  y_p(t) = t^s[A_1\sen(\beta t) + A_2\cos(\beta t)],
\end{equation}
onde $s$ é o menor inteiro tal que $y_p$ não seja solução da equação homogênea associada e $A_1$ e $A_2$ são constantes a determinar.

\begin{ex}
  Vamos calcular uma solução particular para
  \begin{equation}
    y'' + 4y = \cos(2t).
  \end{equation}

  Pelo método dos coeficientes a determinar, buscamos por uma solução particular da forma
  \begin{equation}
    y_p(t) = t[A_1\sen(2t) + A_2\cos(2t)],
  \end{equation}
  observando que $y_1(t) = \cos(2t)$ e $y_2(t) = \sen(2t)$ formam um conjunto fundamental de solução para a equação homogênea associada.
  
  Substituindo $y_p$ na EDO, obtemos
  \begin{align}
    \cos(2t) &= y'' + 4y \\
    &= [A_1t\sen(2t) + A_2t\cos(2t)]'' \\
    &+ 4[A_1t\sen(2t) + A_2t\cos(2t)] \\
    &= 4A_1\cos(2t) - 4A_2\sen(2t)
  \end{align}
  Segue que
  \begin{align}
    4A_1 = 1 &\Rightarrow A_1 = \frac{1}{4},\\
    -4A_2 = 0 &\Rightarrow A_2 = 0.
  \end{align}
  Daí, concluímos que
  \begin{equation}
    y_p(t) = \frac{1}{4}t\sen(2t).
  \end{equation}
  é solução particular da EDO.
\end{ex}

\begin{obs}(Resumo)\label{obs:edolin_o2_mcd}
  \begin{center}
    \begin{tabular}{ll}
      $g(t)$ & $y_p(t)$ \\\hline
      $e^{\alpha t}(c_nt^n + c_{n-1}t^{n-1} + \cdots + c_0)$ & $t^se^{\alpha t}(A_nt^n + \cdots + A_0)$ \\
      $e^{\alpha t}[c_1\sen(\beta t) + c_2\cos(\beta t)]$ & $t^se^{\alpha t}[A_1\sen(\beta t) + A_2\cos(\beta t)]$ \\\hline
    \end{tabular}
  \end{center}
  $s = 0, 1, 2$, sendo o menor valor que garanta que $y_p$ não seja solução da equação homogênea associada.
\end{obs}

\subsection*{Exercícios resolvidos}

\begin{flushright}
  [Vídeo] | [Áudio] | \href{https://phkonzen.github.io/notas/contato.html}{[Contatar]}
\end{flushright}

\begin{flushright}
  [Vídeo] | \href{https://archive.org/details/er-s-3-3}{[Áudio]} | \href{https://phkonzen.github.io/notas/contato.html}{[Contatar]}
\end{flushright}


\begin{exeresol}\label{exeresol:ed2o_nh_vp}
  Use o método da variação dos parâmetros para obter uma solução geral de
  \begin{equation}
    y'' - 2y' - 3y = e^{-t}+\sen(t).
  \end{equation}
\end{exeresol}
\begin{resol}
  Primeiramente, resolvemos a equação homogênea associada
  \begin{equation}
    y'' - 2y' - 3y = 0.
  \end{equation}
  Para tanto, buscamos as raízes da equação característica associada
  \begin{equation}
    r^2 - 2r - 3 = 0,
  \end{equation}
  as quais são
  \begin{equation}
    r = \frac{2 \pm \sqrt{4 - 4\cdot 1\cdot (-3)}}{2},
  \end{equation}
  i.e. $r_1 = -1$ e $r_2 = 3$. Logo,
  \begin{equation}
    y_1(t) = e^{-t}\quad\text{e}\quad y_2(t) = e^{3t}
  \end{equation}
  formam um conjunto fundamental de soluções da EDO homogênea.

  Agora, buscamos por uma solução particular
  \begin{equation}
    y_p(t) = u_1(t)y_1(t) + u_2(t)y_2(t)
  \end{equation}
  para a equação não homogênea. Os parâmetros variáveis $u_1 = u_1(t)$ e $u_2 = u_2(t)$ dependem do wronskiano
  \begin{align}
    W(y_1,y_2;t) &= \left|
    \begin{matrix}
      y_1 & y_2 \\
      y_1' & y_2'
    \end{matrix}
    \right| \\
    &= \left|
    \begin{matrix}
      e^{-t} & e^{3t} \\
      -e^{-t} & 3e^{3t}
    \end{matrix}
    \right| \\
    &= 4e^{2t}.
  \end{align}
  Mais especificamente, eles são dados por
  \begin{align}
    u_1(t) &= -\int \frac{y_2(t)g(t)}{W(y_1,y_2;t)}\,dt \\
    &= -\int \frac{e^{3t}[e^{-t} + \sen(t)]}{4e^{2t}}\,dt \\
    &= -\frac{1}{4}t - \frac{1}{8}e^{t}[\sen(t) - \cos(t)]
  \end{align}
  e
  \begin{align}
    u_2(t) &= \int \frac{y_1(t)g(t)}{W(y_1,y_2;t)}\,dt \\
    &= \int \frac{e^{-t}[e^{-t} + \sen(t)]}{4e^{2t}}\,dt \\
    &= -\frac{1}{16}e^{-4t} - \frac{3}{40}e^{-3t}[\sen(t) + \cos(t)]
  \end{align}
  Com isso, temos que a solução particular é
  \begin{align}
    y_p(t) &= \left\{-\frac{1}{4}t - \frac{1}{8}e^{t}[\sen(t) - \cos(t)]\right\}e^{-t} \\
    &+ \left\{-\frac{1}{16}e^{-4t} - \frac{3}{40}e^{-3t}[\sen(t) + \cos(t)]\right\}e^{3t} \\
    &= -\left(\frac{1}{16}+\frac{1}{4}t\right)e^{-t} + \frac{1}{10}\cos(t) - \frac{1}{5}\sen(t).
  \end{align}
  Concluímos que a solução geral é
  \begin{align}
    y(t) &= c_1e^{-t} + c_2e^{3t} \\
    &- \left(\frac{1}{16}+\frac{1}{4}t\right)e^{-t} + \frac{1}{10}\cos(t) - \frac{1}{5}\sen(t) \\
    &= c_1e^{-t} + c_2e^{3t} \\
    &- \frac{t}{4}e^{-t} + \frac{1}{10}\cos(t) - \frac{1}{5}\sen(t).
  \end{align}
\end{resol}

\begin{exeresol}
  Use o método dos coeficientes a determinar para obter uma solução geral de
  \begin{equation}
    y'' - 2y' - 3y = e^{-t}+\sen(t).
  \end{equation}  
\end{exeresol}
\begin{resol}
  Esta é a mesma equação \eqref{exeresol:ed2o_nh_vp} que foi resolvida no ER. \ref{exeresol:ed2o_nh_vp}. Das contas realizadas, sabemos que
  \begin{equation}
    y_1(t) = e^{-t}\quad\text{e}\quad y_2(t) = e^{3t}
  \end{equation}
  são soluções fundamentais da equação homogênea associada.

  Disso e com base no termo não homogêneo
  \begin{equation}
    g(t) = e^{-t} + \sen(t),
  \end{equation}
  buscamos por uma solução particular da forma
  \begin{equation}
    y_p(t) = Ate^{-t} + B\sen(t) + C\cos(t).
  \end{equation}
  Substituindo na EDO, obtemos
  \begin{align}
    e^{-t} + \sen(t) &= y_p'' - 2y_p' - 3y_p \\
    &= -4Ae^{-t}+(2C-4B)\sen(t)-(2B+4C)\cos(t).
  \end{align}
  Logo, devemos ter
  \begin{equation}
    -4A = 1 \Rightarrow A = -\frac{1}{4}
  \end{equation}
  e
  \begin{align}
    -4B+2C &= 1 \\
    -2B-4C &= 0
  \end{align}
  o que nos leva a $C = 1/10$ e $B = -1/5$.

  Com tudo isso, concluímos que a solução geral é
  \begin{align}
    y(t) &= c_1e^{-t} + c_2e^{3t} \\
    &- \frac{t}{4}e^{-t} - \frac{1}{5}\sen(t) + \frac{1}{10}\cos(t).    
  \end{align}
\end{resol}

\subsection*{Exercício}

\begin{flushright}
  [Vídeo] | [Áudio] | \href{https://phkonzen.github.io/notas/contato.html}{[Contatar]}
\end{flushright}

\begin{exer}
  Resolva
  \begin{equation}
    y'' + y' - 2y = e^{2t}
  \end{equation}
  usando
  \begin{enumerate}[a)]
  \item o método da variação dos parâmetros.
  \item o método dos coeficientes a determinar.
  \end{enumerate}
\end{exer}
\begin{resp}
  $y(t) = c_1e^{-2t} + c_2e^t + \frac{1}{4}e^{2t}$
\end{resp}

\begin{exer}
  Resolva
  \begin{equation}
    y'' + y' - 2y = e^{-2t}.
  \end{equation}
\end{exer}
\begin{resp}
  $y(t) = \left(c_1 - \frac{t}{3}\right)e^{-2t} + c_2e^t$
\end{resp}

\begin{exer}
  Resolva
  \begin{equation}
    y'' + 2y' + y = e^{-t}
  \end{equation}
  usando o método dos coeficientes a determinar.
\end{exer}
\begin{resp}
  $y(t) = \left(c_1 + c_2t + \frac{t^2}{2}\right)e^{-t}$
\end{resp}

\begin{exer}
  Resolva
  \begin{equation}
    y'' + 4y = t\cos(2t).
  \end{equation}
\end{exer}
\begin{resp}
  $y(t) = \left(c_1 + \frac{t}{16}\right)\cos(2t) + \left(c_2 + \frac{t^2}{8}\right)\sen(2t)$
\end{resp}

\begin{exer}
  Mostre que se $y_1 = y_1(t)$ é solução de
  \begin{equation}
    y'' + by' + cy = g_1(t)
  \end{equation}
  e $y_2 = y_2(t)$ é solução de
  \begin{equation}
    y'' + by' + cy = g_2(t),
  \end{equation}
  então $y(t) = y_1(t) + y_2(t)$ é solução de
  \begin{equation}
    y'' + by' + cy = g_1(t) + g_2(t).
  \end{equation}
\end{exer}
\begin{resp}
  Dica: Basta usar que $y_1''+by_1'+cy_1=g_1(t)$ e que $y_2''+by_2'+cy_2=g_2(t)$.
\end{resp}

\begin{exer}
  Resolva
  \begin{align}
    &y'' + 3y' + 2y = t,\quad t>0,\\
    &y(0)=0,\quad y'(0)=0.
  \end{align}
\end{exer}
\begin{resp}
  $y(t) = -\frac{1}{4}e^{-2t} + e^{-t} + \frac{t}{2} - \frac{3}{4}$
\end{resp}

\begin{exer}
  Considere um sistema massa-mola modelado por
  \begin{align}
    & ms'' + \gamma s' + ks = \cos(t),\quad t>0,\\
    & s(0) = s_0,\quad s'(0) = v_0,
  \end{align}
  onde $m>0$ é a massa, $\gamma>0$ é o coeficiente de resistência do meio, $k>0$ é a constante da mola, $s=s(t)$ é posição da massa ($s=0$ posição de repouso, $s>0$ mola esticada, $s<0$ mola contraída), $s_0$ é a posição inicial e $v_0$ é a velocidade inicial da massa.

  Supondo que $\gamma^2-4mk = 0$, o que pode se dizer sobre o comportamento de $s=s(t)$ para valores de $t$ muito grandes. 
\end{exer}
\begin{resp}
  $s(t) \approx \displaystyle \frac{1}{\gamma^2 + (k-m)^2}\left(\frac{k-m}{\gamma}\cos(t) + \sen(t)\right)$
\end{resp}

\section{EDO de ordem mais alta}\label{cap_edolin_sec_edon}

\begin{flushright}
  [Vídeo] | [Áudio] | \href{https://phkonzen.github.io/notas/contato.html}{[Contatar]}
\end{flushright}

Os métodos aplicados para EDOs lineares de segunda ordem podem ser estendidos para tratarmos EDOs lineares de ordem mais alta. Aqui, vamos nos restringir ao caso de tais EDOs com coeficientes constantes, i.e. equações da forma
\begin{equation}
  {\color{blue}a_ny^{(n)} + a_{n-1}y^{(n-1)} + \cdots + a_0y = g(x)},
\end{equation}
onde $y = y(t)$, $a_0$, $a_1$, $\dotsc$, $a_{n}$ são constantes dadas e $a_n\neq 0$.

\subsection{EDO homogênea}

\begin{flushright}
  \href{https://archive.org/details/edo-ordem-mais-alta-homogenea}{[Vídeo]} | [Áudio] | \href{https://phkonzen.github.io/notas/contato.html}{[Contatar]}
\end{flushright}

A \emph{solução geral} de uma EDO \emph{homogênea} da forma
\begin{equation}
  {\color{blue}a_ny^{(n)} + a_{n-1}y^{(n-1)} + \cdots + a_0y = 0},
\end{equation}
é dada por
\begin{equation}
  {\color{blue}y(t) = c_1y_1(t) + c_2y_2(t) + \cdots + c_ny_n(t)},
\end{equation}
sendo que $y_1$, $y_2$, $\dotsc$, $y_n$ formam um conjunto fundamental de soluções, i.e. são soluções tais que o \emph{wronskiano}\footnote{Józef Maria Hoene-Wroński, 1776 - 1853, matemático polonês. Fonte: \href{https://en.wikipedia.org/w/index.php?title=J\%C3\%B3zef_Maria_Hoene-Wro\%C5\%84ski\&oldid=939986060}{Wikipedia}.}
\begin{equation}
  W(y_1,y_2,\dotsc,y_n;t) =
  \begin{vmatrix}
    y_1 & y_2 & \cdots & y_n \\
    y_1' & y_2' & \cdots & y_n' \\
    \vdots & \vdots & \vdots & \vdots \\
    y_1^{(n-1)} & y_2^{(n-1)} & \cdots & y_n^{(n-1)}
  \end{vmatrix} \neq 0.
\end{equation}


Podemos obter soluções particulares da forma
\begin{equation}
  {\color{blue}y(t) = e^{rt}}.
\end{equation}
Substituindo na EDO, vemos que $r$ deve satisfazer a \emph{equação característica}
\begin{equation}
  {\color{blue}a_nr^n + a_{n-1}r^{n-1} + \cdots + a_0 = 0}.
\end{equation}
Cada raiz nos fornece uma solução particular $y_p=y_p(t)$:
\begin{enumerate}[a)]
\item se $r = r_p$ é \emph{raiz simples}, então
  $y_p(t) = e^{r_pt}$
\item se $r = r_p$ \emph{raiz de multiplicidade $m$}, então
  \begin{align}
    y_{p+1}(t) &= e^{r_pt} \\
    y_{p+2}(t) &= te^{r_pt} \\
    &\vdots \\
    y_{p+m} &= t^{m-1}e^{r_pt}
  \end{align}
\item se $r = \lambda_p\pm i\mu_p$ é \emph{raiz complexa}, então
  \begin{align}
    y_{p+1} &= e^{\lambda_pt}\cos(\mu_pt) \\
    y_{p+2} &= e^{\lambda_pt}\sen(\mu_pt).
  \end{align}
\end{enumerate}

\begin{ex}\label{ex:edolin_on_h}
  Vamos calcular a solução geral de
  \begin{equation}
    y''' + 2y'' - y' - 2y = 0.
  \end{equation}
  A \emph{equação característica associada} é
  \begin{equation}
    {\color{blue}r^3 + 2r^2 - r - 2} = 0.
  \end{equation}
  Queremos calcular as raízes do polinômio característico. Para tanto, começamos observando que ${\color{red}r=1}$ é raiz, pois $1^3+2\cdot 1^2 - 1 - 2 = 0$. Logo, o polinômio é divisível pelo monômio ${\color{red}(r-1)}$. Calculando a divisão
  \begin{align}
    &\;\;{\color{blue}r^3 + 2r^2 - r - 2} \,|\underline{{\color{red}r-1}\quad}\\
    &\underline{-r^3 + r^2}\quad\quad\quad\quad\;{\color{cyan}r^2+3r+2}\\
    &\quad\quad\; 3r^2 - r\\
    &\quad\; \underline{-3r^2 + 3r}\\
    &\quad\quad\quad\quad 2r - 2\\
    &\quad\quad\quad\underline{-2r + 2}\\
    &\quad\quad\quad\quad\quad\quad 0
  \end{align}
  obtemos, para $r\neq 1$,
  \begin{equation}
    \frac{{\color{blue}r^3 + 2r^2 - r - 2}}{{\color{red}r-1}} = {\color{cyan}r^2+3r+2}
  \end{equation}
  ou, equivalentemente,
  \begin{equation}
    {\color{blue}r^3 + 2r^2 - r - 2} = {\color{red}(r-1)}{\color{cyan}(r^2+3r+2)}
  \end{equation}
  Portanto, as outras raízes do polinômio característico ocorrem quando
  \begin{gather}
    r^2 + 3r + 2 = 0\\
    r = \frac{-3 \pm \cancelto{1}{\sqrt{9-4\cdot 1\cdot 2}}}{2}\\
    {\color{red}r = -2},\quad\text{ou}\quad {\color{red}r = -1}.
  \end{gather}
  Concluímos que as soluções da equação característica são ${\color{red}r_1 = -2}$, ${\color{red}r_2 = -1}$ e ${\color{red}r_3=1}$. Logo, a \emph{solução geral} é
  \begin{equation}
    y(t) = c_1e^{-2t} + c_2e^{-t} + c_3e^t.
  \end{equation}

  Vamos, ainda, verificar se $y_1(t) = e^{-2t}$, $y_2(t) = e^{-t}$ e $y_3(t) = e^t$ formam um conjunto fundamental de soluções. Por construção, sabemos que estas são soluções particulares. Resta, portanto, verificar que o wronskiano é não nulo. De fato, temos
  \begin{align}
    W(y_1, y_2, y_3; t) &=
    \begin{vmatrix}
      y_1 & y_2 & y_3 \\
      y_1' & y_2' & y_3' \\
      y_1'' & y_2'' & y_3''
    \end{vmatrix} \\
    &=
    \begin{vmatrix}
      e^{-2t} & e^{-t} & e^t \\
      -2e^{-2t} & -e^{-t} & e^t \\
      4e^{-2t} & e^{-t} & e^t
    \end{vmatrix} \\
    &= 6e^{-2t} \neq 0,\quad \forall t.
  \end{align}

  \ifispython
  No \python, podemos computar a solução geral com os seguintes comandos:
  \begin{lstlisting}
    In : from sympy import *
    In : t,r = symbols('t,r')
    In : y = symbols('y', cls=Function)

    In : edo = y(t).diff(t,3) + 2*y(t).diff(t,2) \
    ...:     - y(t).diff(t) - 2*y(t)  
    In : dsolve(edo,y(t))
    Out: Eq(y(t), C1*exp(-2*t) + C2*exp(-t) + C3*exp(t))
  \end{lstlisting}
  Então, para computarmos o wronskiano, podemos usar os seguintes comandos:
  \begin{lstlisting}
    In : y1 = exp(-2*t)
    In : y2 = exp(-t)
    In : y3 = exp(t)
    In : W = Matrix([[y1,y2,y3], \
      [y1.diff(t),y2.diff(t),y3.diff(t)], \
      [y1.diff(t,2),y2.diff(t,2),y3.diff(t,2)]])
    In : W.det()
    Out: 6*exp(-2*t)
  \end{lstlisting}
  \fi
\end{ex}

\begin{ex}
  Vamos encontrar a solução geral de
  \begin{equation}
    y^{(4)} + 2y'' - 8y' + 5y = 0.
  \end{equation}
  As raízes da \emph{equação característica} associada\footnote{Observando que $r=1$ é solução da equação, podemos usar o método de redução do grau utilizado no exemplo anterior.}
  \begin{equation}
    r^4 + 2r^2 - 8r + 5 = 0
  \end{equation}
  são $r_1, r_2 = 1$ ou $r_3, r_4 = -1 \pm 2i$. Logo, temos as seguintes soluções particulares
  \begin{gather}
    y_1(t) = e^t,\quad y_2(t) = te^t, \\
    y_3(t) = e^{-t}\cos(2t),\quad y_4(t) = e^{-t}\sen(2t).
  \end{gather}
  Calculando o wronskiano, temos
  \begin{align}
    W(y_1,y_2,y_3,y_4;t) &=
    \begin{vmatrix}
      y_1 & y_2 & y_3 & y_4 \\
      y_1' & y_2' & y_3' & y_4' \\
      y_1'' & y_2'' & y_3'' & y_4'' \\
      y_1''' & y_2''' & y_3''' & y_4''' \\
    \end{vmatrix}\\
    &= 128 \neq 0.
  \end{align}
  Logo, concluímos que a solução geral é
  \begin{equation}
    y(t) = (c_1 + c_2t)e^t + e^{-t}[c_3\cos(2t) + c_4\sen(2t)].
  \end{equation}
\end{ex}

\subsection{Equação não homogênea}

\begin{flushright}
  [Vídeo] | [Áudio] | \href{https://phkonzen.github.io/notas/contato.html}{[Contatar]}
\end{flushright}

Ambos os métodos da variação dos parâmetros e dos coeficientes a determinar podem ser generalizados para EDOs lineares de ordem mais altas, com coeficientes constantes e não homogêneas. Tais equações têm a forma
\begin{equation}
  y^{(n)} + a_{n-1}y^{(n-1)} + \cdots + a_0y = g(t),
\end{equation}
com $n\geq 1$ e $g(t)\not\equiv 0$. A \emph{solução geral} pode ser escrita na forma
\begin{equation}
  y(t) = c_1y_1(t) + c_2y_2(t) + \cdots + c_ny_n(t) + y_p(t),
\end{equation}
onde $y_1$, $y_2$, $\dotsc$, $y_n$ formam um conjunto fundamental de soluções para a equação homogênea associada e $y_p$ é uma solução particular qualquer da equação não homogênea.

\subsubsection{Método da variação dos parâmetros}

\begin{flushright}
  \href{https://archive.org/details/edo-ordem-n-mvp}{[Vídeo]} | [Áudio]  | \href{https://phkonzen.github.io/notas/contato.html}{[Contatar]}
\end{flushright}

Seja a equação diferencial ordinária
\begin{equation}\label{eq:edolin_on_varpar}
  y^{(n)} + a_{n-1}y^{(n-1)} + \cdots + a_0y = g(t),
\end{equation}
com $n\geq 1$ e $g(t)\not\equiv 0$. Seja, também, $y_1$, $y_2$, $\dotsc$, $y_n$ um conjunto fundamental de soluções da equação homogênea associada. O \emph{método da variação dos parâmetros} consiste em buscar uma solução particular para \eqref{eq:edolin_on_varpar} com o seguinte formato
\begin{equation}
  y_p(t) = u_1(t)y_1(t)+u_2(t)y_2(t)+\cdots +u_n(t)y_n(t),
\end{equation}
onde $u_1$, $u_2$, $\dotsc$, $u_n$ são funções parâmetros a determinar.

Os parâmetros $u_1$, $u_2$, $\dotsc$, $u_n$ devem ser escolhidos de forma que $y_p$ satisfaça \eqref{eq:edolin_on_varpar}. Quando $n>1$, temos um problema subdeterminado (mais variáveis que equações). Para fechar o problema, exigimos que os parâmetros sejam tais que
\begin{align}
  u_1'y_1+u_2'y_2+\cdots +u_n'y_n &= 0 \label{eq:edolin_on_varpar_eq0}\\
  u_1'y_1'+u_2'y_2'+\cdots +u_n'y_n' &= 0 \\
  &\vdots \\
  u_1'y_1^{(n-2)}+u_2'y_2^{(n-2)}+\cdots +u_n'y_n^{(n-2)} &= 0.\label{eq:edolin_on_varpar_eqn-1}
\end{align}
Com isso, ao substituirmos $y_p$ em \eqref{eq:edolin_on_varpar}, obtemos
\begin{align}
  g(t) &= a_0y_p+a_1y_p'+a_2y_p''+\cdots +y_p^{(n)}\\
  &= a_0(u_1y_1+u_2y_2+\cdots +u_ny_n) \\
  &+ a_1(u_1y_1'+u_2y_2'+\cdots +u_ny_n') \\
  &+ a_2(u_1y_1''+u_2y_2''+\cdots +u_ny_n'') \\
  &+ \cdots + \\
  &+ (u_1y_1^{(n)}+u_2y_2^{(n)}+\cdots +u_ny_n^{(n)}) \\
  &+ (u_1'y_1^{(n-1)}+u_2'y_2^{(n-1)}+\cdots +u_n'y_n^{(n-1)}).
\end{align}
Lembrando que $y_1$, $y_2$, $\dotsc$, $y_n$ são soluções da equação homogênea associada, esta última equação é equivalente a
\begin{equation}\label{eq:edolin_on_varpar_eqn}
  u_1'y_1^{(n-1)}+u_2'y_2^{(n-1)}+\cdots +u_n'y_n^{(n-1)} = g(t).
\end{equation}

Desta forma, concluímos que os parâmetros podem ser escolhidos de forma a satisfazerem o sistema de equações \eqref{eq:edolin_on_varpar_eq0}-\eqref{eq:edolin_on_varpar_eqn-1} e \eqref{eq:edolin_on_varpar_eqn}, i.e.
\begin{align}
  u_1'y_1+u_2'y_2+\cdots +u_n'y_n &= 0 \\
  u_1'y_1'+u_2'y_2'+\cdots +u_n'y_n' &= 0 \\
  &\vdots \\
  u_1'y_1^{(n-2)}+u_2'y_2^{(n-2)}+\cdots +u_n'y_n^{(n-2)} &= 0 \\
  u_1'y_1^{(n-1)}+u_2'y_2^{(n-1)}+\cdots +u_n'y_n^{(n-1)} &= g.
\end{align}
Usando o método de Cramer\footnote{Gabriel Cramer, 1704 - 1752, matemático suíço. Fonte: \href{https://en.wikipedia.org/wiki/Gabriel_Cramer}{Wikipedia}.} temos
\begin{equation}\label{eq:edolin_on_varpar_ul}
  u_m'(t) = \frac{g(t)W_m(t)}{W(t)},
\end{equation}
onde $m=1,2,\dotsc,n$, $W(t)$ denota o wronskiano
\begin{equation}
  W(y_1,y_2,\cdots,y_n;t) =
  \begin{vmatrix}
    y_1 & y_2 & \cdots & y_n \\
    y_1' & y_2' & \cdots & y_n' \\
    \vdots & \vdots & \cdots & \vdots \\
    y_1^{(n-1)} & y_2^{(n-1)} & \cdots & y_n^{(n-1)}
  \end{vmatrix}
\end{equation}
e $W_m(t)$ é o determinante obtido de $W$ substituindo a $m$-ésima coluna por vetor $(0, 0, \dotsc, 1)$.

Por fim, integrando \eqref{eq:edolin_on_varpar_ul}, obtemos os parâmetros
\begin{equation}
  u_m(t) = \int \frac{g(t)W_m(t)}{W(t)}\,dt,
\end{equation}
com $m=1,2,\dotsc,n$.

\begin{ex}\label{ex:edlin_on_nh_vv}
  Vamos calcular a solução geral de
  \begin{equation}\label{eq:ex_edolin_on_nh_vv}
    y''' + 2y'' - y' - 2y = e^{2t}.
  \end{equation}

  No Exemplo \ref{ex:edolin_on_h}, vimos que
  \begin{equation}
    y_1(t) = e^{-2t},\quad y_2(t)=e^{-t},\quad y_3(t)=e^t
  \end{equation}
  formam um conjunto fundamental de soluções para a equação homogênea associada. Com isso, nos resta calcular uma solução particular $y_p=y_p(t)$ para a equação não homogênea.

  Pelo método da variação dos parâmetros, podemos calcular como
  \begin{equation}
    y_p(t) = u_1(t)y_1(t) + u_2(t)y_2(t) + u_3(t)y_3(t),
  \end{equation}
  onde
  \begin{equation}
    u_m(t) = \int \frac{g(t)W_m(t)}{W(t)}\,dt,
  \end{equation}
  sendo $g(t) = e^{2t}$ e $m=1,2,3$. Ainda, temos
  \begin{align}
    W(t) &=
    \begin{vmatrix}
      y_1 & y_2 & y_3 \\
      y_1' & y_2' & y_3' \\
      y_1'' & y_2'' & y_3''
    \end{vmatrix} \\
    &=
    \begin{vmatrix}
      e^{-2t} & e^{-t} & e^t \\
      -2e^{-2t} & -e^{-t} & e^t \\
      4e^{-2t} & e^{-t} & e^t
    \end{vmatrix} \\
    &= 6e^{-2t}
  \end{align}
  \begin{align}
    W_1(t) &=
    \begin{vmatrix}
      0 & y_2 & y_3 \\
      0 & y_2' & y_3' \\
      1 & y_2'' & y_3''
    \end{vmatrix} \\
    &= (-1)^{3+1}
    \begin{vmatrix}
      y_2 & y_3 \\
      y_2' & y_3'
    \end{vmatrix} \\
    &=
    \begin{vmatrix}
      e^{-t} & e^t \\
      -e^{-t} & e^t
    \end{vmatrix} \\
    &= 2
  \end{align}
  \begin{align}
    W_2(t) &=
    \begin{vmatrix}
      y_1 & 0 & y_3 \\
      y_1' & 0 & y_3' \\
      y_1'' & 1 & y_3''
    \end{vmatrix} \\
    &= (-1)^{3+2}
    \begin{vmatrix}
      y_1 & y_3 \\
      y_1' & y_3'
    \end{vmatrix} \\
    &= -
    \begin{vmatrix}
      e^{-2t} & e^t \\
      -2e^{-2t} & e^t
    \end{vmatrix} \\
    &= -3e^{-t}
  \end{align}
  \begin{align}
    W_3(t) &=
    \begin{vmatrix}
      y_1 & y_2 & 0 \\
      y_1' & y_2' & 0 \\
      y_1'' & y_2'' & 1
    \end{vmatrix} \\
    &= (-1)^{3+3}
    \begin{vmatrix}
      y_1 & y_2 \\
      y_1' & y_2'
    \end{vmatrix} \\
    &=
    \begin{vmatrix}
      e^{-2t} & e^{-t} \\
      -2e^{-2t} & -e^{-t}
    \end{vmatrix} \\
    &= e^{-3t}
  \end{align}
  Logo,
  \begin{align}
    u_1(t) &= \int \frac{g(t)W_1(t)}{W(t)}\,dt \\
    &= \int \frac{2e^{2t}}{6e^{-2t}}\,dt \\
    &= \frac{1}{3}\int e^{4t}\,dt \\
    &= \frac{1}{12}e^{4t}
  \end{align}
  \begin{align}
    u_2(t) &= \int \frac{g(t)W_2(t)}{W(t)}\,dt \\
    &= \int \frac{e^{2t}\cdot (-3)\cdot e^{-t}}{6e^{-2t}}\,dt \\
    &= -\frac{1}{2}\int e^{3t}\,dt \\
    &= -\frac{1}{6}e^{3t}
  \end{align}
  \begin{align}
    u_3(t) &= \int \frac{g(t)W_3(t)}{W(t)}\,dt \\
    &= \int \frac{e^{2t}e^{-3t}}{6e^{-2t}}\,dt \\
    &= \frac{1}{6}\int e^{t}\,dt \\
    &= \frac{1}{6}e^{t}
  \end{align}
  Com isso,
  \begin{align}
    y_p &= u_1y_1 + u_2y_2 + u_3y_3 \\
    &= \frac{1}{12}e^{4t}e^{-2t} - \frac{1}{6}e^{3t}e^{-t}+\frac{1}{6}e^te^t \\
    &= \frac{1}{12}e^{2t}.
  \end{align}

  Concluímos que a solução geral de \eqref{eq:ex_edolin_on_nh_vv} é
  \begin{equation}
    y(t) = c_1e^{-2t}+c_2e^{-t}+c_3e^t+\frac{1}{12}e^{2t}.
  \end{equation}

  \ifispython
  No \python, podemos computar a solução geral com os seguintes comandos:
  \begin{lstlisting}
    In : from sympy import *
    In : t = symbols('t')
    In : y = symbols('y', cls=Function)

    In : eq = Eq(diff(y(t),t,3) + 2*diff(y(t),t,2) \
    ...:    - diff(y(t),t)-2*y(t), exp(2*t))
    In : dsolve(eq,y(t))
    Out: Eq(y(t), C1*exp(-2*t) + C2*exp(-t) + C3*exp(t) \
    + exp(2*t)/12)
  \end{lstlisting}
  \fi
\end{ex}


\subsubsection{Método dos coeficientes a determinar}

\begin{flushright}
  \href{https://archive.org/details/edo-ordem-n-mcd}{[Vídeo]} | [Áudio] | \href{https://phkonzen.github.io/notas/contato.html}{[Contatar]}
\end{flushright}

A aplicação na resolução de EDOs de ordem mais alta do método dos coeficientes a determinar é análoga ao caso de EDOs de segunda ordem (veja a Subseção \ref{subsec:edolin_o2_mcc}).

\begin{ex}
  Vamos calcular a solução geral de
  \begin{equation}\label{eq:ex_edolin_on_nh_cc}
    y''' + 2y'' - y' - 2y = e^{2t}.
  \end{equation}

  No Exemplo \ref{ex:edolin_on_h}, vimos que
  \begin{equation}
    y_1(t) = e^{-2t},\quad y_2(t)=e^{-t},\quad y_3(t)=e^t
  \end{equation}
  formam um conjunto fundamental de soluções para a equação homogênea associada. Com isso, nos resta calcular uma solução particular $y_p=y_p(t)$ para a equação não homogênea.

  Com base no termo fonte $g(t) = e^{2t}$, buscamos por uma solução particular da forma
  \begin{equation}
    y_p(t) = Ae^{2t},
  \end{equation}
  onde $A$ é um coeficiente a determinar.

  Para determinar $A$, substituímos $y_p$ na \eqref{eq:ex_edolin_on_nh_cc}, donde
  \begin{align}
    e^{2t} &= y_p''' + 2y_p'' - y_p' - 2y_p \\
    &= 8Ae^{2t}+8Ae^{2t}-2Ae^{2t}-2Ae^{2t} \\
    &= 12Ae^{2t}.
  \end{align}
  Com isso, temos $A = 1/12$ e
  \begin{equation}
    y_p(t) = \frac{1}{12}e^{2t}.
  \end{equation}

  Concluímos que a solução geral é
  \begin{equation}
    y(t) = c_1e^{-2t} + c_2e^{-t} + c_3e^t + \frac{1}{12}e^{2t}.
  \end{equation}
\end{ex}


\subsection*{Exercícios resolvidos}

\begin{flushright}
  [Vídeo] | [Áudio] | \href{https://phkonzen.github.io/notas/contato.html}{[Contatar]}
\end{flushright}

\begin{exeresol}\label{exeresol:edolin_on_h}
  Resolva o PVI
  \begin{align}
    y''' + 3y'' + 3y' + y = 0,\quad t>0,\\
    y(0)=1,\quad y'(0)=0,\quad y''(0)=0.
  \end{align}
\end{exeresol}
\begin{resol}
  A equação característica associada
  \begin{equation}
    r^3 + 3r^2 + 3r + 1 = 0
  \end{equation}
  tem raiz tripla $r=-1$. Logo, a solução geral é
  \begin{equation}
    y(t) = (c_1 + c_2t + c_3t^2)e^{-t}.
  \end{equation}
  Vamos agora aplicar as condições iniciais. Começamos com $y(0)=1$. Substituindo $t=0$ na solução geral, obtemos
  \begin{equation}
    y(0) = c_1,
  \end{equation}
  logo, ${\color{blue}c_1 = 1}$. Para usarmos a condição inicial $y'(0)=0$, usamos a derivada da solução geral, i.e.
  \begin{equation}
    y'(t) = (-c_1 + c_2 + (2c_3-c_2)t - c_3t^2)e^{-t}.
  \end{equation}
  Então, temos
  \begin{gather}
    0 = y'(0) \\
    0  = -c_1 + c_2 \\
    0  = -1 + c_2 \\
    {\color{blue}c_2 = 1}.
  \end{gather}
  Por fim, para usarmos a condição inicial $y''(0)=0$, usamos a segunda derivada da solução geral como segue.
  \begin{gather}
    y''(t) = (2c_3-2c_2+c_1 + (c_2 - 4c_3)t + c_3t^2)e^{-t}\\
    0=y''(0)=2c_3-2c_2+c_1\\
    0=2c_3-2+1\\
    2c_3=1\\
    {\color{blue}c_3=\frac{1}{2}}.
  \end{gather}
  Logo, a solução do PVI é
  \begin{equation}
    y(t) = \left(1 + t + \frac{1}{2}t^2\right)e^{-t}.
  \end{equation}
\end{resol}

\begin{exeresol}
  Use o método da variação dos parâmetros para calcular a solução geral de
  \begin{equation}\label{eq:exeresol_edolin_on_nh_vv}
    y''' + 3y'' + 3y' + y = 2t
  \end{equation}
\end{exeresol}
\begin{resol}
  No ER \ref{exeresol:edolin_on_h}, vimos que
  \begin{align}
    y_1(t) = e^{-t},\\
    y_2(t) = te^{-t},\\
    y_3(t) = t^2e^{-t}
  \end{align}
  formam um conjunto fundamental de soluções para a equação homogênea associada a \eqref{eq:exeresol_edolin_on_nh_vv}.

  A aplicação do método da variação dos parâmetros consiste em calcular uma solução particular para \eqref{eq:exeresol_edolin_on_nh_vv} da forma 
  \begin{equation}
    y_p(t) = u_1(t)y_1(t)+u_2(t)y_2(t)+u_3(t)y_3(t).
  \end{equation}

  Para calcularmos os parâmetros $u_1$, $u_2$ e $u_3$, precisamos dos seguintes determinantes:
  \begin{align}
    W(y_1,y_2,y_3;t) &=
    \begin{vmatrix}
      y_1 & y_2 & y_3 \\
      y_1' & y_2' & y_3' \\
      y_1'' & y_2'' & y_3''
    \end{vmatrix}\\
    &= 2e^{-3t}
  \end{align}
  \begin{align}
    W_1(t) &=
    \begin{vmatrix}
      0 & y_2 & y_3 \\
      0 & y_2' & y_3' \\
      1 & y_2'' & y_3''
    \end{vmatrix}\\
    &= t^2e^{-2t}
  \end{align}
  \begin{align}
    W_2(t) &=
    \begin{vmatrix}
      y_1 & 0 & y_3 \\
      y_1' & 0 & y_3' \\
      y_1'' & 1 & y_3''
    \end{vmatrix}\\
    &= -2te^{-2t}
  \end{align}
  \begin{align}
    W_3(t) &=
    \begin{vmatrix}
      y_1 & y_2 & 0 \\
      y_1' & y_2' & 0 \\
      y_1'' & y_2'' & 1
    \end{vmatrix}\\
    &= e^{-2t}
  \end{align}

  Com isso, temos
  \begin{align}
    u_1(t) &= \int \frac{g(t)W_1(t)}{W(t)}\,dt \\
    &= \int \frac{2t\cdot t^2e^{-2t}}{2e^{-3t}}\,dt \\
    &= (t^3 - 3t^2 + 6t -6)e^t
  \end{align}
  \begin{align}
    u_2(t) &= \int \frac{g(t)W_2(t)}{W(t)}\,dt \\
    &= -\int \frac{2t\cdot 2te^{-2t}}{2e^{-3t}}\,dt \\
    &= (-2t^2 + 4t - 4)e^t
  \end{align}
  \begin{align}
    u_3(t) &= \int \frac{g(t)W_3(t)}{W(t)}\,dt \\
    &= \int \frac{2t\cdot e^{-2t}}{2e^{-3t}}\,dt \\
    &= (t-1)e^t
  \end{align}

  Logo, obtemos a seguinte solução particular para \eqref{eq:exeresol_edolin_on_nh_vv}
  \begin{equation}
    y_p(t) = 2t - 6.
  \end{equation}
  Concluímos que
  \begin{equation}
    y(t) = (c_1 + c_2t + c_3t^2)e^{-t} + 2t - 6
  \end{equation}
  é solução geral de \eqref{eq:exeresol_edolin_on_nh_vv}.

  \ifispython
  No \python, podemos resolver este exercício com o seguinte código:
  \begin{lstlisting}
    from sympy import *

    # def. das variaveis
    t = symbols('t')
    y = symbols('y', cls=Function)

    # solucoes fundamentais
    y1 = exp(-t)
    y2 = t*exp(-t)
    y3 = t**2*exp(-t)

    # wronskiano
    WM = Matrix([[y1,y2,y3], \
      [diff(y1,t),diff(y2,t),diff(y3,t)], \
      [diff(y1,t,2),diff(y2,t,2),diff(y3,t,2)]])
    W = simplify(WM.det())
    print('W = ', W)

    WM1 = Matrix([[0,y2,y3], \
      [0,diff(y2,t),diff(y3,t)], \
      [1,diff(y2,t,2),diff(y3,t,2)]])
    W1 = simplify(WM1.det())
    print('W1 = ', W1)

    WM2 = Matrix([[y1,0,y3], \
      [diff(y1,t),0,diff(y3,t)], \
      [diff(y1,t,2),1,diff(y3,t,2)]])
    W2 = simplify(WM2.det())
    print('W2 = ', W2)

    WM3 = Matrix([[y1,y2,0], \
      [diff(y1,t),diff(y2,t),0], \
      [diff(y1,t,2),diff(y2,t,2),1]])
    W3 = simplify(WM3.det())
    print('W3 = ', W3)

    # fonte
    g = 2*t

    # parametros
    u1 = integrate(g*W1/W,t)
    print('u1(t) = ', u1)

    u2 = integrate(g*W2/W,t)
    print('u2(t) = ', u2)

    u3 = integrate(g*W3/W,t)
    print('u3(t) = ', u3)

    # solucao particular
    yp = simplify(u1*y1 + u2*y2 + u3*y3)
    print('yp(t) = ', yp)

    # constantes indeterminadas
    C1,C2,C3 = symbols('C1,C2,C3')

    # sol. geral
    y = simplify(C1*y1 + C2*y2 + C3*y3 + yp)
    print('y(t) = ', y)
  \end{lstlisting}
\end{resol}

\begin{exeresol}
  Use o método dos coeficientes a determinar para obter uma solução particular de
  \begin{equation}\label{eq:exeresol_edolin_on_nh_cd}
    y''' + 3y'' + 3y' + y = \cos(t)
  \end{equation}
\end{exeresol}
\begin{resol}
  Tendo em vista o fonte $g(t) = \cos(t)$, as soluções fundamentais obtidas no ER \ref{exeresol:edolin_on_h} e a Observação \ref{obs:edolin_o2_mcd}, a aplicação do método dos coeficientes a determinar consiste em calcular uma solução particular da forma
  \begin{equation}
    y_p(t) = A\cos(t) + B\sen(t),
  \end{equation}
  onde $A$ e $B$ são coeficientes a determinar.

  Substituindo $y_p$ em \eqref{eq:exeresol_edolin_on_nh_cd}, obtemos
  \begin{align}
    \cos(t) &= y_p''' + 3y_p'' + 3y_p' + y_p \\
    &= (2B-2A)\cos(t) - (2A + 2B)\sen(t).
  \end{align}
  Segue que
  \begin{align}
    2B - 2A &= 1 \\
    2B + 2A &= 0.
  \end{align}
  Resolvendo, obtemos $A = -1/4$ e $B = 1/4$. Concluímos que uma solução particular para \eqref{eq:exeresol_edolin_on_nh_cd} é
  \begin{equation}
    y_p(t) = -\frac{1}{4}\cos(t) + \frac{1}{4}\sen(t).
  \end{equation}

  \ifispython
  No \python, podemos resolver este exercício com os seguintes comandos:
  \begin{lstlisting}
    In : from sympy import *
    In : # def. das variaveis
    In : t,A,B = symbols('t,A,B')
    In : y = symbols('y', cls=Function)
    In : # sol. particular
    In : yp = A*cos(t) + B*sin(t)
    In : # subs. na EDO
    In : Eq(diff(yp,t,3)+3*diff(yp,t,2)+3*diff(yp,t)+yp,cos(t))
    Out: Eq(-2*A*sin(t) + A*cos(t) + B*sin(t) + 2*B*cos(t) \
    - 3*(A*cos(t) + B*sin(t)), cos(t))

    In : factor(_,[cos(t),sin(t)])
    Out: Eq(-2*((A - B)*cos(t) + (A + B)*sin(t)), cos(t))

    In : # resolve o sistema
    In : solve([Eq(-2*(A-B),1),Eq(-2*(A+B),0)])
    Out: {A: -1/4, B: 1/4}
  \end{lstlisting}
  \fi
\end{resol}

\subsection*{Exercícios}

\begin{flushright}
  [Vídeo] | [Áudio] | \href{https://phkonzen.github.io/notas/contato.html}{[Contatar]}
\end{flushright}

\begin{exer}
  Calcule a solução geral de
  \begin{equation}
    y''' + 2y'' - 5y' - 6y = 0.
  \end{equation}
\end{exer}
\begin{resp}
  $y(t) = c_1 e^{- 3 t} + c_{2} e^{- t} + c_{3} e^{2 t}$
\end{resp}

\begin{exer}
  Calcule a solução geral de
  \begin{equation}
    y''' - 3y'- 2y = 0.
  \end{equation}
\end{exer}
\begin{resp}
  $y(t) = (c_1 + c_2t)e^{-t} + c_3e^{2t}$
\end{resp}

\begin{exer}
  Calcule a solução geral de
  \begin{equation}
    y''' - y'' + 2y = 0.
  \end{equation}
\end{exer}
\begin{resp}
  $y(t) = c_1e^{-t} + e^{t}[c_2\cos(t) + c_3\sen(t)]$
\end{resp}

\begin{exer}
  Calcule a solução geral de
  \begin{equation}
    y^{(4)}-2y'''-3y''+8y'-4y=0
  \end{equation}
\end{exer}
\begin{resp}
  $y(t) = c_1e^{-2t} + (c_2+c_3t)e^{t}+c_4e^{2t}$
\end{resp}

\begin{exer}
  Calcule a solução do PVI
  \begin{align}
    y''' + 2y'' - 5y' - 6y = 0, t>0\\
    y(0)=1,\quad y'(0)=-2,\quad y''(0)=0.
  \end{align}
\end{exer}
\begin{resp}
  $y(t) = \frac{4}{3}e^{-t} - \frac{1}{3}e^{2t}$
\end{resp}

\begin{exer}
  Calcule uma solução particular de
  \begin{equation}
    y''' + 2y'' - 5y' - 6y = 2e^t
  \end{equation}
  \begin{enumerate}[a)]
  \item pelo método da variação dos parâmetros.
  \item pelo método dos coeficientes a determinar.
  \end{enumerate}
\end{exer}
\begin{resp}
  $y_p(t) = -\frac{1}{4}e^{t}$.
\end{resp}

\begin{exer}
  Calcule uma solução particular de
  \begin{equation}
    y''' + 2y'' - 5y' - 6y = 2e^{-t}
  \end{equation}
  \begin{enumerate}[a)]
  \item pelo método da variação dos parâmetros.
  \item pelo método dos coeficientes a determinar.
  \end{enumerate}
\end{exer}
\begin{resp}
  $y_p(t) = -\frac{t}{3}e^{-t}$.
\end{resp}

\begin{exer}
  Calcule uma solução particular de
  \begin{equation}
    y''' - y'' + 2y = 10\sen(t).
  \end{equation}
  \begin{enumerate}[a)]
  \item pelo método da variação dos parâmetros.
  \item pelo método dos coeficientes a determinar.
  \end{enumerate}
\end{exer}
\begin{resp}
  $y_p(t) = 3\sen(t) + \cos(t)$.
\end{resp}

\begin{exer}
  Calcule uma solução particular de
  \begin{equation}
    y''' - y'' + 2y = 10(1 + e^t)\sen(t).
  \end{equation}
  \begin{enumerate}[a)]
  \item pelo método da variação dos parâmetros.
  \item pelo método dos coeficientes a determinar.
  \end{enumerate}
\end{exer}
\begin{resp}
  $y_p(t) = 3\sen(t) + \cos(t) - te^{t}(2\cos(t) + \sen(t))$
\end{resp}
