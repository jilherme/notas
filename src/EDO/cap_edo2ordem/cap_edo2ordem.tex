%Este trabalho está licenciado sob a Licença Atribuição-CompartilhaIgual 4.0 Internacional Creative Commons. Para visualizar uma cópia desta licença, visite http://creativecommons.org/licenses/by-sa/4.0/deed.pt_BR ou mande uma carta para Creative Commons, PO Box 1866, Mountain View, CA 94042, USA.

\chapter{EDO linear de segunda ordem}\label{cap_edo2ordem}
\thispagestyle{fancy}

\section{EDO homogênea com coeficientes constantes - parte I}\label{cap_edo2ordem_sec_eqhom_coefconst1}

Uma \emph{EDO de segunda ordem homogênea com coeficientes constantes} tem a forma
\begin{equation}\label{eq:edo2_hom_coefconst}
  {\color{blue}ay'' + by' + cy = 0},
\end{equation}
onde $y = y(t)$ e $a, b, c$ são parâmetros constantes (números reais).

Vamos buscar por soluções da forma ${\color{blue}y(t) = e^{rt}}$, onde $r$ é constante. Substituindo na equação \eqref{eq:edo2_hom_coefconst}, obtemos
\begin{align}
  & a\left(e^{rt}\right)'' + b\left(e^{rt}\right)' + ce^{rt} = 0 \\
  & \Rightarrow ar^2e^{rt} + bre^{rt} + ce^{rt} = 0 \\
  & \Rightarrow {\color{blue}(ar^2 + br + c)e^{rt} = 0}.
\end{align}
Ou seja, $y(t) = e^{rt}$ é solução de \eqref{eq:edo2_hom_coefconst} quando
\begin{equation}
  {\color{blue}ar^2 + br + c = 0}.
\end{equation}
Esta é chamada de \emph{equação característica} de \eqref{eq:edo2_hom_coefconst}.

\begin{ex}\label{ex:ed2o_homcc}
  Vamos buscar por soluções de
  \begin{equation}\label{eq:ex_ed2o_hom_coefconst1}
    y'' - 4y = 0.
  \end{equation}
  Buscamos por $r$ tal que $y(t) = e^{rt}$ seja solução desta equação. Substituindo na equação, obtemos
  \begin{align}
    \left(e^{rt}\right)'' - 4e^{rt} = 0 \Rightarrow (r^2 - 4)e^{rt} = 0
  \end{align}
  o que nos fornece a equação característica
  \begin{equation}
    r^2 - 4 = 0.
  \end{equation}
  As soluções desta equação são $r_1 = -2$ e $r_2 = 2$. Ou seja, obtemos as seguintes \emph{soluções particulares} da EDO
  \begin{equation}
    y_1(t) = e^{-2t}\quad\text{e}\quad y_2(t) = e^{2t}.
  \end{equation}

  Observamos, ainda, que para quaisquer constantes $c_1$ e $c_2$,
  \begin{equation}
    y(t) = c_1e^{-2t} + c_2e^{2t}
  \end{equation}
  também é solução da EDO \eqref{eq:ex_ed2o_hom_coefconst1}. De fato, temos
  \begin{align}
    y'' - 4y &= \left(c_1e^{-2t} + c_2e^{2t}\right)'' -4\left(c_1e^{-2t}+c_2e^{2t}\right) \\
             &= 4c_1e^{-2t} + 4c_2e^{2t} -4c_1e^{-2t} -4c_2e^{2t} \\
             &= 0.
  \end{align}

  Como veremos logo mais,
  \begin{equation}
    y(t) = c_1e^{-2t} + c_2e^{2t}
  \end{equation}
  é a \emph{solução geral} de \eqref{eq:ex_ed2o_hom_coefconst1}.
\end{ex}

\subsection{Conjunto fundamental de solução}

Sejam $y_1 = y_1(t)$ e $y_2 = y_2(t)$ soluções de
\begin{equation}\label{eq:ed2o_homcc}
  ay'' + by' + cy = 0.
\end{equation}
Então,
\begin{equation}
  y(t) = c_1y_1(t) + c_2y_2(t)
\end{equation}
também é solução de \eqref{eq:ed2o_homcc}.

De fato, basta verificar que
\begin{align}
  ay'' + by' + cy &= a(c_1y_1+c_2y_2)'' \\
                  &+ b(c_1y_1 + c_2y_2)' \\
                  &+ c(c_1y_1 + c_2y_2) \\
                  &= c_1(ay_1'' + by_1' + cy_1) \\
                  &+ c_2(ay_2'' + by_2' + cy_2) \\
                  &= 0.
\end{align}

Suponhamos, ainda, que as soluções $y_1 = y_1(t)$ e $y_2 = y_2(t)$ são tais que o chamado \emph{wronskiano}
\begin{equation}
  {\color{blue}W(y_1,y_2;t) := \left|
    \begin{array}{cc}
      y_1 & y_2 \\
      y_1' & y_2'
    \end{array}
\right| \neq 0},
\end{equation}
para todo $t$.

Neste caso, sempre é possível escolher as constantes $c_1$ e $c_2$ tais que
\begin{equation}\label{eq:od2o_homcc_sg}
  y(t) = c_1y_1(t) + c_2y_2(t)
\end{equation}
satisfaça o problema de valor inicial
\begin{align}
  &ay'' + by' + cy = 0,\\
  &y(t_0) = y_0,\quad y'(t_0) = y_0',
\end{align}
para quaisquer dados valores $y_0$ e $y'_0$.

De fato, já sabemos que \eqref{eq:od2o_homcc_sg} satisfaz a EDO. Então, $c_1$ e $c_2$ deve satisfazer o seguinte sistema linear
\begin{align}
  y(t_0) &= c_1y_1(t_0) + c_2y_2(t_0) = y_0\\
  y'(t_0) &= c_1y_1'(t_0) + c_2y_2'(t_0) = y_0'.
\end{align}
Do método de Cramer\footnote{Gabriel Cramer, 1704 - 1752, matemático suíço.}, temos
\begin{equation}
  c_1 = \frac{\left|
      \begin{matrix}
        y_0 & y_2(t_0) \\
        y'_0 & y_2'(t_0)
      \end{matrix}
\right|}{\left|
      \begin{matrix}
        y_1(t_0) & y_2(t_0) \\
        y'_1(t_0) & y_2'(t_0)
      \end{matrix}
\right|}
\end{equation}
e
\begin{equation}
  c_2 = \frac{\left|
      \begin{matrix}
        y_1(t_0) & y_0\\
        y_1'(t_0) & y'_0
      \end{matrix}
\right|}{\left|
      \begin{matrix}
        y_1(t_0) & y_2(t_0) \\
        y'_1(t_0) & y_2'(t_0)
      \end{matrix}
\right|}.
\end{equation}
O \emph{wronskiano não nulo} nos garante a existência de $c_1$ e $c_2$.

Por fim, afirmamos que todas as soluções de \eqref{eq:ed2o_homcc} podem ser escritas como combinação linear de $y_1 = y_1(t)$ e $y_2 = y_2(t)$, i.e. têm a forma
\begin{equation}
  y(t) = c_1y_1(t) + c_2y_2(t).
\end{equation}

De fato, seja $\psi = \psi(t)$ uma solução de \eqref{eq:ed2o_homcc}. Então, $\psi$ é solução do seguinte PVI
\begin{align}
  &ay'' + by' + cy = 0,\\
  &y(t_0) = \psi(t_0),\quad y'(t_0) = \psi'(t_0),
\end{align}
para quaisquer $t_0$ dado. Agora, pelo que vimos acima e lembrando que o wronskiano $W(y_1,y_2;t)\neq 0$, temos que existem constantes $c_1$ e $c_2$ tais que
\begin{equation}
  y(t) = c_1y_1(t) + c_2y_2(t).
\end{equation}
também é solução deste PVI. Da \emph{unicidade de solução}\footnote{Embora não tenha sido apresentada aqui, a unicidade de solução pode ser demonstrada.}, segue que
\begin{equation}
  \psi(t) = c_1y_1(t) + c_2y_2(t).
\end{equation}

Do que vimos aqui, a \emph{solução geral} de \eqref{eq:ed2o_homcc} é
\begin{equation}
  y(t) = c_1y_1(t) + c_2y_2(t)
\end{equation}
dadas quaisquer soluções  $y_1 = y_1(t)$ e $y_2 = y_2(t)$ com wronskiano $W(y_1,y_2;t)\neq 0$ para todo $t$.

\begin{ex}
  No Exemplo \ref{ex:ed2o_homcc}, vimos que
  \begin{equation}
    y_1(t) = e^{-2t}\quad\text{e}\quad y_2(t) = e^{2t}
  \end{equation}
  são soluções particulares de
  \begin{equation}\label{eq:ex_ed2o_homcc}
    y'' - 4y = 0.
  \end{equation}
  Como
  \begin{align}
    W(y_1,y_2;t) &= \left|\begin{matrix}
        y_1 & y_2 \\
        y_1' & y_2' \\
      \end{matrix}\right| \\
                 &= \left|\begin{matrix}
                     e^{-2t} & e^{2t} \\
                     -2e^{-2t} & 2e^{2t} \\
                   \end{matrix}\right| \\
                 &= 4 \neq 0,
  \end{align}
  temos que
  \begin{equation}
    y(t) = c_1e^{-2t} + c_2e^{2t}
  \end{equation}
  é solução geral de \eqref{eq:ex_ed2o_homcc}.
\end{ex}

\subsection{Raízes reais distintas}

Uma EDO da forma
\begin{equation}
  {\color{blue}ay'' + by' + cy = 0}
\end{equation}
tem \emph{solução geral}
\begin{equation}
  y(t) = c_1e^{r_1t} + c_2e^{r_2t}
\end{equation}
quando sua \emph{equação característica}
\begin{equation}
  {\color{blue}ar^2 + br + c = 0}
\end{equation}
tem $r_1$ e $r_2$ como suas raízes reais distintas.

\begin{ex}
  Vamos resolver o seguinte PVI
  \begin{align}
    y'' - 3y' + 2 = 0,\\
    y(0) = 3,\quad y'(0) = 5.
  \end{align}

  Começamos resolvendo a equação característica associada
  \begin{equation}
    r^2 -3r + 2 = 0.
  \end{equation}
  As soluções são
  \begin{align}
    r &= \frac{3 \pm \sqrt{9 - 8}}{2} \\
      &= \frac{3 \pm 1}{2}.
  \end{align}
  Ou seja, $r_1 = 1$ e $r_2 = 2$. Logo,
  \begin{equation}
    y(t) = c_1e^t + c_2e^{2t}
  \end{equation}
  é solução geral da EDO.

  Agora, aplicando as condições iniciais, temos
  \begin{align}
    y(0) = 3 &\Rightarrow c_1 + c_2 = 3,\\
    y'(0) = 5 &\Rightarrow c_1 + 2c_2 = 5.
  \end{align}
  Resolvendo este sistema linear, obtemos $c_1 = 1$ e $c_2 = 2$. Concluímos que
  \begin{equation}
    y(t) = e^t + 2e^{2t}
  \end{equation}
  é a solução do PVI.
\end{ex}

\subsection*{Exercícios resolvidos}

\begin{exeresol}
  Calcule a solução geral de
  \begin{equation}
    2y'' + 2y' - 4 = 0.
  \end{equation}
\end{exeresol}
\begin{resol}
  A equação característica associada é
  \begin{equation}
    2r^2 + 2r -4 = 0.
  \end{equation}
  Suas soluções são
  \begin{align}
    r &= \frac{-2 \pm \sqrt{4 - 4\cdot 2\cdot (-4)}}{2} \\
    &= \frac{-2 \pm 6}{4},
  \end{align}
  i.e. $r_1 = -2$ e $r_2 = 1$. Como a equação característica tem raízes reais distintas, concluímos que
  \begin{equation}
    y(t) = c_1e^{-2t} + c_2e^{t}
  \end{equation}
  é solução geral da EDO.
\end{resol}

\begin{exeresol}
  Mostre que se $y_1(t) = e^{-2t}$ e $y_2(t) = e^t$, então o wronskiano
  \begin{equation}
    W(y_1,y_2; t) \neq 0.
  \end{equation}
\end{exeresol}
\begin{resol}
  Calculamos
  \begin{align}
    W(y_1,y_2; t) &= \left|
                    \begin{matrix}
                      y_1 & y_2 \\
                      y_1' & y_2'
                    \end{matrix}
                             \right|
                          &= \left|
                            \begin{matrix}
                              e^{-2t} & e^t \\
                              -2e^{-2t} & e^{t}
                            \end{matrix} \right| \\
                  &= e^{-2t}e^t + 2e^{-2t}e^t \\
                  &= 3e^{-t}.
  \end{align}
  Como $e^{-t} \neq 0$ para todo $t$, temos que $W(y_1,y_2; t)\neq 0$ para todo $t$.
\end{resol}

\subsection*{Exercícios}

\begin{exer}
  Calcule a solução geral de
  \begin{equation}
    -2y'' + 2y' + 4y = 0.
  \end{equation}
\end{exer}
\begin{resp}
  $y(t) = c_1e^{-t} + c_2e^{2t}$
\end{resp}

\begin{exer}
  Resolva o seguinte PVI
  \begin{align}
    &y'' = 7y' - 12,\\
    &y(0) = 0,\quad y'(0) = -1.
  \end{align}
\end{exer}
\begin{resp}
  $y(t) = e^{3t} - e^{4t}$
\end{resp}

\begin{exer}
  Resolva o seguinte PVI
  \begin{align}
    &y'' - 3y' + 2y = 0,\\
    &y(\ln 2) = -2,\quad y'(\ln 2) = -6.
  \end{align}
\end{exer}
\begin{resp}
  $y(t) = e^t - e^{2t}$
\end{resp}

\begin{exer}
  Calcule o wronskiano de $y_1(t) = \cos(t)$ e $y_2(t) = \sen(t)$.
\end{exer}
\begin{resp}
  $1$
\end{resp}

\begin{exer}
  Mostre que se $r_1$ e $r_2$ são raízes reais distintas da equação
  \begin{equation}
    ar^2 + br + c = 0,
  \end{equation}
  então
  \begin{equation}
    y(t) = c_1e^{r_1t} + c_2e^{r_2t}
  \end{equation}
  é solução geral de
  \begin{equation}
    ay'' + by + c = 0.
  \end{equation}
\end{exer}
\begin{resp}
  Mostre que $y_1(t) = e^{r_1t}$ e $y_2(t) = e^{r_2t}$ são soluções da EDO com $W(y_1,y_2; t) \neq 0$.
\end{resp}


\section{EDO homogênea com coeficientes constantes - parte II}\label{cap_edo2ordem_sec_eqhom_coefconst2}

Na Seção \ref{cap_edo2ordem_sec_eqhom_coefconst1} introduzimos as propriedades fundamentais de EDOs lineares de segunda ordem com coeficientes constantes. Em particular, tratamos o caso em que a equação característica tem raízes reais distintas. Nesta seção, estudar os casos em que a equação característica tem raízes complexas ou raízes duplas.

\subsection{Raízes complexas}

Consideramos
\begin{equation}\label{eq:ed2o_hcc_rc}
  {\color{blue}ay'' + by' + cy = 0},
\end{equation}
cuja \emph{equação característica}
\begin{equation}
  {\color{blue}ar^2 + br + c = 0}
\end{equation}
tem \emph{raízes complexas}
\begin{equation}
  {\color{blue}r_1 = \lambda + i\mu,\quad r_2 = \lambda - i\mu}.
\end{equation}
As soluções particulares associadas são
\begin{align}
  &y_1(t) = e^{r_1t} = e^{(\lambda + i\mu)t} \\
  &y_2(t) = e^{r_2t} = e^{(\lambda - i\mu)t}.
\end{align}

Da \emph{fórmula de Euler}\footnote{Leonhard Euler, 1707-1783, matemático suíço. Fonte: \href{https://en.wikipedia.org/wiki/Leonhard_Euler}{Wikipedia}.}, temos
\begin{align}
  e^{a+bi} &= e^ae^{bi}\\
           &= e^a(\cos b + i\sen b).
\end{align}
Ou seja, as soluções particulares podem ser reescritas da forma\footnote{Lembre-se que seno é uma função ímpar, i.e. $\sen(-x) = -\sen(x)$.}
\begin{align}
  &y_1(t) = e^{\lambda t}\left[\cos(\mu t) + i\sen(\mu t)\right], \\
  &y_2(t) = e^{\lambda t}\left[\cos(\mu t) - i\sen(\mu t)\right]
\end{align}

Agora, se denotarmos
\begin{equation}
  u(t) = e^{\lambda t}\cos(\mu t)\quad\text{e}\quad v(t) = e^{\lambda t}\sen(\mu t),
\end{equation}
temos
\begin{equation}
  y_1(t) = u(t) + iv(t),\quad y_2(t) = u(t) - iv(t).
\end{equation}
Para concentrar a escrita, vamos denotar
\begin{equation}
  y(t) = u(t) \pm iv(t).
\end{equation}
Substituindo $y = y(t)$ na EDO, obtemos
\begin{align}
  0 &= ay'' + by' + cy \\
    &= a(u'' \pm iv'') \\
    &+ b(u' \pm iv') \\
    &+ c(u + pm iv) \\
    &= (au'' + bu' + cu) \\
    &\pm i(av'' + bv' + cv).
\end{align}
Ou seja,
\begin{align}
  au'' + bu' + cu &= 0 \\
  av'' + bv' + cv &= 0.
\end{align}
Desta forma, concluímos que $u(t) = e^{\lambda t}\cos(\mu t)$ e $v(t) = e^{\lambda t}\sen(\mu t)$ são soluções particulares da EDO \eqref{eq:ed2o_hcc_rc}. Do que vimos na Seção \ref{cap_edo2ordem_sec_eqhom_coefconst1}, temos que
\begin{equation}
  {\color{blue}y(t) = e^{\lambda t}\left[c_1\cos(mu t) + c_2\sen(\mu t)\right]}
\end{equation}
é \emph{solução geral} de \eqref{eq:ed2o_hcc_rc}.

\begin{ex}
  Vamos resolver
  \begin{equation}
    y'' + 2y' + 5 = 0.
  \end{equation}

  Começamos identificando a equação característica associada
  \begin{equation}
    r^2 + 2r + 5 = 0.
  \end{equation}
  Suas raízes são
  \begin{align}
    r &= \frac{-2 \pm \sqrt{4 - 4\cdot 1\cdot 5}}{2} \\
      &= -1 \pm 2i.
  \end{align}
  Logo, a solução geral é
  \begin{equation}
    y(t) = e^{-t}\left[c_1\cos(2t) + c_2\sen(2t)\right].
  \end{equation}
\end{ex}

\subsection{Raízes repetidas}

Seja a equação
\begin{equation}\label{eq:ed2o_hcc_rr}
  {\color{blue}ay'' + by' + cy = 0},
\end{equation}
cuja equação característica
\begin{equation}
  {\color{blue}ar^2 + br + c = 0}
\end{equation}
tem raiz dupla\footnote{$b^2-4ac = 0$}
\begin{equation}
  r = \frac{-b}{2a}.
\end{equation}
Neste caso, podemos verificar que
\begin{equation}
  {\color{blue}y_1(t) = e^{-\frac{b}{2a}t}}
\end{equation}
é solução particular de \eqref{eq:ed2o_hcc_rr}.

Vamos usar o \emph{método de redução de ordem} para encontrar uma segunda solução particular $y_2 = y_2(t)$ de \eqref{eq:ed2o_hcc_rr}, lembrando que o wronskiano $W(y_1,y_2; t)$ deve ser não nulo. O método consiste em buscar por uma solução da forma
\begin{align}
  {\color{blue}y_2(t)} &{\color{blue}= u(t)y_1(t)} \\
                      &= u(t)e^{-\frac{b}{2a}t}.
\end{align}
Substituindo $y_2$ na EDO \eqref{eq:ed2o_hcc_rr}, obtemos
\begin{align}
  0 &= ay_2'' + by_2' + cy_2 \\
    &= a(u''y_1 + 2u'y_1' + uy_1'') \\
    &+ b(u'y_1 + uy_1') + cuy_1 \\
    &= au''y_1 + (2ay_1' + by_1)u' \\
    &+ (ay_1'' + by_1' + cy_1)u \\
    &= au''y_1 + \left(-\frac{2ab}{2a}e^{-\frac{b}{2a}} + be^{-\frac{b}{2a}}\right)u' \\
    &= au''y_1.
\end{align}
Segue que
\begin{align}
  u'' = 0  &\Rightarrow u' = c_1 \\
           &\Rightarrow u = c_1 + c_2t.
\end{align}
Podemos escolher $c_1$ e $c_2$ arbitrariamente, desde que o wronskiano
\begin{equation}
  W(y_1,y_2;t) \neq 0.
\end{equation}
A escolha mais simples é $c_1 = 0$ e $C_2 = 1$, donde segue que
\begin{equation}
  {\color{blue}y_2(t) = te^{-\frac{b}{2a}t}}.
\end{equation}
Concluímos que a \emph{solução geral} de \eqref{eq:ed2o_hcc_rr} é
\begin{equation}
  {\color{blue}y(t) = (c_1 + c_2t)e^{-\frac{b}{2a}t}}.
\end{equation}

\begin{ex}
  Vamos resolver
  \begin{equation}
    y'' -2y' + y = 0.
  \end{equation}

  Da equação característica
  \begin{equation}
    r^2 - 2r + 1 = 0
  \end{equation}
  obtemos a raiz dupla
  \begin{equation}
    r = 1.
  \end{equation}
  Logo, a solução geral da EDO é
  \begin{equation}
    y(t) = (c_1 + c_2t)e^{t}.
  \end{equation}
\end{ex}

\subsection*{Exercícios resolvidos}

\begin{exeresol}
  Resolva
  \begin{align}
    &y'' - 4y' + 5 = 0,\\
    &y(0) = 2,\quad y'(0)=0.
  \end{align}
\end{exeresol}
\begin{resol}
  Resolvendo a equação característica
  \begin{equation}
    r^2 - 4r + 5 = 0,
  \end{equation}
  obtemos as raízes
  \begin{align}
    r &= \frac{4 \pm \sqrt{16 - 4\cdot 1\cdot 5}}{2} \\
      &= 2 \pm i.
  \end{align}
  Logo, a solução geral é
  \begin{equation}
    y(t) = e^{2t}\left[c_1\cos(t) + c_2\sen(t)\right].
  \end{equation}

  Por fim, aplicamos as condições iniciais
  \begin{align}
    y(0) = 2 &\Rightarrow e^{2\cdot 0}\left[c_1\cos(0) + c_2\sen(0)\right] = 2 \\
             &\Rightarrow c_1 = 2.
  \end{align}
  e, observando que
  \begin{equation}
    y'(t) = e^{2t}\left[(2c_1 + c_2)\cos(t) + (2c_2 - c_1)\sen(t)\right]
  \end{equation}
  temos
  \begin{align}
    y'(0) = 0 &\Rightarrow e^{2\cdot 0}(2\cdot 2 + c_2) = 0 \\
              &\Rightarrow 4 + c_2 = 0 \\
              &\Rightarrow c_2 = -4.
  \end{align}
  Concluímos que a solução do PVI é
  \begin{equation}
    y(t) = e^{2t}\left[2\cos(t) - 4\sen(t)\right].
  \end{equation}
\end{resol}

\begin{exeresol}
  Resolva
  \begin{align}
    &y'' + 4y' + 4 = 0,\\
    &y(0) = 0,\quad y'(0)=1.    
  \end{align}
\end{exeresol}
\begin{resol}
  Resolvemos a equação característica
  \begin{equation}
    r^2 + 4r + 4 = 0,
  \end{equation}
  de modo que obtemos uma raiz dupla
  \begin{equation}
    r = -2.
  \end{equation}
  Logo, a solução geral da EDO é
  \begin{equation}
    y(t) = (c_1 + c_2t)e^{-2t}.
  \end{equation}

  Agora, aplicamos as condições iniciais
  \begin{equation}
    y(0) = 0 \Rightarrow c_1 = 0
  \end{equation}
  e, observando que
  \begin{equation}
    y'(t) = (c_2-2c_1-2c_2t)e^{-2t}
  \end{equation}
  \begin{align}
    y'(0) = 1 &\Rightarrow (c_2-2\cdot 0-2c_2\cdot 0)e^{-2\cdot 0} = 1 \\
              &\Rightarrow c_2 = 1.
  \end{align}
  Concluímos que a solução do PVI é
  \begin{equation}
    y(t) = te^{-2t}.
  \end{equation}
\end{resol}

\subsection*{Exercícios}

\begin{exer}
  Encontre a solução geral de
  \begin{equation}
    2y'' - 4y + 4 = 0.
  \end{equation}
\end{exer}
\begin{resp}
  $y(t) = [c_1\sen(t) + c_2\cos(t)]e^{t}$
\end{resp}

\begin{exer}
  Resolva
  \begin{align}
    &2y'' + 12y' = -26,\\
    &y(0) = 0,\quad y'(0) = 2.
  \end{align}
\end{exer}
\begin{resp}
  $y(t) = e^{-3t}\sen(2t)$
\end{resp}

\begin{exer}
  Encontre a solução geral de
  \begin{equation}
    3y'' + 27y = 18y'
  \end{equation}
\end{exer}
\begin{resp}
  $y(t) = (c_1 + c_2t)e^{3t}$
\end{resp}

\begin{exer}
  Resolva
  \begin{align}
    &-y = 2y' + y'',\\
    &y(0) = 2,\quad y'(0) = 0.
  \end{align}
\end{exer}
\begin{resp}
  $y(t) = (2 + 2t)e^{-t}$
\end{resp}

\begin{ex}
  Mostre que o wronskiano de $y_1(t) = e^{\lambda t}\cos(\mu t)$ e $y_2(t) = e^{\lambda t}\sen(\mu t)$ é não nulo para qualquer $\mu\neq 0$.
\end{ex}
\begin{resp}
  $W(y_1,y_2;t) = \mu e^{2\lambda t} \neq 0$
\end{resp}

\begin{ex}
  Mostre que o wronskiano de $y_1(t) = e^{rt}$ e $y_2(t) = te^{rt}$ é não nulo para qualquer $r$.  
\end{ex}
\begin{resp}
  $W(y_1,y_2;t) = e^{2rt} \neq 0$
\end{resp}

\section{EDO não homogênea}

Pode-se mostrar que a solução geral de
\begin{equation}\label{eq:ed2o_nh}
  {\color{blue}y'' + ay' + by = g(t)},
\end{equation}
tem a forma
\begin{equation}
  {\color{blue}y(t) = c_1y_1(t) + c_2y_2(t) + y_p(t)},
\end{equation}
onde $y_1$ e $y_2$ formam um conjunto fundamental de soluções\footnote{São soluções da equação homogênea associada e $W(y_1,y_2;t)\neq 0$.} da equação homogênea associada
\begin{equation}
  y'' + ay' + by = 0
\end{equation}
e $y_p$ é uma solução particular qualquer de \eqref{eq:ed2o_nh}.

\subsection{Método da variação dos parâmetros}

O \emph{método da variação dos parâmetros} consiste em calcular uma solução particular de \eqref{eq:ed2o_nh} da forma
\begin{equation}
  {\color{blue}y_p(t) = u_1(t)y_1(t) + u_2(t)y_2(t)},
\end{equation}
onde $y_1$ e $y_2$ é um conjunto fundamental de soluções da equação homogênea associada, enquanto $u_1$ e $u_2$ são funções a serem determinadas.

Observamos que a única condição que temos para determinar $u_1$ e $u_2$ é a equação \eqref{eq:ed2o_nh}. Ou seja, temos uma equação e duas incógnitas. Para fechar o problema, impomos a seguinte condição extra
\begin{equation}\label{eq:ed2o_nh_mv_1}
  {\color{blue}u_1'y_1 + u_2'y_2 = 0}.
\end{equation}

Com isso, temos
\begin{align}
  y_p'(t) &=  u_1'y_1 + u_1y_1' + u_2'y_2 + u_2y_2' \\
          &= u_1y_1' + u_2y_2'
\end{align}
e
\begin{align}
  y_p''(t) &= u_1'y_1' + u_1y_1'' \\
           &= u_2'y_2' + u_2y_2''.
\end{align}

Substituindo $y_p$ em \eqref{eq:ed2o_nh}, temos
\begin{align}
  g(t) &= y_p'' + ay_p' + by_p \\
       &= \left(u_1'y_1' + {\color{blue}u_1y_1''} + u_2'y_2' + {\color{red}u_2y_2''} \right) \\
       &+ a({\color{blue}u_1y_1'} + {\color{red}u_2y_2'}) \\
       &+ b({\color{blue}u_1y_1} + {\color{red}u_2y_2}) \\
       &= u_1'y_1' + u_2'y_2' \\
       &+ \underbrace{{\color{blue}u_1(y_1'' + ay_1' + by_1)}}_{=0} \\
       &+ \underbrace{{\color{red}u_2(y_2'' + ay_2' + by_2)}}_{=0} \\
       &= u_1'y_1' + u_2'y_2'. \label{eq:ed2o_nh_mv_2}
\end{align}

Ou seja, \eqref{eq:ed2o_nh_mv_1} e \eqref{eq:ed2o_nh_mv_2} formam o seguinte sistema de equações
\begin{align}
  u_1'y_1 + u_2'y_2 &= 0\\
  u_1'y_1' + u_2'y_2' &= g(t)
\end{align}
que têm $u_1'$ e $u_2'$ como incógnitas. Aplicando o método de Cramer\footnote{Gabriel Cramer, 1704 - 1752, matemático suíço. Fonte: \href{https://en.wikipedia.org/wiki/Gabriel_Cramer}{Wikipedia}.}, obtemos
\begin{equation}
  u_1' = -\frac{y_2(t)g(t)}{W(y_1,y_2;t)}
\end{equation}
e
\begin{equation}
  u_2' = \frac{y_1(t)g(t)}{W(y_1,y_2;t)}.
\end{equation}
Ou, ainda, por integração temos
\begin{equation}\label{eq:ed2o_nh_vp_u1}
  {\color{blue}u_1(t) = -\int \frac{y_2(t)g(t)}{W(y_1,y_2;t)}\,dt}
\end{equation}
e
\begin{equation}\label{eq:ed2o_nh_vp_u2}
  {\color{blue}u_2(t) = \int \frac{y_1(t)g(t)}{W(y_1,y_2;t)}\,dt}.
\end{equation}

Por tudo isso, concluímos que uma solução particular de \eqref{eq:ed2o_nh} é dada por
\begin{align}
  y_p(t) &= -y_1(t)\int \frac{y_2(t)g(t)}{W(y_1,y_2;t)}\,dt \\
         &+ y_2(t)\int \frac{y_1(t)g(t)}{W(y_1,y_2;t)}\,dt.
\end{align}

\begin{ex}\label{ex:ed2o_nh_vp}
  Vamos calcular a solução geral de
  \begin{equation}\label{eq:ex_ed2o_nh_vp}
    y'' - y = e^{2t}.
  \end{equation}

  Começamos determinando um conjunto fundamental de soluções $y_1 = y_1(t)$ e $y_2 = y_2(t)$ da equação homogênea associada
  \begin{equation}
    y'' - y = 0.
  \end{equation}
  A equação característica associada é
  \begin{equation}
    r^2 - 1 = 0,
  \end{equation}
  cujas raízes são $r_1=-1$ e $r_2=1$. Segue que
  \begin{equation}
    y_1(t) = e^{-t}\quad\text{e}\quad y_2(t) = e^t.
  \end{equation}

  Agora, buscamos por uma solução particular de \eqref{eq:ex_ed2o_nh_vp} da forma
  \begin{equation}
    y_p(t) = u_1(t)y_1(t) + u_2(t)y_2(t),
  \end{equation}
  onde $u_1$ é dada em \eqref{eq:ed2o_nh_vp_u1} e $u_2$ por \eqref{eq:ed2o_nh_vp_u2}. Ambas expressões requer o cálculo do wronskiano
  \begin{align}
    W(y_1,y_2;t) &= \left|
                   \begin{matrix}
                     y_1 & y_2 \\
                     y_1' & y_2'
                   \end{matrix}
                            \right| \\
                 &= y_1y_2' - y_2y_1' \\
                 &= e^{-t}e^t + e^te^{-t} \\
                 &= 2.
  \end{align}
  Com isso, temos
  \begin{align}
    u_1(t) &= -\int \frac{y_2(t)g(t)}{W(y_1,y_2;t)}\,dt \\
           &= -\int \frac{e^te^{2t}}{2}\,dt \\
           &= -\frac{1}{2}\int e^{3t}\,dt \\
           &= -\frac{1}{6}e^{3t}
  \end{align}
  e
  \begin{align}
    u_2(t) &= \int \frac{y_1(t)g(t)}{W(y_1,y_2;t)}\,dt \\
           &= \int \frac{e^{-t}e^{2t}}{2}\,dt \\
           &= \frac{1}{2}\int e^{t}\,dt \\
           &= \frac{1}{2}e^{t}
  \end{align}
  Desta forma, obtemos a solução particular
  \begin{align}
    y_p(t) &= u_1(t)y_1(t) + u_2(t)y_2(t) \\
           &= -\frac{1}{6}e^{3t}e^{-t} + \frac{1}{2}e^{t}e^{t} \\
           &= -\frac{1}{6}e^{2t} + \frac{1}{2}e^{2t} \\
           &= \frac{1}{3}e^{2t}.
  \end{align}
  Observamos que a solução particular é um múltiplo do termo não homogêneo da EDO \eqref{eq:ex_ed2o_nh_vp}. Isso não é apenas um acaso e vamos explorar isso mais adiante no texto.

  Por fim, concluímos que a solução geral de \eqref{eq:ex_ed2o_nh_vp} é
  \begin{align}
    y(t) &= c_1y_1(t) + c_2(t)y_2(t) + y_p(t) \\
         &= c_1e^{-t} + c_2e^t + \frac{1}{3}e^{2t}.
  \end{align}
\end{ex}

\subsection{Método dos coeficientes a determinar}

O métodos dos coeficientes a determinar consiste em buscar por uma solução particular na forma de uma combinação linear de funções elementares apropriadas. Tais funções são inferidas a partir do termo não homogêneo da equação.

\subsubsection{$\pmb{g(t) = ce^{st}}$}

Uma equação da forma
\begin{equation}
  y'' + ay' + by = ce^{st}
\end{equation}
com $s\neq r_1,r_2$, onde $r_1$ e $r_2$ são raízes da equação característica, admite solução particular
\begin{equation}
  y_p(t) = Ae^{st},
\end{equation}
onde $A$ é uma constante a determinar.

\begin{ex}
  Vamos calcular uma solução particular para
  \begin{equation}
    y'' - y = e^{2t}.
  \end{equation}

  Pelo método dos coeficientes a determinar, buscamos por uma solução particular da forma
  \begin{equation}
    y_p(t) = Ae^{2t},
  \end{equation}
  observando que $r_1=-1$ e $r_2=1$ são raízes da equação característica associada.

  Substituindo $y_p$ na EDO, obtemos
  \begin{align}
    e^{2t} &= y'' - y \\
           &= \left(Ae^{2t}\right)'' - Ae^{2t} \\
           &= (4A - A)e^{2t} \\
           &= 3Ae^{2t}.
  \end{align}
  Segue que
  \begin{equation}
    3A = 1 \Rightarrow A = \frac{1}{3}.
  \end{equation}
  Daí, concluímos que
  \begin{equation}
    y_p(t) = \frac{1}{3}e^{2t}
  \end{equation}
  é solução particular da EDO.
\end{ex}

\begin{obs}
  \begin{enumerate}[a)]
  \item $s = r_1$.
    Uma equação da forma
    \begin{equation}
      y'' + ay' + by = ce^{r_1t},
    \end{equation}
    onde $r_1$ é raiz da equação característica associada, admite solução particular
    \begin{equation}
      y_p(t) = Ate^{r_1t}.
    \end{equation}
  \item $s = r$.
    Uma equação da forma
    \begin{equation}
      y'' + ay' + by = ce^{rt},
    \end{equation}
    onde $r$ é raiz dupla da equação característica associada, admite solução particular
    \begin{equation}
      y_p(t) = At^2e^{rt}.
    \end{equation}
  \end{enumerate}
\end{obs}

\subsubsection{$\pmb{g(t) = c_nt^n + c_{n-1}t^{n-1} + \cdots + c_0}$}

Uma equação da forma
\begin{equation}
  y'' + ay' + by = c_nt^n + c_{n-1}t^{n-1} + \cdots + c_0
\end{equation}
admite solução particular
\begin{equation}
  y_p(t) = A_nt^n + A_{n-1}t^{n-1} + \cdots + A_0,
\end{equation}
onde $A_n$, $A_{n-1}$, $\dotsc$, $A_0$ são constantes a determinar.

\begin{ex}
  Vamos calcular uma solução particular para
  \begin{equation}
    y'' - 4y = t.
  \end{equation}

  Pelo método dos coeficientes a determinar, buscamos por uma solução particular da forma
  \begin{equation}
    y_p(t) = A_1t + A_0.
  \end{equation}

  Substituindo $y_p$ na EDO, obtemos
  \begin{align}
    t &= y'' - 4y \\
      &= \left(A_1t + A_0\right)'' - 4(A_1t + A_0) \\
           &= -4A_1t - 4A_0.
  \end{align}
  Segue que
  \begin{align}
    -4A_1 = 1 &\Rightarrow A_1 = -\frac{1}{4},\\
    -4A_0 = 0 &\Rightarrow A_0 = 0.
  \end{align}
  Daí, concluímos que
  \begin{equation}
    y_p(t) = -\frac{1}{4}t
  \end{equation}
  é solução particular da EDO.
\end{ex}

\subsubsection{$\pmb{g(t) = c_1\sen(\beta t) + c_2\cos(\beta t)}$}

Uma equação da forma
\begin{equation}
  y'' + ay' + by = c_1\sen(\beta t) + c_2\cos(\beta t)
\end{equation}
admite solução particular
\begin{equation}
  y_p(t) = t^s[A_1\sen(\beta t) + A_2\cos(\beta t)],
\end{equation}
onde $s$ é o menor inteiro tal que $y_p$ não seja solução da equação homogênea associada e $A_1$ e $A_2$ são constantes a determinar.

\begin{ex}
  Vamos calcular uma solução particular para
  \begin{equation}
    y'' + 4y = \cos(2t).
  \end{equation}

  Pelo método dos coeficientes a determinar, buscamos por uma solução particular da forma
  \begin{equation}
    y_p(t) = t[A_1\sen(2t) + A_2\cos(2t)],
  \end{equation}
  observando que $y_1(t) = \cos(2t)$ e $y_2(t) = \sen(2t)$ formam um conjunto fundamental de solução para a equação homogênea associada.
  
  Substituindo $y_p$ na EDO, obtemos
  \begin{align}
    \cos(2t) &= y'' + 4y \\
             &= [A_1t\sen(2t) + A_2t\cos(2t)]'' \\
             &+ 4[A_1t\sen(2t) + A_2t\cos(2t)] \\
             &= 4A_1\cos(2t) - 4A_2\sen(2t)
  \end{align}
  Segue que
  \begin{align}
    4A_1 = 1 &\Rightarrow A_1 = \frac{1}{4},\\
    -4A_2 = 0 &\Rightarrow A_2 = 0.
  \end{align}
  Daí, concluímos que
  \begin{equation}
    y_p(t) = \frac{1}{4}t\sen(2t).
  \end{equation}
  é solução particular da EDO.
\end{ex}

\begin{obs}(Resumo)
  \begin{center}
  \begin{tabular}{ll}
    $g(t)$ & $y_p(t)$ \\\hline
    $e^{\alpha t}(c_nt^n + c_{n-1}t^{n-1} + \cdots + c_0)$ & $t^se^{\alpha t}(Ac_{n-1}t^{n-1} + \cdots + A_0)$ \\
    $e^{\alpha t}[c_1\sen(\beta t) + c_2\cos(\beta t)]$ & $t^se^{\alpha t}[A_1\sen(\beta t) + A_2\cos(\beta t)]$ \\\hline
  \end{tabular}
\end{center}
  $s = 0, 1, 2$, sendo o menor valor que garanta que $y_p$ não seja solução da equação homogênea associada.
\end{obs}

\subsection*{Exercícios resolvidos}

\begin{exeresol}
  Use o método da variação dos parâmetros para obter uma solução geral de
  \begin{equation}
    y'' - 2y' - 3y = e^{-t}+\sen(t).
  \end{equation}
\end{exeresol}
\begin{resol}
  Primeiramente, resolvemos a equação homogênea associada
  \begin{equation}
    y'' - 2y' - 3y = 0.
  \end{equation}
  Para tanto, buscamos as raízes da equação característica associada
  \begin{equation}
    r^2 - 2r - 3 = 0,
  \end{equation}
  as quais são
  \begin{equation}
    r = \frac{2 \pm \sqrt{4 - 4\cdot 1\cdot (-3)}}{2},
  \end{equation}
  i.e. $r_1 = -1$ e $r_2 = 3$. Logo,
  \begin{equation}
    y_1(t) = e^{-t}\quad\text{e}\quad y_2(t) = e^{3t}
  \end{equation}
  formam um conjunto fundamental de soluções da EDO homogênea.

  Agora, buscamos por uma solução particular
  \begin{equation}
    y_p(t) = u_1(t)y_1() + u_2(t)y_2(t)
  \end{equation}
  para a equação não homogênea. Os parâmetros variáveis $u_1 = u_1(t)$ e $u_2 = u_2(t)$ dependem do wronskiano
  \begin{align}
    W(y_1,y_2;t) &= \left|
                   \begin{matrix}
                     y_1 & y_2 \\
                     y_1' & y_2'
                   \end{matrix}
                            \right| \\
                 &= \left|
                   \begin{matrix}
                     e^{-t} & e^{3t} \\
                     -e^{-t} & 3e^{3t}
                   \end{matrix}
                               \right| \\
                 &= 4e^{2t}.
  \end{align}
  Mais especificamente, eles são dados por
  \begin{align}
    u_1(t) &= -\int \frac{y_2(t)g(t)}{W(y_1,y_2;t)}\,dt \\
           &= -\int \frac{e^{3t}[e^{-t} + \sen(t)]}{4e^{2t}}\,dt \\
           &= -\frac{1}{4}t - \frac{1}{8}e^{t}[\sen(t) - \cos(t)]
  \end{align}
  e
  \begin{align}
    u_2(t) &= \int \frac{y_1(t)g(t)}{W(y_1,y_2;t)}\,dt \\
           &= \int \frac{e^{-t}[e^{-t} + \sen(t)]}{4e^{2t}}\,dt \\
           &= -\frac{1}{16}e^{-4t} - \frac{3}{40}e^{-3t}[\sen(t) + \cos(t)]
  \end{align}
  Com isso, temos que a solução particular é
  \begin{align}
    y_p(t) &= \left\{-\frac{1}{4}t - \frac{1}{8}e^{t}[\sen(t) - \cos(t)]\right\}e^{-t} \\
           &+ \left\{-\frac{1}{16}e^{-4t} - \frac{3}{40}e^{-3t}[\sen(t) + \cos(t)]\right\}e^{3t} \\
           &= -\left(\frac{1}{16}+\frac{1}{4}t\right)e^{-t} + \frac{1}{10}\cos(t) - \frac{1}{5}\sen(t).
  \end{align}
  Concluímos que a solução geral é
  \begin{align}
    y(t) &= c_1e^{-t} + c_2e^{3t} \\
         &- \left(\frac{1}{16}+\frac{1}{4}t\right)e^{-t} + \frac{1}{10}\cos(t) - \frac{1}{5}\sen(t).
  \end{align}
\end{resol}

\emconstrucao

\subsection*{Exercício}

\emconstrucao


