%Este trabalho está licenciado sob a Licença Atribuição-CompartilhaIgual 4.0 Internacional Creative Commons. Para visualizar uma cópia desta licença, visite http://creativecommons.org/licenses/by-sa/4.0/deed.pt_BR ou mande uma carta para Creative Commons, PO Box 1866, Mountain View, CA 94042, USA.

\chapter{EDO linear de segunda ordem}\label{cap_edo2ordem}
\thispagestyle{fancy}

\section{Equação homogênea com coeficientes constantes}\label{cap_edo2ordem_sec_eqhom_coefconst}

Uma \emph{EDO de segunda ordem homogênea com coeficientes constantes} tem a forma
\begin{equation}\label{eq:edo2_hom_coefconst}
  {\color{blue}ay'' + by' + cy = 0},
\end{equation}
onde $y = y(t)$ e $a, b, c$ são parâmetros constantes (números reais).

Vamos buscar por soluções da forma ${\color{blue}y(t) = e^{rt}}$, onde $r$ é constante. Substituindo na equação \eqref{eq:edo2_hom_coefconst}, obtemos
\begin{align}
  & a\left(e^{rt}\right)'' + b\left(e^{rt}\right)' + ce^{rt} = 0 \\
  & \Rightarrow ar^2e^{rt} + bre^{rt} + ce^{rt} = 0 \\
  & \Rightarrow {\color{blue}(ar^2 + br + c)e^{rt} = 0}.
\end{align}
Ou seja, $y(t) = e^{rt}$ é solução de \eqref{eq:edo2_hom_coefconst} quando
\begin{equation}
  {\color{blue}ar^2 + br + c = 0}.
\end{equation}
Esta é chamada de \emph{equação característica} de \eqref{eq:edo2_hom_coefconst}.

\begin{ex}\label{ex:ed2o_homcc}
  Vamos buscar por soluções de
  \begin{equation}\label{eq:ex_ed2o_hom_coefconst1}
    y'' - 4y = 0.
  \end{equation}
  Buscamos por $r$ tal que $y(t) = e^{rt}$ seja solução desta equação. Substituindo na equação, obtemos
  \begin{align}
    \left(e^{rt}\right)'' - 4e^{rt} = 0 \Rightarrow (r^2 - 4)e^{rt} = 0
  \end{align}
  o que nos fornece a equação característica
  \begin{equation}
    r^2 - 4 = 0.
  \end{equation}
  As soluções desta equação são $r_1 = -2$ e $r_2 = 2$. Ou seja, obtemos as seguintes \emph{soluções particulares} da EDO
  \begin{equation}
    y_1(t) = e^{-2t}\quad\text{e}\quad y_2(t) = e^{2t}.
  \end{equation}

  Observamos, ainda, que para quaisquer constantes $c_1$ e $c_2$,
  \begin{equation}
    y(t) = c_1e^{-2t} + c_2e^{2t}
  \end{equation}
  também é solução da EDO \eqref{eq:ex_ed2o_hom_coefconst1}. De fato, temos
  \begin{align}
    y'' - 4y &= \left(c_1e^{-2t} + c_2e^{2t}\right)'' -4\left(c_1e^{-2t}+c_2e^{2t}\right) \\
             &= 4c_1e^{-2t} + 4c_2e^{2t} -4c_1e^{-2t} -4c_2e^{2t} \\
             &= 0.
  \end{align}

  Como veremos logo mais,
  \begin{equation}
    y(t) = c_1e^{-2t} + c_2e^{2t}
  \end{equation}
  é a \emph{solução geral} de \eqref{eq:ex_ed2o_hom_coefconst1}.
\end{ex}

\subsection{Conjunto fundamental de solução}

Sejam $y_1 = y_1(t)$ e $y_2 = y_2(t)$ soluções de
\begin{equation}\label{eq:ed2o_homcc}
  ay'' + by' + cy = 0.
\end{equation}
Então,
\begin{equation}
  y(t) = c_1y_1(t) + c_2y_2(t)
\end{equation}
também é solução de \eqref{eq:ed2o_homcc}.

De fato, basta verificar que
\begin{align}
  ay'' + by' + cy &= a(c_1y_1+c_2y_2)'' \\
                  &+ b(c_1y_1 + c_2y_2)' \\
                  &+ c(c_1y_1 + c_2y_2) \\
                  &= c_1(ay_1'' + by_1' + cy_1) \\
                  &+ c_2(ay_2'' + by_2' + cy_2) \\
                  &= 0.
\end{align}

Suponhamos, ainda, que as soluções $y_1 = y_1(t)$ e $y_2 = y_2(t)$ são tais que o chamado \emph{wronskiano}
\begin{equation}
  {\color{blue}W(y_1,y_2;t) := \left|
    \begin{array}{cc}
      y_1 & y_2 \\
      y_1' & y_2'
    \end{array}
\right| \neq 0},
\end{equation}
para todo $t$.

Neste caso, sempre é possível escolher as constantes $c_1$ e $c_2$ tais que
\begin{equation}\label{eq:od2o_homcc_sg}
  y(t) = c_1y_1(t) + c_2y_2(t)
\end{equation}
satisfaça o problema de valor inicial
\begin{align}
  &ay'' + by' + cy = 0,\\
  &y(t_0) = y_0,\quad y'(t_0) = y_0',
\end{align}
para quaisquer dados valores $y_0$ e $y'_0$.

De fato, já sabemos que \eqref{eq:od2o_homcc_sg} satisfaz a EDO. Então, $c_1$ e $c_2$ deve satisfazer o seguinte sistema linear
\begin{align}
  y(t_0) &= c_1y_1(t_0) + c_2y_2(t_0) = y_0\\
  y'(t_0) &= c_1y_1'(t_0) + c_2y_2'(t_0) = y_0'.
\end{align}
Do método de Cramer\footnote{Gabriel Cramer, 1704 - 1752, matemático suíço.}, temos
\begin{equation}
  c_1 = \frac{\left|
      \begin{matrix}
        y_0 & y_2(t_0) \\
        y'_0 & y_2'(t_0)
      \end{matrix}
\right|}{\left|
      \begin{matrix}
        y_1(t_0) & y_2(t_0) \\
        y'_1(t_0) & y_2'(t_0)
      \end{matrix}
\right|}
\end{equation}
e
\begin{equation}
  c_2 = \frac{\left|
      \begin{matrix}
        y_1(t_0) & y_0\\
        y_1'(t_0) & y'_0
      \end{matrix}
\right|}{\left|
      \begin{matrix}
        y_1(t_0) & y_2(t_0) \\
        y'_1(t_0) & y_2'(t_0)
      \end{matrix}
\right|}.
\end{equation}
O \emph{wronskiano não nulo} nos garante a existência de $c_1$ e $c_2$.

Por fim, afirmamos que todas as soluções de \eqref{eq:ed2o_homcc} podem ser escritas como combinação linear de $y_1 = y_1(t)$ e $y_2 = y_2(t)$, i.e. têm a forma
\begin{equation}
  y(t) = c_1y_1(t) + c_2y_2(t).
\end{equation}

De fato, seja $\psi = \psi(t)$ uma solução de \eqref{eq:ed2o_homcc}. Então, $\psi$ é solução do seguinte PVI
\begin{align}
  &ay'' + by' + cy = 0,\\
  &y(t_0) = \psi(t_0),\quad y'(t_0) = \psi'(t_0),
\end{align}
para quaisquer $t_0$ dado. Agora, pelo que vimos acima e lembrando que o wronskiano $W(y_1,y_2;t)\neq 0$, temos que existem constantes $c_1$ e $c_2$ tais que
\begin{equation}
  y(t) = c_1y_1(t) + c_2y_2(t).
\end{equation}
também é solução deste PVI. Da \emph{unicidade de solução}\footnote{Embora não tenha sido apresentada aqui, a unicidade de solução pode ser demonstrada.}, segue que
\begin{equation}
  \psi(t) = c_1y_1(t) + c_2y_2(t).
\end{equation}

Do que vimos aqui, a \emph{solução geral} de \eqref{eq:ed2o_homcc} é
\begin{equation}
  y(t) = c_1y_1(t) + c_2y_2(t)
\end{equation}
dadas quaisquer soluções  $y_1 = y_1(t)$ e $y_2 = y_2(t)$ com wronskiano $W(y_1,y_2;t)\neq 0$ para todo $t$.

\begin{ex}
  No Exemplo \ref{ex:ed2o_homcc}, vimos que
  \begin{equation}
    y_1(t) = e^{-2t}\quad\text{e}\quad y_2(t) = e^{2t}
  \end{equation}
  são soluções particulares de
  \begin{equation}\label{eq:ex_ed2o_homcc}
    y'' - 4y = 0.
  \end{equation}
  Como
  \begin{align}
    W(y_1,y_2;t) &= \left|\begin{matrix}
        y_1 & y_2 \\
        y_1' & y_2' \\
      \end{matrix}\right| \\
                 &= \left|\begin{matrix}
                     e^{-2t} & e^{2t} \\
                     -2e^{-2t} & 2e^{2t} \\
                   \end{matrix}\right| \\
                 &= 4 \neq 0,
  \end{align}
  temos que
  \begin{equation}
    y(t) = c_1e^{-2t} + c_2e^{2t}
  \end{equation}
  é solução geral de \eqref{eq:ex_ed2o_homcc}.
\end{ex}

\subsection{Raízes reais distintas}

Uma EDO da forma
\begin{equation}
  {\color{blue}ay'' + by' + cy = 0}
\end{equation}
tem \emph{solução geral}
\begin{equation}
  y(t) = c_1e^{r_1t} + c_2e^{r_2t}
\end{equation}
quando sua \emph{equação característica}
\begin{equation}
  {\color{blue}ar^2 + br + c = 0}
\end{equation}
tem $r_1$ e $r_2$ como suas raízes reais distintas.

\begin{ex}
  Vamos resolver o seguinte PVI
  \begin{align}
    y'' - 3y' + 2 = 0,\\
    y(0) = 3,\quad y'(0) = 5.
  \end{align}

  Começamos resolvendo a equação característica associada
  \begin{equation}
    r^2 -3r + 2 = 0.
  \end{equation}
  As soluções são
  \begin{align}
    r &= \frac{3 \pm \sqrt{9 - 8}}{2} \\
      &= \frac{3 \pm 1}{2}.
  \end{align}
  Ou seja, $r_1 = 1$ e $r_2 = 2$. Logo,
  \begin{equation}
    y(t) = c_1e^t + c_2e^{2t}
  \end{equation}
  é solução geral da EDO.

  Agora, aplicando as condições iniciais, temos
  \begin{align}
    y(0) = 3 &\Rightarrow c_1 + c_2 = 3,\\
    y'(0) = 5 &\Rightarrow c_1 + 2c_2 = 5.
  \end{align}
  Resolvendo este sistema linear, obtemos $c_1 = 1$ e $c_2 = 2$. Concluímos que
  \begin{equation}
    y(t) = e^t + 2e^{2t}
  \end{equation}
  é a solução do PVI.
\end{ex}

\subsection*{Exercícios resolvidos}

\begin{exeresol}
  Calcule a solução geral de
  \begin{equation}
    2y'' + 2y' - 4 = 0.
  \end{equation}
\end{exeresol}
\begin{resol}
  A equação característica associada é
  \begin{equation}
    2r^2 + 2r -4 = 0.
  \end{equation}
  Suas soluções são
  \begin{align}
    r &= \frac{-2 \pm \sqrt{4 - 4\cdot 2\cdot (-4)}}{2} \\
    &= \frac{-2 \pm 6}{4},
  \end{align}
  i.e. $r_1 = -2$ e $r_2 = 1$. Como a equação característica tem raízes reais distintas, concluímos que
  \begin{equation}
    y(t) = c_1e^{-2t} + c_2e^{t}
  \end{equation}
  é solução geral da EDO.
\end{resol}

\begin{exeresol}
  Mostre que se $y_1(t) = e^{-2t}$ e $y_2(t) = e^t$, então o wronskiano
  \begin{equation}
    W(y_1,y_2; t) \neq 0.
  \end{equation}
\end{exeresol}
\begin{resol}
  Calculamos
  \begin{align}
    W(y_1,y_2; t) &= \left|
                    \begin{matrix}
                      y_1 & y_2 \\
                      y_1' & y_2'
                    \end{matrix}
                             \right|
                          &= \left|
                            \begin{matrix}
                              e^{-2t} & e^t \\
                              -2e^{-2t} & e^{t}
                            \end{matrix} \right| \\
                  &= e^{-2t}e^t + 2e^{-2t}e^t \\
                  &= 3e^{-t}.
  \end{align}
  Como $e^{-t} \neq 0$ para todo $t$, temos que $W(y_1,y_2; t)\neq 0$ para todo $t$.
\end{resol}

\subsection*{Exercícios}

\begin{exer}
  Calcule a solução geral de
  \begin{equation}
    -2y'' + 2y' + 4y = 0.
  \end{equation}
\end{exer}
\begin{resp}
  $y(t) = c_1e^{-t} + c_2e^{2t}$
\end{resp}

\begin{exer}
  Resolva o seguinte PVI
  \begin{align}
    &y'' = 7y' - 12,\\
    &y(0) = 0,\quad y'(0) = -1.
  \end{align}
\end{exer}
\begin{resp}
  $y(t) = e^{3t} - e^{4t}$
\end{resp}

\begin{exer}
  Resolva o seguinte PVI
  \begin{align}
    &y'' - 3y' + 2y = 0,\\
    &y(\ln 2) = -2,\quad y'(\ln 2) = -6.
  \end{align}
\end{exer}
\begin{resp}
  $y(t) = e^t - e^{2t}$
\end{resp}

\begin{exer}
  Calcule o wronskiano de $y_1(t) = \cos(t)$ e $y_2(t) = \sen(t)$.
\end{exer}
\begin{resp}
  $1$
\end{resp}

\begin{exer}
  Mostre que se $r_1$ e $r_2$ são raízes reais distintas da equação
  \begin{equation}
    ar^2 + br + c = 0,
  \end{equation}
  então
  \begin{equation}
    y(t) = c_1e^{r_1t} + c_2e^{r_2t}
  \end{equation}
  é solução geral de
  \begin{equation}
    ay'' + by + c = 0.
  \end{equation}
\end{exer}
\begin{resp}
  Mostre que $y_1(t) = e^{r_1t}$ e $y_2(t) = e^{r_2t}$ são soluções da EDO com $W(y_1,y_2; t) \neq 0$.
\end{resp}

\emconstrucao

