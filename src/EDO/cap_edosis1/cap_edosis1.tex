%Este trabalho está licenciado sob a Licença Atribuição-CompartilhaIgual 4.0 Internacional Creative Commons. Para visualizar uma cópia desta licença, visite http://creativecommons.org/licenses/by-sa/4.0/deed.pt_BR ou mande uma carta para Creative Commons, PO Box 1866, Mountain View, CA 94042, USA.

\chapter{Sistema de EDOs lineares de ordem 1}\label{cap_edosis1}

Neste capítulo, fazemos uma rápida introdução a sistemas de EDOs de primeira ordem, lineares e com coeficientes constantes. Ou seja, sistemas da forma
\begin{align}
  y_1'(t) &= a_{11}y_1(t) + a_{12}y_2(t) + \cdots + a_{1n}y_n(t) + g_1(t), \\
  y_2'(t) &= a_{21}y_1(t) + a_{22}y_2(t) + \cdots + a_{2n}y_n(t) + g_2(t), \\
  &\vdots \\
  y_n'(t) &= a_{n1}y_1(t) + a_{n2}y_2(t) + \cdots + a_{nn}y_n(t) + g_n(t),
\end{align}
onde $n>1$ é o número de equações, $\pmb{y}(t) = \left(y_1(t), y_2(t),\dotsc, y_n(t)\right)$ é o vetor das incógnitas, $A = [a_{ij}]_{i,j=1}^{n,n}$ é a matriz dos coeficientes e $\pmb{g}(t) = (g_1(t), g_2(t), \dotsc, g_n(t))$ é o vetor das fontes.

\section{Sistema de equações homogêneas}\label{cap_edosis1_sec_eqh}

Nesta seção, discutimos sobre um método de solução para sistemas de EDOs de primeira ordem, lineares, com coeficientes constantes e homogêneas. Ou seja, sistema da forma
\begin{equation}\label{eq:edosis1_eqh}
  {\color{blue}\pmb{y}'(t) = A\pmb{y}(t)},
\end{equation}
onde $\pmb{y}(t) = \left(y_1(t), y_2(t),\dotsc, y_n(t)\right)$, $n>1$, é o vetor das incógnitas e $A = [a_{ij}]_{i,j=1}^{n,n}$ é a matriz dos coeficientes.

O método consiste em buscar soluções da forma
\begin{equation}
  {\color{blue}\pmb{y} = \pmb{v}e^{rt}},
\end{equation}
onde $r$ e o vetor constante $\pmb{v} = (v_1, v_2, \cdots, v_n)$ devem ser determinados.

Substituindo em \eqref{eq:edosis1_eqh}, obtemos
\begin{align}
  r\pmb{v}e^{rt} &= A\pmb{v}e^{rt} \\
  A\pmb{v}e^{rt} - r\pmb{v}e^{rt} &= \pmb{0} \\
  (A - rI)\pmb{v}\underbrace{e^{rt}}_{>0} &= \pmb{0},
\end{align}
ou seja, temos que $r$ e $\pmb{v}$ devem tais que
\begin{equation}\label{eq:edosis1_eqc}
  {\color{blue}(A - rI)\pmb{v} = \pmb{0}}.
\end{equation}
Em outras palavras, $r$ é \emph{autovalor} e $\pmb{v}$ é \emph{autovetor} da matriz $A$.

Com isso, concluímos que se $r_1$ autovalor e $\pmb{v}_1$ autovetor de $A$, então
\begin{equation}
  \pmb{y}_1(t) = \pmb{v}_1e^{r_1t}
\end{equation}
é \emph{solução particular} de \eqref{eq:edosis1_eqh}. A \emph{solução geral} tem a forma
\begin{equation}
  {\color{blue}\pmb{y}(t) = c_1\pmb{y_1}(t) + c_2\pmb{y_2}(t) + \cdots + c_n\pmb{y_n}(t)},
\end{equation}
onde $\pmb{y_1}$, $\pmb{y_2}$, $\dotsc$, $\pmb{y}_n$ formam um conjunto fundamental de soluções de \eqref{eq:edosis1_eqh}, i.e. são soluções linearmente independentes.

\subsection{Autovalores reais distintos}

No caso da matriz dos coeficientes $A$ ter apenas todos os autovalores reais e dois a dois distintos, então a solução geral de \eqref{eq:edosis1_eqh} é
\begin{equation}
  {\color{blue}\pmb{y}(t) = c_1\pmb{v}_1e^{r_1t}+c_2\pmb{v}_2e^{r_2t}+\cdots + c_n\pmb{v}_ne^{r_nt}},
\end{equation}
onde $r_i$ e $\pmb{v}_i$ é autovalor e autovetor de $A$ respectivamente e $i=1, 2, \cdots, n$. A independência linear das soluções é garantida pelo wronskiano
\begin{equation}
  W(e^{r_1t}, \dotsc, e^{r_nt};t)\neq 0.
\end{equation}

\begin{ex}
  Vamos resolver o seguinte sistema
  \begin{align}\label{eq:ex_edosis1_eqh_avd}
    y_1'(t) &= -2y_1(t) + 2y_2(t) \\
    y_2'(t) &= -2y_1(t) + 3y_2(t).
  \end{align}

  Primeiramente, reescrevemos o sistema na sua forma matricial
  \begin{equation}
    \pmb{y}' =
    \underbrace{\begin{bmatrix}
      -2 & 2 \\
      -2 & 3
    \end{bmatrix}}_{A}
    \pmb{y},
  \end{equation}
  onde $\pmb{y}: t\mapsto \pmb{y}(t) = (y_1(t), y_2(t))$. Então, calculamos os autovalores da matriz dos coeficientes $A$. Para tanto, resolvemos sua equação característica
  \begin{gather}
    |A-rI| = 0 \\
    \left|
      \begin{bmatrix}
        -2 & 2 \\
        -2 & 3
      \end{bmatrix}
      -r
      \begin{bmatrix}
        1 & 0 \\
        0 & 1
      \end{bmatrix}
    \right| = 0 \\
    \begin{vmatrix}
      -2 - r & 2 \\
      -2 & 3 - r
    \end{vmatrix} = 0 \\
    (-2-r)(3-r) + 4 = 0 \\
    r^2 - r - 2 =  0,\\
    r = \frac{1 \pm \sqrt{9}}{2} \\
    r_1=-1,\quad r_2=2.
  \end{gather}
  Obtidos os autovalores, calculamos os autovetores $\pmb{v}_1$ e $\pmb{v}_2$. Começamos calculando o autovetor associado a $r_1$.
  \begin{gather}
    (A-r_1I)\pmb{v}_1 = \pmb{0} \\
    \begin{bmatrix}
      -2-r_1 & 2 \\
      -2 & 3-r_1
    \end{bmatrix}
    \begin{bmatrix}
      v_{11} \\
      v_{21}
    \end{bmatrix} =
    \begin{bmatrix}
      0 \\
      0
    \end{bmatrix} \\
        \begin{bmatrix}
      -1 & 2 \\
      -2 & 4
    \end{bmatrix}
    \begin{bmatrix}
      v_{11} \\
      v_{21}
    \end{bmatrix} =
    \begin{bmatrix}
      0 \\
      0
    \end{bmatrix} \\
    \begin{bmatrix}
      -1 & 2 & | & 0 \\
      -2 & 4 & | & 0
    \end{bmatrix}
    \begin{matrix}
      \times 2 \\
      \leftarrow -
    \end{matrix} \\
    \begin{bmatrix}
      -1 & 2 & | & 0 \\
      0 & 0 & | & 0
    \end{bmatrix}    
  \end{gather}
Assim, temos
\begin{gather}
  -v_{11} + 2v_{21} = 0 \\
  v_{11} = 2v_{21},
\end{gather}
donde escolhemos
\begin{equation}
\pmb{v}_1 =
\begin{bmatrix}
  2 \\
  1
\end{bmatrix}.
\end{equation}
Com isso, temos obtido a solução particular
\begin{align}
  \pmb{y}_1(t) &= \pmb{v}_1e^{r_1t} \\
               &=
                 \begin{bmatrix}
                   2\\
                   1
                 \end{bmatrix}e^{-t}
\end{align}


Agora, de forma análoga, calculamos $\pmb{v}_2$, um autovetor associado a $r_2$.
\begin{gather}
  (A-r_2I)\pmb{v}_2 = \pmb{0} \\
  \begin{bmatrix}
    -2-r_2 & 2 & | & 0 \\
    -2 & 3-r_2 & | & 0
  \end{bmatrix} \\
  \begin{bmatrix}
    -4 & 2 & | & 0 \\
    -2 & 1 & | & 0
  \end{bmatrix}
  \begin{matrix}
    \leftarrow - \\
    \times 2
  \end{matrix}\\
  \begin{bmatrix}
    0 & 0 & | & 0 \\
    -2 & 1 & | & 0
  \end{bmatrix}\\
    -2v_{12} + v_{22} = 0 \\
  \pmb{v}_2 =
  \begin{bmatrix}
    1 \\
    2
  \end{bmatrix}
\end{gather}
Com isso, temos a solução particular
\begin{align}
  \pmb{y}_2(t) &= \pmb{v}_2e^{r_2t} \\
               &=
                 \begin{bmatrix}
                   1\\
                   2
                 \end{bmatrix}e^{2t}
\end{align}

De tudo isso, concluímos que a solução geral de \eqref{eq:ex_edosis1_eqh_avd} é
\begin{align}
  \pmb{y}(t) &= c_1\pmb{y}_1(t) + c_2\pmb{y}_2 \\
             &= c_1
               \begin{bmatrix}
                 2\\
                 1
               \end{bmatrix}e^{-t} + c_2
  \begin{bmatrix}
    1\\
    2
  \end{bmatrix}e^{2t}
\end{align}
ou, ainda,
\begin{align}
  y_1(t) &= 2c_1e^{-t} + c_2e^{2t} \\
  y_2(t) &= c_1e^{-t} + 2c_2e^{2t}
\end{align}

\ifispython
No \python, podemos computar os autovalores e autovetores da matriz $A$ com os seguintes comandos\footnote{Veja mais informações em \href{https://docs.sympy.org/latest/tutorial/matrices.html}{SymPy: Matrices}.}:
\begin{verbatim}
In : A = Matrix([[-2,2],[-2,3]])
In : A.eigenvects()
Out: 
[(-1, 1, [Matrix([
   [2],
   [1]])]), (2, 1, [Matrix([
   [1/2],
   [  1]])])]
\end{verbatim}

A solução do sistema \eqref{eq:ex_edosis1_eqh_avd} pode ser computada com os comandos:
\begin{verbatim}
In : t,C1,C2 = symbols('t,C1,C2')
In : y1,y2 = symbols('y1,y2', cls=Function)
In : se = (Eq(diff(y1(t),t),-2*y1(t)+2*y2(t)), \
...:        Eq(diff(y2(t),t),-2*y1(t)+3*y2(t)))
...: 
In : dsolve(se,[y1(t),y2(t)])
Out: 
[Eq(y1(t), 2*C1*exp(-t) + 2*C2*exp(2*t)),
 Eq(y2(t), C1*exp(-t) + 4*C2*exp(2*t))]
\end{verbatim}
\fi
\end{ex}

\subsection{Autovalores reais repetidos}

Um autovalor real duplo $r$ da matriz de coeficientes $A$ nos fornece duas soluções particulares para \eqref{eq:edosis1_eqh}, a saber
\begin{align}
  {\color{blue}\pmb{y}_1(t)} &{\color{blue}= \pmb{v}_1e^{rt}},\\
  {\color{blue}\pmb{y}_2(t)} &{\color{blue}= \pmb{v}_1te^{rt} + \pmb{v}_2e^{rt}},
\end{align}
onde $\pmb{v}_1$ é autovetor associado a $r$. Para encontrar $\pmb{v}_2$, substituímos $\pmb{y}_2$ em \eqref{eq:edosis1_eqh}, donde
\begin{gather}
  \pmb{y}_2' = A\pmb{y}_2 \\
  \pmb{v}_1e^{rt} + r\pmb{v}_1te^{rt} + r\pmb{v}_2e^{rt} = A\pmb{v}_1te^{rt} + A\pmb{v}_2e^{rt} \\
  (A-rI)\pmb{v}_1t + (A-rI)\pmb{v}_2 = \pmb{v}_1
\end{gather}
Segue que
\begin{align}
  {\color{blue}(A-rI)\pmb{v}_1} &{\color{blue}= \pmb{0}} \\
  {\color{blue}(A-rI)\pmb{v}_2} &{\color{blue}= \pmb{v}_1}
\end{align}

\begin{ex}
  Vamos calcular a solução geral de
  \begin{equation}\label{eq:ex_edosis1_eqh_avr}
    \pmb{y}' =
    \begin{bmatrix}
      -1 & 1 & 0 \\
      0 & -1 & -1 \\
      0 & 0 & 1
    \end{bmatrix}
    \pmb{y}.
  \end{equation}

  Neste caso, temos que $r_{1,2} = -1$ é autovalor duplo e $r_3=1$ é autovalor simples da matriz de coeficientes do sistema \eqref{eq:ex_edosis1_eqh_avr}.
  
  Associadas a $r_{1,2}$ buscamos por soluções particulares da forma
  \begin{align}
    \pmb{y}_1 = \pmb{v}_1e^{r_{1,2}t}, \\
    \pmb{y}_2 = \pmb{v}_1te^{r_{1,2}t} + \pmb{v}_2e^{r_{1,2}t}.
  \end{align}
  Calculamos $\pmb{v}_1$ resolvendo
  \begin{gather}
    (A - r_{1,2}I)\pmb{v}_1 = \pmb{0} \\
    \begin{bmatrix}
      -1-r_{1,2} & 1 & 0 & | & 0 \\
      0 & -1-r_{1,2} & -1 & | & 0 \\
      0 & 0 & 1-r_{1,2} & | & 0
    \end{bmatrix}\\
    \begin{bmatrix}
      0 & 1 & 0 & | & 0 \\
      0 & 0 & -1 & | & 0 \\
      0 & 0 &  2 & | & 0
    \end{bmatrix}
  \end{gather}
  Com isso, podemos escolher
  \begin{equation}
    \pmb{v}_1 =
    \begin{bmatrix}
      1 \\
      0 \\
      0
    \end{bmatrix}.
  \end{equation}
  Determinado $\pmb{v}_1$, calculamos $\pmb{v}_2$ com
  \begin{gather}
    (A-r_{1,2}I)\pmb{v}_2 = \pmb{v}_1 \\
    \begin{bmatrix}
      0 & 1 & 0 & | & 1 \\
      0 & 0 & -1 & | & 0 \\
      0 & 0 & 2 & | & 0
    \end{bmatrix}
  \end{gather}
  donde escolhemos
  \begin{equation}
    \pmb{v}_2 =
    \begin{bmatrix}
      1 \\
      1 \\
      0
    \end{bmatrix}.
  \end{equation}

  Agora, associada a $r_3$ temos uma solução particular da forma
  \begin{equation}
    \pmb{y}_3 = \pmb{v}_3e^{r_3t},
  \end{equation}
  onde
  \begin{gather}
    (A-r_3I)\pmb{v}_3 = \pmb{0} \\
    \begin{bmatrix}
      -1-r_3 & 1 & 0 & | & 0 \\
      0 & -1-r_3 & -1 & | & 0 \\
      0 & 0 & 1-r_3 & | & 0 
    \end{bmatrix}\\
    \begin{bmatrix}
      -2 & 1 & 0 & | & 0 \\
      0 & -2 & -1 & | & 0 \\
      0 & 0 & 0 & | & 0 
    \end{bmatrix}
  \end{gather}  
  Com isso, temos $-2v_{13}+v_{23}=0$, $-2v_{23}-v_{33}=0$. Ou seja, podemos escolher
  \begin{equation}
    \pmb{v_3} =
    \begin{bmatrix}
      -1 \\
      -2 \\
      4
    \end{bmatrix}.
  \end{equation}

  Com tudo isso, podemos concluir que a solução geral de \eqref{eq:ex_edosis1_eqh_avr} é
  \begin{align}
    \pmb{y}(t) &= c_1\pmb{v}_1e^{r_{1,2}t}+c_2\left(\pmb{v}_1t+\pmb{v}_2\right)e^{r_{1,2}t} + c_3\pmb{v}_3e^{r_3t} \\
    \pmb{y}(t) &= c_1
                 \begin{bmatrix}
                   1\\
                   0\\
                   0
                 \end{bmatrix}e^{-t} + c_2\left(
    \begin{bmatrix}
      1\\
      0\\
      0
    \end{bmatrix}t +
    \begin{bmatrix}
      1\\
      1\\
      0
    \end{bmatrix}\right)e^{-t} + c_3
    \begin{bmatrix}
      -1\\
      -2\\
      4
    \end{bmatrix}e^t.
  \end{align}
\end{ex}

\subsection{Autovalores complexos}

Vejamos o caso de ${\color{blue}r=\lambda  \pm \mu i}$ serem autovalores complexos da matriz de coeficientes do sistema \eqref{eq:edosis1_eqh}. Associados, temos autovetores da forma
\begin{align}
  {\color{blue}\pmb{v}} &{\color{blue}= \pmb{v}_1 + \pmb{v}_2i},\\
  {\color{blue}\overline{\pmb{v}}} &{\color{blue}= \pmb{v}_1 - \pmb{v}_2i}.
\end{align}
Por substituição direta, podemos verificar que
\begin{align}
  \pmb{u}_1(t) &= \pmb{v}e^{(\lambda + \mu i)t},\\
  \pmb{u}_2(t) &= \overline{\pmb{v}}e^{(\lambda - \mu i)t},
\end{align}
são soluções de \eqref{eq:edosis1_eqh}. Também, verifica-se que as partes real e imaginária de $\pmb{u}_1$ e ${u}_2$ são soluções reais de \eqref{eq:edosis1_eqh}. A fim determiná-las, usamos a fórmula de Euler\footnote{Leonhard Euler, 1707-1783, matemático suíço. Fonte: \href{https://en.wikipedia.org/wiki/Leonhard_Euler}{Wikipedia}.}, calculamos
\begin{align}
  \pmb{u}_1 &= (\pmb{v}_1 + \pmb{v}_2i)e^{(\lambda + \mu i)t} \\
  \pmb{u}_1 &= (\pmb{v}_1 + \pmb{v}_2i)e^{\lambda t}[\cos(\mu t) + i\sen(\mu t)] \\
  \pmb{u}_1 &= e^{\lambda t}[\pmb{v_1}\cos(\mu t)-\pmb{v}_2\sen(\mu t)] \\
            &= ie^{\lambda t}[\pmb{v_1}\sen(\mu t) + \pmb{v}_2\cos(\mu t)].
\end{align}
De forma análoga, verifica-se que $\pmb{u}_1 = \overline{\pmb{u}_2}$. Ou seja, as partes reais e imaginárias de $\pmb{u}_1$ e $\pmb{u}_2$ são
\begin{align}
  {\color{blue}\pmb{y}_1(t)} &{\color{blue}= e^{\lambda t}[\pmb{v_1}\cos(\mu t)-\pmb{v}_2\sen(\mu t)]}, \\
  {\color{blue}\pmb{y}_2(t)} &{\color{blue}= e^{\lambda t}[\pmb{v_1}\sen(\mu t)+\pmb{v}_2\cos(\mu t)]},
\end{align}
as quais são soluções linearmente independentes, particulares de \eqref{eq:edosis1_eqh}.

\begin{ex}
  Vamos calcular a solução geral de
  \begin{equation}\label{eq:ex_edosis1_eqh_avc}
    \pmb{y}' =
    \begin{bmatrix}
      1 & 1 \\
      -1 & 1
    \end{bmatrix}\pmb{y}.
  \end{equation}

  Começamos calculando os autovalores associados a matriz de coeficientes do sistema \eqref{eq:ex_edosis1_eqh_avc}. Podemos fazer isso como segue
  \begin{gather}
    |A-rI| = 0 \\
    \begin{vmatrix}
      1-r & 1 \\
      -1 & 1-r
    \end{vmatrix} = 0 \\
    (1-r)^2 + 1 = 0 \\
    r^2 - 2r + 2 = 0 \\
    r = 1\pm i
  \end{gather}

  Um autovetor associado a $r = 1 + i$ pode ser obtidos resolvendo-se
  \begin{gather}
    (A-rI)\pmb{v} = \pmb{0}
    \begin{bmatrix}
      1-r & 1 & | & 0 \\
      -1 & 1-r & | & 0
    \end{bmatrix} \\
    \begin{bmatrix}
      -i &  1 & | & 0 \\
      -1 & -i & | & 0
    \end{bmatrix}
    \begin{matrix}
      \times i \\
      \leftarrow +
    \end{matrix}\\
    \begin{bmatrix}
      -i & 1 & | & 0 \\
      0 & 0 & | & 0
    \end{bmatrix} \\
    -iw_1 + w_2 = 0.
  \end{gather}
  Com isso, podemos escolher o autovetor
  \begin{align}
    \pmb{v} &=
              \begin{bmatrix}
                1 \\
                i
              \end{bmatrix} \\
            &=
              \underbrace{\begin{bmatrix}
                1\\
                0
              \end{bmatrix}}_{\pmb{v}_1} + i
    \underbrace{\begin{bmatrix}
      0\\
      1
    \end{bmatrix}}_{\pmb{v}_2}
  \end{align}
  
  Desta forma, identificamos as soluções particulares
  \begin{align}
    \pmb{y}_1(t) &= e^t\left\{
    \begin{bmatrix}
      1\\
      0
    \end{bmatrix}\cos(t) -
    \begin{bmatrix}
      0\\
      1
    \end{bmatrix}\sen(t)\right\}\\
        \pmb{y}_2(t) &= e^t\left\{
    \begin{bmatrix}
      1\\
      0
    \end{bmatrix}\sen(t) +
    \begin{bmatrix}
      0\\
      1
    \end{bmatrix}\cos(t)\right\}
  \end{align}

  Concluímos que a solução geral de \eqref{eq:ex_edosis1_eqh_avc} é
  \begin{align}
    \pmb{y}(t) &= c_1\pmb{y}_1(t) + c_2\pmb{y}_2(t) \\
               &= c_1e^t\left\{
    \begin{bmatrix}
      1\\
      0
    \end{bmatrix}\cos(t) -
    \begin{bmatrix}
      0\\
      1
    \end{bmatrix}\sen(t)\right\} \\
    &+ c_2e^t\left\{
    \begin{bmatrix}
      1\\
      0
    \end{bmatrix}\sen(t) +
    \begin{bmatrix}
      0\\
      1
    \end{bmatrix}\cos(t)\right\}
  \end{align}
\end{ex}

\subsection*{Exercícios resolvidos}

\begin{exeresol}
  Resolva o seguinte PVI
  \begin{align}
    \pmb{y}' &=
    \begin{bmatrix}
      -3 & 1 \\
      1 & -3
    \end{bmatrix}\pmb{y},\\
    \pmb{y}(0) &=
                 \begin{bmatrix}
                   -1\\
                   1
                 \end{bmatrix}
  \end{align}
\end{exeresol}
\begin{resol}
  O primeiro passo é encontrar a solução geral de
  \begin{equation}
        \pmb{y}' = \begin{bmatrix}
      -3 & 1 \\
      1 & -3
    \end{bmatrix}\pmb{y}.
  \end{equation}

  Para tanto, calculamos os autovalores da matriz dos coeficientes.
  \begin{gather}
    |A-rI|=0\\
    \begin{vmatrix}
      -3-r & 1 \\
      1 & -3-r
    \end{vmatrix} = 0 \\
    (-3-r)^2-1=0 \\
    r^2+6r+8=0 \\
    r_1 = -4,\quad r_2=-2.
  \end{gather}

  Então, buscamos por autovetores associados.
  \begin{gather}
    (A-r_1I)\pmb{v}_1 = \pmb{0} \\
    \begin{bmatrix}
      1 & 1 & | & 0\\
      1 & 1 & | & 0
    \end{bmatrix}\\
    v_{11}+v_{21}=0\\
    \pmb{v}_1 =
    \begin{bmatrix}
      1\\
      -1
    \end{bmatrix}
  \end{gather}

    \begin{gather}
    (A-r_2I)\pmb{v}_2 = \pmb{0} \\
    \begin{bmatrix}
      -1 & 1 & | & 0\\
      1 & -1 & | & 0
    \end{bmatrix}
    \begin{matrix}
      \times 1 \\
      \leftarrow +
    \end{matrix}\\
    \begin{bmatrix}
      -1 & 1 & | & 0\\
      0 & 0 & | & 0
    \end{bmatrix}
    -v_{12}+v_{22}=0\\
    \pmb{v}_2 =
    \begin{bmatrix}
      1\\
      1
    \end{bmatrix}
  \end{gather}

  Com isso, temos a solução geral da EDO dada por
  \begin{align}
    \pmb{y}(t) &= c_1\pmb{v}_1e^{r_1t} + c_2\pmb{v}_2e^{r_2t} \\
               &= c_1
                 \begin{bmatrix}
                   1 \\
                   -1
                 \end{bmatrix}e^{-4t} + c_2
    \begin{bmatrix}
      1 \\
      1
    \end{bmatrix}e^{-2t}.
  \end{align}

  Agora, aplicamos a condição inicial.
  \begin{gather}
    \pmb{y}(0) =
    \begin{bmatrix}
      -1 \\
      1
    \end{bmatrix}\\
    c_1 \begin{bmatrix}
      1 \\
      -1
    \end{bmatrix} + c_2 \begin{bmatrix}
      1 \\
      1
    \end{bmatrix} =     \begin{bmatrix}
      -1 \\
      1
    \end{bmatrix} \\
    \begin{bmatrix}
      1 & 1 & | & -1 \\
      -1 & 1 & | & 1
    \end{bmatrix}
    \begin{matrix}
      \times 1 \\
      \leftarrow +
    \end{matrix}\\
    \begin{bmatrix}
      1 & 1 & | & -1 \\
      0 & 2 & | & 0
    \end{bmatrix}
  \end{gather}
  Segue que
  \begin{align}
    c_1 + c_2 &= -1\\
    2c_2 &= 0
  \end{align}
  donde, $c_1=-1$ e $c_2 = 0$.

  Concluímos que a solução do PVI é
  \begin{equation}
    \pmb{y}(t) =
    \begin{bmatrix}
      -1 \\
      1
    \end{bmatrix}e^{-4t}.
  \end{equation}

  \ifispython
  No \python, podemos computar a solução deste exercício com os seguintes comandos:
\begin{verbatim}
In : t,C1,C2 = symbols('t,C1,C2')
In : y1,y2 = symbols('y1,y2', cls=Function)
In : ed = (Eq(diff(y1(t),t),-3*y1(t)+y2(t)),\
...:       Eq(diff(y2(t),t),y1(t)-3*y2(t)))
...: 
In : ds = dsolve(ed,[y1(t),y2(t)])
In : cs = solve((Eq(ds[0].rhs.subs(t,0),-1), \
...:             Eq(ds[1].rhs.subs(t,0),1)),[C1,C2])
...: 
In : ds[0].subs(cs)
Out: Eq(y1(t), -exp(-4*t))

In : ds[1].subs(cs)
Out: Eq(y2(t), exp(-4*t))
\end{verbatim}
  \fi
\end{resol}

\begin{exeresol}
  Calcule a solução geral do seguinte sistema
  \begin{align}
    y_1' &= -y_1 + 2y_2 \label{eq:exeresol_edosis1_1}\\
    y_2' &= -2y_1 - y_2 + y_3 \label{eq:exeresol_edosis1_2}\\
    y_3' &= -y_3 + y_4 \label{eq:exeresol_edosis1_3}\\
    y_4' &= -y_4 \label{eq:exeresol_edosis1_4}
  \end{align}
\end{exeresol}
\begin{resol}
  Vamos reescrever o sistema na sua forma matricial.
  \begin{equation}
    \pmb{y}' =
    \begin{bmatrix}
      -1 & 2 & 0 & 0 \\
      -2 & -1 & 1 & 0 \\
      0 & 0 & -1 & 1 \\
      0 & 0 & 0 & -1
    \end{bmatrix}\pmb{y}
  \end{equation}

  Calculamos os autovalores da matriz dos coeficientes.
  \begin{gather}
    |A-rI| = 0 \\
    \begin{vmatrix}
      -1-r & 2 & 0 & 0 \\
      -2 & -1-r & 1 & 0 \\
      0 & 0 & -1-r & 1 \\
      0 & 0 & 0 & -1-r 
    \end{vmatrix} = 0 \\
    r^4 + 4r^3 + 10r^2 + 12r + 5 = 0
  \end{gather}
  Resolvendo esta equação característica, obtemos os autovalores $r_{1,2} = -1$ e $r_3,r_4 = -1 \pm 2i$.

  \begin{enumerate}[a)]
  \item $r_1,r_2=-1$:

    Soluções particulares associadas.
    \begin{align}
      \pmb{y}_1 = \pmb{v}_1e^{r_{1,2}t},\\
      \pmb{y}_2 = \pmb{v}_1te^{r_{1,2}t} + \pmb{v}_2e^{r_{1,2}t}
    \end{align}

   O vetor $\pmb{v}_1$ é autovetor associado a $r_{1,2}$.
    \begin{gather}
      (A-r_1I)\pmb{v}_1 = \pmb{0} \\
      \begin{bmatrix}
        0 & 2 & 0 & 0 & | & 0 \\
        -2 & 0 & 1 & 0 & | & 0 \\
        0 & 0 & 0 & 1 & | & 0 \\
        0 & 0 & 0 & 0 & | & 0
      \end{bmatrix}\\
      \pmb{v}_1 =
      \begin{bmatrix}
        1 \\
        0 \\
        2 \\
        0
      \end{bmatrix}
    \end{gather}

    O vetor $\pmb{v}_2$ é calculado como segue.
    \begin{gather}
      (A-r_{1,2}I)\pmb{v}_2 = \pmb{v}_1 \\
      \begin{bmatrix}
        0 & 2 & 0 & 0 & | & 1 \\
        -2 & 0 & 1 & 0 & | & 0 \\
        0 & 0 & 0 & 1 & | & 2 \\
        0 & 0 & 0 & 0 & | & 0
      \end{bmatrix}\\
      \pmb{v}_2 =
      \begin{bmatrix}
        1 \\
        \frac{1}{2} \\
        2 \\
        2
      \end{bmatrix}
    \end{gather}

  \item $r_3,r_4 = -1 + 2i$:

    Soluções particulares associadas.
    \begin{align}
      \pmb{y}_3 &= e^{-t}[\pmb{v_3}\cos(2t) - \pmb{v}_4\sen(2t)] \\
      \pmb{y}_4 &= e^{-t}[\pmb{v_3}\sen(2t) + \pmb{v}_4\sen(2t)]
    \end{align}

    Os vetores $\pmb{v}_3$ e $\pmb{v}_4$ são, respectivamente, as partes real e imaginária de autovetor associado a $r_3$ ou $r_4$. Usando $r_3$ e denotando o autovetor por $\pmb{v}$, calculamos como segue.
    \begin{gather}
      (A-r_3I)\pmb{v} = \pmb{0} \\
      \begin{bmatrix}
        -2i & 2 & 0 & 0 & | & 0 \\
        -2 & -2i & 1 & 0 & | & 0 \\
        0 & 0 & -2i & 1 & | & 0 \\
        0 & 0 & 0 & -2i & | & 0
      \end{bmatrix}\\
      \pmb{v} =
      \begin{bmatrix}
        1 \\
        i \\
        0 \\
        0
      \end{bmatrix} =
      \underbrace{\begin{bmatrix}
        1 \\
        0 \\
        0 \\
        0
      \end{bmatrix}}_{\pmb{v_1}} + i
    \underbrace{\begin{bmatrix}
        0 \\
        1 \\
        0 \\
        0
      \end{bmatrix}}_{\pmb{v}_2}
    \end{gather}
  \end{enumerate}

  De tudo isso, temos a solução geral
  \begin{align}
    \pmb{y}(t) &= c_1\pmb{y}_1(t) + c_2\pmb{y}_2 + c_3\pmb{y}_3(t) + c_4\pmb{y}_4 \\
    &= c_1
      \begin{bmatrix}
        1 \\
        0 \\
        2 \\
        0
      \end{bmatrix}e^{-t} \\
    &+ c_2\left\{
      \begin{bmatrix}
        1 \\
        0 \\
        2 \\
        0
      \end{bmatrix}t +
    \begin{bmatrix}
      1 \\
      \frac{1}{2} \\
      2 \\
      2
    \end{bmatrix}\right\}e^{-t} \\
    &+ c_3\left\{
      \begin{bmatrix}
        1 \\
        0 \\
        0 \\
        0
      \end{bmatrix}\cos(2t) -
    \begin{bmatrix}
      0 \\ 1 \\ 0 \\ 0
    \end{bmatrix}\sen(2t)\right\} \\
    &+ c_4\left\{
      \begin{bmatrix}
        1 \\
        0 \\
        0 \\
        0
      \end{bmatrix}\sen(2t) +
    \begin{bmatrix}
      0 \\ 1 \\ 0 \\ 0
    \end{bmatrix}\cos(2t)\right\}
  \end{align}
  ou, equivalentemente,
  \begin{align}
    y_1(t) &= c_1e^{-t} + c_2(t + 1)e^{-t} \\
           &+ c_3\cos(2t) + c_4\sen(2t) \\
    y_2(t) &= \frac{c_2}{2}e^{-t} - c_3\sen(2t) + c_4\cos(2t) \\
    y_3(t) &= 2c_1e^{-t} + c_2(2t + 2)e^{-t} \\
    y_4(t) &= 2c_2e^{-t}
  \end{align}
\end{resol}

\subsection*{Exercícios}

\begin{exer}
  Calcule a solução geral de
  \begin{align}
    y_1'(t) &= -y_1(t) + 2y_2(t) \\
    y_2'(t) &= 4y_1(t) + y_2(t)
  \end{align}
\end{exer}
\begin{resp}
  $y_1(t) = c_1e^{-3t} + c_2e^{3t}$, $y_2(t) = -c_1e^{-3t} + 2c_2e^{3t}$
\end{resp}

\begin{exer}
  Calcule a solução do PVI
  \begin{align}
    y_1'(t) &= 2y_1(t) + y_2(t),\quad y_1(0) = 0 \\
    y_2'(t) &= 4y_1(t) - y_2(t),\quad y_2(0) = 5
  \end{align}
\end{exer}
\begin{resp}
  $y_1(t) = e^{3t} - e^{-2t}$, $y_2(t) = e^{3t} + 4e^{-2t}$
\end{resp}

\begin{exer}
  Calcule a solução geral de
  \begin{equation}
    \pmb{y}' =
    \begin{bmatrix}
      1 & 1 \\
      -1 & 3
    \end{bmatrix}\pmb{y}
  \end{equation}
\end{exer}
\begin{resp}
  $\displaystyle \pmb{y}(t) = c_1
  \begin{bmatrix}
    1\\
    1
  \end{bmatrix}e^{2t} + 2c_2
  \begin{bmatrix}
    1\\
    1
  \end{bmatrix}te^{2t} + c_2
  \begin{bmatrix}
    -1\\
    1
  \end{bmatrix}e^{2t}$
\end{resp}

\begin{exer}
  Calcule a solução do PVI
  \begin{align}
    y_1'(t) &= -3y_1(t) + y_2(t),\quad y_1(0) = 2 \\
    y_2'(t) &= -y_1(t) - y_2(t),\quad y_2(0) = 1
  \end{align}
\end{exer}
\begin{resp}
  $y_1(t) = (-t+2)e^{-2t}$, $y_2(t) = (-t+1)e^{-2t}$
\end{resp}

\begin{exer}
  Encontre a solução geral de
  \begin{align}
    y_1' &= 2y_1 - y_2 \\
    y_2' &= 5y_1 - 2y_2
  \end{align}
\end{exer}
\begin{resp}
  $y_1(t) = -c_1\sen(t) - c_2\cos(t)$, $y_2(t) = (-2c_1-c_2)\sen(t) + (c_1 -2c_2)\cos(t)$
\end{resp}

\begin{exer}
  Encontre a solução geral de
  \begin{align}
    y_1' &= y_1 + 4y_2 - 6y_3 \\
    y_2' &= -y_1 -3y_2 + 3y_3 \\
    y_3' &= -y_1 -2y_2 + 2y_3
  \end{align}
\end{exer}
\begin{resp}
  $y_1(t) = -2c_1e^{-t}-2c_2te^{-t}+3c_2e^{-t}-2c_3e^{2t}$, $y_2(t)=c_1e^{-t}+c_2te^{-t}+c_3e^{2t}$, $y_3(t) = c_2e^{-t}+c_3e^{2t}$
\end{resp}


\section{Sistema de equações não homogêneas}\label{cap_edosis1_sec_eqnh}

\emconstrucao

\subsection{Método dos coeficientes a determinar}

\emconstrucao

\subsection*{Exercícios resolvidos}

\emconstrucao

\subsection*{Exercícios}

\emconstrucao
