%Este trabalho está licenciado sob a Licença Atribuição-CompartilhaIgual 4.0 Internacional Creative Commons. Para visualizar uma cópia desta licença, visite http://creativecommons.org/licenses/by-sa/4.0/deed.pt_BR ou mande uma carta para Creative Commons, PO Box 1866, Mountain View, CA 94042, USA.

\chapter{Introdução}\label{cap_intro}
\thispagestyle{fancy}

Neste capítulo, introduzimos conceitos e definições elementares sobre Equações Diferenciais (ED). Por exemplo, definimos tais equações, apresentamos alguns exemplos de modelagem matemática e problemas relacionados.

\ifispython
\begin{obs}\label{obs:python}
Ao longo das notas de aula, contaremos com o suporte de alguns códigos \python~ com o seguinte preâmbulo:
\begin{verbatim}
from sympy import *
\end{verbatim}
\end{obs}
\fi


\section{Equações diferenciais}\label{cap_intro_sec_ed}

\emph{Equação Diferencial (ED)} é o nome dado a qualquer equação que tenha pelo menos um termo envolvendo a diferenciação (derivação) de uma incógnita.

\begin{ex}\label{ex:ed}
  São exemplos de equações diferenciais:
  \begin{enumerate}[a)]
  \item Modelo de queda de um corpo com resistência do ar.
    \begin{equation}\label{eq:ex_ed_queda}
      \frac{dv}{dt} = g - \frac{k}{m}v^2.
    \end{equation}
    Nesta equação, temos a velocidade $v = v(t)$ ($v$ função de $t$) como \emph{incógnita}. O tempo é descrito por $t$ como uma variável independente. As demais letras correspondem a parâmetros dados (constantes). Mais especificamente, $g$ corresponde à gravidade, $k$ à resistência do ar e $m$ à massa do corpo.
  \item Equação de Verhulst (Equação Logística)
    \begin{equation}\label{eq:ex_ed_Verhulst}
      \frac{dy}{dt} = r\left(1 - \frac{y}{K}\right)y.
    \end{equation}
    Esta equação é um clássico modelo de crescimento populacional. Aqui, $y = y(t)$ é o tamanho da população (incógnita) no tempo $t$ (variável independente). As demais letras correspondem a parâmetros dados.
  \item Equação de Schrödinger.
    \begin{equation}\label{eq:ex_ed_Schroedinger}
      -\frac{\hbar}{2m}\frac{d^2\psi}{dx^2} + \frac{kx^2}{2}\psi = E\psi.
    \end{equation}
    Esta equação modela a função de onda $\psi$ (incógnita) de uma partícula em função de sua posição $x$ (modelo unidimensional). Neste modelo quântico, $\hbar$, $m$, $k$ e $E$ são parâmetros.
  \item Modelagem da corrente em um circuito elétrico.
    \begin{equation}
      L\frac{d^2I}{dt^2} + R\frac{dI}{dt} + \frac{1}{C}I = E.
    \end{equation}
    Aqui, a incógnita é função corrente $I$ em função do tempo. O modelo refere-se a um circuito elétrico com os seguintes parâmetros: $L$ indutância, $R$ resistência, $C$ capacitância e $E$ voltagem do gerador.
  \item Equação do calor.
    \begin{equation}
      \frac{\p u}{\p t} = \alpha\frac{\p^2 u}{\p x^2}.
    \end{equation}
    Esta equação modela a distribuição de temperatura (incógnita) $u = u(t,x)$ como função do tempo e da posição (variáveis independentes). O parâmetro é o coeficiente de difusão térmica $\alpha$.
  \end{enumerate}
\end{ex}

\emph{Equação Diferencial Ordinária (EDO)} é aquela em a incógnita é função apenas de uma variável independente. Desta forma, todas as derivadas que aparecem na equação são ordinárias. No Exemplo \ref{ex:ed}, as equações diferenciais a), b), c) e d) são ordinárias. A equação e) não é ordinária, pois a incógnita $u = u(t,x)$ é função das varáveis independentes $t$ e $x$, portanto, os termos diferenciais são parciais (derivadas parciais). Equações como esta são chamadas de equações diferenciais parciais.

Toda EDO pode ser escrita na seguinte forma geral
\begin{equation}\label{eq:edo_geral}
  F(t, y, y', y'', \cdots, y^{(n)}) = 0.
\end{equation}
Aqui, $F$ é uma função envolvendo a variável independente $t$ e a variável dependente $y = y(t)$ (incógnita, função de $t$) e pelo menos uma derivada ordinária de $y$ em relação a $t$\footnote{Lembre-se que $\displaystyle y' = \frac{dy}{dt}$, $y'' = \frac{d^2y}{dt^2}$ e assim por diante.}. O índice $n$ corresponde a \emph{ordem} da derivada de maior ordem que aparece na equação, sendo $n\geq 1$. Quando $F$ é função linear das variáveis $y$, $y'$, $\cdots$, $y^{(n)}$, então a EDO é dita ser \emph{linear}, caso contrário, é \emph{não linear}. Quando $F$ não dependente explicitamente de $t$, a equação é dita ser \emph{autônoma}.

\begin{ex}
  Vejamos os seguintes casos:
  \begin{enumerate}[a)]
  \item A equação
    \begin{equation}
      y'' + y = 0
    \end{equation}
    é uma EDO de ordem $2$, linear e autônoma. Aqui, temos $F(y, y'') = y'' + y$.
  \item As equações \eqref{eq:ex_ed_queda} e \eqref{eq:ex_ed_Verhulst} são EDOs de \emph{primeira ordem} (de ordem $1$), autônomas e não lineares.
  \item A Equação de Schrödinger \eqref{eq:ex_ed_Schroedinger} é uma EDO de \emph{segunda ordem}, linear e não autônoma.
  \end{enumerate}
\end{ex}

Uma \emph{solução} de uma EDO \eqref{eq:edo_geral} é uma função $y = y(t)$ que satisfaça a equação para todos os valores de $t$\footnote{Em várias situações o domínio de interesse de $t$ é também informado junto com a equação. Veremos isso mais adiante.}.

\begin{ex}
  As funções $y_1(t) = e^t$ e $y_2(t) = e^{-t}$ são soluções da equação diferencial ordinária
  \begin{equation}
    y'' - y = 0.
  \end{equation}
  De fato, tomando $y = y_1(t) = e^t$, temos $y'' = e^t$ e
  \begin{equation}
    y'' - y = e^t - e^t = 0
  \end{equation}
  para todo $t$.
  Também, tomando $y = y_2(t) = e^{-t}$, temos $y'' = e^{-t}$ e
  \begin{equation}
    y'' - y = e^{-t} - e^{-t} = 0, \quad\forall t.
  \end{equation}

  \ifispython
  No \python, podemos usar:
\begin{verbatim}
In : t = symbols('t')
In : y = symbols('y', cls=Function)
In : de = Eq(Derivative(y(t),t,t)-y(t) ,0)
In : y1 = exp(t)
In : checkodesol(de,y1)
Out: (True, 0)
In : y2 = exp(-t)
In : checkodesol(de,y2)
Out: (True, 0)
\end{verbatim}
  \fi
\end{ex}

\subsection*{Exercícios resolvidos}

\begin{exeresol}
  Determine a ordem e diga se a seguinte EDO é linear ou autônoma. Justifique suas respostas.
  \begin{equation}
    t^2\frac{dy}{dt} + (1+y^2)\frac{d^2y}{dt} + y = e^t.
  \end{equation}
\end{exeresol}
\begin{resol}
  \begin{enumerate}[a)]
  \item Ordem 2.

    A equação tem ordem 2, pois o termo diferencial de maior ordem é uma derivada de segunda ordem.

  \item EDO é não linear.

    A equação tem um termo $\displaystyle y^2\frac{d^2y}{dt^2}$, o qual não é linear em $y$.

  \item EDO não é autônoma.

    A equação não é autônoma, pois a variável independente $t$ aparece explicitamente. A saber, no primeiro termo do lado esquerdo e no termo fonte da equação.
  \end{enumerate}
\end{resol}

\begin{exeresol}
  Determine os valores de $r$ para os quais $y = e^{rt}$ é solução da equação
  \begin{equation}
    y'' - y = 0.
  \end{equation}
\end{exeresol}
\begin{exeresol}
  Para que $y = e^{rt}$ seja solução da equação dada, devemos ter
  \begin{align}
    y'' - y = 0 &\Rightarrow \left(e^{rt}\right)'' - e^{rt} = 0 \\
                &\Rightarrow r^2e^{rt} - e^{rt} = 0 \\
                &\Rightarrow (r^2 - 1)\cdot \underbrace{e^{rt}}_{>0} = 0\\
                &\Rightarrow r^2 - 1 = 0 \\
                &\Rightarrow r = \pm 1.
  \end{align}
  \ifispython
  No \python, podemos computar a solução deste problema com os seguintes comandos:
\begin{verbatim}
In : r = symbols('r')
In : eq = diff(exp(r*t),t,2)-exp(r*t)
In : solve(eq,r)
Out: [-1, 1]
\end{verbatim}
  \fi
\end{exeresol}

\subsection*{Exercícios}

\begin{exer}
  Determine quais das seguintes são EDOs. Justifique sua resposta.
  \begin{enumerate}[a)]
  \item $\displaystyle y = y''$.
  \item $\displaystyle \frac{\p y}{\p t} = \frac{1}{2}\frac{\p y}{\p x}$.
  \item $\displaystyle y\cdot \frac{d^5y}{dx^5} = x\ln(y) + \frac{d}{dx}e^{x^2}$.
  \item $u_{tt} = \alpha^2u_{xx}$, sendo $\alpha$ um parâmetro.
  \end{enumerate}
\end{exer}
\begin{resp}
  a), c)
\end{resp}

\begin{exer}\label{exer:edo_ordem}
  Determine a ordem das seguintes EDOs. Justifique sua resposta.
  \begin{enumerate}[a)]
  \item $\displaystyle t^2y' = e^{t}$.
  \item $\displaystyle \frac{d^2y}{dt^2} = \frac{d^3y}{dt^3}$.
  \item $\displaystyle y\cdot y'' - 3y'' = y - y'$.
  \item $\displaystyle \left(\frac{d^2y}{dt^2}\right)^2 = e^t$.
  \end{enumerate}
\end{exer}
\begin{resp}
  a)~$1$; b)~$3$; c)~$2$; d)~$2$.
\end{resp}

\begin{exer}
  Determine quais das equações do Exercício \ref{exer:edo_ordem} não são autônomas. Justifique sua resposta.
\end{exer}
\begin{resp}
  a), d).
\end{resp}

\begin{exer}
  Determine quais das equações do Exercício \ref{exer:edo_ordem} são lineares. Justifique sua resposta.
\end{exer}
\begin{resp}
  a), b).
\end{resp}

\begin{exer}
  Para cada equação a seguir, calcule os valores de $r$ para os quais $y = e^{rt}$ seja solução da equação.
  \begin{enumerate}[a)]
  \item $y'' + y' - 6y = 0$.
  \item $y''' = 3y''$.
  \end{enumerate}
\end{exer}
\begin{resp}
  a)~$\{-3, 2\}$; b)~$\{0, 3\}$
\end{resp}

\begin{exer}
  Calcule os valores de $\alpha$ para os quais $y = t^\alpha$, $t>0$, seja solução da equação
  \begin{equation}
    t^2y'' = 2y.
  \end{equation}
\end{exer}
\begin{resp}
  $\{-1, 2\}$.
\end{resp}


\section{Problemas de valores iniciais e de valores de contorno}\label{cap_intro_sec_pv}

Uma Equação Diferencial Ordinária (EDO) pode ter infinitas soluções.

\begin{ex}
A EDO
\begin{equation}
  y' = 1
\end{equation}
tem soluções
\begin{equation}
  \int y'\,dt = \int 1\cdot dt \Rightarrow y = t + c,
\end{equation}
onde $c$ é uma constante indeterminada.
\end{ex}

Afim de fixar uma solução única para tais EDOs, comumente define-se uma \emph{condição inicial} apropriada, i.e. o valor da solução para um dado valor da variável independente. O problema de resolver uma EDO com condição inicial dada é chamado de \emph{Problema de Valor Inicial} (PVI).

\begin{ex}
  No exemplo anterior, $t$ é a variável independente. Assim, por exemplo,
\begin{equation}
  y(t_0) = y(0) = 1
\end{equation}
é um exemplo de uma condição inicial. Neste caso, determinamos a constante $c$ com
\begin{align}
  y(t) = t + c &\Rightarrow y(0) = 0 + c = 1 \\
               &\Rightarrow c = 1.
\end{align}
Ou seja, a solução deste problema de valor inicial é $y(t) = t + 1$.
\end{ex}


EDOs de segunda ordem podem requer duas condições iniciais.

\begin{ex}
Consideramos o seguinte problema de valores iniciais
\begin{align}
  &y'' = 1,\\
  &y(1) = 0,\quad y'(1) = 1.
\end{align}
Integrando a EDO, obtemos
\begin{equation}
  \int y''\,dt = \int 1\cdot dt \Rightarrow y' = t + c_1.
\end{equation}
Integrando novamente
\begin{equation}
  \int y'\,dt = \int t + c_1\,dt \Rightarrow y = \frac{t^2}{2} + c_1t + c_2.
\end{equation}
Com isso, obtemos a chamada \emph{solução geral} desta EDO
\begin{equation}
  y(t) = \frac{t^2}{2} + c_1t + c_2.
\end{equation}
Agora, aplicando as condições de contorno, obtemos
\begin{align}
  y(1) = 0 &\Rightarrow \frac{1}{2} + c_1 + c_2 = 0\\
  y'(1) = 1 &\Rightarrow 1 + c_1 = 1.
\end{align}
Da segunda condição, obtemos $c_1 = 0$. Logo, da primeira, obtemos $c_2 = -\frac{1}{2}$. Portanto, a solução deste PVI de ordem 2 é:
\begin{equation}
  y(t) = \frac{t^2}{2} - \frac{1}{2}.
\end{equation}
\end{ex}


\begin{obs}
  Observe que o número de condições iniciais é igual à ordem da EDO.
\end{obs}

No caso de EDOs de ordem 2, também podemos fixar uma solução através da aplicação de \emph{condições de contorno}. Neste caso, estamos interessados em obter a solução para valores da variável independente restritos a um intervalo fechado $[t_0, t_1]$. A solução é fixada pela determinação de seus valores nos pontos $t_0$ e $t_1$. O problema de encontrar a solução de uma EDO com condições de contorno, é chamado de \emph{Problema de Valor de Contorno (PVC)}.

\begin{ex}
  Consideramos o seguinte problema de valores de contorno
  \begin{align}
    &y'' = 1,\quad 0 < t < 1,\\
    &y(0) = 1,\quad y(1) = \frac{1}{2}.
  \end{align}
  Integrando duas vezes a EDO, obtemos a solução geral
  \begin{equation}
    y(t) = \frac{t^2}{2} + c_1t + c_2.
  \end{equation}
  Agora, aplicando as condições de contorno, obtemos
  \begin{align}
    &y(0) = 1 \Rightarrow c_2 = 1,\\
    &y(1) = \frac{1}{2} \Rightarrow \frac{1}{2} + c_1 = \frac{1}{2} \Rightarrow c_1 = 0.
  \end{align}
  Desta forma, temos que a solução do PVC é
  \begin{equation}
    y(t) = \frac{t^2}{2} + 1.
  \end{equation}
\end{ex}

\begin{obs}
  O número de constantes indeterminadas na solução geral está relacionado à ordem da EDO.
\end{obs}

\subsection*{Exercícios resolvidos}

\begin{exeresol}
  Encontre a solução do seguinte problema de valor inicial (PVI)
  \begin{align}
    &y' = t + 1,\quad t>0,\\
    &y(0) = 2.
  \end{align}
\end{exeresol}
\begin{resol}
  Integrando a EDO obtemos
  \begin{equation}
    \int y'\,dt = \int t+1\,dt \Rightarrow y(t) = \frac{t^2}{2} + t + c,
  \end{equation}
  a qual é a solução geral da EDO.

  Então, aplicando a condição inicial $y(0) = 2$, obtemos
  \begin{equation}
    c = 2.
  \end{equation}
  Logo, a solução do PVC é $y(t) = \frac{t^2}{2} + t + 2$.
\end{resol}

\begin{exeresol}
  Encontre a solução do seguinte problema de valor de contorno (PVC)
  \begin{align}
    y'' = t + 1,\quad -1 < t < 1, \\
    y(-1) = y(1) = 0.
  \end{align}
\end{exeresol}
\begin{resol}
  Integrando duas vezes a EDO, obtemos
  \begin{align}
    y'' = t + 1 &\Rightarrow \int y''\,dt = \int t+1\,dt \\
                &\Rightarrow y' = \frac{t^2}{2} + t + c_1 \\
                &\Rightarrow \int y'\,dt = \int \frac{t^2}{2} + t + c_1 \,dt \\
                &\Rightarrow y(t) =  \frac{t^3}{6} + \frac{t^2}{2} + c_1t + c_2.
  \end{align}
  Obtida a solução geral da EDO, aplicamos as condições de contorno
  \begin{align}
    &y(-1) = 0 \Rightarrow -\frac{1}{6} + \frac{1}{2} - c_1 + c_2 = 0\\
    &y(1) = 0 \Rightarrow \frac{1}{6} + \frac{1}{2} + c_1 + c_2 = 0.
  \end{align}
  Ou seja, precisamos resolver o seguinte sistema linear
  \begin{align}
    -c_1 + c_2 = -\frac{1}{3} \\
    c_1 + c_2 = \frac{2}{3}.
  \end{align}
  Resolvendo, obtemos $c_1 = \frac{1}{2}$ e $c_2 = \frac{1}{6}$.
\end{resol}

\begin{exeresol}
  Determine o valor de $x$ para o qual a solução do PVI
  \begin{align}
    &\frac{dy}{dx} = \frac{2-e^x}{1 + y^2},\quad x>0,\\
    &y(0)=0,
  \end{align}
  atinge seu valor máximo.
\end{exeresol}
\begin{resol}
  Lembramos que a monotonicidade de $y = y(x)$ pode ser analisada a partir do estudo de sinal de $dy/dx$. Fazendo o estudo de sinal de
  \begin{equation}
    \frac{dy}{dx} = \frac{2-e^x}{1+y^2},
  \end{equation}
  vemos que $dy/dx > 0$ para $x\in (0, \ln 2)$ e $dy/dx < 0$ para $x\in (\ln 2, \infty)$. Logo, temos que $y = y(x)$ é crescente em $[0, \ln 2]$ e decrescente em $[\ln 2, \infty)$. Desta forma, concluímos que a solução do PVI atinge seu valor máximo em $x = \ln 2$.
\end{resol}

\subsection*{Exercícios}

\begin{exer}
  Resolva o seguinte PVI
  \begin{equation}
    y' = 0,\quad y(-1) = 1.
  \end{equation}
\end{exer}
\begin{resp}
  $y(t) = 1$.
\end{resp}

\begin{exer}
  Resolva o seguinte PVI
  \begin{equation}
    y' = t,\quad y(-1) = 1.
  \end{equation}
\end{exer}
\begin{resp}
  $y(t) = \frac{t^2}{2} + \frac{1}{2}$.
\end{resp}

\begin{exer}
  Resolva o seguinte PVC
  \begin{align}
    &y'' = 1,\\
    &y(0)=1,\quad y(1)=-1.
  \end{align}
\end{exer}
\begin{resp}
  $y(t) = \frac{t^2}{2} - \frac{5}{2}t + 1$.
\end{resp}

\begin{exer}
  Resolva o seguinte PVC
  \begin{align}
    &y'' = \sen(t),\\
    &y(-\pi)=y(\pi)=0.
  \end{align}
\end{exer}
\begin{resp}
  $y(t) = -\sen(t)$.
\end{resp}